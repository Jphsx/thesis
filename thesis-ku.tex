%% LyX 2.3.1 created this file.  For more info, see http://www.lyx.org/.
%% Do not edit unless you really know what you are doing.
\documentclass[12pt,english,letterpaper]{kuthesis}
%\usepackage{mathptmx}
\renewcommand{\sfdefault}{lmss}
\renewcommand{\ttdefault}{lmtt}

\usepackage[T1]{fontenc}
\usepackage[utf8]{inputenc}
\usepackage{geometry}
\geometry{verbose,tmargin=1in,bmargin=1in,lmargin=1in,rmargin=1in}
\setcounter{secnumdepth}{3}
\setcounter{tocdepth}{3}
%\usepackage{color}
\usepackage{babel}
\usepackage{url}
\usepackage{graphicx}
\usepackage{setspace}
%\usepackage{esint}
\usepackage{chngpage}
\usepackage{multirow}



%% packages ripped from AN
\usepackage{xspace}
\usepackage{xcolor}
%\usepackage{cancel}
%\usepackage{graphicx}
%%\usepackage{subcaption}
\usepackage{amssymb}
\usepackage{amsmath}
%\usepackage[section]{placeins}
%\usepackage{hepnames}
%\usepackage{hyperref}
%\usepackage{cite}
%\usepackage{makecell}
%\usepackage{ulem}
%\usepackage{epstopdf}
%\usepackage{multirow}
%\usepackage{longtable}
%%
\usepackage{booktabs}% http://ctan.org/pkg/booktabs
\newcommand{\tabitem}{~~\llap{\textbullet}~~}
\usepackage{tabularx}

\usepackage{makecell}
%\renewcommand{\cellalign}{vh}
%\renewcommand{\theadalign}{vh}

\newcommand{\ttbar}{\ensuremath{{\PQt{}\PAQt}}\xspace}
\newcommand{\ttjets}{\ensuremath{{\PQt{}\PAQt}+\mathrm{jets}}\xspace}
\newcommand{\ttx}{\ensuremath{{\PQt{}\PAQt}\mathrm{X}+\mathrm{jets}}\xspace}
\newcommand{\zdy}{\ensuremath{\PZ/\PGg^{*}+\mathrm{jets}}\xspace}
\newcommand{\Wjets}{\ensuremath{\PW{}+\mathrm{jets}}\xspace}
\newcommand{\Znunu}{\ensuremath{\PZ(\to\nu\bar{\nu})+\mathrm{jets}}\xspace}
\newcommand{\et}{E_{\mathrm{T}}}
\newcommand{\alqed}{\alpha_{\mathrm{em}}}
\newcommand{\GF}{G_{\mathrm{F}}}
\newcommand{\mtop}{M_{\PQt{}}}
\newcommand{\mH}{M_{\PH{}}}
\newcommand{\mX}{M_{\mathrm{X}}}
\newcommand{\mW}{M_{\PW{}}}
\newcommand{\mZ}{M_{\PZ{}}}
\newcommand{\GZ}{\ensuremath{\Gamma_{\PZ}}}
\newcommand{\GW}{\ensuremath{\Gamma_{\PW}}}
\newcommand{\WW}{\ensuremath{\PWp\PWm}}
\newcommand{\ff}{\mathrm{f}\mathrm{\overline{f}}}
\newcommand{\bb}{\mathrm{b}\mathrm{\overline{b}}}
\newcommand{\Gt}{\Gamma_{\mathrm{t}}}
\newcommand{\mt}{m_{\mathrm{t}}}
\newcommand{\alr}{A_{LR}}
\newcommand{\ee}{\ensuremath{\Pep\Pem}}
\newcommand{\nZhad}{\ensuremath{n_{\PZ^{\mathrm{had}}}}}
\newcommand{\mumu}{\ensuremath{\Pgmp{}\Pgmm}}
\newcommand{\pipi}{\ensuremath{\Pgpp{}\Pgpm}}
\newcommand{\pip}{\ensuremath{\Pgpm{}\Pp}}
\newcommand{\kpi}{\ensuremath{\PKm{}\Pgpp}}
\newcommand{\sweff}{{\sin^2{\theta}^{\ell}_{\mathrm{eff}}}}
\newcommand{\sw}{\sin^2{\theta}_{\mathrm{W}}}
\newcommand{\ECM}{\sqrt{s}}
\newcommand{\percmsqs}{\mathrm{cm}^{-2}\mathrm{s}^{-1}}
\newcommand{\fbinv}{\mbox{\ensuremath{\,\text{fb}^{-1}}}\xspace}
\newcommand{\abinv}{\mbox{\ensuremath{\,\text{ab}^{-1}}}\xspace}
\newcommand{\MeV}{\ensuremath{\,\text{Me\hspace{-.08em}V}}\xspace}
\newcommand{\GeV}{\ensuremath{\,\text{Ge\hspace{-.08em}V}}\xspace}
\newcommand{\TeV}{\ensuremath{\,\text{Te\hspace{-.08em}V}}\xspace}
\newcommand{\xprime}{s^{\prime}/s}
\newcommand{\sqrtsp}{\sqrt{s}_{p}}
\newcommand{\pt}{\ensuremath{p_{\mathrm{T}}}\xspace}
\newcommand{\SingleWminus}{\ensuremath{\PWm\Pep\nu_{\Pe}}}
\newcommand{\SingleWplus}{\ensuremath{\PWp\Pem\overline{\nu}_{\Pe}}}
\newcommand{\gOneZ}{g_{1}^{\mathrm{Z}} }
\newcommand{\kgamma}{\kappa_{\gamma}}
\newcommand{\lgamma}{\lambda_{\gamma}}

%% File with various sparticle combinations
%
% The PS based names are in hepnames.sty/hepnicenames.sty/heppennames.sty
%
\def \conentwo {\PScharginoOnepm \PSneutralinoTwo}
\def \conecone {\PScharginoOneplus \PScharginoOneminus}
%\def \c1c1 {\PScharginoOneplus \PScharginoOneminus}
\def \nonentwo {\PSneutralinoOne \PSneutralinoTwo}
\def \ntwontwo {\PSneutralinoTwo \PSneutralinoTwo}
\usepackage[authoryear]{natbib}
\doublespacing
\usepackage[unicode=true,
 bookmarks=true,bookmarksnumbered=false,bookmarksopen=false,
 breaklinks=true,pdfborder={0 0 0},pdfborderstyle={},backref=false,colorlinks=true]
 {hyperref}
\hypersetup{pdftitle={University of Kansas Thesis Template},
 pdfauthor={Anonymous},
 pdfsubject={A Thesis},
 urlcolor={black},citecolor={black},allcolors={black}}

\makeatletter

%%%%%%%%%%%%%%%%%%%%%%%%%%%%%% LyX specific LaTeX commands.
%\providecommand{\LyX}{\texorpdfstring%
%  {L\kern-.1667em\lower.25em\hbox{Y}\kern-.125emX\@}
%  {LyX}}
%% Because html converters don't know tabularnewline
%\providecommand{\tabularnewline}{\\}

%%%%%%%%%%%%%%%%%%%%%%%%%%%%%% User specified LaTeX commands.
%%%FUNCTIONS

\newcommand\FigOne[3]{%
\begin{figure}[!htbp]%
\centering
\includegraphics[width=0.6\textwidth]{fig/#1}\hfill
\caption{#2}
\label{#3}
\end{figure}}

\newcommand\FigOneScale[4]{%
\begin{figure}[!htbp]%
\centering
\includegraphics[width=#4\textwidth]{fig/#1}\hfill
\caption{#2}
\label{#3}
\end{figure}}

\newcommand\FigTwo[4]{%
\begin{figure}[!htbp]%
\centering
\includegraphics[width=0.45\textwidth]{fig/#1}
\includegraphics[width=0.45\textwidth]{fig/#2}\hfill
\caption{#3}
\label{#4}
\end{figure}}

\newcommand\FigTwoScale[6]{%
\begin{figure}[!htbp]%
\centering
\includegraphics[width=#5\textwidth]{fig/#1}
\includegraphics[width=#6\textwidth]{fig/#2}\hfill
\caption{#3}
\label{#4}
\end{figure}}

\newcommand\FigThree[5]{%
\begin{figure}[!htbp]%
\centering
\includegraphics[width=0.45\textwidth]{fig/#1}
\includegraphics[width=0.45\textwidth]{fig/#2}\hfill
\includegraphics[width=0.45\textwidth]{fig/#3}\hfill
\caption{#4}
\label{#5}
\end{figure}}

\newcommand\FigThreeScale[8]{%
\begin{figure}[!htbp]%
\centering
\includegraphics[width=#6\textwidth]{fig/#1}
\includegraphics[width=#7\textwidth]{fig/#2}\hfill
\includegraphics[width=#8\textwidth]{fig/#3}\hfill
\caption{#4}
\label{#5}
\end{figure}}


\newcommand\FigFour[6]{%
\begin{figure}[!htbp]%
\centering
\includegraphics[width=0.49\textwidth]{fig/#1}
\includegraphics[width=0.49\textwidth]{fig/#2}\hfill
\includegraphics[width=0.49\textwidth]{fig/#3}
\includegraphics[width=0.49\textwidth]{fig/#4}\hfill
\caption{#5}
\label{#6}
\end{figure}}


\newcommand\FigFive[7]{%
\begin{figure}[!htbp]%
\centering
\includegraphics[width=0.45\textwidth]{fig/#1}
\includegraphics[width=0.45\textwidth]{fig/#2}\hfill
\includegraphics[width=0.45\textwidth]{fig/#3}
\includegraphics[width=0.45\textwidth]{fig/#4}\hfill
\includegraphics[width=0.45\textwidth]{fig/#5}\hfill
\caption{#6}
\label{#7}
\end{figure}}
%%% END FUNC



%used to align decimals in tables according to APA style
\usepackage{dcolumn}
\usepackage{booktabs}
\usepackage{placeins}
% Set the title and author info
\title{Search for Weak Scale Supersymmetric Particles in Compressed Scenarios }
\author{Justin Anguiano}

% Following is OPTIONAL list of previous degrees earned. 
% If there are more than 5 or so, then title pagelayout may become too crowded,
% depending on the number of committee members. 
\priorcreds{B.S. Engineering Physics, University of Kansas, 20XX}{M.S. Computational Physics and Astronomy, University of Kansas, 20XX}
% It is acceptable to delete \priorcreds if it is not desired on title page


\dept{Department of Physics and Astronomy}
\degreetitle{Doctor of Philosophy}
\papertype{Dissertation} %or Thesis (Choose whatever word you want to put on p.2)

%% Committee member names are required for the title page. We make space
%% for as many as 7 members, with various roles/titles.
%% It is required to have 7 entries, even if some are empty for committee and role
\committee{Graham Wilson}{Alice Bean}{Christopher Rogan}{Ian Lewis}{Zsolt Talata}{}{}
\role{Chairperson}{}{}{}{External Reviewer}{}{}
%At Most 7 members allowed, last here is blank on purpose to demonstrate
%flexibility

%%BOTH dates must be included. 
\@printd@testrue
\datedefended{July 02, 2019}
\dateapproved{August 06, 2019}

%% These settings are now in the kuthesis.cls file, but users are free
% to customize. listings has great documentation online
%% When listings are used, break lines
%\lstset{
 %    breaklines=true,  % sets automatic line breaking
 %    breakindent=2em,
 %    breakatwhitespace=true,  % sets if automatic breaks should
 %   breakautoindent=true
%}

\@ifundefined{showcaptionsetup}{}{%
 \PassOptionsToPackage{caption=false}{subfig}}
\usepackage{subfig}
\makeatother

\usepackage{listings}
\renewcommand{\lstlistingname}{\inputencoding{latin9}Listing}

\begin{document}
\begin{romanpages}

\maketitle

\begin{abstractlong}

A generic search for supersymmetric particles with an emphasis on compressed scenarios is performed withproton-proton collisions at $\sqrt{s} = 13$ TeV and the CMS detector using a data sample with integrated luminosity of 138 fb$^{-1}$. Potential supersymmetric events with initial-state-radiation recoiling against a massive invisible sparticle system are organized based the Recursive Jigsaw Reconstruction method. Events are then further categorized based on physics object combinatorics such as jet multiplicity, lepton multiplicity, b-tags, and kinematics that are sensitive to generic compressed sparticle topologies. This work focuses on a subset of many pieces of a larger analysis where the underlying compressed strategy is discussed, the selection of leptons and the calibration of simulation through the study of systematic effects, and the implementation and optimization of the data driven fit. Finally expected limits with a 95\% C.L. are placed on processes that include the production of electroweakinos, sleptons, and stops. 

\end{abstractlong}

\begin{acknowledgementslong}
%%if you want a "quote" environment for acknowledgements,
%% use acknowledgements instead of acknowledgementslong
First I would like to acknowledge the collaborative effort and work done by my research group. This work would not be possible without the extensive work of my advisors and peers which has been included alongside my own work for completeness. Much of the content is based on, or modeled after our internal Analysis Note \cite{AN}, so thanks to everyone that has contributed to this massive analysis.  
I am also extremely grateful for my supervisor, Professor Graham Wilson, for his invaluable mentorship and patience that has pushed me to be the best version of myself. I'm also grateful for everyone in my research group and their excellent advice and support, with a special thanks to the students which are always happy to team up and solve problems. I would be much worse off without the community we've developed. Finally, I would like to express my gratitude to my parents which have been essential in helping me survive the final stretch of grad school.


\end{acknowledgementslong}

\tableofcontents{}

\listoffigures

\listoftables

\end{romanpages}

%

\setcounter{secnumdepth}{3}
\setcounter{tocdepth}{3}
\setlength{\parskip}{\smallskipamount}
\setlength{\parindent}{0pt}


\makeatletter


\providecommand{\tabularnewline}{\\}


\makeatother

%\usepackage{babel}
%\begin{document}

\chapter{The Standard Model and Supersymmetry}

%\begin{chapterabstract}
%Here we outline the fundamental concepts of particle physics, we introduce the set of fundamental particles, fields, and interactions which %are described by the standard model. Since the standard model does not solve all of the problems of particle physics I introduce %supersymmetry, a well motivated extension to the standard model. The implications of SUSY model space on various phenomoglly of processes %as well as experimental observations.  The motivation for the SUSY extension is reserved for the following chapter.

%\end{chapterabstract}

\section{Introduction}

The fundamental building blocks of matter and their interactions expressed through three of the four fundamental forces of nature via the Standard Model (SM). The fourth force, or gravity, is left to General Relativity. The SM  is the culmination of over a century of work by many theorists and experimentalists, with roots in the late 19th century. The experimental starting point begins with the establishment of the sub-atomic world through the discovery of the electron by JJ Thomson in 1897 \cite{Thomson:1897cm}, then proton by Ernest Rutherford in 1917 \cite{Rutherford:1911zz}, followed by further development with the solution to the mystery of beta decay by the prediction of the neutrino by Enrico Fermi in 1934 \cite{Fermi:1934hr}, and its discovery in 1957 by Clyde Cowan and Frederick Reines \cite{Reines:1956rs}.  After the neutrino,  the theory of strong interactions is formulated as quantum chromodynamics by Murray Gell-Mann and others in the 1960s, providing a description of how protons and neutrons are held together in the nucleus of an atom \cite{GellMann:1964nj} ushering in the modern Standard Model era. In this era, the theory describing weak interactions from Enrico Fermi in the 1930s is combined with electromagnetic interactions in electroweak theory by Sheldon Glashow, Abdus Salam, and Steven Weinberg in the 1960s \cite{GLASHOW1961579}\cite{Salam:1968rm}\cite{Weinberg:1967tq} and later confirmed by the UA1 experiments with the discovery of the W and Z bosons in 1983\cite{arnison1983experimental}\cite{glashow1984future}. Following the estalishment of electroweak theory, the term Standard Model is coined in the 1970s describing the collection particle physics theories over the last century. The most recent and milestone is added to the SM is the discovery of the Higgs boson in 2012 \cite{hCMS:2012qbp}\cite{ATLAS:2012yve}.  Outlining the innerworkings from this history of particle physics, this chapter will introduce concepts of the Standard Model, bthe fundamental particles, fields, their basic properties, and interactions. Expanding from the  SM we will discuss potential new physics with an extension of the SM which proposes a new symmetry between fermions and bosons called Supersymmetry. Finally, we delve into the specifics of supersymmetric simplified models of and the challenges associated with experimental discovery.



\section{The Standard Model}

The Standard Model is a collection of theories used to predict and reproduce experimental data. The theory itself incorporates four major concepts: Quantum Field theory (QFT), the Dirac equation, the gauge principle, and the Higgs mechanism. These four principles are constrained by physical data and describe the set of elementary particles, known as fermions and bosons. The SM generally refers to the SM Lagrangian, an equation with different sectors that describe different subsets of particles, fields, and their interactions. The SM Lagrangian itself consists of 26 free parameters which are input by hand. These parameters are: the masses of the 12 fermions, 3 coupling constants, $g, g', g_s$, to describe gauge interactions,  2 parameters to desribe the Higgs potential, $m_h, v$, being the higgs mass $m_h$ and the vacuum expectation value (vev), and 9 mixing angles describing mixing of different fermionic fields. The 12 fermion mass parameters are subdivided by three neutrinos $m_{\nu_i}$, three charged leptons $m_{\ell_i}^\pm$, and six quarks $m_{q_i}$ \cite{Thomson:2013zua}.

QFT provides a description for both known and theoretical particles by combining quantum theory, the field concept, and relativity \cite{Peskin:1995ev}. The gauge theory aspect describes the exact nature of QFT interactions and provides the mechanisms for the electromagnetic, strong, and weak forces.  We know of three gauge fields:  $\vec{G}$ which transforms under $SU(3)$ and govern strong interactions, $\vec{W}$ and $B$ which transform under $SU(2)_L \times U(1)$ and govern electromagnetic and weak interactions. The combination of the gauge fields,fermion fields, and the Dirac equation yields eigenstates that represent fermionic matter particles. These matter particles would technically be massless if not for the inclusion of the complex scalar Higgs field.  The spontaneous symmetry breaking of the Higgs field, due to the Yukawa coupling, creates the non-zero vev resposible for generating the masses of the electroweak gauge bosons and SM fermions \cite{Higgs:1966ev}\cite{Bernardi:2008zz}.

The set of standard model elementary particles is divided into two subgroups: fermions and bosons.  The fermions consist of both charged and neutral leptons as well as fractionally charged quarks. There are three flavors of charged leptons $(\ell)$, the electron $(e)$, the muon $(\mu)$, and the tau $(\tau)$. Each charged lepton has a flavor pairing neutral neutrino $\nu_\ell$.  As for the quarks, there are also three generations of pairs of quarks.. The lighest set of quarks are the up $(u)$ and down $(d)$ quarks, followed by the charm $(c)$ and strange $(s)$, and lastly the bottom $(b)$ and extremely massive top quark $(t)$.  The bosons are the force carrying particles which represent the gauge fields. They are comprised of the vector bosons - the photon $(\gamma)$, gluon $(g)$, the $W^\pm$, and the $Z^0$ - along with the singular scalar Higgs boson $(h)$ \cite{ParticleDataGroup:2020ssz}. The elementary particles masses, generations, and spins are summarized in Figure \ref{fig:smfig}.

%The $e$ and $\mu$ are also generally considered as "light" leptons due to their small mass relative to the $\tau$. The term lepton, depending on context, often refers to only the charged particles.

\FigOneScale{Intro_figs/Standard_Model_of_Elementary_Particles.png}{The elementary particles of the standard model which includes the color coded categories of fermions and bosons as well as their nominal mass (or mass upper limit), charge, and spin.\cite{particle-physics-wikipedia}}{fig:smfig}{0.6}


%The SM Lagrangian is composed of constituent sectors which describe diffrent groups/fields/particles. The main SM sectors are, the quantum chromodynamics (QCD) sector, the electroweak sector, the Higgs sector, and the Yukawa sector. QCD describes colored interactions of quarks mediated by gluons with the strong force. The electroweak sector unifies both the electromagnetic and weak interactions via exchange of W or Z bosons as well as electromagnetic interactions via $\gamma$. The Higgs sector introduces the complex scalar higgs field (citation needed). Interaction of bosons with the Higgs field causes the bosons to have mass and the Yukawa coupling describes the interaction of fermions with the higgs field which also allows the fermions to have mass (citation needed).



The SM is an asymmetric chiral theory, combining three groups $SU(3)_L \times SU(2)_L \times U(1)$. The $L$, or left handed, subscript indicates that mirrored fields (with different chiralities)  transform differently under the Lorentz group and the EW gauge group.  The consequence of chiralilty is that the possible combinations between interaction vertices is limited \cite{Thomson:2013zua}. This peculiar property shows up with the $W$ boson, which only couples to left handed particles or right handed antiparticles. Extensions of the standard model also often extend chiral or symmetrical properties.  %Helcity is the defined by the projection of a particles spin ontion its direction of motion (thompson). A particle is considred right-handed if the direction of its spin and motion is parallel. %It is left-handed if spin and motion antiparallel. (cite wikipedia helicity). This peculiar property shows up with the W boson, which only couples to left handed particles or right handed antiparticles.

\section{Supersymmetry}

Supersymmetry (SUSY) is an extension of the standard model that adds a generator rotating the spin between bosons and fermions. This introduces a bosonic degree of freedom for every fermionic degree of freedom, and in turn, a super partner for each particle differing by spin one-half \cite{Baer:2007izw}.  The resulting set of mirrored elementary particles are referred to as sparticles. Each bosonic sparticle carries the same name as its fermion partner but with an `s' prefix e.g. sfermion, squark, selectron. As for the bosons, with the gauge fields $B$ and $\vec{W}$, these are accompanied by three super symmetric fields - the Higgsino $\tilde{H}$, and gauginos represented by the Bino $\tilde{B}$, and Wino $\tilde{W}$. The set of boson superpartners can be obtained with the same approach as the SM where the mixture of the B and $\vec{W}$ SM fields are represented by particle matrix.  The diagnolization of particle matrix leads to mass eigenstates representing the SM particles $\gamma, \, \, Z, \, \, W^\pm$. Similarly, the Higgsino, Bino, and Wino super fields mix to produce four neutral and two charged eignestates, the neutralinos ($\tilde{\chi}^0_1, \tilde{\chi}^0_2, \tilde{\chi}^0_3, \tilde{\chi}^0_4$)  and charginos ($\tilde{\chi}^\pm_1, \tilde{\chi}^\pm_2$) \cite{DJOUADI_2008}. SUSY also requires an additional Higgs doublet to give mass to up-type and down-type fermions,  leading to five higgs boson states consisting of two charged Higgs and three neutral Higgs \cite{Adam:2021rrw}. The lightest neutral Higgs of the three neutral options is chosen to represent the SM Higgs boson. The full set of SM particles alongside their SUSY partners are illustrated in Figure \ref{fig:smandsusyfig}. The addition of another Higgs doublet also introduces a second vev. The ratio between the two vevs is denoted as $v_1/v_2 = \tan \beta$ and is an important parameter in experimental searches. An important bookeeping parameter, similar to lepton number or baryon number conservation, is R-parity. This parameter tallies the total number of SM particles (+1) and sparticles (-1) and expects the net total between particles to be conserved in the initial and final states. R-parity conservation then requires sparticles to be produced in pairs. If R-parity is violated, the common consequence is that the lightest supersymmetric particle (LSP) is unstable or the models allows protons to decay \cite{Farrar:1978xj}. 

\FigOneScale{Intro_figs/smandsusy.png}{The elementary particles of the standard model with their supersymmetric partners.}{fig:smandsusyfig}{0.8}



Supersymmetry is an expansive model and intractable to experimentally test without significant well motivated simplifications. The most experimentally common simplified SUSY model is the Minimally Super Symmetric Standard Model (MSSM). The MSSM contains the smallest number of new particle states and new interactions which are consistent with phenomenology \cite{Baer:2007izw}. The MSSM is still experimentally inaccesible due to the presence of over 100 parameters, where small changes in parameter space can completely morph the model structure and experimental signatures. To reduce the problem's dimensionality, further simplification is needed, resulting in a popular simplified model: the phenomological MSSM (pMSSM). The pMSSM contains 19 parameters which include the masses of each generation of squark and slepton, parameters to control the mixing of $\tilde{H}, \tilde{W}, \tilde{B}$, and dials for the higgs doublet \cite{MSSMWorkingGroup:1998fiq}.  The pMSSM is still borderline too complicated to attack directly, so, the pMSSM is boiled down into a simplified model of four parameters $M_1$,$M_2$, $\mu$, and $\tan\beta$. $M_1$ and $M_2$ are the gaugino mass parameters, $\mu$ is the Higgsino mass parameter, and $\tan\beta$ is the previously mentioned vev ratio \cite{Fuks_2018}.  A model point from this four parameter space is referred to as Realistic simplified gaugino-higgsino model, and targets specific regions of MSSM parameter space and experimental topologies. Realistic simplified models are typcially used in most experimental searches, including this one.

To effectively grasp the structure of SUSY and various models, either in the pMSSM or simplified models, there are a couple key elements to condsider. First is the mass scale of the relative SUSY sectors i.e. how massive are the -inos versus sleptons versus squarks. If the mass scales are well separated, the sectors are effectively decoupled. If the mass scales are similar then it may introduce complicated cross-talk between sectors. A SUSY search with a 4 parameter simplified model can be further simplified by assuming squarks and slepton masses sit at the several TeV scale while the targeted electroweak-inos are at detectable TeV and sub-TeV scale.  By decoupling sectors, outside the sector-of-interest, we remove the interaction between these groups, so for example, if sleptons are decoupled from the charginos then complicated dependencies, like cascading between the two decays are avoided. The other key element is the dominant type of super field coupling or the composition of the LSP, typically $\tilde{\chi}^0_1$.  A model point can be denoted by the field that dominates the overall LSP mixing, i.e. a Higgsino model LSP would be composed of a majority $\tilde{H}$ \cite{ATLAS:2015wrn}. The characteristic take away from simplified model types is that $\tilde{H},\tilde{W},\tilde{B}$ controls the nature of the model by governing the cross sections, topological infrastructure, and how the sparticles interact amongst themselves and SM particles. Two pMSSM examples comparing the mass structure between two arbitrary mass points of a Wino model versus Higgsino model is shown in Figure \ref{fig:mass_modelpoint}. For both models in Figure \ref{fig:mass_modelpoint} the Higgs and slepton sectors are decoupled at a multi-TeV scale while the squark and gaugino sectors are at an accesible TeV and sub-TeV scale. Note that small changes in pMSSM model space results in differing LSP content and large variations in the relative mass structure and orderings. The difference in cross sections between the same two model points for Neutralino/Chargino pair production combinations are show in in Figure \ref{fig:xsec_modelpoint}. This relative differences in cross section illustrates that the same tweak in parameter space can induce order of magnitude changes sparticle production.

\FigTwoScale{Intro_figs/wino_modelpoint_mass.png}{Intro_figs/higgsino_modelpoint_mass.png}{Left: Wino-like LSP from PMSSM model point 18898934. Right: Higgsino-like LSP from PMSSM model 6755879 \cite{ATLAS:2015wrn}. Both models have a somewhate similar relative mass structure but order of magnitude differences in the higgs, squark, and slepton sectors.}{fig:mass_modelpoint}{0.49}{0.49}


\FigTwoScale{Intro_figs/wino_modelpoint_xsec.png}{Intro_figs/higgsino_modelpoint_xsec.png}{Comparison of the Wino LSP and Higgsino LSP models from Figure \ref{fig:mass_modelpoint} sparticle pair production cross sections.}{fig:xsec_modelpoint}{0.75}{0.75}
%Example of differing model points


In addition to the mass structure and cross sections, the decay specifics of each gaugino and higgsino mixing also varies. The variation in decay modes has a significant impact on the experimental channels and signatures of interest. In an experimental search we would expect the heavier sparticles to decay to both SM particles along with the LSP. If the LSP happens to be close in mass to its parent, say O(100) GeV or less, the model would be considered as a compressed scenario. This scenario is considered compressed because the observable energy of the SM particle involved in a sparticle decay is compressed to a very small amount due to the majority of the available energy being used by the rest mass of the sparticles. Of the 3 types of models, the most likely candidates for compression are the Higgsino-like and Bino-like models. Wino models by far have the largest cross sections but are the least likely to have compressed states. Particularly interesting topologies for these compressed models involve decay signatures of processes like $\tilde{\chi}^0_2 \rightarrow Z^*\tilde{\chi}^0_1 $, $\tilde{\chi}^0_2\rightarrow \tilde{\chi}^\pm_1 \tilde{\chi}^0_1 $, $\tilde{\chi}^\pm_1\rightarrow W^\pm \tilde{\chi}^0_1$, $\tilde{t}\rightarrow t \tilde{\chi}^0_1$, $\tilde{\ell}\rightarrow\ell \tilde{\chi}^0_1$. To further complicate the experimental topoligies, the nature of sparticle decay is not only dependent on the superfield mixing but also on the level compression. Figure \ref{fig:n2decaymodes} shows the average decay modes for $\tilde{H}, \tilde{W},$ or $\tilde{B}$ LSPs as a function of mass splittings from a selection of pMSSM models \cite{ATLAS:2015wrn}.

\FigThreeScale{Intro_figs/winoN2decay.pdf}{Intro_figs/binoN2decay.pdf}{Intro_figs/hinoN2decay.pdf}{Example average neutralino branching fractions for the combined set of pMSSM model points from \cite{ATLAS:2015wrn}. The LSP content for each figure is Wino-like top-left, Bino-like top-right, and Higgsino-like bottom. Branching fractions are shown as a function of the mass splitting $\Delta m = m_{\tilde{\chi}_2^0 } - m_{\tilde{\chi}_1^0}$ . The model with Wino LSP is subject to a $W$ channel enhancement while $Z$ is suppressed. In the Bino case, the $W/Z$ channel are opposite of Wino. The higgsino LSP models can be be dominated by either the W or Z channel depending on the mass splitting.}{fig:n2decaymodes}{0.49}{0.49}{0.49}


Note that between each model type in Figure \ref{fig:n2decaymodes} the $Z^*$ and $W^\pm$ modes can be highly suppressed or enhanced. In some cases even, specific modes like $\tilde{\chi}^0_2\rightarrow \tilde{\chi}_1^\pm W^\mp$ can be either kinematically forbidden, or excluded to streamline MC production and enhance the statistical power of different targeted final states. Alongside these branching fraction complications, the decay phase space of the final state particles is also model dependent. For instance, in the case of $\tilde{\chi}^0_2 \rightarrow Z^*\tilde{\chi}^0_1 $ the shape of $Z$ dilepton mass distribution $m_{\ell\ell}$  changes depending on the sign of the $\tilde{\chi}_2^0$ and $\tilde{\chi}_1^0$ eigenstates. Experimentally this problem is then divided into two possible scenarios: cases where the eigenstates are the same sign and cases where the eigenstates are the opposite sign. The distribution that showcases the $m_{\ell\ell}$ differences under two different model interpretations is shown in Figure \ref{fig:atlasmllwbh} from a previous ATLAS search. Overall, with the complications of model dependent decays, inherently rare production, varying mass orderings, and relative scale between sectors, the search for SUSY is an extraordinary challenge. To discover SUSY, a reasonable approach is to try to be genericallly sensitive to all of the formerly discussed complicated nuances. 


\FigOneScale{Intro_figs/atlas_wbh_mll.png}{Example reweighting of simplified model from an ATLAS search which covers the different dilepton mass line shapes in the same-sign and opposite-sign eigenstate cases \cite{ATLAS:2019lng}.}{fig:atlasmllwbh}{0.7}

%What particles are in susY?
%For each elementary standard model particle there is a super partner. For the quarks and leptons, the pairing is simple, there is just the equivalent slepton and squark partner. The gauge bosons are slightly more complicated, these are generally denoted with an "ino" suffix. There are also 3 super fields which mix in specific quantities to yield varying instances of particles with particular properties. These mixings define the characteristics of the model point by influecing things like decay mode, cross seection, and couplings.  (WhY?) There are four neutralinos $\chi^0_i$ and two charginos $\chi^\pm_j$. There are also 4 Hiigs bosons, a charged pair $H^\pm$ and a neutral pair $H^0_{u,d}$. (Why?)  The electroweakinos, i.e chargino or neutralinos, increase in mass with increasing index but the structure of reletavie masses depends specifically on the model. The $\chi^0_0$ is generally the lightest supersymmetric partilce (LSP) and in many popular models is stable. The instances of stable LSP depend on R-parity conservation. (Define R-parity conservation) If this is violated the LSP will decay into SM particles.



%Incldue a plot with mass hierarchies. Since there are so many possible parameters, varying sets of paramters can produce significant diffences in experimental signatures and topologies. Typically for a model we decouple specific sectors when generatting monte carlo, For instance if we are searching for sleptons, the squark or electroweakino sector will be chosen to be significantly heavier (out of current experimental range) effectively decoupling it from the slepton sector. Then from a simplified model with everything else decoupled we scan various topologies with particular mass values. 

%SUSY chirality

%Talk about higgsino/wino bino model structure decay modes etc


%
\setcounter{secnumdepth}{3}
\setcounter{tocdepth}{3}

\setlength{\parindent}{1 em}


\makeatother


\chapter{Motivating the Search for SUSY}

%\begin{chapterabstract}
%Introduce some of the issues of the SM and opening up with the basic motivations for susy, like solving the hierarchty problem and dark matter candidate, next we look %at a theoretical motivation for SUSY via the higgs mass. Motivate simplified models with naturalness etc, talk about how susy needs to be at a few TeV scale to work %out. Then we visit two recent experimental measurements which strongly motivate the search for susy and more speficically this body of work.
%\end{chapterabstract}

\section{Introduction}

The Standard Model is a remarkable theory describing a wide variety of sub-atomic phenomenon which also has consistently held up to tests over many orders of magnitude in energy. However, it's not a perfect theory. There are a few  experimental and theoretical problems that the SM can not yet explain such as: how to incorporate gravity,  how can we explain neutrino mass and mass orderings, or why is the universe made up of matter and not antimatter?  
A significant problem in physics, which connects both galactic scale and sub-atomic scale physics, is dark matter. Cold dark matter (CDM) is a type of matter that is thought to have played a crucial role in the formation of large-scale structures in the Universe, such as galaxies and galaxy clusters \cite{Garrett:2010hd}. The evidence for CDM begins with observations of Zwicky in 1933 where he found that the observed motion of the galaxies in the Coma Cluster could not be explained by the gravitational interactions of visible matter in the cluster, but could be explained by adding additonal invisble mass in the form of ``dark matter'' \cite{Zwicky:1933gu}. Additional evidence for CDM has compiled over the years such as: gravitational lensing data that disfavors being completely explained by effects black holes or condensed baryonic matter \cite{Massey:2007lens}, large scale structure formation where CDM can explain the formation and evolution of galaxy clusters \cite{Springel:2005}, and temperature fluctuations in the cosmic microwave background that suggest the universe is composed of approximately 85\% dark matter \cite{Planck:2018vyg}. Despite overwhelming evidence for the existence of CDM there are no suitable SM dark matter candidates to explain the abundance of this potential cosmic particle. SUSY offers an attractive solution with the introduction of new particles that can explain dark matter directly via massive invisible particles, such as the neutralino, $\chi_1^0$, or sneutrino $\tilde{\nu}$. The neutralino can handle the CDM problem with models capable of producing the expected CDM relic density of the universe and, in fact, this is used to constrain SUSY model space and simplify searches. Aside from these leading motivations, other more detailed motivations will be discussed in this chapter, the first being the ``naturalness problem'' with its theoretically aesthetic improvement adding a symmetry to protect against divergent terms in the perturbative expansion of the Higgs mass. The next motivations are experimental, where SUSY offers an explanation to the signficant deviation observed in the muon $(g-2)_\mu$ factor from recent FNAL result, as well as the deviation observed in the $W$ boson mass at CDF II. It should be noted that the divergent higgs mass - known as the hierarchy problem - satisfies most SUSY scenarios up to the few TeV scale \cite{Barbieri:1987fn}, but, the two latter experimental measurements specifically motivate SUSY compressed scenarios.



\section{Stabilizing the Higgs mass}

An aesthetic attribute of theoretical models is naturalness. We should expect a model to function naturally if the ratio of free parameters in a model are of $O(1)$. Large swings between parameters would be considered fine-tuning and could indicate issues with the underlying theory. So, naturally, if fine-tuning exists in a model, it strongly motivates building extensions to the model to eliminate fine-tuning. %. We would expect with some improved theory with new physics one would balance out finely tuned parameters providing a natural solution to whatever is being modeled. 
One such fine tuning arises in the hierarchy problem, specifically in the Higgs self interaction terms. The SM Higgs Lagrangian terms that involve self interaction are illustrated in equation \ref{eq-higgslagrange}.
\begin{equation}
\label{eq-higgslagrange}
\mathcal{L}=\frac{gm_h}{4M_W}H^3 - \frac{g^2m_h^2}{32M_W^2}H^4
\end{equation}
$H$ represents the scalar higgs field, $m_h$ the higgs mass, and $m_W$ the W boson  mass. A correction to the higgs mass can be calculated using standard perturbation theory by evaluating the second term of the Higgs Lagrangian \cite{Baer:2007izw}. 
\begin{equation}
\label{eq:higgpert}
\begin{split}
\Delta m_h^2 = \langle H | \frac{g^2m_h^2}{32M_W^2} H^4 | H  \rangle = 12\frac{g^2m_h^2}{32M_W^2}\int \frac{d^4 k}{(2\pi)^2} \frac{i}{k^2 - m_h^2}\\
= 12\frac{g^2m_h^2}{32M_W^2} \frac{1}{16\pi^2}\big( \Lambda^2 - m_h^2\log\frac{\Lambda^2}{m_h^2} + O(\frac{1}{\Lambda^2})\big)
\end{split} 
\end{equation}
 
The equation \ref{eq:higgpert} intergal term is the propagator for the exchange of a virtual Higgs and is integrated over phase space. The $\Lambda$ is known as the scale cutoff parameter and should be interpreted as the scale at which the SM breaks down, possibly near the planck scale $O(10^{19})$ GeV. Notice the leading term $\Lambda^2$ indicates that the expansion is quadratically divergent. The divergent mass correction means there needs to be extremely large cancellations, around 20 orders of magnitude, to maintain $\Delta m_h \sim O(m_h)$. This divergent phenomenon can also be observed with fermion masses, but, chiral symmetry protects the fermion mass from divergence by cancelling out high order $\Lambda$ terms. SUSY offers a similar protection to the Higgs mass by introducing a symmetry with the additional fermionic and bosonic degrees of freedom leading to similar cancellations and producing a more natural model. 


\section{The Muon Anomalous Magnetic Moment}

A major experimental motivation for SUSY lies within the measurement of the muon anomalous magnetic moment.  Multiple meausurements between two labs, Brookhaven National Lab (BNL) and Fermi National Accelerator Lab (FNAL) have shown significant disagreement with the SM. These experiments measure the muon $g$ factor, or specifically, its deviation from two, $(g-2)_\mu$ .  The $g$ factor is related to the electromagnetic coupling of charged particles with the photon and largely depends on the tree level lepton-photon coupling, but, gets small quantum corrections from higher order loops. The largest correction is the single photon loop shown in Figure \ref{fig:gm2fig}. To predict the $g$ factor, an SM calculation is performed with three types of quantum corrections: Quantum Electrodynamic (QED), Electroweak (EW), and Hadronic. Corrections from the Higgs are neglected because the effects are not experimentally observable. %the mass disparity $m_h >> m_{e,\mu}$ and the mass dependence in the Higgs coupling  effects that are smaller than what is experimentally observable. 
The g-factor prediction starts at exactly 2, with QED, and then involves quantum corrections up to $O(10^{-11})$. The prediction is compared with an experimental measurement at a very high level of precision. If the observation were to deviate from the SM prediction, it can indicate new and unaccounted for physics interactions with the SM leptons.
The current best $a_\mu = \frac{g-2}{2}$ prediction is reported as $a_\mu= a_\mu^{QED}+ a_\mu^{EW}+a_\mu^{\text{Hadronic}} =  116 591 810(43) \times 10^{-11}$ \cite{Muong-2:2021ojo}.
\FigOneScale{Motivation_figs/g2_diagrams_pdg.png}{Diagrams which contribute to $(g-2)_\mu$. Left is the single photon Schwinger loop that contributes the largest deviation from two. The middle diagrams are electroweak contributions and the far right diagram is the hadronic vacuum polarization the involves a loop with hadrons \cite{ParticleDataGroup:2020ssz} }{fig:gm2fig}{0.95}
 For each of the $a_\mu$ components, the QED compenent enters at the $O(10^{-3})$ and is known to $O(10^{-11})$. the EW component enters the sum at $O(10^{-9})$ and is known to $O(10^{-10})$. Finally the most complicated hadronic component, contributes at $O(10^{-8})$ and is known up to $O(10^{-9})$. The hadronic contributions arise from Hadronic vacuum polarization (HVP) and light by light scattering (LBL) with the former diagram also illustrated in Figure \ref{fig:gm2fig}. The $a_\mu^{\text{Hadronic}}$ precision dominiates the overall $a_\mu$ error and is constrained by data driven measurements alongside the limitations of the computational approach using QCD lattice theory. The BNL measurement of $a_\mu$ yields a difference with the SM prediction of $\Delta a_\mu := a_\mu^{BNL} - a_\mu^{SM} = 279(76) \times 10^{-11}$ which carries significance of $3.7\sigma$. The most recent $a_\mu$ measurement from FNAL confirms the BNL measurement within $1\sigma$ and the combined experimental average increases the SM deviation with a significance of $4.2\sigma$ \cite{Muong-2:2021ojo}.


%Experimentally the deviations from 2 are the most interesting, and are written in the form $a_\ell = \frac{g-2}/2$ and referred to as $(g-2)_\ell$. These small contribtutions are interesting because they encapsulate the current theory and provide a test bed for our current understanding.  If observations were to deviate from the SM prediction, it would be an indication of new and unaccounted physics interactions with the SM leptons. The $g$ factor can be extracted by measuring the anomalous magnetic moment of any generation of charged lepton.  The current best candidate to both test the SM and search for new physics is by measuring $(g-2)_\mu$ or $a_\mu$ because of  experimental precision potential. The electron measurement is already known to the highest precision and is expected to have the smallest contributions from new physics (cite youtube citation). The $(g-2)_\tau$ is not yet experimentally tractable competitive precision to $\mu$ or $e$.%so $(g-2)_\mu$ has been measured at both at Brookhaven National Lab (BNL) and again at Fermi National Accelerator Laboratory (FNAL).
%The current best SM prediction of $a_\mu$ from (CITE g-2 collab) includes QED, Electroweak(EW) and Hadronic contributions and is reported as $a_\mu^{SM} = a_\mu^{QED}+ a_\mu^{EW}+a_\mu^{\text{Hadronic}} = 116 591 810(43) \times 10^{-11}$. For each of the $a_\mu$ components, the QED compenent enters at the $O(10^{-3})$ and is known to $O(10^{-11})$. the EW component enters the sum at $O(10^{-9})$ and is known to $O(10^{-10})$. Finally the most complicated component, hadronic, contributes at $O(10^{-8})$ and is known up to $O(10^{-9})$, the  main sub components that contribute to the $a_\mu^{\text{Hadronic}}$ is the Hadronic vacuum polarization and light by light scattering, diagrams illustrated in Figure X. The hadronic precision is constrained by data driven measurements and computation approaches -- QCD lattice theory, this error dominates the overall uncertainty of $a_\mu$. The BNL measurement of $a_\mu$ yields a difference with the SM prediction of $\Delta a_\mu := a_\mu^{BNL} - a_\mu^{SM} = 279(76) \times 10^{-11}$ which is a significance of $3.7\sigma$. The most recent $a_\mu$ measurement from FNAL confirms the BNL measurement within $1\sigma$ and the combined experimental average increases the SM deviation with a significance of $4.2\sigma$.

%What could this deviation mean?
The $4.2\sigma$  is a compeling sign for new physics, but not a smoking gun. It is possible to reduced or eliminate the discrepancy by improving the calculations of the HVP and LBL contributions. New and updated calculations are being performed attempting to resolve the discrepancy by a few $\sigma$, but do not yet fully resolve the differences between observations and theory. If computational improvements can't bring the theory into focus, new particles would introduce  quantum corrections bringing experiment and theory into agreement. Several models qualify and successfully explain the $a_\mu$ SM deviation, one being SUSY, where for example, contributes additional diagrams via the smuon-muon coupling illustrated in Figure \ref{fig:gm2susy}.

\FigOneScale{Motivation_figs/g2_susy_loop.png}{Example muon diagrams which include sparticle loops that would contribute to $(g-2)_\mu$\cite{SvenTalkgm2}}{fig:gm2susy}{0.9}

%g-2 is an experiment designed to measure the anomolaus magnetic dipole moment of the muon. The spin magnetic moment of a charged, spin-1/2 particle that does not possess any internal structure (a Dirac particle) is given by (wiki direct quote \url{https://en.wikipedia.org/wiki/G-factor_(physics)}) ${\displaystyle {\boldsymbol {\mu }}=g{e \over 2m}\mathbf {S} }$. where g is the particles g-factor, $\mu$ is the magetic moment, $m$ is the particle mass and $S$ is the spin. The g-factor in quantum electrodynamics is close to 2 so typically the reported measurement is the difference from 2 or g-2 or as a signficance $a_\mu = g-2/2$. The difference from 2 arises from higher order contributions in quantum field theory

\section{The W boson mass}
%the most recent w mass measurement yielded a heavy W, this higher mass is more favorable for light higgsino and compressed susy models

%What is the W boson
The W boson is an important and peculiar particle, it is the electrically charged boson and couples only with left handed particles. The decay modes follow two channels: (1) the hadronic mode with different flavor quark pairs and (2) the leptonic mode with a charged lepton and neutrino. Measuring the W mass directly is challenging at the LHC due to either high levels of QCD di-jet background or missing energy from the neutrino. The mass parameter itself, $m_W$, underpins many important parameters in the SM as well. In fact, $m_W$ is related to the Higgs vev, implying that coupling of the Higgs field to all particles is effectively tuned by $m_W$. Similary, $m_W$ is related to the $g$ factor from $(g-2)_\mu$, so, both $m_W$ and $g$ can be used to constrain new physics. The W mass can be paramterized at tree level in terms the fine structure constant $\alpha$, the Fermi constant $G_\mu$ and the Z-boson mass $m_Z$, with higher order radiative corrections coming from $\Delta r$ shown in equation \ref{eq:mwequation} \cite{Awramik:2003rn}.
\begin{equation}
\label{eq:mwequation}
m_W^2 = m_Z^2\Bigg(\frac{1}{2} + \sqrt{\frac{1}{4} - \frac{\pi\alpha}{\sqrt{2}G_\mu m_{Z}^2 }(1+\Delta r) } \Bigg)
\end{equation}

%What is the current status of the W boson?
There is no exact SM prediction of the W mass, but, since there is an interdependence of many parameters such as $v$, $m_z$, $G_\mu$,$\alpha$ , the SM prediction is constrained by experimentally well measured parameters. The most recent measurement of $m_W$ was performed by CDF II at the Tevatron where $m_W$ was obtained by fitting the kinematic distributions of light leptonic decays recoiling against a system of jets. This measurement is $50\%$ more precise than the previous measurement by ATLAS and heavier than the SM prediction. The combination of a large deviation with very small error bars results in a significance of $7\sigma$ \cite{CDF:2022hxs}.  

%What could this deviation mean?
If follow up experiments confirm the excess in the W mass, it is a definite sign of new physics. The new physics would express itself as new particles in the radiative corrections via equation \ref{eq:mwequation}. Numerous SUSY models can explain the excessive mass of the W boson, but in general, a slightly heavier W favors light SUSY models, potentially at the electroweak scale, illustrated in Figure \ref{fig:cdfw}. A light SUSY also implies light Higgsinos which favor compressed scenarios.  To illustrate the SUSY capability to satisfy both heavy $m_W$ and deviations in $(g-2)_\mu$, an abundance of model points are shown in Figure \ref{fig:gm2mw}.
% Due to the interdependence of $m_W$ and $(g-2)_\mu$, these parameters both constrain compressed SUSY and spotlight a critical area to search. An example of model points of sleptons and gaug-ino models which satisfy the newest $g-2$ and W mass constraints in shown in FIGURE Z (cite Wmass and g-2 sven paper)
\FigOneScale{Motivation_figs/wmass_cdf.png}{The mass of the W boson as a function of top quark mass which displays the CDF II measurement with 68\% C.I. compared to the SM prediction and LEP2/Tevatron measurement with 68\% C.I \cite{CDF:2022hxs}}{fig:cdfw}{0.6}
\FigOneScale{Motivation_figs/gm2_mw.png}{Illustration of various color coded model points which correspond to different scenarios and -ino model types that favor the measured $(g-2)_\mu$ and W mass that exceeds the SM predction. The grey band represents the SM prediction, the green line is combines all the experimental results for the W mass, and the blue line indicates the combined $(g-2)_\mu$ experimental results \cite{Bagnaschi:2022qhb}}{fig:gm2mw}{0.7}
\section{The current status of SUSY}
%drop the most recent limits here, start with multi TeV excluded gluino and squark models which leaves the a good place to search in the weak scale sector with electoweakinos. Talk about electroweak limits and how alot of these are excluded already one of the remaining places to search is the compressed corridor where mass splittings are small. link this limit motivation with how both g-2 and W mass favor compressed scenarios

There have been many searches for SUSY particles, starting from  LEP and still ongoing at the LHC. There is not yet observed evidence of SUSY, but, there also is not enough lack of observation to fully reject the SUSY hypothesis. The most comprehensively searched region SUSY space is related to strong production of SUSY particles. The large expected cross sections for squarks and gluinos compared to the inos or sleptons sectors offer the most low hanging fruit for potential discovery.
 %ut there is still plenty of room to keep searching. 
%The strong production of SUSY has the largest excluding limits due to the large expected cross sections compared to other sectors gauginos/sleptons.
Simplified model searches in ATLAS and CMS have excluded $\tilde{g}$ and $\tilde{q}$ (not including $\tilde{t}$) up to around 2 TeV with the most recent limits are shown in Figure \ref{fig:run2stronglim} and \ref{fig:run2squarklim} with their sister exclusions of stop squarks around 1 TeV shown in \ref{fig:stoplim}. The area inside the limit lines in each figure indicates that the 2-D mass points of the sparticle and LSP pair are ruled at a 95\% confidence level for the associated simplified model.
\FigTwoScale{Motivation_figs/gluinoPair_lim.png}{Motivation_figs/atlas_gluinopair_lim.png}{The CMS (left) and ATLAS (right) gluino limits from multiple competing channels and models \cite{cmslims}\cite{atlaslims}}{fig:run2stronglim}{0.48}{0.51}

\FigTwoScale{Motivation_figs/squarkPair_lim.png}{Motivation_figs/altas_squark_lim.png}{The CMS (left) and ATLAS (right) squark limits from multiple competing channels and models \cite{cmslims}\cite{atlaslims}}{fig:run2squarklim}{0.48}{0.51}

\FigTwoScale{Motivation_figs/cms_stop_limit.png}{Motivation_figs/Atlas_stop_limit.png}{The CMS (top) and ATLAS (bottom) limits on stop production \cite{cmslims}\cite{atlaslims}}{fig:stoplim}{0.73}{0.85}

The CMS and ATLAS electroweak limits are shown in Figure \ref{fig:ewlims}. Note that the electroweak limits have sufficient data to only reach the TeV scale while SUSY remains valid at the few TeV. This leaves significant room in the 2-D mass plane to either discover or exclude SUSY by adding more data. One particular simplified model which is nearly undaddressed by both CMS and ATLAS is chargino pair production associated with final states with two oppositely charged $W$ bosons.  Most simplified models assume a mass degeneracy with $m_{\tilde{\chi}^0_2} = m_{\tilde{\chi}^\pm_1}$ but, there is no reason to believe that $\tilde{\chi}^0_2$ can not be decoupled from $\tilde{\chi}^\pm_1$. If this were the case, the limits would not apply in excluding charginos of any mass, So, it is important to address specific final state. The slepton limits  for both CMS and ATLAS are shown in Figure \ref{fig:sleplims}. These are generally the weakest limits of all the aforementioned processes, but, potentially the most important in association with $(g-2)_\mu$.



\FigTwoScale{Motivation_figs/c1n2_lim.png}{Motivation_figs/Atlas_chiwz_lim.png}{The mass limits on chargino pair production with degenerate masses $m_{\tilde{\chi}_2^0}= m_{\tilde{\chi}_1^\pm}$ for CMS (top) and ATLAS (bottom). There are no published limits on compressed chargino pairs in CMS  \cite{cmslims}\cite{atlaslims}}{fig:ewlims}{0.75}{0.8}

\FigTwoScale{Motivation_figs/slep_lim.png}{Motivation_figs/Atlas_slepton_lim.png}{The slepton mass limits for CMS (top) and ATLAS (bottom) which assume the same masses for L and R sleptons and combine smuon and selectron production \cite{cmslims}\cite{atlaslims}}{fig:sleplims}{0.8}{0.83}

For all of the previously presented limits from CMS and ATLAS, excluding gluinos and squarks, a common thread is that the weakest exclusion regions are the compressed regions. The compressed region varies from process to process. For example, compressed relative to stops is such that the mass difference $\Delta m = m_{NLSP} - m_{LSP}$ is less than the top mass $\Delta m < m_t$. Compression with eletroweakinos would have a $\Delta m$  below the W or Z pole. For sleptons decaying directly to leptons, there are no intermediate heavy particles like a W,Z or t, so, the compressed region is more ambiguous and is considered to be ``soft'' interpreted as $O(20-30)$ GeV or less. Several of the compressed scenarios, are unadressed by CMS, but are complemented by dedicated compressed searches in ATLAS. However, each ATLAS summary result combines different searches with the common feature of large gaps between the results. So, based on the current status of all SUSY results, there is strong motivation to confirm compressed ATLAS results with a CMS compressed search but also extend the current limits and cover gaps between searches.



%\setcounter{secnumdepth}{3}
\setcounter{tocdepth}{3}
\setlength{\parskip}{\smallskipamount}
\setlength{\parindent}{0pt}


\makeatletter


\providecommand{\tabularnewline}{\\}


\makeatother


\chapter{The CMS experiment}

\section{Introduction} The Compact Muon Solenoid (CMS) experiment consists of a detector housed at the Large Hadron Collider (LHC). Two synchronous bunches of high energy protons counter rotate through the LHC accelerator ring and collide at a center point in the CMS detector.  The protons are collided with a significantly large energy with the expectation that more massive and potentially new particles can be produced. The intermediate particles decay or interact and can then measured by the detector, where different regions of the detector specialize in the measurement of specific signatures or features of different particles. The overall p-p collision is then reconstructed or essentially reverse engineered through final state particles interpeted through observable quantities such as energy and momentum.   

%Here talk about accelrator concept + hardware, detector concept + hardware, the anatomy of physics events the interpretation of the events through measureed observables, 
we accelerate particles to collided them and produce new particles


\section{The Large Hadron Collider}
The LHC is a circular collider designed to collide proton beams with a centre-of-mass energy of 14 TeV and an instantaneous luminosity of $1034 \text{cm}^{-2}\text{s}^{-1}$.(cite lhc paper direct quote). The main accelerator ring consists of two counter rotating proton beams which are incased in an ultra high vacuum to protect the beam from interactions. The beams are accelerated with cryogenic electro-magnets which operate at -273C and are cooled by liquid helium. There are two types of magnets present, 1232 dipole magnets which bend the beam around the ring and 392 quadrapole magnets which focus the beams.the beams are structured etc bunches TALK AbOUt BEAMS HERE. Currently there are two completed data taking periods at the LHC denoted as Run I, and Run II. The integrated luminosity respectively per run is $58\text{fb}^{-1}$ and $138\text{fb}^{-1}$ with an expected cumulative integrated luminosity of up to $500\text{fb}^{-1}$ Including the presently ongoing Run III.  

%%TODO add bunch structure 23ns info

%machine layout and goals: \\
%rings magnets, bends beam, quadrapole focuses beam, rf cavities add energy to beam


\section{The CMS Detector}

The CMS detector is a hermetic shell that encapsulates the two counter rotating proton beams. The beams collide at the center of the detector and produce outgoing showers of particles that travel transverse to the beam axis. The observable outgoing particles, depending on the type of particle, are then measured in one of the specialized concentric layers of the detector. The initial transverse depiction of p-p interaction and intermediate particles can then be reconstructed from the energy measured in the detector. The total final state energy longitudinal to the beam axis is unknown because the momentum fraction of the initial quarks is unknown as well as some final state particles travel outside detector acceptance along the beam-line.  The detector also not generally record every collision but works on a triggering system meaning that the detector records and event and if the event is sufficiently interesting, say due the presence of a muon, a snap shot of the event is taken and the event is permanently recorded.

The reconstruction story of a particle traversing the detector is as follows. Primary particles are produced at a primary interaction point, other less energetic p-p interactions can occur in the same detector snapshot and are considered the underlying-event or pile-up which could be considered as a form of noise obfuscating the primary interaction. Both Charged and neutral particles traverse the first region of the detector, the silicon tracker. The silicon tracker consists of concentric thin electronic sensors that register "hits" from only charged particles. Each sequence of hits can be connected into a track which is a reconstruction of the path and origin of the charged particle.  The next region encounted by traveling particles is the Electromagnetic Calorimeter (ECAL). In the ECAL consists of scintillating pbW04 crystals that stop electrons and photons and measure their deposited energy. The energy deposits from photons and electrons are distinguished by tracks that connect to ECAL showers. Following the ECAL, is the hadronic calorimeter (HCAL) which consists of brass and plastic scintillators. The HCAL stops the heavy, remaining particles which are yet to interact and measures their energy. The last two regions of the detector are generally only traversed by the muon and a the centerpieces of CMS. The muons first stop would be through the solenoidal magnet, which generates a 4 Tesla uniform magnetic field throughout all of the inner regions of the detector. The magnetic field serves two purposes, first charge particles path bends in the presence of a magnetic field and that bend is either clockwise or counter clockwise around the beam axis depending on the charge, and thereby distinguishing the charge of the particle.  Secondly the momentum of tracks is measured their radius of curvature, or how much they bend in the magnetic field. The final stop for a muon is the muon chamber, which similar to the tracker, registers a sequence of hits via drift tubes or cathode strips. The tracks in both the tracker and muon chambers can then be combined to precisely measure the momemntum of the muon. resistive plate chambers reduntant triggering system 




%\section{The Anatomy of a physics event}
%	Reconstruction of particles
%	observables


%
\setcounter{secnumdepth}{3}
\setcounter{tocdepth}{3}
\setlength{\parskip}{\smallskipamount}
\setlength{\parindent}{0pt}


\makeatletter


\providecommand{\tabularnewline}{\\}


\makeatother

%\usepackage{babel}
%\begin{document}

\chapter{Compressed SUSY Search}

%\begin{chapterabstract}
%This chapter summarizes the approach for a compressed susy search and pertinent sensitive kinematic variables that the analysis is based on.
%\end{chapterabstract}

\section{Compressed Search Introduction}
In accordance with the strong experimental and phenomenological motivation for compressed spectra accompanied by the pursuit to comprehensively test SUSY, we conduct a search designed to be generic sensitive to many compressed final states. The work in this chapter is based largely on the previous work by C. Rogan and the Recursive Jigsaw Reconstruction (RJR) framework \cite{AN}. The targeted processes include, but are ultimately not limited to the production of stops, electroweakinos, and sleptons with diagrams included in Figure \ref{susyfeyn}.  
\FigFive{RJR_figs/T2tt.pdf}{RJR_figs/T6bbWW.pdf}{RJR_figs/TChiWW.pdf}{RJR_figs/TChiWZ.pdf}{RJR_figs/TSlepSlep.pdf}{Feynman diagrams for SUSY pair produce processes. Top row involves di-stop pairs where the intermediate state on the top left undergoes stop to top decay and the top right instead undergoes a stop to chargino decay. The middle left diagram show Chargino pairs decaying to two W boson and the middle right is a Neutralino and Chargino pair which decay into a W and Z boson final state. The bottom diagram shows dislepton production with sleptons decaying directly to SM leptons}{fig:susyfeyn}
The common thread between between all of these processes is a pair produced visible system alongside a massive invisible system. In the case of a compressed scenario, most of the energy available in the system is used by the rest mass of the LSP. These small mass splittings leads to low momemtum visible products that are difficult to reconstruct or are undetectable. In the case of intermediate massive particles, such as W or Z boson, these are forced off-shell so the visible products receive even less momentum, as the available energy goes into the mass of the intermediate particle. In order to identify these type of events we study cases with significant initial-state radiation (ISR). The ISR system recoils against, or boosts, the sparticle system, leading to high missing transverse momentum which is a tractable experimental signature. A depiction of this type of event is shown in Figure \ref{fig:isrrecoil}. With the ISR assisted pair produced topology we categorize and subdivide the visible and invisible system using RJR to approximate various rest frames \cite{PhysRevD.96.112007}. From those rest frames, we compute a basis of kinematic observables that describe features aid in the discrimination of compressed SUSY against SM processes.

\FigOneScale{RJR_figs/ISRrecoil.png}{Illustration of a visible ISR system recoiling against a Sparticle system which can be decomposed into soft visible subsystem and massive invisible subsystem}{fig:isrrecoil}{0.4}

%mentally all supersymmetry searches tend to exhibit their weakest limits in corridor regions with low mass differ-
%ences; if supersymmetry is to be tested comprehensively this region must also be explored. Phenomenologically,
%the lowest lying states in the electroweakino sector \\
%Given the lack of compelling conclusive evidence against supersymmetry, we conduct a broad search
%with many final states that involve missing energy. This is well motivated and does not rely strongly on theoretical
%assumptions.


%A compressed system is defined by a sparticle such as a neutralino 2 or stop in which the mass difference with this particle and the lightest supersymmetrical particle is small. The mass difference is considered small when the sparticle decays to intermediate standard model particles like W,Z,t such that the intermediate particle is forced off shell. For example the smallest targeted mass splittings can range between 3 to 10 GeV in neutralino 2 to W/Z decays. The intermediate decays will be difficult to detect or separate from standard model backgrounds. To assist in identifying compressed topologies we look for ISR assisted events. The signature of the event then becomes an ISR jet back to back with sparticle system which consists of mostly missing transverse energy from the LSP and soft SM particles.


\section{ RJR Reconstruction}
 The ISR assisted topology involves a collimated invisible and soft visible system together recoiling against another visible system of ISR jets. Each event is organized by imposing a decay tree onto the visible $(V)$ and invisible objects $(I)$ and assigning the visible components to either the ISR or sparticle $(S)$ side of the event. From the initial assignment, the sparticle system is further subdivided into subsystems A and B where each sub system has a visible $(V_{a/b})$ and invisible  $(I_{a/b})$ component. The three decay trees that contains these sets of objects in different reference frames are illustrated in Figure \ref{fig:decaytrees}. 
\FigThreeScale{RJR_figs/tree_CM.pdf}{RJR_figs/tree_ISR.pdf}{RJR_figs/tree_S.pdf}{A depiction of the three reference frames used in this RJR analyis. The top left shows the lab frame which is composed of the visible and invisible systems. The top right shows the CM frame in which the visible objects are divided into either ISR or visible objects associated with the sparticle system. The bottom figure further breaks down the sparticle system into a system composed of sparticle pairs of which both have their own visible and invisible subsystem.}{fig:decaytrees}{0.4}{0.4}{0.4}
In order to resolve the two invisible four momenta, kinematic and combinatoric unknowns need to be estimated. For example, the visible products are indistinguishable, so, there needs to be a metric which dictates the assignment of the visible objects to either the ISR or sparticle system. Similarly, the sparticle subsystem partitioning, both visible and invisible, must be determined when only the transverse invisible momentum is known. Thus, the combinatoric assignment and the estimated four momenta depend on each other, so, we  apply a set of rules to simultaneously determine both.  RJR provides the framework and rules to organize and evaluate each event \cite{PhysRevD.96.112007}. The set of rules used by this analysis are as follows :
\begin{itemize}
\item[1.] Assign charged leptons to the S system
\begin{itemize}
	\item Target leptonic sparticle signatures and fixing the leptonic system to always recoil against ISR
\end{itemize}
\item[2.] Fully determine the set of $\{V,ISR\}$ objects by assigning other visible objects to either the S or ISR systems by maximizing the momentum of the sparticle system in the CM frame
\begin{equation}
\{V,ISR\} =  \underset{V,ISR}{\arg\max} \, p_S^{CM}
\end{equation}
\item[3.] The visible S system objects are assigned to $V_a$ or $V_b$ by minimizing the mass of the sparticle subsystems i.e. grouping objects that traveling in similar directions
\begin{equation}
\{V_a,V_b\} = \underset{V_a,V_b}{\arg\min} \, M_{P_a}^2 + M_{P_b}^2
\end{equation}
\item[4.] Adjust the total mass of the invisible system according the visible systems where the individual invisible masses are constrained to zero.
\begin{equation}
M_I^2 = M_V^2 - 4M_{V_a}M_{V_b}
\end{equation}
\item[5.] Estimate the longitudinal component of invisible momentum by minimizing CM mass, i.e. determining the transverse mass of V+I systems
\begin{equation}
 \vec{\beta}_{CM,z}^{\text{lab}} = \underset{\vec{\beta}_{CM,z}^{\text{lab}} } {\arg\min} \, M_{CM}
\end{equation}
\item[6.] Determine the full kinematics of $I_A$ and $I_B$ and momentum partitioning  by evaluating the S frame velocities also through minimizing the mass of the sparticle subsystems
\begin{equation}
\vec{\beta}_{P_a}^S, \, \vec{\beta}_{P_b}^S = \underset{\vec{\beta}_{P_a}^S, \, \vec{\beta}_{P_b}^S}{\arg\min} \, M_{P_a}^2 + M_{P_b}^2
\end{equation}
\end{itemize}

By recursively iterating through the various combinations of objects and determining  the four momenta of the groupings from Figure \ref{fig:decaytrees}, the optimal organization for the event is determined that satisfies the aforementioned RJR prescription. The consequences of this organization is discussed in the following section where we construct a basis of kinematic variables to exploit characteristics of compressed SUSY and discriminate against SM processes.



%An isr assisted event is divided into multiple reference frames. The CM frame consists of the particles measured in the lab e.g. the isr jet against the met system. The sparticle frame consists two subsystems A and B. THe sparticles are expected to be pair produced if r parity is conserved.

%The kinematic variables the form basis of the search is RISR and MPERP.
%Risr. RISR is process independent and peaks at the ratio of sparticle/lsp masses. mperp is the transverse mass of the sparticle frame with respect to the sparticle frame boost axis.  
\section{Compressed Kinematics}

The main observables are designed to be sensitive to the properties of compressed SUSY. One of these properties is the mass of the invisible particle in the event. With the ISR assisted system recoiling against a massive invisible particle, the invisible gets a large momentum kick from ISR.  This characterisic can be exploited with the variable $R_{ISR}$ which is defined as:
\begin{equation}
R_{ISR} = \frac{|\vec{p}_I^{CM} \cdot \hat{p}_{ISR}^{CM}|}{|\vec{p}_{ISR}^{CM}|} \sim \frac{m_I}{m_P}
\end{equation}

In a compressed scenario, the fraction of the invisible momemtum to the total ISR kick would be expected to be close to one, as the visible system uses a small fraction of the momentum of the event and the ISR vs invisible system would be nearly anti parallel. The peak of the $R_{ISR}$ distribution can be approximated by the ratio of the true invisible mass to its sparticle parent mass.  This behavior of $R_{ISR}$ is that is spread in the distribution is reduced and the peak approaches one as the level of compression increases. Similarly the $R_{ISR}$ behavior doesn't depend specifically on the underlying SUSY process, but only that there is a heavy invisible system recoiling against ISR.  SM backgrounds do not exhibit the same behavior in $R_{ISR}$, so, the result is strong discrimination between compressed SUSY and SM at high $R_{ISR}$ which actually improves as the mass splittings $\Delta m = m_P - m_I$ decrease. An illustration of the $R_{ISR}$ shapes comparing signal to background is show in Figure \ref{fig:risrshape}

\FigTwo{RJR_figs/RISR_1L_T2tt.png}{RJR_figs/RISR_1L_bkg.png}{$R_{ISR}$ shapes with a basic 1 lepton selection for stop pairs on the left comparing  a range of mass splittings. The right distribution is the $R_{ISR}$ shape for the components of the full SM background with the same basic 1 lepton selection}{fig:risrshape}

Another observable which is uncorrelated with ISR but also sensitive to compressed topologies is $M_\perp$. $M_\perp$ is constructed from the average squared masses of the sparticle subsytems $M_{P_{a/b}}$ and is explicitly defined as
\begin{equation}
M_\perp = \sqrt{\frac{M_{P_a\perp}^2 + M_{P_b\perp}^2}{2}}
\end{equation}
The indivdual invisible masses are constrained to zero, and, in the case of massive invisible particles is an incorrect assumption. The consequence is then that $M_\perp$ is sensitive to the inherent mass splittings between the parent sparticle and LSP. The behavior of the $M_\perp$ distribution is that it exhibits a kinematic endpoint or edge at the $\Delta m = m_P - m_I$ and gives the strongest discrimination against SM processes at larger values of $M_\perp$, which may still be compressed e.g. in case of intermediate top quarks. An example of the shape of the $M_\perp$ distributions in with signal versus background is shown in Figure \ref{fig:mperpshape}

\FigTwo{RJR_figs/Mperp_0L_T2tt.png}{RJR_figs/Mperp_0L_bkg.png}{$M_\perp$ shapes with no leptons for di stop pairs on the left comparing a range of mass splittings. The right distributions is the $M_{\perp}$ shape for the compoents of the full SM background which also is absent of reconstructed leptons. }{fig:mperpshape}

The combination of both $R_{ISR}$ and $M_\perp$ form a 2-D plane in which to conduct a bump hunt which is illustrated in Figure \ref{fig:risrperp2d}. Here the localization of the bump in the plane depends on the sparticle masses, the larger both masses the less over spread in the distribution. Also the smaller the mass splitting between sparticles, the peak of the distribution is pushed sharply towards 1 in $R_{ISR}$ and the stronger the discrimination against SM backgrounds becomes. Binning in this 2D plane gives us the most sensitive signal region at high $R_{ISR}$  while the opposite, low $R_{ISR}$ provides a background rich region to constrain the background yields in the sensitive region. 

\FigTwo{RJR_figs/RISR_v_Mperp_2L_TChiWZ_250_240.png}{RJR_figs/RISR_v_Mperp_2L_ttbar.png}{2D distributions of $R_{ISR}$ and ${M_\perp}$ with a basic 2 lepton selction for Neutralino-Chargino pairs with LSP and NLSP mass splitting of 10 GeV on the left and total SM background on the right. The bump from the left distribution develops a more resolved peaked and better signal to background at its center in cases with increasing compression.}{fig:risrmperp2d}


Two additional kinematic variables are utilized to aid in discrimination against SM backgrounds, both are less powerful than $R_{ISR}$ and $M_\perp$ but still very useful. The first quantity is complementary to $R_{ISR}$ and is the transverse momentum of the ISR system $p_T^{ISR}$. The more momentum in the ISR system means the sparticle system gets kicked harder - leading to better resolution in the $R_{ISR}$ distribution from the $p_T^{ISR}$ $R_{ISR}$ correlation. Fortunately the $R_{ISR}$  distributions from backgrounds are anti correlated with $p_T^{ISR}$ which means that the combination of both high  $p_T^{ISR}$ $R_{ISR}$ provides a rich signal region. Both correlations for signal and background can are visualized in Figure \ref{fig:ptisrrisr}

\FigTwo{RJR_figs/PTISR_v_RISR_2L_T2tt_500_480-gif-converted-to.pdf}{RJR_figs/PTISR_v_RISR_2L_Wjets-gif-converted-to.pdf}{ptisr risr distributions}{fig:ptisrrisr}

 The other kinematic variable is $\gamma_\perp$ which is a measure of the symmetry of the di-sparticle system and is defined as:
\begin{equation}
\gamma_\perp = \frac{2M_\perp}{M_{S\perp}}
\end{equation}
Here $M_S$ is the mass composed of the transverse four momenta of all objects from  both sparticle subsystems. The behavior of the mass ratio $\gamma_\perp$ is that it tends to larger values asymmetry in the final state, as illustrated in Figure \ref{fig:gammat}. This is useful because it better isolates signals and backgrounds with have pairs of W or Z bosons. Since both complementary variables $p_T^{ISR}$ and $\gamma_\perp$ don't have the discriminating power or $R_{ISR}$ and $M_\perp$ we categorize events. With $p_T^{ISR}$ we have high or low categories, where the lower edge and low to high pivot depends on the lepton and jet multiplicity of the event. In the case of $\gamma_\perp$ we also construct high and low categories about the value $\gamma_\perp= 0.5$. The combination of high and low and the divides selected for each pair of categories is designed to create more signal sensitive high regions with background yields that are constrained by low background rich categories.

\FigTwo{RJR_figs/gammaT_2L_bkg.png}{RJR_figs/gammaT_2L_TChiWZ-gif-converted-to.pdf}{gamT dist}{fig:gammat}
  


%The analysis is generalized to deal with a broad range of signal models but the three targeted compressed signal processes incldue stop, neutralino/chargino, slepton production. the signals include T2tt, T2bW, TChiWZ, TChiWW, TSlepSlep


%
\setcounter{secnumdepth}{3}
\setcounter{tocdepth}{3}
\setlength{\parskip}{\smallskipamount}
\setlength{\parindent}{0pt}


\makeatletter


\providecommand{\tabularnewline}{\\}


\makeatother

%\usepackage{babel}
%\begin{document}

\chapter{Analysis Description }

\section{Introduction and Strategy}
The full analysis is built on the compressed kinematics described in the previous chapter. This goal of this chapter is to outline specfic details and the strategies used to potentially discover SUSY. This includes the description of events selected to analyze and the objects that compose these events. This search casts a wide net to capture a wide variety of signatures and final states, the consequence of this strategy yields a large number of categories and bins that cover many multiplicites of lepton and jet final states. Finally, I will discuss the data driven approach to constrain and predict background events in the most sensitive regions by conducting a series of fits to construct a robust fit model.


\section{ Data and Simulation}
The analysis invovles the full Run II dataset which is divided into three subsets by the years 2016,2017,2018 and total integrated luminosity of $138 \text{fb}^{-1}$. Each year is comprised of $36.31 \, \text{fb}^{-1} \, \pm 1.2\%$ (cite lumi 17 003), $41.48 \, \text{fb}^{-1} \, \pm 2.3\%$ (cite lumi 17 004), $59.83 \, \text{fb}^{-1} \, \pm2.5\%$ (cite lumi 18-002) in 2016, 2017, and 2018 respectively. The data is modeled by MC that represents the full SM background and is qualitatively grouped by process and final state. These grouping of SM backgrounds is defined in table \ref{tab:bkgsigtab} along with the partnered signal models.

%insert bg list and description
\begin{table}
\label{tab:bkgsigtab}
\caption{table caption}
\begin{tabular}{c|c}
\hline 
Bkg. Label & Bkg. Composition \\ 
\hline 
\hline

W + jets & \makecell{Single W boson, a dominant background that composes \\ about $50\%$ of the total background} \\ 
  & \\
tt+jets & \makecell{$t\bar{t}$ which can be accompanied by a W,Z,h, or $\gamma$, \\ the other dominant background composes about $50\%$ of the total background} \\ 
  & \\
ZDY & Z+jets and Drell Yan, an intermediate background \\ 
Di-boson (DB)& WW,ZZ,WZ,Wh,Zh,  an intermediate background \\ 

ST & Single top processes including tW, a rare background \\ 

Tri-boson (TB) & WWW,ZZZ,WWZ, WZZ, WZ$\gamma$, WW$\gamma$, a rare background  \\ 
\hline 
Signal Label & Signal Composition \\
\hline
\hline
T2tt & $pp \rightarrow \tilde{t} \tilde{\bar{t}}; \, \, \tilde{t}\rightarrow t \tilde{\chi}_1^0$ \\
TChiWZ & $pp\rightarrow \tilde{\chi}_2^0 \tilde{\chi}_1^\pm; \, \, \tilde{\chi}_2^0 \rightarrow Z \tilde{\chi}^0_1; \, \, \tilde{\chi}_1^\pm \rightarrow W^\pm \tilde{\chi}^0_1$ \\
TSlepSlep & $pp\rightarrow \tilde{\ell} \tilde{\ell}; \, \, \tilde{\ell} \rightarrow \ell \tilde{\chi}^0_1$ \\
TChipmWW & $pp\rightarrow \tilde{\chi}_1^\pm \tilde{\chi}_1^\mp; \, \,  \tilde{\chi}_1^\pm \rightarrow W^\pm \tilde{\chi}^0_1$  \\
\end{tabular} \\
\end{table}

The signals that will be addressed in this work include mutiple different sparticles and final states, which also will include multiple interpretation of their analysis results. A list of the signals is provided in table y. Each signal is produced according to an (LSP,NLSP) mass grid for each year.  The raw number of events per mass points and grid spacings for the signals shown in table \ref{tab:bkgsigtab} are displayed in Figure \ref{fig:grids} with all years combined. 

%insert signal list and description
\FigFour{Analysis_figs/T2tt_EventCount.pdf}{Analysis_figs/TChiWZ_EventCount.pdf}{Analysis_figs/TSlepSlep_EventCount.pdf}{Analysis_figs/TChipmWW_EventCount.pdf}{grids}{fig:grids}

The majority of signal and backgrounds use the MadGraph \cite{madgraph} generator to model at LO and NLO. The ST backgrounds use PowHEG 2.0 \cite{powheg} to model at NLO. Parton shower and fragmentaion for all samples is done with PYTHIA 7 \cite{pythia}. Each year is subjected to an underlying event tunes with CUETP8M1 for 2016, CP2 for 2017 and 2018 signals and CP5, for 2017 and 2018 backgrounds \cite{erich tunes}. The detector conditions and response are simulated for all samples with GEANT4 \cite{geant}.
 
\section{Event Selection and Physics Objects}
 For events to be qualified for analysis, they pass a handful of selected triggers and preselection which reflect the compressed kinematic description provided in the previous chapter. The triggers used PFMET PFMHT cross triggers. PFMET is the particle flow missing transverse energy which is expected to capture the $p_T^{Miss}$ from the LSP. PFMHT is expected to trigger on events with significant jet activity and multiplicity which are likely candidates for ISR events.  In general, the compressed SUSY topology is are somewhat rare organization of an event, due to the uncommon nature of these types of events the MC modeling does not always sufficiently and precisely describe data, so,  a data driven approach is utilized for physics objects to compensate for any disagreement. This approach compares the overall efficiency or behavior of selected objects and computes data driven scale factors, while also providing a platform to model and understand systematic effects. These calibrations are then applied to MC to bring data and MC into agreement. The instance of Scale factor generation arises in the modeling of the efficiencies of the trigger turn ons, with efficiency defined as events that pass the trigger and preselection versus only preselection. The comparisons of data with  which is shown in Figure \ref{fig:metsf} with the efficiency shapes modeled by a Gaussian CDF.  

%insert trigger turn ons
\FigOne{Analysis_figs/SF_Plot_HT-Le600--SingleMuontrigger-E1--Nmu-E1_SingleMuon_2016.pdf}{met trig sf}{fig:metsf}

The aforementioned pre selection which is a set of carefully studied criteria to remove background and mismodeled events as well as select events with objects that are expected to populated targeted final states. The pre selection consists of the criteria listed in the following Table Z:

%insert preselection table  
\begin{tabular}{c|c}
\hline 
\multicolumn{2}{|c|}{Preselection Requirements} \\ 
\hline 
Criteria & Description \\ 
\hline 
\hline
$N_V \geq 1$ & At least one visible object assigned to the S system \\ 
$N_j^{ISR} \geq 1$ & At least one jet assigned to the ISR system \\ 

$p_T^{miss} > 150$GeV &\makecell{ Minimum transverse missing energy based on trigger efficiency}  \\ 

$p_T^{ISR} > 250 $GeV & \makecell{Minimum ISR kick to resolve massive invisible particles} \\ 

$R_{ISR}$ > 0.5 & Target Massive LSPs \\ 

$|\Delta \phi_{\vec{p}_T^{miss}, V}| < \pi/2$ &  Ensures visible and invisible system are traveling in the same direction \\ 

$p_T^{CM} < 200$  & Rejects mismodeled events \\ 

veto $f(\Delta\phi_{CM,I}, p_T^{CM})$& 2D function to also reject mismodeled events - See Fig \ref{fig:cleancut}\\
\hline 
\end{tabular} \\

\FigOne{Analysis_figs/cleaningCuts.pdf}{cleaning cuts}{fig:cleancut}

The pre selection forms the basis for an event to be analyzed. Following preselection, the physics objects can be selected, classified, and categorized. The possible object composition can consist of jets, b-tagged jets, soft secondary vertices (SVs), and leptons. The discussion of the leptons and their classification will be reserved for the following chapter alongside the calculation of lepton scale factors. A summary of these physics objects and their kinematic requirements are listed in the following Table X. The various types of jets involve working points (WP) from their respective physics object groups and are standard objects used in CMS physics analysis (cite jet stuff ak4 and ids). The b-tagging is done by a standard NN based tagger which only identifies b-jets down to 20 GeV. A complementary SV tagger was developed specifically for this analysis and a detailed description of this tool is described in this analysis's sister thesis (cite erich thesis). The purpose of the SV tagger is to efficiently extend the b-tagging range down to 2 GeV because a final state topology with something like compressed stops often includes soft b-jets.


\begin{table}
\centering
\label{tab:physicsobjects}
\caption{table caption2}
\begin{tabular}{c|c}
\hline 
\multicolumn{2}{c}{Visible Physics Objects} \\ 
\hline 
\hline
Jets & \makecell{AK4 PF Jets \\ Tight ID \\ $p_T^{jet} > 20$ GeV \\ $|\eta|>2.4$} \\ 
\hline
B-tagged Jets & \makecell{AK4 PF Jets \\ DeepJet Medium WP \\ $p_T^{b-jet} > 20 $ GeV}  \\ 
\hline
SVs & $2<p_T^{SV}<20$ GeV \\ 
\hline
Leptons & \makecell{Very Loose ID \\ $p_T^{\mu^\pm} > 3$ GeV \\ $p_T^{e^\pm} > 5 $GeV \\ Gold/Silver/Bronze quality classes} \\ 
\hline 
\end{tabular} 
\end{table}

\section{Categorization and Fit Strategy}

Once an event passes all the preselection and the objects are classified, the event is then categorized based on object composition, object multplicity, and kinematic characteristics. 

% dont forget bin optimization


%\newcommand{\ID}{\text{ID}}
\newcommand{\Prompt}{\text{Prompt}}
\newcommand{\Isolated}{\text{Isolated}}
\newcommand{\Gold}{\text{Gold}}
\newcommand{\Silver}{\text{Silver}}
\newcommand{\Bronze}{\text{Bronze}}


\newcommand\FigureOne[3]{%
\begin{figure}[!htbp]%
\centering
\includegraphics[width=0.5\textwidth]{fig/Lep_Obj_plots/#1}\hfill
\caption{#2}
\label{#3}
\end{figure}}


\newcommand\FigureFour[6]{%
\begin{figure}[!htbp]%
\centering
\includegraphics[width=0.5\textwidth]{fig/Lep_Obj_plots/#1}\hfill
\includegraphics[width=0.5\textwidth]{fig/Lep_Obj_plots/#2}\hfill
\includegraphics[width=0.5\textwidth]{fig/Lep_Obj_plots/#3}\hfill
\includegraphics[width=0.5\textwidth]{fig/Lep_Obj_plots/#4}\hfill
\caption{#6}
\label{#5}
\end{figure}}

\newcommand\FigureThree[5]{%
\begin{figure}[!htbp]%
\centering
\includegraphics[width=0.4\textwidth]{fig/Lep_Obj_plots/#1}
\includegraphics[width=0.4\textwidth]{fig/Lep_Obj_plots/#2}\hfill
\includegraphics[width=0.4\textwidth]{fig/Lep_Obj_plots/#3}\hfill
\caption{#4}
\label{#5}
\end{figure}}

\newcommand\FigureTwo[4]{%
\begin{figure}[!htbp]%
\centering
\includegraphics[width=0.5\textwidth]{fig/Lep_Obj_plots/#1}\hfill
\includegraphics[width=0.5\textwidth]{fig/Lep_Obj_plots/#2}\hfill
\caption{#4}
\label{#3}
\end{figure}}

\newcommand\FigureStack[4]{%
\begin{figure}[!htbp]%
\centering
\includegraphics[width=0.75\textwidth]{fig/Lep_Obj_plots/#1}\hfill\\
\includegraphics[width=0.75\textwidth]{fig/Lep_Obj_plots/#2}\hfill
\caption{#4}
\label{#3}
\end{figure}}

\setcounter{secnumdepth}{3}
\setcounter{tocdepth}{3}
\setlength{\parskip}{\smallskipamount}
\setlength{\parindent}{0pt}


\makeatletter


\providecommand{\tabularnewline}{\\}


\makeatother

%\usepackage{babel}
%\begin{document}

\chapter{The Tag-and-Probe}

\begin{chapterabstract}
The Tag-and-Probe is a method used to measure the selection efficiencies of an object using data. In the context of this compressed SUSY analysis, the Tag-and-probe measures the efficiencies separately of each light lepton($e/\mu$) selection critera. The total lepton selection efficiency is then computed by combining factorized efficiency components. The same general method is used for both electrons and muons, however, Muons utilize  the $J/\psi$ di-muon trigger which allow more precise efficiency measurements from data at lower $p_T$.
\end{chapterabstract}

\section{Introduction and Methodology}
An important element of a lepton based search is properly modeling the efficiency of selected leptons. A purely Monte-Carlo driven approach is inadequate in perfectly describing nuances in data due to imperfections in modeling. Instead of trying to model exactly all physics and detector effects with simulation, the efficiencies can be directly measured from data by using the Tag-and-Probe method. 

The Tag-and-Probe method is used to measure a selection criteria by using a well known resonance such as a $Z$, $J/\psi$, or $\Upsilon$ and counting the number of probes that pass that criteria. Each counted instance of the Tag-and-Probe consists of two selected leptons. One of the selected leptons is the tag and the other is the probe.  The tag passes tight selection requirement to give high confidence that it isn't a fake lepton. Fake leptons fall into two possible categories: reducible and irreducible. A reducible fake lepton is a particle that fakes the signature of a lepton such as a charged pion. An irreducible fake lepton is an actual lepton which coincidentally passes some selection criteria but is not the targeted leptons of interest e.g. an isolated muon from a jet accompanying a leptonic Z decay of interest.  The  second lepton in the Tag-and-Probe is the probe. The probe is subjected to the selection criteria whose efficiency is being measured. The invariant mass of the pair of leptons is calculated and required to fall within a defined range around the resonance. A particular event may have multiple lepton pairs but the tag and the probe are not allowed to switch positions and be counted twice, as double counting would lead to a bias in the efficiency measurement \cite{AN111-2009}. To avoid bias, the tag and probe are required to be the opposite charge and same flavor where the tag is randomly selected. If multiple same flavor lepton pairs occur in single event i.e. there are multiple probes to a single tag, the treatment for selecting the pairs differs between electrons and muons. There is no specific study which led to justifying the differing arbitration approaches in flavors, only that the choice reflects the default choices implemented in the existing code bases.  For muons, no arbitration is used, all pairs are utilized which means an additional pair not truly from the resonance will then contribute as combinitorial background in a single event. For electrons, only a single probe is selected per event which has the highest \pt. The selected probes can either pass or fail their selection which leads to the formation of three distributions, one with a passing probe, one with a failing probe, and one with all probes. An example of all three distributions is shown in Figure \ref{tnpexp}.  The probability of observing $k$ passing probes in $n$ Tag-and-Probe pair trials is dependent on the selection efficiency $\varepsilon$ and can be expressed as a likelihood from the binomial probabilty density $P(k|\varepsilon,n) = \binom{n}{k}\varepsilon^k(1-\varepsilon)^{n-k}$. The MLE estimator for efficiency is then the fraction of passing probes to the total number of pairs, or $\varepsilon = k/n$. Technical documentation for the Tag-and-Probe in CMS is scarce, but, an early strategy for fitting efficiency is defined in \cite{Berryhill_2010}. The legacy code base as of \url{CMSSW_10_6_X}  uses a binned maximum likelihood between the observed passing probes and failing probes where the efficiency extracted is an explicit fit parameter. The two simultaneously fit functions are:
\begin{equation}
	N^{\text{Pass}} = N_{\text{Total}} (\varepsilon \cdot f^{\text{sig}}_{\text{All}} ) +  \varepsilon_{\text{bkg}} \cdot (1-f^{\text{sig} }_{\text{All}}) )
\end{equation} 
\begin{equation}
	N^{\text{Fail}} = N_{\text{Total}} ( (1-\varepsilon) \cdot f^{\text{sig}}_{\text{All}} +   (1-\varepsilon_{\text{bkg}}) \cdot (1-f^{\text{sig}}_{\text{All}}) )
\end{equation}

$N^{\text{Pass/Fail}}$ is the total number of observed probes that either pass or fail the selection criteria while $N_{\text{Total}}$ is the total number of Tag-and-Probe pairs.
The binomial estimator for efficiency, $\varepsilon$, enters the fit functions as the first term but is accompanied by a second term that describes the background contribution with its own efficiency $\varepsilon_{\text{bkg}}$.  The term $f^{\text{sig}}_{\text{All}}$ is the fraction of background subtracted signal events over the allowed dilepton mass range.  $f^{\text{sig}}_{\text{All}}$ depends on the defined signal and background pdfs. The nominal pdfs chosen for reported fits uses a 5 parameter Voigtian+Voigtian signal model which share a common mean but use independent $\Gamma$ and $\sigma$.  The signal model is combined with an Exponential background model. 


\FigureThree{PassingProbes_Med16_lowpt.png}{FailingProbes_Med16_lowpt.png}{AllProbes_Med16_lowpt.png}{Example Tag-and-Probe Z di-muon fits for passing,failing, and all probes with the Medium Id, $|\eta|<1.2$, and $p_T < 20$ GeV   }{tnpexp}



%The distributions are fit simultaneously with a combined signal and background model. To extract the probe criteria efficiecy we divide the resonance distributions by the all probes distribution. The uncertainty on an efficiency is a combination of a statistical and systematic uncertainties. The systematic uncertainties are defined by repeating the simultaneous fit with varying mass windows, number of bins, and fit models and measuring the maximum spread of the central values.


%The fit code is located here: \url{https://github.com/cms-sw/cmssw/blob/CMSSW_10_6_X/PhysicsTools/TagAndProbe/src/TagProbeFitter.cc} around line 600. confirm this with the fit model from the electron paper. \cite{Berryhill_2010}

\FloatBarrier
\section{Lepton Object Definitions}


Leptons are selected according to the minimium requirement ``VeryLoose'' which depend kinematic and topological quantities which are shown in Table \ref{tab:veryloose}. The electrons use an additional loose MVA requirement: MVA VLooseFO ID \cite{?}. The set of VeryLoose leptons are further subdivided by quality into three mutually exclusive categories: Gold, Silver, and Bronze. Each category has a measure of three main quantities, the first being the quality of the pre-determined Id. The Id's differ per flavor and are the standard working points defined by the corresponding physics object group. The muons use the Medium Id \cite{muMediumId} and electrons use a more strict selection, due to their messy nature, with the Tight Id \cite{eTightID}. The second quantity is the ``promptness'' or distance of the lepton production point from the primary vertex. Promptness is measured by the significance of the 3D impact parameter (SIP3D) which is defined as the impact parameter normalized by its measured error. A SIP3D $> 1$ is associated with a secondary particle which is not produced at the primary vertex. The last component is the  isolation, a measure of the density of particles in a cone around the lepton. Two similar but complimentary absolute isolations are used: PFIso \cite{murun2baseline} and MiniIso \cite{miniIso}. Both isolations are an energy sum of neighboring particles inside a cone, but, PFIso has a fixed cone size of $R=0.4$ cm  and miniIso cone sizes varies inversely with lepton \pt as shown in \ref{isoeq}.
\begin{equation}
R_{\text{miniIso}}=
    \begin{cases}
      0.2 & \pt^\ell < 50 \text{GeV}\\
      \frac{10}{\pt^\ell} & 50 \text{GeV} \leq \pt^\ell \leq 200 \text{GeV} \\
      0.05 & \pt^\ell > 200 \text{GeV}
    \end{cases}
    \label{isoeq}
\end{equation}

Mini isolation also includes effective area pile-up corrections provided in a look up table of bins of \pt and $\eta$ in the CMSSW Producer/Ntuplizing stage. The implementation of mini-isolation and their corrections utilize the same IsoValueMap producer as used in NANO AOD as of \url{CMSSW_10_6_X}.


The explicit flavor independent formulas for Gold, Silver, and Bronze can be generalized by the product of three components which are the measured efficiences of the three previously mentioned quantities. The efficiencies take the form of conditional probabilities to be measured independently in sequence relative to each other:
\begin{equation}\label{eq:efflep_general}
\begin{split}
\epsilon_{\Gold}& = \epsilon_{\ID}\times \epsilon_{\Isolated|\ID} \times \epsilon_{\Prompt|(\ID \cap \Isolated)} \\
\epsilon_{\Silver}& = \epsilon_{\ID} \times \epsilon_{\Isolated|\ID} \times (1-\epsilon_{\Prompt|(\ID \cap \Isolated)}) \\
\epsilon_{\Bronze}& = 1-(\epsilon_{\ID} \times \epsilon_{\Isolated|\ID)} )
\end{split}
\end{equation}

The subscript for an efficiency, e.g. $\epsilon_{\Prompt|(\ID \cap \Isolated)}$, reads as the efficiency to pass the SIP3D requirement given the lepton passes the Id and Isolation requirements. From equation \ref{eq:efflep_general} the Gold, Silver, and Bronze effiencies can be read off as Gold passes all criteria, Silver fails only the SIP3D requirement, and Bronze fails either the Id or isolation and is agnostic to SIP3D. While isolation and vertexing requirements are physically uncorrelated, there is an intersection between the two, meaning a lepton can be both prompt and isolated. This intersection then demands the necessity for conditional efficiencies.  The order of the conditional efficiencies is also chosen to minimize the number of measured efficiencies by reusing efficiencies across Gold, Silver, and Bronze.  





\begin{table}[htbp]
\centering
\caption{\label{tab:veryloose} The criteria that define the minimum requirements for an accepted lepton. The electron and muon requirements are equivalent in terms of pseudorapidity, vertexing, and isolation but vary in \pt threshold and the MVA VLooseFO working point. The MVA VLooseFO ID also varies between years.}

\begin{tabular}{c|c|c}
\hline
Criteria & Electron & Muon \\
\hline
\hline
\pt & $\geq 5$ GeV & $\geq 3$ GeV \\

$|\eta|$ & $<2.4$ & $<2.4$ \\
\hline

$\text{IP}_{3D}/\sigma_{\text{IP}_{3D}}$ & $<8$ & $<8$ \\

$|d_{xy}|$ & $<0.05$ cm & $<0.05$ cm \\

$|d_z|$ & $<0.1$ cm & $<0.1$ cm \\

\hline
$\text{PFIso}_{\text{abs}}$ & $<20 + (300/\pt)$ GeV & $<20 + (300/\pt)$ GeV \\

\hline
MVA VLooseFO ID & \checkmark  & --\\
\end{tabular}
\end{table}


The advantage of having various lepton quality categories allows for robust sensitivity to a wide range of signal processes. This strategy boosts the overall modeling statistics and provides control regions for multiple scenarios. %but also provide fake rich selection in control region that helps stabilize the overall fit and extract fake rates into the sensitive regions.  
The populations of different truth selected objects are shown in Figure \ref{andresPurity} and the overall efficiency for Gold, Silver, and Bronze on truth matched objects are shown in Figure \ref{andresEff}.  The gold region is mainly populated by prompt and isolated leptons that are produced within the primary vertex. This region also coincides with the signature of many targeted electroweakino models. The silver selection accomodates both leptonically decaying taus, providing an ideal region for stau's, and assists in recovering efficiency of isolated b decays in stop production. The bronze selection is rich in fake leptons and provides the best regions to extract overall fake rates for other regions as well as a surplus of events to anchor the fit. 



\FigureThree{gold_ele_TTJets_Fall17.pdf}{silver_ele_TTJets_Fall17.pdf}{bronze_ele_TTJets_Fall17.pdf}{Gold (Top-Left), Silver (Top-Right) and Bronze (Bottom) MC truth matching in TTJets sample 2017. Signal is defined here as prompt electrons from a $W$ decay.}{andresPurity}

\FigureOne{sigEff_ele_TTJets_Fall17.pdf}{Gold, Silver, and Bronze efficiency on truth matched prompt electrons as signal and secondary electons as Fakes.}{andresEff}

\FloatBarrier
\section{Electron Tag-and-Probe }

The electron tag and probe is done by using the Z resonance over the entire \pt range of selected electrons. The selected binnings follow the $\pt$ and $\eta$ binning conventions from the electron physics object group and are $ p_T \in [5, 10, 20, 30, 40, 70, 100]$ and $|\eta| \in [ 0, 0.6, 1.4, 2.4]$. The electron Tag-and-Probe tools uses a centrally curated CMSSW PhysicsTools in \url{CMSSW_10_2_X}. The software pipeline consists of two steps, an ntuplizing stage and a fitting stage. The  Ntupilizing stage selects Tag-and-Probe pairs along with all potential variables of interest and loads them onto an ntuple using \url{TnPTreeProducer} \cite{ElTnPGit}. The samples used in the Ntuplizing stage are listed in Table \ref{tab:electronTnPSamples}. In the fitting stage, a random subset of of TnP pairs are sampled with \url{TnPTreeAnalyzer} \cite{ElTnPAnaGit}. The analyzer performs all of the fitting and efficiency measurements according to the specified selection criteria. 

%notes to add back in later 2016B, 2017C, 2018A
\begin{table}
\caption{ Data and MC samples for each year used for the electron Tag-and-Probe. }
\label{tab:electronTnPSamples}
\scriptsize
\begin{tabular}{|c|c|c|}
\hline 
Type & Year & Sample Name \\ 
\hline 
Data & 2016 & \tiny \url{/SingleElectron/Run2016-17Jul2018_ver2-v1/MINIAOD}  \\  
Data & 2017 & \tiny \url{/SingleElectron/Run2017-31Mar2018-v1/MINIAOD} \\  
Data & 2018 & \tiny \url{/EGamma/Run2018-PromptReco-v1/MINIAOD} \\ 
\hline 
MC & 2016 & \tiny \url{/DYJetsToLL_Pt-100To250_TuneCUETP8M1_13TeV-amcatnloFXFX-pythia8/RunIISummer16MiniAODv3-PUMoriond17_94X_mcRun2_asymptotic_v3_ext5-v2/MINIAODSIM} \\ 
MC & 2017 & \tiny \url{/DYJetsToLL_Pt-100To250_TuneCP5_13TeV-amcatnloFXFX-pythia8/RunIIFall17MiniAODv2-PU2017_12Apr2018_94X_mc2017_realistic_v14-v1/MINIAODSIM} \\ 
MC & 2018 & \tiny \url{/DYJetsToLL_Pt-100To250_TuneCP5_13TeV-amcatnloFXFX-pythia8/RunIIAutumn18MiniAOD-102X_upgrade2018_realistic_v15-v1/MINIAODSIM} \\ 
\hline
\end{tabular} 
\end{table}




A general selection is applied for electron TnP candidates. The selection for electrons differs between the tag and probe, but, both depend on super cluster (SC) kinematics. The super clusters are expected to fall within the calorimeter acceptance which includes vetoing super clusters in the endcap gaps. The invariant mass of the electron of the pair also is required to fall within a specified Z-window. The selection specifics are listed in Table \ref{tab:eleTnPSelect}.  The tag is also required to pass a trigger requirement to reflect the inherit trigger bias which is not applied in simulation by default. The triggers selected are HLT electron collections and are grouped by specific paths and filters. The electrons are matched to trigger objects in the path/filter combination and passed based on the OR of triggers in the collection. The probes are not subjected to trigger matching. The chosen trigger combinations are \url{HLT_Ele27_eta2p1_WPTight_Gsf_v*}, \url{HLT_Ele32_WPTight_Gsf_L1DoubleEG_v*}, \url{HLT_Ele32_WPTight_Gsf_v*} for 2016 through 2018 respectively.\\

\begin{table}
\caption{selection}
\label{tab:eleTnPSelect}
\begin{tabular}{|c|c|c|c|}
\hline 
\multicolumn{4}{|c|}{Tag-and-Probe Electron Candidate Selection Criteria} \\ 
\hline 
Tag & Probe & Super Cluster & Pair \\ 
\hline 
$|\eta_{SC}| \leq 2.1$ & $|\eta_{SC}| \leq 2.5$  & $|\eta|<2.5 $ & $50 \text{GeV} < m_{ee} < 130 \text{GeV} $ \\
veto $ 1.4442 \leq |\eta_{SC}| \leq 1.566 $ & $E_{ECAL}\sin(\theta_{SC}) > 5.0 $ GeV & $E_T > 5.0 $ GeV &  \\
 $\pt \geq 30.0$ GeV &  &  &  \\
 Passes Tight Id &  &  & \\
\hline 
\end{tabular} 
\end{table}



%\begin{center}
%\begin{tabular}{@{}l@{}} 
%\tabitem 2016: \url{HLT_Ele27_eta2p1_WPTight_Gsf_v*} \\
%\tabitem 2017: \url{HLT_Ele32_WPTight_Gsf_L1DoubleEG_v*}\\
%\tabitem 2018: \url{HLT_Ele32_WPTight_Gsf_v*} \\
%\end{tabular} 
%\end{center}


%\end{center}
The measurments of the gold silver and bronze efficiencies components, based on Equations \ref{eq:efflep_general}, are shown in Figure \ref{fig:el2deff}. The relative efficiencies per component range from approximately $75\%$ to $95\%$ with a slight dependence on $|\eta|$ which is the strongest lower \pt. The largest combined systematic and statistical errors are $O(4\%)$ and occur in data with the lowest \pt bins. The data and MC agreement is within a few percent for both the Id and Isolation but the average data and MC agreement in SIP3D averages closer to $O(10\%)$ with the highest \pt bins discrepancies about $20\%$ and a consistent deficit in data efficiency. The product of the efficiency components into their corresponding Gold, Silver, and Bronze category is shown in Figure \ref{fig:elgsb}. The efficiency for Very Loose is also included separately but is factored into the denominator efficiencies components, so, the Gold, Silver, and Bronze efficiencies represent the overall electron efficiency for that particular lepton ranking. The range of efficiencies for each ranking are $(50-70)\%$, $(10-20)\%$, and $(10-30)\%$ for Gold, Silver, and Bronze respectively. The component combined agreement for all three ranks ranges around $10\%$ to $20\%$ but large discrepancies can be seen at the highest and lowest \pt bins for Silver and Bronze. Better measurements could be obtained by using a different resonance such as $J/\psi \rightarrow ee$ to measure the lower \pt ranges, however, data triggers with electrons for $J/\psi$ are not available.

\FigureThree{h_2017_1_eleff.pdf}{h_2017_2_eleff.pdf}%
           {h_2017_3_eleff.pdf}%
         {2017 efficiencies} {fig:el2deff}


\FigureFour{h_2017_0_el.pdf}{h_2017_1_el.pdf}%
           {h_2017_2_el.pdf}{h_2017_3_el.pdf}%
          {fig:elgsb}{2017 electron GSB efficiency and SF }

\FloatBarrier
\section{Muon Tag-and-Probe}
The muon Tag-and-Probe tools also uses a centrally curated CMSSW PhysicsTools in \url{CMSSW_10_6_X}. The software pipeline is identical to electons in that it consists of an ntuplizing \cite{MuTnPTwiki} and fitting \cite{MuTnPAnaTwiki} stage. The code bases for muons and electrons are separate but functionally identical. The samples chosen for Z measurements are shown in Table \ref{tab:mutnpsamples}. The $J/\psi$ ntuples are available from a central repository of standard Tag-and-Probe selection variables which use the pre-ultra legacy samples for each year \cite{MuTnPCentralSamps}.  The muon Tag-and-Probe efficiencies are measured above 20 GeV using the Z boson while below 20 GeV benefits from the $J/\psi$ meson for Id measurements. The $\eta$ bins are divided into a central and forward regions around the endcaps at $|\eta| = 2.1$. In total there are three sets of binnings: The low \pt $J/\psi$ binning $J/\psi^{L}$ for muon Id below 20 GeV, the high \pt Z binning $Z^{H}$ above 20 GeV, and the low \pt Z binning $Z^{L}$ used to extrapolate isolation and impact parameter efficiencies down to 3 GeV.  The explicit bin edges for each range are defined in Table \ref{tab:mubin}.


%notes to add in later 2016C, 2017C, 2018A
\begin{table}
\caption{}
\label{tab:mutnpsamples}
\scriptsize
\begin{tabular}{|c|c|c|}
\hline 
Type & Year & Sample Name \\ 
\hline 
Data & 2016 & \tiny \url{/SingleMuon/Run2016-17Jul2018-v1/MINIAOD}  \\  
Data & 2017 & \tiny \url{/SingleMuon/Run2017-31Mar2018-v1/MINIAOD} \\  
Data & 2018 & \tiny \url{/SingleMuon/Run2018-17Sep2018-v2/MINIAOD} \\ 
\hline 
MC & 2016 & \tiny \url{/DYJetsToLL_M-50_TuneCUETP8M1_13TeV-madgraphMLM-pythia8/RunIISummer16MiniAODv3-PUMoriond17_94X_mcRun2_asymptotic_v3_ext2-v2/MINIAODSIM} \\ 
MC & 2017 & \tiny \url{/DYJetsToLL_M-50_TuneCP5_13TeV-madgraphMLM-pythia8/RunIIFall17MiniAODv2-PU2017RECOSIMstep_12Apr2018_94X_mc2017_realistic_v14_ext1-v1/MINIAODSIM} \\ 
MC & 2018 & \tiny \url{/DYJetsToLL_M-50_TuneCP5_13TeV-madgraphMLM-pythia8/RunIIAutumn18MiniAOD-102X_upgrade2018_realistic_v15-v1/MINIAODSIM} \\ 
\hline
\end{tabular} 
\end{table}

\begin{table}
\caption{add ref to this table later, premade jpsi tnp trees for id}
\label{tab:jpsimutnpsamples}
\scriptsize
\begin{tabular}{|c|c|c|}
\hline 
Type & Year & Sample Name \\ 
\hline 
Data & 2016 & \tiny \url{TnPTreeJPsi_LegacyRereco07Aug17_Charmonium_Run2016Bver2_GoldenJSON.root}  \\  
Data & 2017 & \tiny \url{TnPTreeJPsi_17Nov2017_Charmonium_Run2017Cv1_Full_GoldenJSON.root} \\  
Data & 2018 & \tiny \url{TnPTreeJPsi_Charmonium_Run2018Dv2_GoldenJSON.root} \\ 
\hline 
MC & 2016 & \tiny \url{TnPTreeJPsi_80X_JpsiToMuMu_JpsiPt8_Pythia8.root} \\ 
MC & 2017 & \tiny \url{TnPTreeJPsi_94X_JpsiToMuMu_Pythia8.root} \\ 
MC & 2018 & \tiny \url{TnPTreeJPsi_102XAutumn18_JpsiToMuMu_JpsiPt8_Pythia8.root} \\ 
\hline
\end{tabular} 
\end{table}

\begin{table}
\caption{muon binning}
\label{tab:mubin}
\begin{tabular}{|c|c|c|}
\hline 
\multicolumn{3}{|c|}{Muon Binning} \\ 
\hline 
Range & $p_T$ GeV & $|\eta|$ \\ 
\hline 
$J/\psi^{L}$ & [3.0, 4.0,  5.0, 6.0, 7.0, 9.0, 14.0,  20.0] & [0, 1.2, 2.4] \\ 

$Z^{H}$ &  [10, 20, 30, 40, 60, 100] & [0, 1.2, 2.4] \\ 

$Z^{L}$ & [6,8,10,14,18,22,28,32,38,44,50] & [0, 1.2, 2.4] \\ 
\hline 
\end{tabular} 
\end{table}

%\begin{itemize}
%\item $J/\psi^{L}$  
%	\begin{itemize}
%		\item[] $p_T \in [3.0, 4.0,  5.0, 6.0, 7.0, 9.0, 14.0,  20.0]$
%		\item[]  $\eta| \in [0, 1.2, 2.4]$
%	\end{itemize}
%\item $Z^{H}$
%	\begin{itemize}
%		\item[] $ p_T \in [10, 20, 30, 40, 60, 100]$
%		\item[] $|\eta| \in [ 0, 1.2, 2.4]$
%	\end{itemize}
%\item $Z^{L}$
%	\begin{itemize}
%		\item[] $ p_T \in [6,8,10,14,18,22,28,32,38,44,50]$ 
%		\item[] $|\eta| \in [0, 1.2, 2.4]$
%	\end{itemize}
%\end{itemize}


Topological dependecies for isolation and impact parameters prevent measurement using the $J/\psi$. About $30\%$ of prompt $J/\psi$ are produced from higher mass states $\chi_c$ and $\Psi(2S)$ thus $J/\psi$ will be produced from a cascade inside jets and likely be unisolated \cite{Lansberg:2006dh}. Similary another $10\%$ of all $J/\psi$ are produced within b-jets and leading to  non-prompt  unisolated events \cite{LHCb:2013itw}.

The exact criteria chosen for the tag and probe vary between physics processes but are identical across the two $Z$ ranges. The selections follow the standards defined from the centrally produced muon Tag-and-Probe efficiencies.\\

\begin{table}
\small
\begin{tabular}{|c|c|c|}
\hline 
\multicolumn{3}{|c|}{Tag-and-Probe Muon Candidate Selection Criteria} \\ 
\hline 
%\multicolumn{3}{|c|}{$J/\psi$} \\
\hline
Tag & Probe & Pair \\ 
\hline 
%isGlobalMuon & Matches hltTracksIter   & $2.8 \text{GeV} < m_{\mu\mu} < 3.4 \text{GeV} $ \\
%numberOfMatchedStations$>1$  & OR &  $|z_{\mu_1} - z_{\mu_2}| <1 $ cm\\
% $\pt > 5$ GeV & Matches hltMuTrackJpsiEffCtfTrackCands &    \\
% Matches hltIterL3MuonCandidates &   & \\
%\hline
\multicolumn{3}{|c|}{$Z$} \\
\hline
passes tightID & No requirement & $m_{\mu\mu} > 60$ GeV \\
$\sum \pt^{ch} / \pt < 0.2$ & &  $|z_{\mu_1} - z_{\mu_2}| <4 $ cm \\
$\pt > 15$ GeV &   &   \\
\hline 
\end{tabular} 
\end{table}
The muon data will also  have an implicit selection due to triggering. To reflect this selection in MC, the tag is required to pass a chosen trigger in the efficiency denominator in addition to HLT object matching. The triggers available vary from year to year for $Z$ using \url{IsoTkMu22} in 2016 and \url{isoMu24eta2p1} in 2017 and 2018. A single $J/\psi$ triggers is available for all years which is \url{Mu7p5Tk2}.\\

%\begin{center}
%\begin{tabular}{@{}l@{}} 
%\tabitem $J/\psi$ 2016,2017,2018: \url{Mu7p5Tk2} \\
%\tabitem $Z$ 2016: \url{IsoTkMu22}\\
%\tabitem $Z$ 2017, 2018: \url{isoMu24eta2p1} \\
%%\end{tabular} 
%\end{center}





 The Gold, Silver, and Bronze efficiency definitions are split based on \pt and reflect the high and low binning separations shown in Table \ref{tab:mubin}. The low \pt muons include the Id measured by $J/\psi$ as well as the extrapolated efficiencies from SIP3D and isolation fits in $Z_{L}$. The high \pt muons are composed of all the factors directly measured in $Z_H$.

\begin{itemize}
\item[] $\pt \in [3,20)$
\end{itemize}
\begin{equation}\label{eq:effcomp_J}
\begin{split}
\epsilon_{\Gold}& = \epsilon_{\ID}^{J/\psi}\times \epsilon_{\Isolated|\ID}^{Z_L} \times \epsilon_{\Prompt|(\ID \cap \Isolated)}^{Z_L} \\
\epsilon_{\Silver}& = \epsilon_{\ID}^{J/\psi} \times \epsilon_{\Isolated|\ID}^{Z_L} \times (1-\epsilon_{\Prompt|(\ID \cap \Isolated)}^{Z_L}) \\
\epsilon_{\Bronze}& = 1-(\epsilon_{\ID}^{J/\psi} \times \epsilon_{\Isolated|\ID}^{Z_L})
\end{split}
\end{equation}
%\quad \quad \\
\begin{itemize}
\item[] $\pt \in [20,100 ]$
\end{itemize}
\begin{equation}\label{eq:effcomp_Z}
\begin{split}
\epsilon_{\Gold}& = \epsilon_{\ID}^{Z_H}\times\epsilon_{\Isolated|\ID}^{Z_H}\times\epsilon_{\Prompt|(\ID \cap \Isolated)}^{Z_H} \\
\epsilon_{\Silver}& = \epsilon_{\ID}^{Z_H}\times\epsilon_{\Isolated|\ID}^{Z_H}\times(1-\epsilon_{\Prompt|(\ID \cap \Isolated)}^{Z_H})\\
\epsilon_{\Bronze}& = 1-(\epsilon_{\ID}^{Z_H} \times \epsilon_{\Isolated|\ID}^{Z_H})
\end{split}
\end{equation}

 The 2017 Id efficiency with statistical errors for both data and MC are shown in Figure \ref{fig:jpsiZ17-ideff-ratio}. The other efficiencies for each year for all \pt ranges are included in the appendix. The overlapping bins between $J/\psi$ and $Z$ do not all match within statistical uncertainties. However, the average deviation of the efficiency central values are $0.02\%$ for MC and $1\%$ for data.  The relative efficiencies per component range from approximately $88\%$ to $98\%$ and are fairly uniform between the central tracker and endcaps. The efficiencies for the isolation ranges from $(90 - 95)\%$ where the encaps generally are about $5\%$ more efficient. As for SIP3D, the efficiency ranges from about $(80 - 93)\%$ with another $5\%$ $|eta|$ based efficiency gap, however, in the SIP3D case, the central tracks are more efficient as opposed to isolation.  The extrapolation of the vertexing and isolation efficiencies below 20 GeV is done by fitting a quadratic polynomial to the efficiencies on the $Z_L$ interval.  Both data and MC are shown in Figure \ref{fig:eff17-extraps}.  The errors for each bin are the combined statistical and systematic errors from Table \ref{tab:musyst} and are adjusted before the polynomial fit. Any efficiencies below 20 GeV are then reported from the fit model. The fit errors are the 68\% confidence interval combined with the systematic errors. The worst observed right tail P-value from all fits is $\approx 2\%$, the median P-value from the Figure \ref{fig:eff17-extraps} is $84\%$. The fits in each year behave qualitatively the same as 2017.
The product of the efficiency components into their corresponding Gold, Silver, and Bronze category is shown in Figure \ref{fig:2017-mu1-gsbvl}. Similar to electrons, the efficiency for Very Loose is also included separately but is factored into the denominator efficiencies components, so, the Gold, Silver, and Bronze
efficiencies represent the overall electron efficiency for that particular lepton ranking. The
range of efficiencies for each ranking are $(70 - 80)\%$, $(5 - 15)\%$, and $(4 - 20)\%$ for Gold, Silver, and Bronze respectively. The Data and MC agreement for all three ranks is better than electrons with the largest discrepancy in Gold being $2\%$ and the average deviation in Silver and Bronze begin approximately $(5-10)\%$.

%The very loose and the efficiency components combined into Gold, Silver, and Bronze are summarized in Figure \ref{fig:2017-mu1-gsbvl}, the other years are included in %the appendix. The tool used to store/calculate efficiencies can be found at \url{https://github.com/Jphsx/LepTool}.  The cumulative efficiency for a muon also includes %the efficiency of the VeryLoose selection and is defined as:
%\begin{equation}
%\epsilon_\mu = \epsilon_{\text{VeryLoose}} \times \epsilon_{\text{Gold/Silver/Bronze}}
%\end{equation}



\FigureFour{canvas0J2017b.pdf}{canvas1J2017e.pdf}%
           {canvas0Z2017b.pdf}{canvas1Z2017e.pdf}%
          {fig:jpsiZ17-ideff-ratio}{Tag-and-Probe efficiencies for the Medium Id in 2017. The left plots show the barrel while the right plots show the endcaps. The top fits use $J/\psi$ resonance while the bottom use the Z resonance. }

\FigureTwo{effFit_2.pdf}{effFit_5.pdf}%
          {fig:eff17-extraps}{The fitted muon isolation and SIP3D efficiencies for 2017. Includes both data and MC which are separated between barrel and endcap.  }



\FigureFour{h_2017_1_mu.pdf}{h_2017_2_mu.pdf}
		   {h_2017_3_mu.pdf}{h_2017_0_mu.pdf}
		   {fig:2017-mu1-gsbvl}{The combined efficiency components from equations \ref{eq:effcomp_J} and \ref{eq:effcomp_Z} and Very Loose for 2017. The low-\pt region ($<20$ GeV) includes the contributions from $J/\psi$ as well as the isolation and SIP3D extrapolations. Propagated errors are treated as uncorrelated.}

\FloatBarrier
\section{Lepton Systematics and Scale Factors}

The systematic error for the electron and muon efficiencies are derived by varying the Tag-and-Probe signal and background models, slimming and widening the mass window , and increasing and decreasing the number of bins used in the fit. The systematic error is defined as the maximum spread in efficiencies between the modeling variations with an example spread shown in Figure \ref{fig:systspread}.  Rather than compute the systematic error for every bin, similarities between neighboring bins motivates using a simplified bin approach which was chosen qualitatively by the background shape. The shape of the \pt based mass distributions is illustrated in Figure \ref{fig:systplots}. The same $\eta$ bins are utilized according to lepton flavor, but the \pt bins are consolidated into a high and low bin pivoting on $20$ GeV. A high and low systematic is derived for each selection criteria per flavor per year and is applied to the efficiencies that fall within the corresponding \pt and $\eta$ range. 
%todo syst spread fig
%todo syst tnp dist
\FigureStack{muons_lowpt_syst.png}{muons_highpt_syst.png}%
	{fig:systplots}{Tag-and-Probe di-muon mass distributions for both passing and failing probes. The top set of plots consist of probes below 20 GeV and the bottom set are about 20 GeV.}
	


Scale factors are derived bin by bin for each criteria per flavor per year by finding the ratio of efficiencies in data to Monte Carlo. The scale factor variance is propagated by combining both the statistical error from the Tag-and-Probe in quadrature with the systematic error. The full 2017 set of systematics electrons and muons is shown in Table \ref{tab:elesysts} and Table \ref{tab:musysttable}. Additional scale factors are also  needed adjusting the differences between samples which are either created with a full simulation or fast simulation. The Fast to Full factor is obtained by extracting the criteria efficiency ratio between  full and fast sim ttbar samples.

\FigureOne{2017data_medID_msyst.pdf}{Example systematic spread from various fit models and binnings for muons. Includes the four combinations of regions either low or high pt and central and forward eta.}{fig:systspread}
	
\begin{table}
\caption{The electron systematic error derived from the Tag-and-Probe for 2017 data and split into \pt and $|\eta|$ regions. }
\label{tab:elesysts}
\begin{tabular}{|c|ccc|}
\hline
ID & $0\leq |\eta|<0.8$ & $0.8\leq |\eta|<1.479$ & $|\eta|\geq1.479$ \\
\hline
$\pt < 20$ [GeV] & 0.003 & 0.001 & 0.005 \\
$\pt \geq 20$ [GeV] & 0.001 & 0.001 & 0.002  \\
 &  &  &    \\
\hline
Iso $|$ ID  &  &  &   \\
\hline
$\pt < 20$ [GeV] & 0.002 & 0.003 & 0.003   \\
$\pt \geq 20$ [GeV] & 0.001 & 0.001 & 0.002 \\
 &  &  &   \\
\hline
SIP $|$ Iso $\cap$ ID &  &  &  \\
\hline
$\pt < 20$ [GeV]& 0.006 & 0.004 & 0.007 \\
$\pt \geq 20$ [GeV]& 0.002 & 0.002 & 0.0006  \\
\hline
VeryLoose &  &  &  \\
\hline
$\pt < 20$ [GeV]& 0.002 & 0.007 & 0.03 \\
$\pt \geq 20$ [GeV]& 0.003 & 0.0001 & 0.0007 \\
\hline
\end{tabular}
\end{table}


\begin{table}[htbp]
\centering
\caption{The muon systematic error derived from the Tag-and-Probe data and split into \pt and $|\eta|$ regions. }
\label{tab:musysttable}
\begin{tabular}{|c|cc|}
\hline
\hline
ID & $|\eta|<1.2$ & $|\eta|\geq 1.2$  \\
\hline
$\pt < 20$ [GeV](J) & 0.001 & 0.001  \\

$\pt \geq 20$ [GeV](Z) &  0.001& 0.0003 \\

 &  & \\
\hline
Iso $|$ ID  &  &  \\
\hline
$\pt < 20$ [GeV]  & 0.007 & 0.004  \\

$\pt \geq 20$ [GeV] & 0.007 & 0.002  \\

 &  &  \\
\hline
SIP $|$ Iso $\cap$ ID &  &  \\
\hline
$\pt < 20$ [GeV]& 0.005 & 0.003 \\

$\pt \geq 20$ [GeV]& 0.001 & 0.002 \\
 & &  \\
\hline
Very Loose & &  \\
\hline
$\pt < 20 $ [GeV]  & 0.001 & 0.0003 \\
$\pt \geq 20$ [GeV]  & 0.001 & 0.001 \\
\hline
\end{tabular}
\label{tab:musyst}
\end{table}




%\setcounter{secnumdepth}{3}
\setcounter{tocdepth}{3}
\setlength{\parskip}{\smallskipamount}
\setlength{\parindent}{0pt}


\makeatletter


\providecommand{\tabularnewline}{\\}


\makeatother

\chapter{Modeling}

\section{Introduction}
%introduce actual fit, fit regions, systematics treatment, results of fit stages
This analysis is counting experiment that is evaluated with a Poisson likelihood. The MC model is designed to be data driven since background dominates the majority of regions tuning the data and MC agreement in the fit. The anchoring provided by background righ regions translates into well constrained background predictions in sensitive regions with an ABCD-like approach. To obtain a robust MC model before unblinding, three regions are defined, the Control Region (CR) which has no signal, a Validation Region (VR) which has mild sensitivity tests modeling in regions untouched by the CR , and finally the Signal Region (SR) which is comprised of the high $R_{ISR}$ bins and is sensitive to all signals. The fits are conducted in three stages starting with the CR only, then CR+VR, and finally the full fit combining all three fit regions CR+VR+SR.

\section{Fit Strategy and Fit Region Definitions}
%CR VR SR
The fits are carried out in three stages combining the Control Region, Validation Region, and Signal Region. The control region is guaranteed to have no signal sensitivity and covers the majority of bins. The expected signal contamination in the control region from stops, sleptons, or electroweakinos is $<1\%$ in for sparticle masses that are not excluded. The categories and bins that compose the CR are easily described as the low $R_{ISR}$ region and more specifically the lowest two $R_{ISR}$ bins of almost every category. The only category that has been specifically excluded from the CR is 2L high $p_T^{ISR}$ categories. The high $p_T^{ISR}$ is very sensitive to stop processes and electroweakinos. The CR region is comprised of 1298 bins which hold $72\%$ of total expected Run II events. This means that the CR dominates the behavior of the fit, but does not guarantee decent modeling at high $R_{ISR}$. To deal with this short coming, we introduce the Validation Region. This region is fit with the CR and extends the fit sampling all categories and kinematic ranges. The VR is composed of the remaining bronze category $R_{ISR}$ not covered by the CR. The signal presence in VR bins is at the few percent level with signals showing up in the most sensitive categories, like $\mu\mu$. Together the CR+VR fit covers 1517 total bins increasing the expected number of events seen by the fit $10\%$. The remaining region is the SR, it has 1576 total bins, but only $19\%$ of total expected events. The sensitivity to every signal is high in this region examples of the S/B significance using the Z-binomial statistic if Figures \ref{fig:zbi1} and \ref{fig:zbi2}.

\FigTwoScale{Model_figs/0L_4J_ratioZbin_highDm_RISR_Mperp.pdf}{Model_figs/1L_G_0J_ratioZbin_chargeSep_RISR_Mperp.pdf}{Distributions that show the relative sensitivity of compressed stop processes in each of $R_{ISR}$ and $M_\perp$ bins. The left distribution shows 0 lepton and 4 S-jets with color-coded b-tag counting categories. The $k^+$ denotes high $p_T^{ISR}$ and high $\gamma_\perp$ categories. The right distribution shows 1 gold lepton and 0 S-jets with color-coded lepton flavor categories. The X in both distribution indicates the integration over all sub-categories not explicitly listed.}{fig:zbi1}{0.8}{0.8}

\FigTwoScale{Model_figs/zbi_2lgold.png}{Model_figs/zbi_2lbron.png}{A comparison of three different signal processes in a 2 lepton selection for bins of $R_{ISR}$ and $M_\perp$. The left plot shows gold regions with 0 S-jets which are split by sign or include SVs. The right plot shows bronze regions with 1 S-jet split by flavor combinations. The Bronze categories are still able to be sensitive in high $R_{ISR}$ despite being considered background rich regions. The X in both distributions indicates integration of all categories not shown.}{fig:zbi2}{0.8}{0.8}




\section{Fit Implementation and Model Defintion}
%Poisson likelihood
The fitting framework is provided by the \url{HiggsCombine} tool which generates datacards that encodes all the components of the fit into a standard format which is then processed by \url{CombineHarvester} and RooFit/RooStats packages \cite{Antcheva:2009zz}\cite{moneta2011roostats}. The fit can be represented by a Poisson likelihood defined as:
\begin{equation}
\label{eq:fit}
\mathcal{L}(\vec{\alpha}|\vec{x}) = \bigg[ \prod_i^N \text{Pois}(x_i|\lambda_i(\vec{\alpha})) \bigg] \bigg[\prod_j^M \pi_j(\alpha_j) \bigg]
\end{equation}
Equation \ref{eq:fit} extend over the range of all $N$ analysis bins where each $i$-th bin is composed of a count of observed events $x_i$ and expected events $\lambda_i$. The expected events are subject to the set of nuisance parameters $\vec{\alpha}$ of which some are conditioned by prior probability distributions $\pi_j(\alpha_j)$. The ideal model for $\lambda(\vec{\alpha})$ is found by maximizing the likelihood with the minimal set of nuisance parameters $\vec{\alpha}$ that is sensitive to the signal+background or background only hypothesis.  There are three types of nuisances implemented in the fit, free floating rate parameters, log-normal constrained parameters, and shape parameters.  Freely floating parameters contribute to a factor $\kappa$, with a starting value of 1, that is multiplied against the expected bin yield $\lambda$ adjusting the yield by some fraction with respect to the nominal value. The free parameters have no associated penalty with their adjustment and are fully determined by data. Individual bins $i$ are mapped together by common processes, $k$, which are all associated under a common nuisance $j$. The selection of processes associated to a nuisance parameter can either be the contribution from an non-fake background process or flavor and source separated fake leptons background process. The definition of a free rate parameter can then be defined as 
\begin{equation}
\label{eq:rateparam}
\kappa_{ijk}(\alpha_j) = \alpha_j
\end{equation}  
The log-normal parameters also uses a $\kappa$ factor that is applied to the expected events of the associated bin. The log-normal parameter is different from the freely floating parameters in such that it is penalized for moving from the nominal value with based on normally distributed prior $\pi(\alpha_j)$. The prior uncertainty associated with a process $j$ and nuisance $k$, is written as $\sigma_{jk}$. The log-normal definition follows with:
\begin{equation}
\label{eq:logparam}
\kappa_{ijk}(\alpha_j) = (1+\sigma_{ijk})^{\alpha_j}
\end{equation}

The third type of nuisance is different from the first two because it adjusts expected bin yields based on the underlying shapes of the $R_{ISR}$ and $M_\perp$ distributions. The $\kappa$ factor for shapes is then a function of up and down variations of one of the kinematic variables and is encoded with a normally distributed prior $\pi(\alpha_j)$. The $\kappa$ definition is based on the interpolation $-1<\alpha_j<1$ and is written as follwing based on a predefined shape treatment \cite{Conway:2011in}
\begin{equation}
\label{eq:shapeparam}
\kappa_{ijk}(\alpha_j)= 1 + \frac{1}{2}((\delta^+ - \delta^-)\alpha_j + \frac{1}{8}(\delta^+ + \delta^-)(3\alpha_j^6-10\alpha_j^4+15\alpha_j^2))
\end{equation}
the $\delta^\pm$ components are ratios of the up and down shape variations of the nominal shape $\lambda^{nominal}$ such that $\delta^+ = \lambda^{up}/\lambda^{nominal}$ and $\delta^- = \lambda^{down}/\lambda^{nominal}.$

The Likelihood Equation \ref{eq:fit} combines the three types of nuisances from equations \ref{eq:rateparam}, \ref{eq:logparam}, \ref{eq:shapeparam}. Each nuisance is mapped to either a set of processes or shapes in conjuction with a mapping to a set bins. The fit adjusts the three $\kappa$ factors to maximize the agreement between the observed data $\vec{x}$ and $\vec{\lambda}$. 


\section{Definitions of Modeling Systematics}
The set of nuisances has gone through an extensive evolution, beginning with very early fits with only 10 nuisances in \cite{erich}. The first fits only used a single nuisance to describe b-tag systematics, one for MET trigger systematics, one for luminosity and one for each background process rate. The final configuration consists of over 200 nuisances which are divided into 5 subcategories: kinematic, process normalizations, lepton fakes, lepton categorization, and b-tagging. The optimizations of these nuisances, that is, their bin association, process mapping, and  allowed degrees of freedom has undergone extensive study. The complete list of systematics, their type, and prior uncertainties are listed in Tables \ref{tab:kinnuisance}, \ref{tab:btagnuisance}, \ref{tab:lcatnuisance}, \ref{tab:procnuisance}, \ref{tab:fakenuisance}, \ref{tab:svnuisance}, and \ref{tab:othernuisance}.  The kinematic nuisances from Table \ref{tab:kinnuisance} contribute 27 factors which serve the purpose accounting for systematic effects between the high and low $p_T^{ISR}$ and $\gamma_\perp$ for each lepton and jet multiplicity. The kinematic nuisances, and nuisances in general, that appear to be missing e.g. $\gamma_\perp$ 0L 1J are merged with neighboring jet multiplicites if they are determined to be extraneous degrees of freedom or are highly correlated with another nuisance. Table \ref{tab:btagnuisance} describes 73 nuisances designed to accomodate systematic effects from categorization of b-tagged jets in either the S or ISR system for each lepton and jet multiplicity. Regions with few b-tags or are mapped to all background processes, otherwise, each nuisance mapping is split into two tt+jets or everything else. There are 21 lepton categorization nuisances which account for systematically different rates in gold categories versus silver or bronze categories. Table \ref{tab:procnuisance} shows the rates for each background process normalizations, which includes a special hierarchy parameterization that will be discussed later. The rule of thumb for splitting background process degrees of freedom is that dominant processes are split by lepton and jet multilpicity, intermediate backgrounds are split by lepton, and rare backgrounds are mapped globally with a single nuisance. The fake lepton nuisances are comprised of global rates for each flavor and source, a single nuisance to account for the global rate of  charge misidentification, and the fake shapes. The fake shapes are split by flavor, lepton multiplicity, and jet multiplicity. The fake shapes were originally split by source but this introduced many extraneous parameters where heavy flavor and light flavor sources were highly correlated. The fake shapes target systematic effects from the shapes of either $R_{ISR}$, $M_\perp$ and are based on $1\sigma$ up and down variations applied to the shape template in Equation \ref{eq:shapeparam}. The last two tables \ref{tab:svnuisance} and \ref{tab:othernuisance} account for systematics from SV effiency and rapidity, as well as, other sources of systematics.

% table of systematics
\begin{table}
\centering
\caption{Kinematic nuisances mappings which are applied to the higher of the two available bins. Bracketed jet mappings indicate all integer jet multiplicities between the listed inclusive edges. All factors are assigned a $20\%$ prior uncertainty.}
\begin{tabular}{ccc}
\hline 
Category Mapping & $N_L$ Mapping & $N_{jets}^S$  Mapping \\ 
\hline 
\hline
$p_T^{ISR}$  & 0 & $[1,\geq5]$ \\ 
$p_T^{ISR}$ & 1 & $[0,\geq4]$ \\ 
$p_T^{ISR}$ (QCD) & 0 & $[0,\geq5]$ \\ 
$\gamma_\perp$ & 0 & $[2,\geq4]$ \\ 
$\gamma_\perp$ & 1 & $[1,\geq4]$ \\ 
$\gamma_\perp$ & 2 & $[0,\geq2]$ \\  
$\gamma_\perp$ (QCD) & 0 & $[2,\geq4]$ \\ 
\hline 
\end{tabular} 
\label{tab:kinnuisance}
\end{table}


\begingroup

\begin{table}
\centering
\caption{The log-normal nuisance mapping for all b-tag counting categories. Includes all combinations of Process $\times$ Category $\times$ $(N_\ell,N_{jet}^S)$ multiplicities. All factors are assigned a prior $20\%$ uncertainty.}
\setlength{\tabcolsep}{10pt} % Default value: 6pt
\renewcommand{\arraystretch}{1.5} % Default value:
\begin{tabular}{lc}

\multicolumn{2}{|l}{Process Mapping per $(N_\ell,N_{jet}^S)$ } \\ 
\hline 
 & (Combined/All) or ($tt+jets$) or (not $tt+jets$)  \\ 
\multicolumn{2}{|l}{Category Mapping per $(N_\ell,N_{jet}^S)$ } \\ 
\hline 
 & $(N_{b-tag}^{ISR},N_{b-tag}^S)=\{(0,1),(1,0),(1,1),( \text{inclusive} ,\geq2) \}$ \\ 
\multicolumn{2}{|l}{Combined/All Nuisances}  \\ 
\hline 
 & $(N_\ell,N_{jet}^S)=\{(0,1),(1,1),(2,1),(2,2) \}$ \\ 

\end{tabular} 
\label{tab:btagnuisance}
\end{table}
\endgroup

\begin{table}
\centering
\caption{Lepton category nuisance mapping. The complete set of nuisances is represented by the product of the $N_\ell \times N_{jet}^S$ with an assigned prior uncertainty or $20\%$. The category !Gold indicates the combined Silver and Bronze categories. }
\begin{tabular}{cc|ccc}

 &  & \multicolumn{3}{c}{$N_\ell$ Mapping} \\  
 &  & 1$\ell$ & 2$\ell$ & 3$\ell$ \\ 
\hline 
\multirow{6}{*}{
\rotatebox[origin=c]{90}{$N_{jet}^S$ Mapping}}  & Inclusive &  & \makecell{$(ee,\mu\mu,e\mu)\times$!Gold \\ \quad } & \makecell{$(Z*,noZ*) $ \\ \quad } \\ 
 
 & 0J & $(e,\mu)\times$(Gold,!Gold) & \makecell{$(OS,SS)\times$ Gold \\ $(\ell\ell)\times$(Gold,!Gold) \\ \quad } &   \\ 
 
 & 1J & $\ell$ Gold & $(Z*,noZ*)\times$ Gold &  \\ 
 
 & 2J & $\ell$ Gold & $(Z*,noZ*)\times$ Gold &  \\ 
 
 & 3J & $\ell$ Gold &  &  \\ 
 
 & 4J & $\ell$ Gold &  &  \\ 

\end{tabular} 
\label{tab:lcatnuisance}

\end{table}


\begin{table}
\centering
\caption{The nuisance mapping split by lepton and jet multiplicity for background processes. The dominant backgrounds W+jets and tt+jets are implented with a hierarchy parameterization of factors.  Global factors indicate mapping to every bin and $N_{jet}^S$ implies each possible number of jets for a given lepton multiplicity.}
 
\begin{tabular}{ccc}
\hline 
Category Mapping & Process Mapping & Param. Details \\ 
\hline 
\hline
per  $(N_\ell =1,2,3  \, ,N_{jet}^S)$ & W+jets & hierarchy \\ 

per $(N_\ell = 0 \, , N_{jet}^S)$ & (W+jets)+(ZDY) & hierarchy \\ 

per $(N_\ell, N_{jet}^S)$ & tt+jets & hierarchy \\ 
 
per $(N_\ell=0,1,2 \,, N_{jet}^S)$ & QCD & $0\ell$ floating otherwise $20\%$ prior \\ 
per $N_\ell=1,2,3$ & (ZDY)+(DB) & $20\%$ prior \\ 
per $N_\ell=0,1,2$ & ST & $20\%$ prior \\ 
global & TB & $20\%$ prior \\ 
global & ZDY & free floating \\ 
global & DB & free floating \\ 
\hline 
\end{tabular} 
\label{tab:procnuisance}

\end{table}

\begin{table}
\centering
\caption{The nuisances associated to fake leptons. The global or silver or bronze rates are split between MC matched source of either Heavy flavor or Light flavor. The shape systematics are applied per lepton and S-jet combinations but combine HF/LF sources and split by flavor only.}
\begin{tabular}{ccc}

Category Mapping & Process Mapping & Parameter Details \\ 
\hline 
\hline
$\ell^\pm\ell^\pm$ & Global & Free floating \\ 
 
Global & $(e,\mu)\times(HF,LF)$ & Free floating \\ 
 
Silver & $(e,\mu)\times(HF,LF)$ & $20\%$ prior \\ 
 
Bronze & $(e,\mu)\times(HF,LF)$ & $20\%$ prior \\ 
 
per $(N_\ell^S,N_{jet}^S)$ & $(e,\mu)\times(R_{ISR}^{shape}, M_\perp^{shape})$ & $5\%$ prior \\ 
\hline 
\end{tabular} 
\label{tab:fakenuisance}

\end{table}


\begin{table}
\centering
\caption{The set of nuisances associated to the categories that involve tagged SVs. The SV counting rates are mapped globally to every bin and the kinematic $\eta$ separation is split between SVs associated with tt+jets versus SVs associated with anything else.}
\begin{tabular}{cc}
Category mapping & process mapping \\ 
\hline 
\hline
$N_{SV}^S=1$ & All \\ 
 
$N_{SV}^S \geq 1$ & All \\ 
 
$|\eta_{SV}^f|$ & tt+jets \\ 
 
$|\eta_{SV}^f|$ & other \\ 
\hline 
\end{tabular} 
\label{tab:svnuisance}

\end{table}

\begin{table}
\centering
\caption{Additional secondary systematics which account for various scale factors or systematic effects of jet reconstruction and clustering.  The nuisances with a (*) are not currently implemented but will be added in later in the final model. }
\begin{tabular}{|c|c|}
\hline 
Systematic Source and Scale Factors & Parameter Type \\ 
\hline 
Luminosity & log-normal \\  
$e,\mu$ Efficiency & log-normal \\ 
b-tag efficiency & log-normal \\ 
*Factorization, Renormalization, PDF, and $Q^2$ & log-normal \\ 
*JES \& type-I MET & shape \\ 
*Unclustered Energy & shape \\ 
MET Trigger Efficiency & shape \\ 
Lepton FastSIM SF & log-normal \\ 
SV FastSIM SF & log-normal \\ 
b-jet FastSim SF & log-normal \\ 
*MET FastSIm Correction & shape \\ 
\hline 
\end{tabular} 
\label{tab:othernuisance}

\end{table}

%begin hierarchy
The process normalization hierarchy listed in Table \ref{tab:procnuisance} is a special parameterization of nuisance parameters organized into a tree of factors. This parameterizatin is implemented separately for the two dominant backgrounds W+jets and tt+jets. The factor at the top of the hierarchy is the root factor and is the region with the highest statisical power and purity for that process. This factor, for a particular lepton and S jet multiplicity, informs neighboring jet multiplicity factors and in turn those neighbors then inform their neighbors. This allows for the root parameters for each process to be interpreted as an estimate of a singular normalization scale factor for that group of processes, while also allowing for independent factors in different multiplicities to be interpreted and constrained relative to the higher factor. 

For example: The root factor for W+jets is 1L 0J, chosen because it is $14\%$ of the total process yield and $93\%$ of the total background for that particular region. 0L 2J, 1L 1J, and 1L 2J have comparable stats with up to $20\%$ more events but loses $20\%$ purity (the fraction of W+jets to other backgrounds). The $\kappa^{Wjets}_{1L, 0J}$ ``root'' scale factor maps to every analysis bin for the W+jets process and governs the overall rate. This means that the root factor is multiplied against every other scale factor in each lepton multiplicity. An example low level parameter would be $\kappa^{Wjets}_{1L, 4J}$, which is multiplied by $\kappa^{Wjets}_{1L, 0J}\kappa^{Wjets}_{1L, 1J}\kappa^{Wjets}_{1L ,2J}\theta^{Wjets}_{1L, 3J}$ and interpreted as relative to each preceding nuisance at a higher level in the hierarchy.  A similar hierarchial parameterization is evaluated for tt+jets, and both hierarchy trees are shown in Figure \ref{fig:hier}. The 0L hierarchy for W+jets is special, due to very high correlations with the ZDY process from the similarities between $W$ and $Z(\rightarrow\nu\nu)$ decays. Here, the process mapping combines the backgrounds ZDY and W+jets while the ZDY, DB factor for 0L has been removed. 
%hier diagrams
\FigTwoScale{Model_figs/wjets_hierarchy.png}{Model_figs/ttbar_hierarchy.png}{Pair of k-nary trees that illustrate the hierarchical organization of background rates for W+jets and tt+jets.}{fig:hier}{0.49}{0.49}
%end hierarchy

\section{Development of Modeling Systematics}


The optimization of each group of systematics was conducted primarily on CR fits to 2016 data and then expanded to include multiple years accounting for systematic effects induced from different run conditions in each year. The statistical evaluation of each fit is performed by comparing metrics such as the $\log\mathcal{L}$, $\chi^2$, z-score, and impacts. The impacts are a series of separate fits that independently vary all of the nuisance parameters to assess their impact of each on the POI and is a tool provided by \url{CombineHarvester}. Multiple definitions of z-scores are used with the primary one being $ (O-E)/(E+\sigma_{\text{post-fit}})$ where the $O$ is the observed data, $E$ is the expected events post-fit, the denonimator includes the expected post-fit events acting as the Poisson variance. The total error of the denominator is the sum of the Poisson variance and the post-fit variance, $\sigma_{\text{post-fit}}$. Similar z-score defintions are used such as the same evaulation without the post-fit variance or evaluated with respect to the pre-fit values. The fit evaluation is performed on all bins and subsets of each individual lepton multiplicity,or gold, silver, or bronze. The statisical errors for each bin are assumed to be gaussian with sufficient number of events. The cases with bins having very few events are assumed to be Poisson distributed, this distinction is important when combining different bins varying in events by orders of magnitude, into a z-score. If a deviation of a few observed events were to occur in a bin with only a few expected events, the z-score, would appear to be a significant outlier when it is not. To correct this issue, each bin is given the ``Poisson treatmen''. A new z-score is calculated from the Poisson probality of the original observation given the expectation. The recipe for calculating the adjusted z-scores is as follows
\begin{itemize}
\item[1.] Generate N trials, each with new expectation $E_i' \sim \mathcal{N}(E,1)$
\item[2.] Generate new observations $O_i'\sim \text{Poisson}(E_i')$
\item[3.] Count $k$ successes such that $O',E'$ follow the original observation $(O<E \,\, \text{or} \, \, O>E)$
\item[4.] Compute the Poisson probability $P=k/N$ 
\item[5.] Translate P into a z-score with normal distribution quantile
\item[6.] Compute error on z-score with up/down variations of binomial error
\item[7.] Sign the new z-score based on $O-E$ convention
\end{itemize} 

The effect of the Poisson treatment is that the z-score significance is reduced in low statistics cases where $O>E$ and increased in cases with $O<E$ due to the asymmetery of the Poisson distribution having a long tail tending to higher values. Comparisons of the final fit model with and without the Poisson treatment is shown Figure \ref{fig:comparePoissonPull}. Example pull distributions that compare the early fit model with the most final model are shown in Figure \ref{fig:compareBuildPull}. The progression of the fit model's likelihood and nuisances development is summarized Figure \ref{fig:likes} and accompanied by Table \ref{tab:builds} with brief descriptions of each milestone in the fit configurations. 
 
%build 62 no poisson vs build 62 w/ poisson
\FigTwoScale{Model_figs/build62_pull_run2_nopoisson.pdf}{Model_figs/build62_pull_run2_poisson.pdf}{Comparison of Run II CR fits with Build 62 of \ref{tab:builds} without Poisson treatment (left) and with Poisson treatment (right). The improvement by implementing is the Poisson treatment shown in the RMS and fitted $\sigma$ of the right distribution by reducing the number of large outliers due to bins with low statistics. }{fig:comparePoissonPull}{0.49}{0.49}
 
%build 8 pull with last 16 build no poisson\
\FigTwoScale{Model_figs/build8_pull_2016.pdf}{Model_figs/build62_pull_2016.pdf}{A comparison of the Poisson corrected builds from \ref{tab:builds} with the first Build 8 on the left and the final Build 62 on the right. Both distributions use only 2016 data and MC scaled to 138 fb$^{-1}$. }{fig:compareBuildPull}{0.49}{0.49}



%plot of likelihoods
\FigOneScale{Model_figs/likelihood_progress.pdf}{The progression of the CR fit log likelihood following the model version milestones listed in the Build Table \ref{tab:builds}.}{fig:likes}{0.6}

%table of builds
\begin{table}
\caption{Listing of all the model verison milestones with a brief description of each build, its Id number and number of nuisances. }
\begin{tabular}{ll}
\hline 
Build 8 & N Nuisances = 186 \\ 
\hline
 & \makecell[l]{Used 3 shape sytematics for W+jets, QCD, Fakes. All other backgrounds \\ are grouped together under ``other'' and split by $(N_\ell^S,N_{jet}^S)$} \\ 
 & \\
\hline 
Build 17 & N Nuisances = 179 \\
\hline
 &\makecell[l]{Removed W+jets and QCD shapes due to over fitting. Added in W+jets \\ hierarchy.  Split up ``other'' into 3 groups \{(tt+jets, ST),(ZDY),(DB,TB)\}\\ each split by $(N_\ell^S,N_{jet}^S)$} \\
 & \\
\hline
Build 23 & N Nuisances = 193 \\
\hline
 & \makecell[l]{Added a simplified b-tag configuration with a splitting by $(N_{b-tag}^S,N_{b-tag}^{ISR})$\\ and further split by $(N_\ell^S,N_{jet}^S)$ }\\
 & \\ 
\hline
Build 24 & N Nuisances = 194 \\
\hline
 & \makecell[l]{Added a nuisance to adjust the rate of same-sign lepton pairs.}  \\
 & \\
\hline
Build 25 & N Nuisances = 194 \\
\hline
 & \makecell[l]{Added tt+jets hierarchy. Reconfigured background process grouping to \\ \{(ZDY, DB),(ST,TB)\} with full $(N_\ell^S,N_{jet}^S)$ splitting } \\
 & \\
\hline
Build 30 & N Nuisances = 229 \\
\hline
 & \makecell[l]{Implemented lepton category nuisances from Table \ref{tab:lcatnuisance} and the Bronze \\ and Silver global fake rates from Table \ref{tab:fakenuisance} } \\
 & \\
\hline
Build 37 & N Nuisances = 209 \\
\hline
 & \makecell[l]{Reworked background process grouping to the final configuration in Table \\ \ref{tab:procnuisance}  and consolidated extraneous degrees of freedom. \\Consolidated Fake shapes to no longer split between HF and LF. } \\
 & \\
\hline 
Build 62 & N Nuisances = 227 \\
\hline
 & \makecell[l]{Reworked b-tagging parameters to include process splitting from Table \ref{tab:btagnuisance}. \\ This build reflects the final configuration of all previously described nuisances \\ and those implemented from Table \ref{tab:othernuisance}. } \\
\end{tabular} 
\label{tab:builds}
\end{table}
\section{Control Region Fit Results}
The results of the control region fit and the control region plus validation region fit using the fit model described in the previous section lead to good agreement between data and MC. Distributions summarizing the effects of the various nuisances, with simplified categorization splitting over a targeted category and integrated over sub categories, are shown in Figures \ref{fig:crsummary}\ref{fig:crprockin}\ref{fig:cr2lsummary}, and \ref{fig:crbtagsummary}. Figure \ref{fig:crsummary}, shows a summary of the data and MC agreement for each lepton multiplicity, this demonstrates the benefit of the jet multiplicity splitting, and also the effect of kinematic factors in 2L and 3L. The following figure Figure \ref{fig:crprockin} shows splitting by ptisr, gamT, and the region with the W+jets root factor. Figure \ref{fig:cr2lsummary} shows the lepton category splitting and a fake dominated bronze category and the final CR summary in Figure \ref{fig:crbtagsummary} illustates the splitting by SV and b-tag counting.

%0L jet split plot
%1L 2L 3L
\FigFour{Model_figs/0L_S_Summary.pdf}{Model_figs/1L_J_Summary_fakesIncl_b-fit.pdf}{Model_figs/2L_GSBJ_b-fit.pdf}{Model_figs/3L_Summary_fakesIncl_b-fit}{CR post-fit summary plots split by number of S-jets for each lepton  multiplicity. The bottom two figures for 2 lepton and 3 lepton selections include both jet or quality and jet or lepton category respectively. The two included types of splitting in 2 and 3 lepton are not mutually exclusive.}{fig:crsummary}

\FigTwoScale{Model_figs/1L_G_0J_ChargeSep_fakesIncl_b-fit.pdf}{Model_figs/0L_3J_PTISRgamT_fakesIncl_b-fit.pdf}{CR post-fit summary plots with the category associated with the root factor of W+jets (left) and a 0 lepton selection splitting by kinematic categories $p_T^{ISR}$ and $\gamma_\perp$ (right).}{fig:crprockin}{0.49}{0.49}

\FigTwoScale{Model_figs/2L_Cat_Summary.pdf}{Model_figs/2L_B_2J_DefaultPlot_b-fit}{CR post-fit summary plots demonstrating the effects on 2 leptons which are split by lepton categories (left) or a Bronze only region which is strongly associated with lepton fake factors (right).}{fig:cr2lsummary}{0.49}{0.49}

\FigTwoScale{Model_figs/1L_G_4J_DefaultPlot_fakesIncl_b-fit.pdf}{Model_figs/SV_Summary.pdf}{CR post-fit summary examples which show the agreement of data to the fitted background model for b-tagging and SV categories.}{fig:crbtagsummary}{0.49}{0.49}


In addition to the control region only fit, a control region combined with the validation region fit was performed. This fit uses the final systematic configuration but excludes the fake shapes. Fake shapes are excluded because their implementation requires consistent numbers of bins across gold, silver, and bronze. This uniform binning requirement disqualifies the validation region because the bronze bins cover the entire $R_{ISR}$ range. The CR+VR fit result is reasonable, but, has apparent systematic mismodeling from the absence of fake shapes.  To ensure that the high $R_{ISR}$ modeling is good enough, a bronze only fit is performed. This fit includes only the bronze categories and the complete set of systematics. Results are shown in Figure \ref{fig:crvrsummary} which show reasonable modeling in high $R_{ISR}$ regions.

%bronze only fit
\FigTwo{Model_figs/1L_B_0J_b-fit.pdf}{Model_figs/2L_B_2J_b-fit.pdf}{The Bronze only category fit with fake dominated 1 lepton and 2 lepton bronze categories.}{fig:crvrsummary}
\section{Bias tests}

One possible danger of having a fit with two many degrees of freedom is that the model would be too flexible and fit away potential signal. Testing for overfitting we perform and ensemble of pseudeoexperiments with and without signal injection to test if the fit model recovers the correct hypothesis.  If the fit is unbiased and sensitive to various signals then the signal strenth parameter, $r$, should be evaluated as $r=0$ in a background only fit and $r=1$ when signal is injected.  The hypotheses are tested by generating observed data in each bin centered on the MC expectation, fitting all regions simultaneously, and extracting the $r$ value. The pseudoexperiment is performed with a handful of points around the edge of the expected mass limits for T2tt, TChiWZ, and TSlepSlep. The results are shown in Figure \ref{fig:biasstudy} which shwow no fit bias by recovering of the correct hypothesis. This means that the final set of nuisances is sufficienctly small and the fit does not fit is sensitive to discovery.

\FigThreeScale{Model_figs/T2tt_7500670_r0_r1.pdf}{Model_figs/TChiWZ_3250315_r0_r1.pdf}{Model_figs/TSlepSlep_2500245_r0_r1.pdf}{Signal injection and bias test for three different signal grid points. The top left is T2tt with $m_{\tilde{t}}=750$ and $m_{\tilde{\chi}_1^0}=670$. The top right is TChiWZ with $m_{\tilde{\chi}_2^0}=325$ and $m_{\tilde{\chi}_1^0}=315$. The bottom figure uses TSlepSlep with $m_{\tilde{\ell}}=250$ and $m_{\tilde{\chi}_1^0}=245$ \textit{Source R. Salvatico} \cite{AN}}{fig:biasstudy}{0.49}{0.49}{0.49}
 




\setcounter{secnumdepth}{3}
\setcounter{tocdepth}{3}
\setlength{\parskip}{\smallskipamount}
\setlength{\parindent}{0pt}


\makeatletter


\providecommand{\tabularnewline}{\\}


\makeatother


\chapter{Results}


\section{Asymptotic Limits}

This search is designed to be generically sensitive to SUSY with an emphasis on compressed scenarios. The consequence is sensitivity to a wide variety of models and final states, so, we present expected upper limits on the cross sections for stop, electroweakino, and slepton processes. The results use the full Run II data-set alongside the full SM MC background combined with the data driven fit model described in the previous chapter. The limits are calculated using the asymptotic method for profile-likelihood the test statistic \cite{Cowan:2010js}.

\FigOneScale{Results_figs/allyear_limit_t2tt_138.pdf}{Run II expected cross section upper limits for di stop production, T2tt, excluding stop masses to the left of the 95\% CL line. }{fig:limt2tt}{0.6}

\FigOneScale{Results_figs/t2bw_lim138.pdf }{Run II equivalent expected cross section upper limits for di stop production with an intermediate chargino, T2bW. The MC uses the 2016 samples with 2016 luminosity and 2017 samples scaled to the combined 2017 and 2018 integrated luminosity. The chargino mass is assumed to be halfway between the LSP and stop mass for each grid point.}{fig:limt2bw}{0.58}


\FigOneScale{Results_figs/TChiWZ161718_138_lim1.pdf}{Run II expected cross section upper limits for Neutralino and Chargino production, TChiWZ, which excludes chargino masses to the left of the line with 95\% CL. Shown as a function of the chargino mass and chagino-LSP mass difference. The simplified model uses the Wino-like cross-section and assumes that the initial sparticle pair are mass degenerate $m_{\tilde{\chi}_2^0}=m_{\tilde{\chi}_1^\pm}$}{fig:limtchiwz}{0.58}

\FigOneScale{Results_figs/TChiWW_limit_corrected4-24-23.pdf}{2017 expected cross section upper limits scaled to the full Run II integrated luminosity of 138 fb$^{-1}$ for chargino pairs decaying to an oppositely signed W boson pair. Chargino masses are excluded to the left of the line with 95\% CL. Limits are based on wino corss-sections and a sample which is filtered to purely leptonic decays. }{fig:limtchiww}{0.58}


\FigOneScale{Results_figs/allyear_limit_tslepslep_138.pdf}{Run II expected cross section upper limits that exclude slepton masses to the left of the line at 95\% CL. Shown as a function of slepton mass and slepton-LSP mass splitting. The L/R indicates the super partners of the SM left or right handed partner which are assumed to be degenerate in mass.}{fig:limslep}{0.58}


 
\FloatBarrier

\section{Model Dependent Interpretation}
The TChiWZ limits from the previous section use a simplified model which is described in Chapter 1. The decay kinematics in this simplified model  use a flat matrix element resulting in $W$ and $Z$ boson 3 body uniform phase space decays. However, the 3 body decays phase space depends on the sign of the eigenstates of the neutralino mass matrix. A reweighting of events in the simplified TChiWZ model is performed to assess the difference in kinematics of the Z decay. This model reinterpretation has been done by other analyses in ATLAS and CMS \cite{CMS:2021edw}\cite{ATLAS:2019lng} but are restricted to $Z(\rightarrow \ell\ell)$. We extend the reweighting strategy to the more general case of di-fermion pairs where neutralino A undergoes a three body decay to a di fermion pair and neutralino B. 
\begin{equation}
\label{eq:3bd}
A \rightarrow f \bar{f} B
\end{equation}
The expression for the matrix element of the process in Equation \ref{eq:3bd} uses a phase space parameterization $x,y,z$ according to \cite{Nojiri:1999ki} which can be used to express the partial width 
\begin{equation}
\label{eq:dgam}
\frac{d\Gamma^\pm}{xy}\sim \frac{(1-x)(x-r_B^2)+(1-y)(y-r_B^2)\pm 2|r_B|z}{(z-r_z^2)^2}
\end{equation}
$x,y,z$ are represented by mass ratios
\begin{equation}
\begin{split}
x=(m_{\bar{f}B}/m_A)^2 \\
y=(m_{fB}/m_A)^2 \\
z=(m_{f\bar{f}}/m_A)^2 
\end{split}
\end{equation} 
and the $r$ parameters represent mass ratios of particle B or Z mass with particle A
\begin{equation}
\begin{split}
r_B = m_B/m_A \\
r_Z = m_Z/m_A 
\end{split}
\end{equation}
By sampling the $x,y,z$ space from Equation \ref{eq:dgam} along the allowed boundaries:
\begin{equation}
\begin{split}
r_B^2 \leq x \leq 1 \\
r_B^2 \leq x \leq 1 \\
z(xy-r_B^2) \geq 0
\end{split}
\end{equation}
we can generate the four momentum of all particles in the 3 body decay according to the two eigenstate cases, OS ($\Gamma^-$) and SS ($\Gamma^+$). Using the model dependent $(x,z)$ distribution a weight can be calculated mapping the original phase space of the simplified model to the OS or SS case. The  Dalitz distributions which fully describe the differences in Z decays in terms of $x$ and $z$ is shown in Figure \ref{fig:dalitzplot} for a TChiWZ grid point with $m_A = 300$ GeV and $m_B= 270$ GeV.

\FigThree{Results_figs/ThreeBody-PS.png}{Results_figs/ThreeBody-OS.png}{Results_figs/ThreeBody-SS.png}{Dalitz distributions of the phase space parameters x and z. Compares the uniform phase space simplified model against the model dependent scenarios OS and SS which distribute the four momenta of the decay components differently. }{fig:dalitzplot}

The relative differences in the final state kinematics for each model is shown in Figure \ref{fig:rwt} which uses the same TChiWZ grid point.

\FigTwoScale{Results_figs/DifermionMass.png}{Results_figs/Plot1DpNeutralino.png}{The reweighed difermion invariant masses (left) and an illustration of the momentum partitioning for each reweighting scenario (right)}{fig:rwt}{0.49}{0.49}

The resulting limits for the  model dependent cases are shown in Figure \ref{fig:rwtlim}. The exclusion boundary between the two model dependent cases are similar but the OS scenario is able to exclude higher masses likely due to the momentum enhancement of $A$ (LSP) which produces a larger missing momentum signature as seen on the red curve in Figure \ref{fig:rwt} right.

\FigTwoScale{Results_figs/OS16_lim.pdf}{Results_figs/SS16_lim.pdf}{The expected cross section upper limits for 2016 MC scaled to 138 fb$^{-1}$. Compares the OS reweighting (top) and SS reweighting (bottom). }{fig:rwtlim}{0.75}{0.75}




\setcounter{secnumdepth}{3}
\setcounter{tocdepth}{3}
\setlength{\parskip}{\smallskipamount}
\setlength{\parindent}{0pt}


\makeatletter


\providecommand{\tabularnewline}{\\}


\makeatother


\chapter{Summary}


This dissertation outlines a generic search for supersymmetric particles in compressed scenarios. The search is performed with proton-proton collisions at $\sqrt{s} = 13$ TeV and the CMS detector using the full Run II dataset with integrated luminosity of 138 fb$^{-1}$. A generic compressed SUSY topology is identified as ISR recoiling against an energetic invisible and soft visible system. This topology is organized into a set of decay trees by labeling topological components as either part of the ISR or Sparticle systems. The sparticle system is further subdivided into a di-sparticle system with visible and invisble components partitioned into each. The decay tree reference frames are approximated using a rule set from the Recursive Jigsaw Reconstruction framework that guides the optimal partitioning for each refrence frame. Following the construction of each event's decay tree, kinematic mass sensitive variables are derived to discriminate against backgrounds. Each event is then further categorized in bins of the mass sensitive kinematic variables $R_{ISR}$ and $M_\perp$ as well as physics objects combinatorics such as lepton multiplicity, jet multiplicity, b-tagging, and other complementary variables. The organization of all categories and the optimization of those categories has been discussed where the multitude of different complementary regions act as cross constraints for the fit. The lepton selection is also defined and the efficiencies of that selection are modeled with the Tag-and-Probe method. The Tag-and-Probe measurements are used to  model and correct systematic effects through scale factors for each component of the Gold, Silver, and Bronze in each year separately. The Tag-and-Probe scale factors adn systematics are a minor contribution to the overall fit which is composed of over 200 nuisance parameters. The nuisance parameters were derived by studying control region fits and statistical metrics such as the Poisson likelihood, z-score, or chi-squared. The set of nuisances is found to be satisfactory in describing both the control region fit and the validation region fit. But, the set of nuisances have been studied to have few enough degrees of freedom to not reject a signal hypothesis in an signal injected fit. From the establishment of this fit, expected limits are shown for T2tt, T2bW, TChiWZ, TChipmWW, and TSlepSlep  which extend the current exclusion threshold significantly.   



  
\global\long\def\bibname{References}%

%\bibliographystyle{plain}
\bibliographystyle{unsrt}
\bibliography{Biblio/allcites}


%\appendix
%
\chapter{My Appendix, Next to my Spleen}

There could be lots of stuff here

\end{document}
