%% LyX 2.3.1 created this file.  For more info, see http://www.lyx.org/.
%% Do not edit unless you really know what you are doing.
\documentclass[12pt,english,letterpaper]{kuthesis}
%\usepackage{mathptmx}
\renewcommand{\sfdefault}{lmss}
\renewcommand{\ttdefault}{lmtt}

\usepackage[T1]{fontenc}
\usepackage[utf8]{inputenc}
\usepackage{geometry}
\geometry{verbose,tmargin=1in,bmargin=1in,lmargin=1in,rmargin=1in}
\setcounter{secnumdepth}{3}
\setcounter{tocdepth}{3}
%\usepackage{color}
\usepackage{babel}
\usepackage{url}
\usepackage{graphicx}
\usepackage{setspace}
%\usepackage{esint}
\usepackage{chngpage}
\usepackage{multirow}



%% packages ripped from AN
\usepackage{xspace}
\usepackage{xcolor}
%\usepackage{cancel}
%\usepackage{graphicx}
%%\usepackage{subcaption}
\usepackage{amssymb}
\usepackage{amsmath}
%\usepackage[section]{placeins}
%\usepackage{hepnames}
%\usepackage{hyperref}
%\usepackage{cite}
%\usepackage{makecell}
%\usepackage{ulem}
%\usepackage{epstopdf}
%\usepackage{multirow}
%\usepackage{longtable}
%%
\usepackage{booktabs}% http://ctan.org/pkg/booktabs
\newcommand{\tabitem}{~~\llap{\textbullet}~~}
\usepackage{tabularx}

\usepackage{makecell}
%\renewcommand{\cellalign}{vh}
%\renewcommand{\theadalign}{vh}

\newcommand{\ttbar}{\ensuremath{{\PQt{}\PAQt}}\xspace}
\newcommand{\ttjets}{\ensuremath{{\PQt{}\PAQt}+\mathrm{jets}}\xspace}
\newcommand{\ttx}{\ensuremath{{\PQt{}\PAQt}\mathrm{X}+\mathrm{jets}}\xspace}
\newcommand{\zdy}{\ensuremath{\PZ/\PGg^{*}+\mathrm{jets}}\xspace}
\newcommand{\Wjets}{\ensuremath{\PW{}+\mathrm{jets}}\xspace}
\newcommand{\Znunu}{\ensuremath{\PZ(\to\nu\bar{\nu})+\mathrm{jets}}\xspace}
\newcommand{\et}{E_{\mathrm{T}}}
\newcommand{\alqed}{\alpha_{\mathrm{em}}}
\newcommand{\GF}{G_{\mathrm{F}}}
\newcommand{\mtop}{M_{\PQt{}}}
\newcommand{\mH}{M_{\PH{}}}
\newcommand{\mX}{M_{\mathrm{X}}}
\newcommand{\mW}{M_{\PW{}}}
\newcommand{\mZ}{M_{\PZ{}}}
\newcommand{\GZ}{\ensuremath{\Gamma_{\PZ}}}
\newcommand{\GW}{\ensuremath{\Gamma_{\PW}}}
\newcommand{\WW}{\ensuremath{\PWp\PWm}}
\newcommand{\ff}{\mathrm{f}\mathrm{\overline{f}}}
\newcommand{\bb}{\mathrm{b}\mathrm{\overline{b}}}
\newcommand{\Gt}{\Gamma_{\mathrm{t}}}
\newcommand{\mt}{m_{\mathrm{t}}}
\newcommand{\alr}{A_{LR}}
\newcommand{\ee}{\ensuremath{\Pep\Pem}}
\newcommand{\nZhad}{\ensuremath{n_{\PZ^{\mathrm{had}}}}}
\newcommand{\mumu}{\ensuremath{\Pgmp{}\Pgmm}}
\newcommand{\pipi}{\ensuremath{\Pgpp{}\Pgpm}}
\newcommand{\pip}{\ensuremath{\Pgpm{}\Pp}}
\newcommand{\kpi}{\ensuremath{\PKm{}\Pgpp}}
\newcommand{\sweff}{{\sin^2{\theta}^{\ell}_{\mathrm{eff}}}}
\newcommand{\sw}{\sin^2{\theta}_{\mathrm{W}}}
\newcommand{\ECM}{\sqrt{s}}
\newcommand{\percmsqs}{\mathrm{cm}^{-2}\mathrm{s}^{-1}}
\newcommand{\fbinv}{\mbox{\ensuremath{\,\text{fb}^{-1}}}\xspace}
\newcommand{\abinv}{\mbox{\ensuremath{\,\text{ab}^{-1}}}\xspace}
\newcommand{\MeV}{\ensuremath{\,\text{Me\hspace{-.08em}V}}\xspace}
\newcommand{\GeV}{\ensuremath{\,\text{Ge\hspace{-.08em}V}}\xspace}
\newcommand{\TeV}{\ensuremath{\,\text{Te\hspace{-.08em}V}}\xspace}
\newcommand{\xprime}{s^{\prime}/s}
\newcommand{\sqrtsp}{\sqrt{s}_{p}}
\newcommand{\pt}{\ensuremath{p_{\mathrm{T}}}\xspace}
\newcommand{\SingleWminus}{\ensuremath{\PWm\Pep\nu_{\Pe}}}
\newcommand{\SingleWplus}{\ensuremath{\PWp\Pem\overline{\nu}_{\Pe}}}
\newcommand{\gOneZ}{g_{1}^{\mathrm{Z}} }
\newcommand{\kgamma}{\kappa_{\gamma}}
\newcommand{\lgamma}{\lambda_{\gamma}}

%% File with various sparticle combinations
%
% The PS based names are in hepnames.sty/hepnicenames.sty/heppennames.sty
%
\def \conentwo {\PScharginoOnepm \PSneutralinoTwo}
\def \conecone {\PScharginoOneplus \PScharginoOneminus}
%\def \c1c1 {\PScharginoOneplus \PScharginoOneminus}
\def \nonentwo {\PSneutralinoOne \PSneutralinoTwo}
\def \ntwontwo {\PSneutralinoTwo \PSneutralinoTwo}
\usepackage[authoryear]{natbib}
\doublespacing
\usepackage[unicode=true,
 bookmarks=true,bookmarksnumbered=false,bookmarksopen=false,
 breaklinks=true,pdfborder={0 0 0},pdfborderstyle={},backref=false,colorlinks=true]
 {hyperref}
\hypersetup{pdftitle={University of Kansas Thesis Template},
 pdfauthor={Anonymous},
 pdfsubject={A Thesis},
 urlcolor={black},citecolor={black},allcolors={black}}

\makeatletter

%%%%%%%%%%%%%%%%%%%%%%%%%%%%%% LyX specific LaTeX commands.
%\providecommand{\LyX}{\texorpdfstring%
%  {L\kern-.1667em\lower.25em\hbox{Y}\kern-.125emX\@}
%  {LyX}}
%% Because html converters don't know tabularnewline
%\providecommand{\tabularnewline}{\\}

%%%%%%%%%%%%%%%%%%%%%%%%%%%%%% User specified LaTeX commands.
%%%FUNCTIONS

\newcommand\FigOne[3]{%
\begin{figure}[!htbp]%
\centering
\includegraphics[width=0.6\textwidth]{fig/#1}\hfill
\caption{#2}
\label{#3}
\end{figure}}

\newcommand\FigOneScale[4]{%
\begin{figure}[!htbp]%
\centering
\includegraphics[width=#4\textwidth]{fig/#1}\hfill
\caption{#2}
\label{#3}
\end{figure}}

\newcommand\FigTwo[4]{%
\begin{figure}[!htbp]%
\centering
\includegraphics[width=0.45\textwidth]{fig/#1}
\includegraphics[width=0.45\textwidth]{fig/#2}\hfill
\caption{#3}
\label{#4}
\end{figure}}

\newcommand\FigTwoScale[6]{%
\begin{figure}[!htbp]%
\centering
\includegraphics[width=#5\textwidth]{fig/#1}
\includegraphics[width=#6\textwidth]{fig/#2}\hfill
\caption{#3}
\label{#4}
\end{figure}}

\newcommand\FigThree[5]{%
\begin{figure}[!htbp]%
\centering
\includegraphics[width=0.45\textwidth]{fig/#1}
\includegraphics[width=0.45\textwidth]{fig/#2}\hfill
\includegraphics[width=0.45\textwidth]{fig/#3}\hfill
\caption{#4}
\label{#5}
\end{figure}}

\newcommand\FigThreeScale[8]{%
\begin{figure}[!htbp]%
\centering
\includegraphics[width=#6\textwidth]{fig/#1}
\includegraphics[width=#7\textwidth]{fig/#2}\hfill
\includegraphics[width=#8\textwidth]{fig/#3}\hfill
\caption{#4}
\label{#5}
\end{figure}}


\newcommand\FigFour[6]{%
\begin{figure}[!htbp]%
\centering
\includegraphics[width=0.49\textwidth]{fig/#1}
\includegraphics[width=0.49\textwidth]{fig/#2}\hfill
\includegraphics[width=0.49\textwidth]{fig/#3}
\includegraphics[width=0.49\textwidth]{fig/#4}\hfill
\caption{#5}
\label{#6}
\end{figure}}


\newcommand\FigFive[7]{%
\begin{figure}[!htbp]%
\centering
\includegraphics[width=0.45\textwidth]{fig/#1}
\includegraphics[width=0.45\textwidth]{fig/#2}\hfill
\includegraphics[width=0.45\textwidth]{fig/#3}
\includegraphics[width=0.45\textwidth]{fig/#4}\hfill
\includegraphics[width=0.45\textwidth]{fig/#5}\hfill
\caption{#6}
\label{#7}
\end{figure}}
%%% END FUNC



%used to align decimals in tables according to APA style
\usepackage{dcolumn}
\usepackage{booktabs}
\usepackage{placeins}
% Set the title and author info
\title{Search for Weak Scale Supersymmetric Particles in Compressed Scenarios }
\author{Justin Anguiano}

% Following is OPTIONAL list of previous degrees earned. 
% If there are more than 5 or so, then title pagelayout may become too crowded,
% depending on the number of committee members. 
\priorcreds{B.S. Engineering Physics, University of Kansas, 20XX}{M.S. Computational Physics and Astronomy, University of Kansas, 20XX}
% It is acceptable to delete \priorcreds if it is not desired on title page


\dept{Department of Physics and Astronomy}
\degreetitle{Doctor of Philosophy}
\papertype{Dissertation} %or Thesis (Choose whatever word you want to put on p.2)

%% Committee member names are required for the title page. We make space
%% for as many as 7 members, with various roles/titles.
%% It is required to have 7 entries, even if some are empty for committee and role
\committee{Graham Wilson}{Alice Bean}{Christopher Rogan}{Ian Lewis}{Zsolt Talata}{}{}
\role{Chairperson}{}{}{}{External Reviewer}{}{}
%At Most 7 members allowed, last here is blank on purpose to demonstrate
%flexibility

%%BOTH dates must be included. 
\@printd@testrue
\datedefended{July 02, 2019}
\dateapproved{August 06, 2019}

%% These settings are now in the kuthesis.cls file, but users are free
% to customize. listings has great documentation online
%% When listings are used, break lines
%\lstset{
 %    breaklines=true,  % sets automatic line breaking
 %    breakindent=2em,
 %    breakatwhitespace=true,  % sets if automatic breaks should
 %   breakautoindent=true
%}

\@ifundefined{showcaptionsetup}{}{%
 \PassOptionsToPackage{caption=false}{subfig}}
\usepackage{subfig}
\makeatother

\usepackage{listings}
\renewcommand{\lstlistingname}{\inputencoding{latin9}Listing}

\begin{document}
\begin{romanpages}

\maketitle

\begin{abstractlong}

A generic search for supersymmetric particles with an emphasis on compressed scenarios is performed withproton-proton collisions at $\sqrt{s} = 13$ TeV and the CMS detector using a data sample with integrated luminosity of 138 fb$^{-1}$. Potential supersymmetric events with initial-state-radiation recoiling against a massive invisible sparticle system are organized based the Recursive Jigsaw Reconstruction method. Events are then further categorized based on physics object combinatorics such as jet multiplicity, lepton multiplicity, b-tags, and kinematics that are sensitive to generic compressed sparticle topologies. This work focuses on a subset of many pieces of a larger analysis where the underlying compressed strategy is discussed, the selection of leptons and the calibration of simulation through the study of systematic effects, and the implementation and optimization of the data driven fit. Finally expected limits with a 95\% C.L. are placed on processes that include the production of electroweakinos, sleptons, and stops. 

\end{abstractlong}

\begin{acknowledgementslong}
%%if you want a "quote" environment for acknowledgements,
%% use acknowledgements instead of acknowledgementslong
First I would like to acknowledge the collaborative effort and work done by my research group. This work would not be possible without the extensive work of my advisors and peers which has been included alongside my own work for completeness. Much of the content is based on, or modeled after our internal Analysis Note \cite{AN}, so thanks to everyone that has contributed to this massive analysis.  
I am also extremely grateful for my supervisor, Professor Graham Wilson, for his invaluable mentorship and patience that has pushed me to be the best version of myself. I'm also grateful for everyone in my research group and their excellent advice and support, with a special thanks to the students which are always happy to team up and solve problems. I would be much worse off without the community we've developed. Finally, I would like to express my gratitude to my parents which have been essential in helping me survive the final stretch of grad school.


\end{acknowledgementslong}

\tableofcontents{}

\listoffigures

\listoftables

\end{romanpages}

%

\setcounter{secnumdepth}{3}
\setcounter{tocdepth}{3}
\setlength{\parskip}{\smallskipamount}
\setlength{\parindent}{0pt}


\makeatletter


\providecommand{\tabularnewline}{\\}


\makeatother

%\usepackage{babel}
%\begin{document}

\chapter{The Standard Model and Supersymmetry}

%\begin{chapterabstract}
%Here we outline the fundamental concepts of particle physics, we introduce the set of fundamental particles, fields, and interactions which %are described by the standard model. Since the standard model does not solve all of the problems of particle physics I introduce %supersymmetry, a well motivated extension to the standard model. The implications of SUSY model space on various phenomoglly of processes %as well as experimental observations.  The motivation for the SUSY extension is reserved for the following chapter.

%\end{chapterabstract}

\section{Introduction}

The fundamental building blocks of matter and their interactions expressed through three of the four fundamental forces of nature via the Standard Model (SM). The fourth force, or gravity, is left to General Relativity. The SM  is the culmination of over a century of work by many theorists and experimentalists, with roots in the late 19th century. The experimental starting point begins with the establishment of the sub-atomic world with the discovery of the electron and proton in by JJ Thomson in 1897 \cite{Thomson:1897cm} and Ernest Rutherford in 1917 \cite{Rutherford:1911zz}. One of the next major milestones is the prediction of the neutrino based on observations of beta decay by Enrico Fermi in 1934 \cite{Fermi:1934hr} and its discovery in 1957 by Clyde Cowan and Frederick Reines in 1956 \cite{Reines:1956rs}.  Following the neutrino, one of the fundamental forces is established under the theory of strong interactions, formulated as quantum chromodynamics by Murray Gell-Mann and others in the 1960s, which provides a description of how protons and neutrons are held together in the nucleus of an atom \cite{GellMann:1964nj}. Entering the modern Standard Model era, the theory describing weak interactions from Enrico Fermi in the 1930s is combined with electromagnetic interactions in electroweak theory by Sheldon Glashow, Abdus Salam, and Steven Weinberg in the 1960s \cite{GLASHOW1961579}\cite{Salam:1968rm}\cite{Weinberg:1967tq} and later confirmed by the UA1 experiments with the discovery of the W and Z bosons in 1983\cite{arnison1983experimental}\cite{glashow1984future}.
In this chapter, concepts of the Standard Model are introduced, including the fundamental particles, fields, and their basic properties and interactions. Then expanding from the core SM we will discuss an extension of the Standard Model with supersymmetry, which proposes a new symmetry between fermions and bosons. Finally, we delve into the specifics of simplified models of supersymmetry and the challenges associated with detecting these models experimentally.



\section{The Standard Model}

The Standard Model is a collection of adhoc theories used to predict and reproduce experimental data. The theory itself incorporates four major concepts: Quantum Field theory (QFT), the Dirac equation, the gauge principle, and the Higgs mechanism. These four principles are constrained by physical data and describe the set of elementary particles, known as fermions and bosons. The SM generally refers to the SM Lagrangian, an equation with different sectors that describe different subsets of particles, fields, and their interactions. The SM Lagrangian itself consists of 26 free parameters which are input by hand. These parameters are: the masses of the 12 fermions, 3 coupling constants that describe gauge interactions: $g, g', g_s$, 2 parameters to desribe the Higgs potential i.e. the higgs mass $m_h$ and the vacuum expectation value (vev), and 9 mixing angles which describe the  PMNS and CKM matrices or the mixing of different fermionic fields. The 12 fermion paramters are subdivided by three neutrinos $m_{\nu_i}$, three charged leptons $m_{\ell_i}^\pm$, and six quarks $m_{q_i}$ \cite{Thomson:2013zua}.

QFT provides a description for both known and theoretical particles by combining quantum theory, the field concept, and relativity \cite{Peskin:1995ev}. The gauge theory aspect describes the exact nature of QFT interactions and provides the mechanisms for the electromagnetic, strong, and weak forces.  We know of three gauge fields:  $\vec{G}$ which transforms under $SU(3)$ and govern strong interactions, $\vec{W}$ and $B$ which transform under $SU(2)_L \times U(1)$ and govern electromagnetic and weak interactions. The combination of the gauge fields and fermion fields along with the Dirac equation yields eigenstates that represent fermionic matter particles. These particles would be massless if not for the inclusion of the complex scalar Higgs field.  The spontaneous symmetry breaking of the Higgs field, due to the Yukawa coupling, creates the non-zero vev responsible for generating the masses of the electroweak gauge bosons. Additionally, the interaction between the fermionic fields and the non zero-vev generates the masses of SM fermions \cite{Higgs:1966ev}\cite{Bernardi:2008zz}.

The set of standard model elementary particles is divided into two subgroups: fermions and bosons.  The fermions consist of both charged and neutral leptons as well as fractionally charged quarks. There are three flavors of charged leptons $(\ell)$, the electron $(e)$, the muon $(\mu)$, and the tau $(\tau)$. Each charged lepton has a flavor pairing neutral neutrino $\nu_\ell$. The $e$ and $\mu$ are also generally considered as "light" leptons due to their small mass relative to the $\tau$. The term lepton, depending on context, often refers to only the charged particles. As for the quarks, there are also three generations of pairs of quarks.. The lighest set of quarks are the up $(u)$ and down $(d)$ quarks, followed by the charm $(c)$ and strange $(s)$, and lastly the bottom $(b)$ and extremely massive top quark $(t)$.  The bosons are the force carrying particles which represent the gauge fields. They are comprised of the vector bosons - the photon $(\gamma)$, gluon $(g)$, the $W^\pm$, and the $Z^0$ - along with the singular scalar boson the Higgs $(h)$ \cite{ParticleDataGroup:2020ssz}. The elementary particles masses, generations, and spins are summarized in Figure \ref{fig:smfig}.

\FigOne{Intro_figs/Standard_Model_of_Elementary_Particles.png}{particles figure cite wiki}{fig:smfig}


%The SM Lagrangian is composed of constituent sectors which describe diffrent groups/fields/particles. The main SM sectors are, the quantum chromodynamics (QCD) sector, the electroweak sector, the Higgs sector, and the Yukawa sector. QCD describes colored interactions of quarks mediated by gluons with the strong force. The electroweak sector unifies both the electromagnetic and weak interactions via exchange of W or Z bosons as well as electromagnetic interactions via $\gamma$. The Higgs sector introduces the complex scalar higgs field (citation needed). Interaction of bosons with the Higgs field causes the bosons to have mass and the Yukawa coupling describes the interaction of fermions with the higgs field which also allows the fermions to have mass (citation needed).



The SM is an asymmetric chiral theory, combining three groups $SU(3)_L \times SU(2)_L \times U(1)$. The $L$, or left handed, subscript indicates that mirrored fields (with different chiralities)  transform differently under the Lorentz group and the EW gauge group (cite slides).  The consequence of chiralilty is that the possible combinations between interaction vertices is limited \cite{Thomson:2013zua}. This peculiar property shows up with the $W$ boson, which only couples to left handed particles or right handed antiparticles. Extensions of the standard model also often extend chiral or symmetrical properties.  %Helcity is the defined by the projection of a particles spin ontion its direction of motion (thompson). A particle is considred right-handed if the direction of its spin and motion is parallel. %It is left-handed if spin and motion antiparallel. (cite wikipedia helicity). This peculiar property shows up with the W boson, which only couples to left handed particles or right handed antiparticles.

\section{Supersymmetry}

Supersymmetry (SUSY) is an extension of the standard model. It adds a generator that rotates the spin between bosons and fermions. This then introduces a bosonic degree of freedom for every fermionic degree of freedom  which generates a super partner for each particle with spin differing by a half integer \cite{Baer:2007izw}.  The resulting set of mirrored elementary particles are referred to as sparticles. Each bosonic sparticle carries the same name as its fermion partner but with an "s" prefix e.g. sfermion, squark, selectron. As for the bosons, with the gauge fields $B$ and $\vec{W}$, these are accompanied by three super symmetric fields - the Higgsino $\tilde{H}$, Bino $\tilde{B}$, and Wino $\tilde{W}$. The mixture of the B and $\vec{W}$ SM fields can be represented by particle matrix. One can obtain the  mass eigenstates representing the SM particles $\gamma, \, \, Z, \, \, W^\pm$ through the diagnolization of particle matrix. Similarly, the Higgsino, Bino, and Wino mix to produce four neutral and two charged eignestates, the neutralinos ($\tilde{\chi}^0_1, \tilde{\chi}^0_2, \tilde{\chi}^0_3, \tilde{\chi}^0_4$)  and charginos ($\tilde{\chi}^\pm_1, \tilde{\chi}^\pm_2$) \cite{DJOUADI_2008}. SUSY also requires an additional Higgs doublet to give mass to up-type and down-type fermions,  leading to five higgs boson states consisting of two charged Higgs and three neutral Higgs \cite{Adam:2021rrw}. The lightest neutral higgs of the three neutral options represents the SM Higgs boson. The full set of SM particles alonglide their SUSY partners are illustrated in Figure \ref{smandsusyfig}. The addition of another higgs doublet also introduces a second vev. The ratio between the two vev's  is commonly denoted as $v_1/v_2 = \tan \beta$ and is an important parameter in experimental searches. Another important bookeeping parameter, similar to lepton number or baryon number conservation, is R-parity. This parameter tallies the total number of SM particles (+1) and sparticles (-1) and expects the net total between particles to be conserved in the initial and final states. R-parity conservation then requires sparticles to be produced in pairs. If R-parity is violated, the common consequence is that the lightest supersymmetric particle (LSP) is unstable or the proton decays \cite{Farrar:1978xj}. 

\FigOne{Intro_figs/elementary_sparticles.png}{stolen from this springer thesis book, probably make my own figure later \url{https://link.springer.com/chapter/10.1007/978-3-030-25988-4_4}}{fig:smandsusyfig}



Supersymmetry is an extremely expansive model and intractable to experimentally test without significant well motivated simplifications. The most experimentally common simplified SUSY model is the Minimally Super Symmetric Standard Model (MSSM). The MSSM contains the smallest number of new particle states and new interactions which are consistent with phenomenology \cite{Baer:2007izw}. The MSSM is still experimentally inaccesible due to the presence of over 100 parameters, where small changes in parameter space can completely morph the model structure and experimental signatures. To reduce the problem's dimensionality, further simplification is needed, resulting in a popular simplified model: the phenomological MSSM (pMSSM). The pMSSM contains 19 parameters which include the masses of each generation of squark and slepton, parameters to control the mixing of $\tilde{H}, \tilde{W}, \tilde{B}$, and dials for the higgs doublet \cite{MSSMWorkingGroup:1998fiq}.  The pMSSM is still borderline too complicated to attack directly, so, the pMSSM is boiled down into a simplified model of four parameters $M_1$,$M_2$, $\mu$, and $\tan\beta$. $M_1$ and $M_2$ are the gaugino mass parameters, $\mu$ is the Higgsino mass parameter, and $\tan\beta$ is the previously mentioned vev ratio \cite{Fuks_2018}.  A model point from this four parameter space is referred to as Realistic simplified gaugino-higgsino model, and targets specific regions of MSSM parameter space and experimental topologies.

To effectively grasp the structure of SUSY and various models, either in the pMSSM or simplified models, there are a couple key elements to condsider. The first elements is the mass scale of the relative SUSY sectors i.e. how massive are the gauginos versus sleptons versus squarks. If the mass scales are well separated, the sectors are effectively decoupled. If the mass scales are similar then it may introduce complicated cross-talk between sectors. In an electroweak SUSY search with a 4 parameter simplified model, the model can be further simplified by assuming squarks and slepton masses sit at the several TeV scale while the targeted electroweak-inos are at detectable  TeV and sub-TeV scale.  By decoupling sectors outside the sector-of-interest we remove the interaction between these groups, so, if sleptons are decoupled from the gauginos complicated dependencies, like cascading decays are avoided. The other key element is the composition of the LSP, typically $\tilde{\chi}^0_1$. Each unique model point is composed of a specific mixing of $\tilde{H},\tilde{W},\tilde{B}$ with an LSP that reflects that mixing. The model point is denoted by the field that dominates the overall mix, so a Higgsino model has an LSP composed of mostly $\tilde{H}$ \cite{ATLAS:2015wrn}. The characteristic take away from simplified model types is that H,W,B  can control the nature of the model by governing the overall cross sections for sparticles, the topological infrastructure, and how the sparticles interact and amongst themselves and SM particles. Two pMSSM examples comparing the mass structure between two arbitrary mass points of a Wino model versus Higgsino model is shown in Figure \ref{fig:mass_modelpoint}. For both models the Higgs and slepton sectors are decoupled at a multi-TeV scale while the squark and gaugino sectors are at an accesible TeV and sub-TeV scale. Note that small changes in pMSSM model space results in differing LSP content and large variations in the relative mass structure and orderings. The difference in cross sections between the same two model points for gaugino pair production combinations are show in in Figure \ref{fig:xsec_modelpoint}. This relative differences in cross section illustrates that the same tweak in parameter space can induce order of magnitude changes sparticle production.

\FigTwo{Intro_figs/wino_modelpoint_mass.png}{Intro_figs/higgsino_modelpoint_mass.png}{mass structure winno vs higgsino modelpoints}{fig:mass_modelpoint}


\FigTwo{Intro_figs/wino_modelpoint_xsec.png}{Intro_figs/higgsino_modelpoint_xsec.png}{xsec strucutre wino vs higgsino modelpoints}{fig:xsec_modelpoint}
%Example of differing model points


In addition to the mass structure and cross sections, the decay nature of H/W/B models also varies. The variation in decay modes has a significant impact on the experimental channels and signatures of interest. In an experimental search we would expect the heavier sparticles to decay to both SM particles along with the LSP. If the LSP happens to be close in mass to its parent, say O(100) GeV or less, the model would be considered as a compressed scenario. This scenario is considered compressed because the observable energy of the SM particle involved in a sparticle decay is compressed to a very small amount due to the majority of the available energy being used by the rest mass of the sparticles. Of the 3 types of models, the most likely candidates for compression are the Higgsino-like and Bino-like models. Wino models by far have the largest cross sections but are the least likely to have compressed states. Particularly interesting topologies for these compressed models involve decay signatures of processes like $\tilde{\chi}^0_2 \rightarrow Z^*\tilde{\chi}^0_1 $, $\tilde{\chi}^0_2\rightarrow \tilde{\chi}^\pm_1 \tilde{\chi}^0_1 $, $\tilde{\chi}^\pm_1\rightarrow W^\pm \tilde{\chi}^0_1$, $\tilde{t}\rightarrow t \tilde{\chi}^0_1$, $\tilde{\ell}\rightarrow\ell \tilde{\chi}^0_1$. The nature of sparticle decay is not only dependent on the H/W/B nature of the model but also on the degree of compression. Figure \ref{fig:n2decaymodes} shows the average decay modes for H W or B from a selection of pMSSM models \cite{ATLAS:2015wrn}.

\FigThree{Intro_figs/wino_n2decaymodes.png}{Intro_figs/bino_n2decaymodes.png}{Intro_figs/higgsino_n2decaymodes.png}{N2 BFs}{fig:n2decaymodes}  

Note that between each model type in Figure \ref{fig:n2decaymodes} the $Z^*$ and $W^\pm$ modes can be highly suppressed or enhanced. In some cases even, specific modes like $\tilde{\chi}^0_2\rightarrow \tilde{\chi}_1^\pm W^\mp$ can be either kinematically forbidden, or excluded to streamline MC production and enhance the statistical power of different targeted final states. Alongside the decay specific complications, the phase space of the final state particles is model dependent. For instance, in the case of $\tilde{\chi}^0_2 \rightarrow Z^*\tilde{\chi}^0_1 $ the shape of $Z$ dilepton mass distribution $m_{\ell\ell}$  changes depending on the sign of the gaugino eigenstates. Experimentally this problem is divided into two possible scenarios: cases where the eigenstates are the same sign and cases where the eigenstates are the opposite sign. The distribution that showcases the $m_{\ell\ell}$ differences under two different model interpretations is shown in Figure \ref{fig:atlasmllwbh}. Overall, with the complications of model dependent decays, inherently rare production, varying mass orderings, and relative scale between sectors, the search for SUSY is an extraordinary challenge. To discover SUSY one should design a search to encompass a large generalized model space and target generic features rather than highly specific corners. 


\FigOne{Intro_figs/atlas_wbh_mll.png}{mll reweight with w/b or H interpretations from altas paper in grahams talk \cite{ATLAS:2019lng}}{fig:atlasmllwbh}

%What particles are in susY?
%For each elementary standard model particle there is a super partner. For the quarks and leptons, the pairing is simple, there is just the equivalent slepton and squark partner. The gauge bosons are slightly more complicated, these are generally denoted with an "ino" suffix. There are also 3 super fields which mix in specific quantities to yield varying instances of particles with particular properties. These mixings define the characteristics of the model point by influecing things like decay mode, cross seection, and couplings.  (WhY?) There are four neutralinos $\chi^0_i$ and two charginos $\chi^\pm_j$. There are also 4 Hiigs bosons, a charged pair $H^\pm$ and a neutral pair $H^0_{u,d}$. (Why?)  The electroweakinos, i.e chargino or neutralinos, increase in mass with increasing index but the structure of reletavie masses depends specifically on the model. The $\chi^0_0$ is generally the lightest supersymmetric partilce (LSP) and in many popular models is stable. The instances of stable LSP depend on R-parity conservation. (Define R-parity conservation) If this is violated the LSP will decay into SM particles.



%Incldue a plot with mass hierarchies. Since there are so many possible parameters, varying sets of paramters can produce significant diffences in experimental signatures and topologies. Typically for a model we decouple specific sectors when generatting monte carlo, For instance if we are searching for sleptons, the squark or electroweakino sector will be chosen to be significantly heavier (out of current experimental range) effectively decoupling it from the slepton sector. Then from a simplified model with everything else decoupled we scan various topologies with particular mass values. 

%SUSY chirality

%Talk about higgsino/wino bino model structure decay modes etc


%
\setcounter{secnumdepth}{3}
\setcounter{tocdepth}{3}

\setlength{\parindent}{1 em}


\makeatother


\chapter{Motivating the Search for SUSY}

%\begin{chapterabstract}
%Introduce some of the issues of the SM and opening up with the basic motivations for susy, like solving the hierarchty problem and dark matter candidate, next we look %at a theoretical motivation for SUSY via the higgs mass. Motivate simplified models with naturalness etc, talk about how susy needs to be at a few TeV scale to work %out. Then we visit two recent experimental measurements which strongly motivate the search for susy and more speficically this body of work.
%\end{chapterabstract}

\section{Introduction}

The Standard Model is a remarkable theory describing a wide variety of sub-atomic phenomenon which also has consistently held up to tests over many orders of magnitude in energy. However, it's not a perfect theory. There are a few  experimental and theoretical problems that the SM can not yet explain such as: how to incorporate gravity,  how can we explain neutrino mass and mass orderings, or why is the universe made up of matter and not antimatter?  
A significant problem in physics, which connects both galactic scale and sub-atomic scale physics, is dark matter. Cold dark matter (CDM) is a type of matter that is thought to have played a crucial role in the formation of large-scale structures in the Universe, such as galaxies and galaxy clusters \cite{Garrett:2010hd}. The evidence for CDM begins with observations of Zwicky in 1933 where he found that the observed motion of the galaxies in the Coma Cluster could not be explained by the gravitational interactions of visible matter in the cluster, but could be explained by adding additional invisible mass in the form of ``dark matter'' \cite{Zwicky:1933gu}. Additional evidence for CDM has compiled over the years such as: gravitational lensing data that disfavors being completely explained by effects black holes or condensed baryonic matter \cite{Massey:2007lens}, large scale structure formation where CDM can explain the formation and evolution of galaxy clusters \cite{Springel:2005}, and temperature fluctuations in the cosmic microwave background that suggest the universe is composed of approximately 85\% dark matter \cite{Planck:2018vyg}. Despite overwhelming evidence for the existence of CDM there are no suitable SM dark matter candidates to explain the abundance of this potential cosmic particle. SUSY offers an attractive solution with the introduction of new particles that can explain dark matter directly via massive invisible particles, such as the neutralino, $\chi_1^0$, or sneutrino $\tilde{\nu}$. The neutralino can handle the CDM problem with models capable of producing the expected CDM relic density of the universe and, in fact, this is used to constrain SUSY model space and simplify searches. Aside from these leading motivations, other more detailed motivations will be discussed in this chapter, the first being the ``naturalness problem'' with its theoretically aesthetic improvement adding a symmetry to protect against divergent terms in the perturbative expansion of the Higgs mass. The next motivations are experimental, where SUSY offers an explanation to the significant deviation observed in the muon $(g-2)_\mu$ factor from recent FNAL result, as well as the deviation observed in the $W$ boson mass at CDF II. It should be noted that the divergent Higgs mass - known as the hierarchy problem - satisfies most SUSY scenarios up to the few TeV scale \cite{Barbieri:1987fn}, but, the two latter experimental measurements specifically motivate SUSY compressed scenarios.



\section{Stabilizing the Higgs mass}

An aesthetic attribute of theoretical models is naturalness. We should expect a model to function naturally if the ratio of free parameters in a model are of $O(1)$. Large swings between parameters would be considered fine-tuning and could indicate issues with the underlying theory. So, naturally, if fine-tuning exists in a model, it strongly motivates building extensions to the model to eliminate fine-tuning. %. We would expect with some improved theory with new physics one would balance out finely tuned parameters providing a natural solution to whatever is being modeled. 
One such fine tuning arises in the hierarchy problem, specifically in the Higgs self interaction terms. The SM Higgs Lagrangian terms that involve self interaction are illustrated in equation \ref{eq-higgslagrange}.
\begin{equation}
\label{eq-higgslagrange}
\mathcal{L}=\frac{gm_h}{4M_W}H^3 - \frac{g^2m_h^2}{32M_W^2}H^4
\end{equation}
$H$ represents the scalar Higgs field, $m_h$ the Higgs mass, and $m_W$ the W boson  mass. A correction to the Higgs mass can be calculated using standard perturbation theory by evaluating the second term of the Higgs Lagrangian \cite{Baer:2007izw}. 
\begin{equation}
\label{eq:higgpert}
\begin{split}
\Delta m_h^2 = \langle H | \frac{g^2m_h^2}{32M_W^2} H^4 | H  \rangle = 12\frac{g^2m_h^2}{32M_W^2}\int \frac{d^4 k}{(2\pi)^2} \frac{i}{k^2 - m_h^2}\\
= 12\frac{g^2m_h^2}{32M_W^2} \frac{1}{16\pi^2}\big( \Lambda^2 - m_h^2\log\frac{\Lambda^2}{m_h^2} + O(\frac{1}{\Lambda^2})\big)
\end{split} 
\end{equation}
 
The equation \ref{eq:higgpert} integral term is the propagator for the exchange of a virtual Higgs and is integrated over phase space. The $\Lambda$ is known as the scale cutoff parameter and should be interpreted as the scale at which the SM breaks down, possibly near the Planck scale $O(10^{19})$ GeV. Notice the leading term $\Lambda^2$ indicates that the expansion is quadratically divergent. The divergent mass correction means there needs to be extremely large cancellations, around 20 orders of magnitude, to maintain $\Delta m_h \sim O(m_h)$. This divergent phenomenon can also be observed with fermion masses, but, chiral symmetry protects the fermion mass from divergence by canceling out high order $\Lambda$ terms. SUSY offers a similar protection to the Higgs mass by introducing a symmetry with the additional fermionic and bosonic degrees of freedom leading to similar cancellations and producing a more natural model. 


\section{The Muon Anomalous Magnetic Moment}

A major experimental motivation for SUSY lies within the measurement of the muon anomalous magnetic moment.  Multiple measurements between two labs, Brookhaven National Lab (BNL) and Fermi National Accelerator Lab (FNAL) have shown significant disagreement with the SM. These experiments measure the muon $g$ factor, or specifically, its deviation from two, $(g-2)_\mu$ .  The $g$ factor is related to the electromagnetic coupling of charged particles with the photon and largely depends on the tree level lepton-photon coupling, but, gets small quantum corrections from higher order loops. The largest correction is the single photon loop shown in Figure \ref{fig:gm2fig}. To predict the $g$ factor, an SM calculation is performed with three types of quantum corrections: Quantum Electrodynamics (QED), Electroweak (EW), and Hadronic. Corrections from the Higgs are neglected because the effects are not experimentally observable. %the mass disparity $m_h >> m_{e,\mu}$ and the mass dependence in the Higgs coupling  effects that are smaller than what is experimentally observable. 
The g-factor prediction starts at exactly 2, with QED, and then involves quantum corrections up to $O(10^{-11})$. The prediction is compared with an experimental measurement at a very high level of precision. If the observation were to deviate from the SM prediction, it can indicate new and unaccounted for physics interactions with the SM leptons.
The current best $a_\mu = \frac{g-2}{2}$ prediction is reported as $a_\mu= a_\mu^{QED}+ a_\mu^{EW}+a_\mu^{\text{Hadronic}} =  116 591 810(43) \times 10^{-11}$ \cite{Muong-2:2021ojo}.
\FigOneScale{Motivation_figs/g2_diagrams_pdg.png}{Diagrams which contribute to $(g-2)_\mu$. Left is the single photon Schwinger loop that contributes the largest deviation from two. The middle diagrams are electroweak contributions and the far right diagram is the hadronic vacuum polarization the involves a loop with hadrons \cite{ParticleDataGroup:2020ssz} }{fig:gm2fig}{0.95}
 For each of the $a_\mu$ components, the QED component enters at the $O(10^{-3})$ and is known to $O(10^{-11})$. the EW component enters the sum at $O(10^{-9})$ and is known to $O(10^{-10})$. Finally the most complicated hadronic component, contributes at $O(10^{-8})$ and is known up to $O(10^{-9})$. The hadronic contributions arise from Hadronic vacuum polarization (HVP) and light by light scattering (LBL) with the former diagram also illustrated in Figure \ref{fig:gm2fig}. The $a_\mu^{\text{Hadronic}}$ precision dominates the overall $a_\mu$ error and is constrained by data driven measurements alongside the limitations of the computational approach using QCD lattice theory. The BNL measurement of $a_\mu$ yields a difference with the SM prediction of $\Delta a_\mu := a_\mu^{BNL} - a_\mu^{SM} = 279(76) \times 10^{-11}$ which carries significance of $3.7\sigma$. The most recent $a_\mu$ measurement from FNAL confirms the BNL measurement within $1\sigma$ and the combined experimental average increases the SM deviation with a significance of $4.2\sigma$ \cite{Muong-2:2021ojo}.


%Experimentally the deviations from 2 are the most interesting, and are written in the form $a_\ell = \frac{g-2}/2$ and referred to as $(g-2)_\ell$. These small contribtutions are interesting because they encapsulate the current theory and provide a test bed for our current understanding.  If observations were to deviate from the SM prediction, it would be an indication of new and unaccounted physics interactions with the SM leptons. The $g$ factor can be extracted by measuring the anomalous magnetic moment of any generation of charged lepton.  The current best candidate to both test the SM and search for new physics is by measuring $(g-2)_\mu$ or $a_\mu$ because of  experimental precision potential. The electron measurement is already known to the highest precision and is expected to have the smallest contributions from new physics (cite youtube citation). The $(g-2)_\tau$ is not yet experimentally tractable competitive precision to $\mu$ or $e$.%so $(g-2)_\mu$ has been measured at both at Brookhaven National Lab (BNL) and again at Fermi National Accelerator Laboratory (FNAL).
%The current best SM prediction of $a_\mu$ from (CITE g-2 collab) includes QED, Electroweak(EW) and Hadronic contributions and is reported as $a_\mu^{SM} = a_\mu^{QED}+ a_\mu^{EW}+a_\mu^{\text{Hadronic}} = 116 591 810(43) \times 10^{-11}$. For each of the $a_\mu$ components, the QED compenent enters at the $O(10^{-3})$ and is known to $O(10^{-11})$. the EW component enters the sum at $O(10^{-9})$ and is known to $O(10^{-10})$. Finally the most complicated component, hadronic, contributes at $O(10^{-8})$ and is known up to $O(10^{-9})$, the  main sub components that contribute to the $a_\mu^{\text{Hadronic}}$ is the Hadronic vacuum polarization and light by light scattering, diagrams illustrated in Figure X. The hadronic precision is constrained by data driven measurements and computation approaches -- QCD lattice theory, this error dominates the overall uncertainty of $a_\mu$. The BNL measurement of $a_\mu$ yields a difference with the SM prediction of $\Delta a_\mu := a_\mu^{BNL} - a_\mu^{SM} = 279(76) \times 10^{-11}$ which is a significance of $3.7\sigma$. The most recent $a_\mu$ measurement from FNAL confirms the BNL measurement within $1\sigma$ and the combined experimental average increases the SM deviation with a significance of $4.2\sigma$.

%What could this deviation mean?
The $4.2\sigma$  is a compelling sign for new physics, but not a smoking gun. It is possible to reduced or eliminate the discrepancy by improving the calculations of the HVP and LBL contributions. New and updated calculations are being performed attempting to resolve the discrepancy by a few $\sigma$, but do not yet fully resolve the differences between observations and theory. If computational improvements can't bring the theory into focus, new particles would introduce  quantum corrections bringing experiment and theory into agreement. Several models qualify and successfully explain the $a_\mu$ SM deviation, one being SUSY, where for example, contributes additional diagrams via the smuon-muon coupling illustrated in Figure \ref{fig:gm2susy}.

\FigOneScale{Motivation_figs/g2_susy_loop.png}{Example muon diagrams which include sparticle loops that would contribute to $(g-2)_\mu$\cite{SvenTalkgm2}}{fig:gm2susy}{0.9}

%g-2 is an experiment designed to measure the anomolaus magnetic dipole moment of the muon. The spin magnetic moment of a charged, spin-1/2 particle that does not possess any internal structure (a Dirac particle) is given by (wiki direct quote \url{https://en.wikipedia.org/wiki/G-factor_(physics)}) ${\displaystyle {\boldsymbol {\mu }}=g{e \over 2m}\mathbf {S} }$. where g is the particles g-factor, $\mu$ is the magetic moment, $m$ is the particle mass and $S$ is the spin. The g-factor in quantum electrodynamics is close to 2 so typically the reported measurement is the difference from 2 or g-2 or as a signficance $a_\mu = g-2/2$. The difference from 2 arises from higher order contributions in quantum field theory

\section{The W boson mass}
%the most recent w mass measurement yielded a heavy W, this higher mass is more favorable for light higgsino and compressed susy models

%What is the W boson
The W boson is an important and peculiar particle, it is the electrically charged boson and couples only with left handed particles. The decay modes follow two channels: (1) the hadronic mode with different flavor quark pairs and (2) the leptonic mode with a charged lepton and neutrino. Measuring the W mass directly is challenging at the LHC due to either high levels of QCD di-jet background or missing energy from the neutrino. The mass parameter itself, $m_W$, underpins many important parameters in the SM as well. In fact, $m_W$ is related to the Higgs vev, implying that coupling of the Higgs field to all particles is effectively tuned by $m_W$. Similarly, $m_W$ is related to the $g$ factor from $(g-2)_\mu$, so, both $m_W$ and $g$ can be used to constrain new physics. The W mass can be parameterized at tree level in terms the fine structure constant $\alpha$, the Fermi constant $G_\mu$ and the Z-boson mass $m_Z$, with higher order radiative corrections coming from $\Delta r$ shown in equation \ref{eq:mwequation} \cite{Awramik:2003rn}.
\begin{equation}
\label{eq:mwequation}
m_W^2 = m_Z^2\Bigg(\frac{1}{2} + \sqrt{\frac{1}{4} - \frac{\pi\alpha}{\sqrt{2}G_\mu m_{Z}^2 }(1+\Delta r) } \Bigg)
\end{equation}

%What is the current status of the W boson?
There is no exact SM prediction of the W mass, but, since there is an interdependence of many parameters such as $v$, $m_z$, $G_\mu$,$\alpha$ , the SM prediction is constrained by experimentally well measured parameters. The most recent measurement of $m_W$ was performed by CDF II at the Tevatron where $m_W$ was obtained by fitting the kinematic distributions of light leptonic decays recoiling against a system of jets. This measurement is $50\%$ more precise than the previous measurement by ATLAS and heavier than the SM prediction. The combination of a large deviation with very small error bars results in a significance of $7\sigma$ \cite{CDF:2022hxs}.  

%What could this deviation mean?
If follow up experiments confirm the excess in the W mass, it is a definite sign of new physics. The new physics would express itself as new particles in the radiative corrections via equation \ref{eq:mwequation}. Numerous SUSY models can explain the excessive mass of the W boson, but in general, a slightly heavier W favors light SUSY models, potentially at the electroweak scale, illustrated in Figure \ref{fig:cdfw}. A light SUSY also implies light Higgsinos which favor compressed scenarios.  To illustrate the SUSY capability to satisfy both heavy $m_W$ and deviations in $(g-2)_\mu$, an abundance of model points are shown in Figure \ref{fig:gm2mw}.
% Due to the interdependence of $m_W$ and $(g-2)_\mu$, these parameters both constrain compressed SUSY and spotlight a critical area to search. An example of model points of sleptons and gaug-ino models which satisfy the newest $g-2$ and W mass constraints in shown in FIGURE Z (cite Wmass and g-2 sven paper)
\FigOneScale{Motivation_figs/wmass_cdf.png}{The mass of the W boson as a function of top quark mass which displays the CDF II measurement with 68\% C.I. compared to the SM prediction and LEP2/Tevatron measurement with 68\% C.I \cite{CDF:2022hxs}}{fig:cdfw}{0.6}
\FigOneScale{Motivation_figs/gm2_mw.png}{Illustration of various color coded model points which correspond to different scenarios and -ino model types that favor the measured $(g-2)_\mu$ and W mass that exceeds the SM prediction. The grey band represents the SM prediction, the green line is combines all the experimental results for the W mass, and the blue line indicates the combined $(g-2)_\mu$ experimental results \cite{Bagnaschi:2022qhb}}{fig:gm2mw}{0.7}
\section{The current status of SUSY}
%drop the most recent limits here, start with multi TeV excluded gluino and squark models which leaves the a good place to search in the weak scale sector with electoweakinos. Talk about electroweak limits and how alot of these are excluded already one of the remaining places to search is the compressed corridor where mass splittings are small. link this limit motivation with how both g-2 and W mass favor compressed scenarios

There have been many searches for SUSY particles, starting from  LEP and still ongoing at the LHC. There is not yet observed evidence of SUSY, but, there also is not enough lack of observation to fully reject the SUSY hypothesis. The most comprehensively searched region SUSY space is related to strong production of SUSY particles. The large expected cross sections for squarks and gluinos compared to the inos or sleptons sectors offer the most low hanging fruit for potential discovery.
 %ut there is still plenty of room to keep searching. 
%The strong production of SUSY has the largest excluding limits due to the large expected cross sections compared to other sectors gauginos/sleptons.
Simplified model searches in ATLAS and CMS have excluded $\tilde{g}$ and $\tilde{q}$ (not including $\tilde{t}$) up to around 2 TeV with the most recent limits are shown in Figure \ref{fig:run2stronglim} and \ref{fig:run2squarklim} with their sister exclusions of stop squarks around 1 TeV shown in \ref{fig:stoplim}. The area inside the limit lines in each figure indicates that the 2-D mass points of the sparticle and LSP pair are ruled at a 95\% confidence level for the associated simplified model.
\FigTwoScale{Motivation_figs/gluinoPair_lim.png}{Motivation_figs/atlas_gluinopair_lim.png}{The CMS (left) and ATLAS (right) gluino limits from multiple competing channels and models \cite{cmslims}\cite{atlaslims}}{fig:run2stronglim}{0.48}{0.51}

\FigTwoScale{Motivation_figs/squarkPair_lim.png}{Motivation_figs/altas_squark_lim.png}{The CMS (left) and ATLAS (right) squark limits from multiple competing channels and models \cite{cmslims}\cite{atlaslims}}{fig:run2squarklim}{0.48}{0.51}

\FigTwoScale{Motivation_figs/cms_stop_limit.png}{Motivation_figs/Atlas_stop_limit.png}{The CMS (top) and ATLAS (bottom) limits on stop production \cite{cmslims}\cite{atlaslims}}{fig:stoplim}{0.73}{0.85}

The CMS and ATLAS electroweak limits are shown in Figure \ref{fig:ewlims}. Note that the electroweak limits have sufficient data to only reach the TeV scale while SUSY remains valid at the few TeV. This leaves significant room in the 2-D mass plane to either discover or exclude SUSY by adding more data. One particular simplified model which is nearly unaddressed by both CMS and ATLAS is chargino pair production associated with final states with two oppositely charged $W$ bosons.  Most simplified models assume a mass degeneracy with $m_{\tilde{\chi}^0_2} = m_{\tilde{\chi}^\pm_1}$ but, there is no reason to believe that $\tilde{\chi}^0_2$ can not be decoupled from $\tilde{\chi}^\pm_1$. If this were the case, the limits would not apply in excluding charginos of any mass, So, it is important to address specific final state. The slepton limits  for both CMS and ATLAS are shown in Figure \ref{fig:sleplims}. These are generally the weakest limits of all the aforementioned processes, but, potentially the most important in association with $(g-2)_\mu$.



\FigTwoScale{Motivation_figs/c1n2_lim.png}{Motivation_figs/Atlas_chiwz_lim.png}{The mass limits on chargino pair production with degenerate masses $m_{\tilde{\chi}_2^0}= m_{\tilde{\chi}_1^\pm}$ for CMS (top) and ATLAS (bottom). There are no published limits on compressed chargino pairs in CMS  \cite{cmslims}\cite{atlaslims}}{fig:ewlims}{0.75}{0.8}

\FigTwoScale{Motivation_figs/slep_lim.png}{Motivation_figs/Atlas_slepton_lim.png}{The slepton mass limits for CMS (top) and ATLAS (bottom) which assume the same masses for L and R sleptons and combine smuon and selectron production \cite{cmslims}\cite{atlaslims}}{fig:sleplims}{0.8}{0.83}

For all of the previously presented limits from CMS and ATLAS, excluding gluinos and squarks, a common thread is that the weakest exclusion regions are the compressed regions. The compressed region varies from process to process. For example, compressed relative to stops is such that the mass difference $\Delta m = m_{NLSP} - m_{LSP}$ is less than the top mass $\Delta m < m_t$. Compression with eletroweakinos would have a $\Delta m$  below the W or Z pole. For sleptons decaying directly to leptons, there are no intermediate heavy particles like a W,Z or t, so, the compressed region is more ambiguous and is considered to be ``soft'' interpreted as $O(20-30)$ GeV or less. Several of the compressed scenarios, are unaddressed by CMS, but are complemented by dedicated compressed searches in ATLAS. However, each ATLAS summary result combines different searches with the common feature of large gaps between the results. So, based on the current status of all SUSY results, there is strong motivation to confirm compressed ATLAS results with a CMS compressed search but also extend the current limits and cover gaps between searches.



%\setcounter{secnumdepth}{3}
\setcounter{tocdepth}{3}
\setlength{\parskip}{\smallskipamount}
\setlength{\parindent}{0pt}


\makeatletter


\providecommand{\tabularnewline}{\\}


\makeatother


\chapter{The CMS experiment}

\section{Introduction} The Compact Muon Solenoid (CMS) experiment consists of a detector housed at the Large Hadron Collider (LHC). Two synchronous bunches of high energy protons counter rotate through the LHC accelerator ring and collide at a center point in the CMS detector.  The protons are collided with a significantly large energy with the expectation that more massive and potentially new particles can be produced. The intermediate particles decay or interact and can then measured by the detector, where different regions of the detector specialize in the measurement of specific signatures or features of different particles. The overall p-p collision is then reconstructed or essentially reverse engineered through final state particles interpeted through observable quantities such as energy and momentum.   

Here talk about accelrator concept + hardware, detector concept + hardware, the anatomy of physics events the interpretation of the events through measureed observables, 
we accelerate particles to collided them and produce new particles


\section{The Large Hadron Collider}
The LHC is designed to collide proton beams with a centre-of-mass energy of 14 TeV and an instantaneous luminosity of $1034 \text{cm}^{-2}\text{s}^{-1}$.(cite lhc paper direct quote). The main accelerator ring consists of two counter rotating proton beams which are incased in an ultra high vacuum to prevent interactions. The beams are accelrated with cryogenic magnets which consist of dipole and quadrapole magnets

machine layout and goals: \\
rings magnets, bends beam, quadrapole focuses beam, rf cavities add energy to beam


\section{The CMS Detector}


\section{The Anatomy of a physics event}
	Reconstruction of particles
	observables


%
\setcounter{secnumdepth}{3}
\setcounter{tocdepth}{3}
\setlength{\parskip}{\smallskipamount}
\setlength{\parindent}{0pt}


\makeatletter


\providecommand{\tabularnewline}{\\}


\makeatother

%\usepackage{babel}
%\begin{document}

\chapter{Compressed SUSY Search}

%\begin{chapterabstract}
%This chapter summarizes the approach for a compressed susy search and pertinent sensitive kinematic variables that the analysis is based on.
%\end{chapterabstract}

\section{Compressed Search Introduction}
In accordance with strong experimental and phenomenological motivation for compressed spectra accompanied by the pursuit to comprehensively test SUSY, we conduct a search with a generic approach with many final states that involve missing energy. Some targeted processes include, but are not limited to, stops, electroweakinos, and sleptons with diagrams included in Figure X.  The common thread between between all of these processes is a pair produced visible system alongside a massive invisible system. In the case of a compressed scenario, most of the energy available in the system is used by the rest mass of the LSP. These small mass splittings leads to low momemtum visible products that are difficult to reconstruct or are undetectable. In the case of intermediate massive particles, such as W or Z boson, these are forced off-shell so the visible products receive even less momentum, as the available energy goes into the mass of the intermediate particle. In order to identify these type of events we study cases with significant initial-state radiation (ISR). Due to momentum conservation, the ISR system recoils against, or boosts, the sparticle system, leading to high missing transverse momentum which is a tractable experimental signature. With the ISR assisted pair produced topology we categorize and subdivide the visible and invisible system using Recursive Jigsaw Reconstruction (RJR) to approximate various rest frames. Using the various rest frames we compute a basis of kinematic observables that describe features that are unique to compressed scenarios and process independent and use them to discriminate against SM processes.

%mentally all supersymmetry searches tend to exhibit their weakest limits in corridor regions with low mass differ-
%ences; if supersymmetry is to be tested comprehensively this region must also be explored. Phenomenologically,
%the lowest lying states in the electroweakino sector \\
%Given the lack of compelling conclusive evidence against supersymmetry, we conduct a broad search
%with many final states that involve missing energy. This is well motivated and does not rely strongly on theoretical
%assumptions.


%A compressed system is defined by a sparticle such as a neutralino 2 or stop in which the mass difference with this particle and the lightest supersymmetrical particle is small. The mass difference is considered small when the sparticle decays to intermediate standard model particles like W,Z,t such that the intermediate particle is forced off shell. For example the smallest targeted mass splittings can range between 3 to 10 GeV in neutralino 2 to W/Z decays. The intermediate decays will be difficult to detect or separate from standard model backgrounds. To assist in identifying compressed topologies we look for ISR assisted events. The signature of the event then becomes an ISR jet back to back with sparticle system which consists of mostly missing transverse energy from the LSP and soft SM particles.


\section{ RJR Reconstruction}
 The ISR assisted topology involves a collimated invisible and soft visible system together recoiling against another visible system of ISR jets. Each event is organized by imposing a decay tree onto the visible $(V)$ and invisible objects $(I)$ and assigning the visible components to either the ISR or sparticle $(S)$ side of the event. From the initial assignment, the sparticle system is further subdivided into subsystems A and B where each sub system has a visible $(V_{A/B})$ and invisible  $(I_{A/B})$ component. In order to resolve the invisible four momentum of each system both kinematic and combinatoric unknowns need to be estimated. For example, the visible products are indistinguishable, so figure out how to correctly assign them to either the ISR or sparticle subsystem. Similaary we need partition momentum between two invisible systems where only the magnitude of missing momentum is known. But, the combinatoric assignment and the estimated four vectors depend on each other, so, we use a apply a set of rules estimate and simultaneously determine both.  The RJR framework provides a set of rules to organize and evaluate an event. The set of rules used by this analysis are as follows:
\begin{list}
\item[1.] Leptons are always assigned to the S system
\item[2.] Other visible objects can be assigned to either the S or ISR
\item[3.] The kinematics of $I_{A/B}$ and momentum partitioning are determined
\item[4.] The visible S system objects are assigned to $V_A$ or $V_B$
\end{list}



%An isr assisted event is divided into multiple reference frames. The CM frame consists of the particles measured in the lab e.g. the isr jet against the met system. The sparticle frame consists two subsystems A and B. THe sparticles are expected to be pair produced if r parity is conserved.

%The kinematic variables the form basis of the search is RISR and MPERP.
%Risr. RISR is process independent and peaks at the ratio of sparticle/lsp masses. mperp is the transverse mass of the sparticle frame with respect to the sparticle frame boost axis.  
\section{}

\section{Signal Models}
%The analysis is generalized to deal with a broad range of signal models but the three targeted compressed signal processes incldue stop, neutralino/chargino, slepton production. the signals include T2tt, T2bW, TChiWZ, TChiWW, TSlepSlep


%
\setcounter{secnumdepth}{3}
\setcounter{tocdepth}{3}
\setlength{\parskip}{\smallskipamount}
\setlength{\parindent}{0pt}


\makeatletter


\providecommand{\tabularnewline}{\\}


\makeatother

%\usepackage{babel}
%\begin{document}

\chapter{Analysis Description }

\section{Introduction and Strategy}
The full analysis is built on the compressed kinematics and RJR strategy described in the previous chapter. The goal of this chapter is to build on this strategy with specific details about the objects and categorization used to potentially discover SUSY. This includes the description of events selected to analyze and their associated physics objects. This being a general search, it casts a wide net to capture a variety of signatures and final states with the consequence being a large number of categories and bins. Finally, I will discuss the data driven strategy to constrain and predict background events in the most sensitive regions with a series of fits to construct and test a fit model.

 
\section{Event Selection and Physics Objects}
 For events to be qualified for analysis, they pass a handful of selected triggers and a preselection that reflects the compressed kinematic description provided in the previous chapter. The triggers used PFMET and PFMHT cross triggers. PFMET is particle flow missing transverse energy which is expected to capture the $p_T^{\text{miss}}$ from the LSP. The threshold for the PFMET trigger is that the missing transverse energy is above 120 GeV. The definition of missing momentum is the negative sum of all visible momentum and is as follows: 
\begin{equation}
p_T^{\text{miss}} = -\sum{p_T^{\text{vis}}}
\end{equation} 
PFMHT is expected to trigger on events with significant jet activity which are likely candidates for ISR events. The threshold for PFMHT is that the scalar sum of all visible transverse momentum is above 120 GeV.  In general, the compressed SUSY topology is a somewhat rare organization of an event, due to the uncommon nature of these events, the MC modeling does not always sufficiently and precisely describe data, so,  a data driven approach is utilized for physics objects to compensate for any disagreement. This approach compares the overall efficiency or behavior of selected objects and computes data driven scale factors, while also providing a platform to model and understand systematic effects. These scale factor calibrations are then applied to MC to bring data and MC into agreement. The instance of scale factor generation arises in the modeling of the trigger efficiency, with efficiency defined as events that pass the trigger and preselection versus only preselection. The comparisons of data and MC missing $E_T$ efficiency are shown in Figure \ref{fig:metsf} with the efficiency shapes modeled by a Gaussian CDF.  

%insert trigger turn ons
\FigOneScale{Analysis_figs/SF_Plot_HT-Le600--SingleMuontrigger-E1--Nmu-E1_SingleMuon_2016.pdf}{Example MET trigger efficiency comparison for Single Muon triggers between data and MC  \cite{AN}.}{fig:metsf}{0.75}

The preselection is a set of carefully studied criteria to remove background and mismodeled events while selecting events with objects associated with signals. The preselection consists of the criteria listed in Table \ref{tab:presel} and introduces three quantities $p_T^{\text{CM}}$ which is the vector sum of the transverse momentum of the CM frame, $\Delta \phi_{\vec{p}_T^{\text{miss}}, V}$ the angle between the visible and invisible system, and $\Delta\phi_{CM,I}$ the angle between $p_T^{\text{CM}}$ and the invisible system.

%insert preselection table  
\begin{table}
\caption{Kinematic and combinatorial event requirements.}
\begin{tabular}{c|c}
\hline 
\multicolumn{2}{|c|}{Preselection Requirements} \\ 
\hline 
Criteria & Description \\ 
\hline 
\hline
$N_V \geq 1$ & At least one visible object assigned to the S system \\ 
$N_j^{ISR} \geq 1$ & At least one jet assigned to the ISR system \\ 

$p_T^{\text{miss}} > 150$ GeV &\makecell{ Minimum transverse missing energy based on trigger efficiency}  \\ 

$p_T^{ISR} > 250 $ GeV & \makecell{Minimum ISR kick to resolve massive invisible particles} \\ 

$R_{ISR}$ > 0.5 & Target Massive LSPs \\ 

$|\Delta \phi_{\vec{p}_T^{miss}, V}| < \pi/2$ &  Ensures visible and invisible system are traveling in the same direction \\ 

$p_T^{CM} < 200$ GeV  & Rejects mismodeled events \\ 

veto $f(\Delta\phi_{CM,I}, p_T^{CM})$& 2D function to also reject mismodeled events - See Fig \ref{fig:cleancut}\\
\hline 
\end{tabular} \\
\label{tab:presel}
\end{table}

\FigOneScale{Analysis_figs/cleaningCuts.pdf}{Cleaning cuts designed to veto events in the red regions with significantly large data to MC $R_{ISR}$ ratios. The accepted region corresponds to events passing the last two criteria of Table \ref{tab:presel} \cite{AN}.}{fig:cleancut}{0.75}

The preselection forms the basis for an event to be analyzed. Following preselection, the physics objects can be selected, classified, and categorized. The possible object composition can consist of jets, b-tagged jets, soft secondary vertices (SVs), and leptons. The discussion of the leptons and their classification will be reserved for the following chapter alongside the calculation of lepton scale factors. A summary of these physics objects and their kinematic requirements are listed in Table \ref{tab:physicsobjects}. The AK4 jets are clustered with the anti-$k_t$ algorithm and $\Delta R = 0.4$ \cite{Cacciari:2008gp} and selected based on working points (WP) defined by their physics object group which are standard objects used in CMS physics analysis \cite{CMS:2010xta}. The b-tagging is done by the MVA based DeepJet NN \cite{Stoye:2018qgr} which only identifies b-jets $p_T \geq 20$ GeV. SVs above 20 GeV are tagged using Inclusive Secondary Vertex Finder \cite{CMS:2011yuk}. A complementary SV tagger was developed which efficiently extends the SV tagging range $2 \leq p_T \leq 20$ GeV for final state topologies with very soft b-jets \cite{erich}.


\begin{table}
\centering
\caption{Kinematic requirements and working points for visible physics objects.}
\begin{tabular}{c|c}
\hline 
\multicolumn{2}{c}{Visible Physics Objects} \\ 
\hline 
\hline
Jets & \makecell{AK4 PF Jets \\ Tight ID \\ $p_T^{jet} > 20$ GeV \\ $|\eta| <2.4$} \\ 
\hline
B-tagged Jets & \makecell{AK4 PF Jets \\ DeepJet Medium WP \\ $p_T^{b-jet} > 20 $ GeV}  \\ 
\hline
SVs & $2<p_T^{SV}<20$ GeV \\ 
\hline
Leptons & \makecell{Very Loose ID \\ $p_T^{\mu^\pm} > 3$ GeV \\ $p_T^{e^\pm} > 5 $ GeV \\ Gold/Silver/Bronze quality classes} \\ 
\hline 
\end{tabular} 
\label{tab:physicsobjects}
\end{table}

\section{Categorization}

Once an event passes the preselection and the leptons are classified, the event is then categorized based on object composition, object multiplicity, and kinematic characteristics. The goal of categorization is to finely split up background and create generically sensitive regions for any type of signal. The regions with no signal function as control regions and constrain the background prediction in the sensitive regions. The most fundamental categories for the analysis are the lepton and jet multiplicities, events can contain either 0, 1, 2, or 3 leptons and are accompanied by a range of sparticle system jets (S jets) from $0\leq N_{S_{jet}} \leq 5$ where the upper limit of S jet counting is dependent on the lepton multiplicity. The type of jets are also counted i.e. whether or not the jets have been tagged as b-jets. The b-jet counting is dependent on the lepton multiplicity and restricted by the maximum number of S jets per jet multiplicity category.  ISR system can have jets can have 0 or $\geq 1$ b-jets and the S system can be composed of 0, 1, or $\geq2$ b-jets. SV tagging is similar to ISR jet counting in the S system, but, SVs are not counted in the ISR system. Events with SVs are further classified based on their orientation being either central or forward with respect to $|\eta|=1.5$ with the forward range at of maximum of $|\eta|<2.4$. The counting of b-jets and SVs is very important because it creates categories that isolate tt+jets and stop signals and similarly the opposite regions are created with no b-jets that isolate backgrounds like W+jets and signals such as TChiWZ. The benefit of having different object counting regions is that they can cross-constrain each other and is illustrated in Figure \ref{fig:bcount}, showing the presence and absence of tt+jets or W+jets backgrounds based on S system b-jet counting.
\FigTwo{Analysis_figs/NbS_v_NbISR_0L_ttbar-gif-converted-to.pdf}{Analysis_figs/NbS_v_NbISR_1L_Wjets-gif-converted-to.pdf}{Example of relative background presence for tt+jets (left) and W+jets (right) with b-tag counting.}{fig:bcount}

Aside from jet counting, there is lepton and kinematic categorization. The electrons and muons are separated by flavor, charge, quality, and $Z/\text{no} Z$ candidate.  A $Z$ candidate is defined as an opposite sign, same flavor pair (OSSF) in events with 0 S jets and OSSF in the same A or B hemisphere in events with S jets. A compressed scenario expects an off-shell $Z$ so there is no mass requirement on the OSSF lepton pair in a $Z$ category. The complementary compressed kinematic observables $p_T^{ISR}$ and $\gamma_\perp$ both have a high and low category. A complete table of all the categories is included in Table \ref{tab:cats}. 

\begin{table}
\caption{Organization of categories across all lepton multiplicities. The bracketed jet ranges imply counting for all integer numbers of jets within the inclusively listed edges. The b-tag counting is limited based on the allowed number of jets in the category. The SV and $\gamma_\perp$ splitting indicates two categories of either forward and central $\eta$ or low and high, respectively. The numbers listed in $p_T^{ISR}$ define the low $p_T^{ISR}$ category edges while the high $p_T^{ISR}$ bin is inclusive for everything above the low bin upper edge. The checkmarks indicate regions that are split by either $\gamma_\perp$ or $SV_\eta$. }
\begin{adjustwidth}{-0.7in}{-1in}

\begin{tabular}{|c|c|c|c|c|c|c|c|c|c|}
\multicolumn{7}{c|}{Obj. Combinatorics} & \multicolumn{3}{c}{Obj. Kinematics} \\
\hline 
$N_\ell$ &  $\ell$ Type & $\ell$ Quality & $N_{jets}^{S}$ & $N_{b-tag}^{S}$ & $N_{SV}^S$ & $N_{b-tag}^{ISR}$ & $SV_\eta$ & $\gamma_\perp$ & $p_T^{ISR}$ \\ 
\hline
\hline 
0 & - & - & 0 & 0 & $[1,\geq 2]$ & - & $\checkmark$ & - & $\geq 350$ \\ 
0 & - & - & 1 & - & $\geq 1$ &    - & $\checkmark$ & - & $\geq 400$ \\
0 & - & - & 1 & - &  0       &    - &  -           & - & $[400,\geq 550]$ \\
0 & - & - & $[2,\geq 5]$ & $[0,\geq 2]$ & 0 & $[0,\geq 1]$ & - & $\checkmark$ & $[350,\geq500]$ \\
\hline 
1 & $e^+,e^-,\mu^+,\mu^-$& Gold & 0 & 0 & 0 & $[0,\geq 1]$ & - & $\checkmark$ & $[350, \geq 500]$ \\
1 & $\ell^+, \ell^-$  & Gold & 0 & 0 & 1 & - & $\checkmark$ & - & $\geq 350$ \\
1 & $e, \mu$ & Gold & 1 & 0 &  $\geq 1$ & - & $\checkmark$ & - & $\geq 350$ \\
1 & $\ell$ & Gold & $[1,\geq 4]$ & $[0,\geq 2]$ & 0 & $[0,\geq 1]$ & - & $\checkmark$ & $[350, \geq 500]$ \\
1 & $e, \mu$ & Silver/Bronze & $[0,\geq 1]$ & 0 & 1 & - & $\checkmark$ & - & $\geq 350$ \\
1 & $e, \mu$ & Silver/Bronze & $[2, \geq 4]$ & - & - & - & - & $\checkmark$ & $[350, \geq 500]$ \\
\hline
2 & $e^\pm e^\mp, \mu^\pm \mu^\mp, e^\pm \mu^\mp $ & Gold & 0 & 0 & 0 & $[0,\geq 1]$ & - & $\checkmark$ & $[250,\geq350]$  \\
2 & $Z, no Z$ & Gold & $[1,\geq 2]$ & $[0, \geq 1]$& - & $[0,\geq 1]$ & - & $\checkmark$ & $[250,\geq350]$  \\
2 & $\ell^\pm \ell^\pm$ & Gold & $[0,\geq 2]$ & - & - &  - & - & - & $[250,\geq350]$ \\
2 & $ee, \mu\mu, e\mu$ & Silver/Bronze & $[0,\geq 2]$ & - & - & - & - & - & $\geq350$ \\
2 & $\ell \ell$ & Gold/Silver/Bronze & 0 & 0 & $\geq 1 $ & - & $\checkmark$ & - & $\geq 250$\\
\hline
3 & $Z, no Z$ & Gold/Silver/Bronze & $[0,\geq 1]$ & - & - & - & - & - & $\geq 250$ \\
3 & $\ell^\pm \ell^\pm \ell^\pm$ & Gold/Silver/Bronze & - & - & -& - & - & -& $\geq 250$ \\
\hline
\end{tabular} 
\end{adjustwidth}
\label{tab:cats}
\end{table}



Each category is divided into a 2D set of $(R_{ISR}, M_\perp)$ bins. The number of bins and bin edges are optimized for each combination of lepton and jet multiplicity. An example binning of the sensitive variables with the total SM background in each lepton multiplicity is shown in Figure \ref{fig:kinbin} and a full table of all binnings is included in Table \ref{tab:allkinbin}.
\FigFour{Analysis_figs/0L_2J_totalBkg.png}{Analysis_figs/1L_G_1J_1sv_totalBkg.png}{Analysis_figs/2L_G_0J_totalBkg.png}{Analysis_figs/3L_B_1J_totalBkg.png}{The dotted lines show example 2D $R_{ISR}$ and $M_\perp$ binning for each lepton multiplicity with the full SM MC background. The bin edges for each figure are based on the contour of the total background and the number of edges reflect the amount of background \cite{AN}. }{fig:kinbin}

% insert RISR MP bins table here
\begin{table}
\caption{The complete set of $R_{ISR}$ and $M_\perp$ bin edges. A set of 2D bins have been designed for each allowed lepton and jet multiplicity.}
\begin{adjustwidth}{-0.5in}{-1in}
%\begin{singlespace}
\begin{tabular}{c|cc|cc|cc|cc|}
- & \multicolumn{2}{c}{0L} & \multicolumn{2}{c}{1L} & \multicolumn{2}{c}{2L} & \multicolumn{2}{c}{3L}   \\
\hline
$N_{jets}^S$ & $R_{ISR}$ & $M_\perp$ & $R_{ISR}$ & $M_\perp$ & $R_{ISR}$ & $M_\perp$ & $R_{ISR}$ & $M_\perp$ \\
\hline
0 & [0.95,0.985] & [$\geq 0$]  &[0.9,0.96] & [$\geq 0$]  &[0.6,0.7] & [$0,\geq 50$]  &[0.6,0.7] & [$\geq 0$]  \\
0 & [0.985,1] & [$0,5,\geq 10$]  &[0.96,0.98] & [$0,\geq 10$]  & [0.7,0.8]  & [$0,\geq 40$] &[0.7,0.8] & [$\geq 0$]  \\
0 &  &  &[0.98,1] & [$0,5,\geq 10$]  &  [0.8,0.9]  & [$0,\geq 30$] &[0.8,0.9] & [$\geq 0$]  \\
0 &  &   &        &                &[0.9,0.95] &  [$0,\geq 20$]  &[0.9,1] & [$\geq 0$]  \\
0 &   &   &   &   & [0.95,1] & [$0,\geq 15$]  & &   \\
\hline
1 & [0.8,0.9] & [$\geq 0$]  &[0.65,0.75] & [$0,\geq 50$]  &[0.5,0.6] & [$0,\geq 100$]  &[0.55,0.7] & [$\geq 0$]  \\
1 & [0.9,0.93] & [$0,\geq 20$]  &[0.75,0.85] & [$0,\geq 40$]  &[0.6,0.7] & [$0,\geq 80$]  &[0.7,0.85] & [$\geq 0$]  \\
1 & [0.93,0.96] & [$0,\geq 20$]  & [0.85,0.9] & [$0,\geq 30$] &[0.7,0.8]  & [$0,\geq 60$]  &[0.85,1] & [$\geq 0$]  \\
1 & [0.96,1] & $0,\geq 15$]   &[0.9,0.95] & [$0,\geq 20$]   &[0.8,0.9] & [$0,\geq 40$]   & &   \\
1 &  &   &[0.95,1]   & [$\geq 0$]   &[0.9,1]& [$0,\geq 30$]   & &   \\
\hline
2 & [0.65,0.75] & [$0,\geq 50$]  &[0.55,0.65] & [$0,\geq 110$]  &[0.5,0.65] & [$0,\geq 100$]  & &   \\
2 & [0.75,0.85] & [$0,\geq 40$]  &[0.65,0.75] & [$0,\geq 90$]  &[0.65,0.75] & [$0,\geq 80$]  & &   \\
2 & [0.85,0.9] & [$0,\geq 30$]  &[0.75,0.85] & [$0,\geq 70$]  &[0.75,0.85] & [$0,\geq 60$]  & &   \\
2 & [0.9,0.95] & [$0,\geq 20$]  &[0.85,0.9] & [$0,\geq 50$]  &[0.85,1] & [$\geq 0$]  & &   \\
2 & [0.95,1] & [$\geq 0$]  &[0.9,1] & [$\geq 0$]  &  &   &  &   \\
\hline
3 & [0.55,0.65] & [$0,\geq 110$]  &[0.55,0.65] & [$0,\geq 150$]  &[0.5,0.65] & [$0,\geq 130$]  & &   \\
3 & [0.65,0.75] & [$0,\geq 90$]  &[0.65,0.75] & [$0,\geq 100$]  &[0.65,0.8] & [$0,\geq 100$]  & &   \\
3 & [0.75,0.85] & [$0,\geq 70$]  &[0.75,0.85] & [$0,\geq 80$]  &[0.8,1] & [$\geq 0$]  & &   \\
3 & [0.85,0.9] & [$0,\geq 50$]  &[0.85,1] & [$\geq 0$]  &  &   &  &   \\
3 & [0.9,1] & [$\geq 0$]  &  &   &   &   &  &   \\
\hline
4 & [0.55,0.65] & [$0,\geq 150$]  &[0.5,0.6] & [$0,\geq 210$]  &  &   &  &   \\
4 & [0.65,0.75] & [$0,\geq 100$]  &[0.6,0.7] & [$0,\geq 180$]  &  &   &  &   \\
4 & [0.75,0.85] & [$0,\geq 80$]  &[0.7,0.8] & [$0,\geq 150$]  &  &   &  &   \\
4 & [0.85,1] & [$\geq 0$]  &[0.8,1] & [$\geq 0$]  &  &   &  &   \\
\hline
5 & [0.5,0.6] & [$0,\geq 210$]  &  &   &   &   &  &   \\
5 & [0.6,0.7] & [$0,\geq 180$]  &  &   &   &   &  &   \\
5 & [0.7,0.8] & [$0,\geq 150$]  &  &   &   &   &  &   \\
5 & [0.8,1] & [$\geq 0$]  &  &   &   &   &  &   \\
\hline
\end{tabular}
\end{adjustwidth}
\label{tab:allkinbin}
\end{table}

In total, there are 392 categories and 3093 total bins, the number of categories contributing to each lepton multiplicity is 84 categories from 0L, 178 categories from 1L, 115 categories from 2L, and 15 categories from 3L. The optimization of all the numerous bins and categories was a large undertaking and was performed using two statistical metrics: the Cousins Z-binomial method \cite{Cousins_2008} which is a measure of the signal to background significance with a $20\%$ assumption on systematic error, and the median asymptotic limit based on a profile likelihood ratio \cite{Cowan:2010js}, comparing the signal+background versus background only hypotheses but the observed data is taken as the nearest floor rounded integer from the MC model. Both metrics were tested with the presence of multiple signal types being either, stop, slepton, or electroweakinos. The optimization efforts targeted $R_{ISR}$, $M_\perp$, $p_T^{ISR}$, $\gamma_\perp$, the number of bins and bin edges, as well as object counting such as the  maximum allowed number of S jets or counted b-jets per lepton multiplicity. The categories and bins were also consolidated to guarantee non-zero expected background events in every bin. The bin by bin statistics for guaranteeing non-zero expected events was studied with MC separately for each year scaled to 138 fb$^{-1}$ as well as with the full Run II combined MC. An example of this study is the flavor consolidation of two lepton categories with $N_{jets}^S > 0$ from the original implementation with flavor separated categories. This consolidation instance was found to have too few events with electrons which led to pathologically good exclusion limits from the empty bins. Similarly 2L $N_{jets}^S \geq 3$  was reduced to $N_{jets}^S \geq 2$ to boost the statistics of high jet multiplicity events in 2L. Example limits comparing many different optimization tests are illustrated in Figure \ref{fig:opto}.

%1d limit
\FigTwoScale{Analysis_figs/TChiWZ_consol_opto.png}{Analysis_figs/T2bWopto.png}{Example internal sensitivity optimization plots. The y-axis is the median asymptotic limit with smaller values indicating stronger expected exclusion of the signal grid points on the x-axis. The x-axis label conventions are $m_{NLSP}\_m_{LSP}\_(m_{NLSP}-m_{LSP})$. The top distribution shows tests performed on TChiWZ with bin consolidation and jet multiplicity consolidation. The bottom distribution tests b-tag counting and $\gamma_T$ bin optimization on T2tt. }{fig:opto}{0.95}{0.95}


%\newcommand{\ID}{\text{ID}}
\newcommand{\Prompt}{\text{Prompt}}
\newcommand{\Isolated}{\text{Isolated}}
\newcommand{\Gold}{\text{Gold}}
\newcommand{\Silver}{\text{Silver}}
\newcommand{\Bronze}{\text{Bronze}}


\newcommand\FigureOne[3]{%
\begin{figure}[!htbp]%
\centering
\includegraphics[width=0.5\textwidth]{fig/Lep_Obj_plots/#1}\hfill
\caption{#2}
\label{#3}
\end{figure}}


\newcommand\FigureFour[6]{%
\begin{figure}[!htbp]%
\centering
\includegraphics[width=0.5\textwidth]{fig/Lep_Obj_plots/#1}\hfill
\includegraphics[width=0.5\textwidth]{fig/Lep_Obj_plots/#2}\hfill
\includegraphics[width=0.5\textwidth]{fig/Lep_Obj_plots/#3}\hfill
\includegraphics[width=0.5\textwidth]{fig/Lep_Obj_plots/#4}\hfill
\caption{#6}
\label{#5}
\end{figure}}

\newcommand\FigureThree[5]{%
\begin{figure}[!htbp]%
\centering
\includegraphics[width=0.4\textwidth]{fig/Lep_Obj_plots/#1}
\includegraphics[width=0.4\textwidth]{fig/Lep_Obj_plots/#2}\hfill
\includegraphics[width=0.4\textwidth]{fig/Lep_Obj_plots/#3}\hfill
\caption{#4}
\label{#5}
\end{figure}}

\newcommand\FigureTwo[4]{%
\begin{figure}[!htbp]%
\centering
\includegraphics[width=0.5\textwidth]{fig/Lep_Obj_plots/#1}\hfill
\includegraphics[width=0.5\textwidth]{fig/Lep_Obj_plots/#2}\hfill
\caption{#4}
\label{#3}
\end{figure}}

\newcommand\FigureStack[4]{%
\begin{figure}[!htbp]%
\centering
\includegraphics[width=0.75\textwidth]{fig/Lep_Obj_plots/#1}\hfill\\
\includegraphics[width=0.75\textwidth]{fig/Lep_Obj_plots/#2}\hfill
\caption{#4}
\label{#3}
\end{figure}}

\setcounter{secnumdepth}{3}
\setcounter{tocdepth}{3}
\setlength{\parskip}{\smallskipamount}
\setlength{\parindent}{0pt}


\makeatletter


\providecommand{\tabularnewline}{\\}


\makeatother

%\usepackage{babel}
%\begin{document}

\chapter{Lepton Selection and Efficiency Measurement}

%\begin{chapterabstract}
%The Tag-and-Probe is a method used to measure the selection efficiencies of an object using data. In the context of this compressed SUSY %analysis, the Tag-and-probe measures the efficiencies separately of each light lepton($e/\mu$) selection critera. The total lepton selection %efficiency is then computed by combining factorized efficiency components. The same general method is used for both electrons and muons, %however, Muons utilize  the $J/\psi$ di-muon trigger which allow more precise efficiency measurements from data at lower $p_T$.
%\end{chapterabstract}
\section{Lepton Object Definitions}


Electrons and muons are selected according to the minimum requirement ``VeryLoose'' which depends on kinematic and topological quantities in Table \ref{tab:veryloose}. The electrons use an additional loose requirement, the MVA VLooseFO Id \cite{elecMVA}. The set of VeryLoose leptons are further subdivided by quality into three mutually exclusive categories: Gold, Silver, and Bronze. Each category has a measure of three main quantities, the first is the quality of the pre-determined Id. The Id's differ per flavor and are the standard working points defined by the corresponding physics object group.  Id's range from Loose to Medium to Tight and gain stricter requirements for each tighter requirement. The Loose Id has the highest selection efficiency and misidentification rate while the Tight Id is the least efficient and has the lowest misidentification rate. The muons use the Medium Id \cite{muMediumId} and electrons use a more strict selection with the Tight Id \cite{eTightID}. The second quantity is the ``promptness'' or distance of the lepton production point from the primary vertex. Promptness is measured by the significance of the 3D impact parameter (SIP3D) defined as the impact parameter normalized by its measured error. The impact parameter is the distance of the track's point of closest approach to the primary vertex. There are three relevant impact parameters, $d_{xy}$, $d_{z}$, and $IP_{3D}$. The first two parameters, $d_{xy}$ and $d_{z}$ are the distance in the $x-y$ and $r-z$ planes, respectively. $IP_{3D}$ is the three dimensional distance from the primary vertex to track's point of closest approach.  All three impact parameters are signed based on the same convention. For a track with direction $\hat{t}$ and distance vector $\hat{d}$ directed from the primary vertex to point of closes approach, the sign of the impact parameter follows $\hat{t}\cdot\hat{d}$.  The last component is isolation, a measure of the density of particles in a cone around the lepton. Two similar but complimentary absolute isolations are used: PFIso \cite{murun2baseline} and MiniIso \cite{miniIso}. Both isolations are an energy sum of neighboring particles inside a cone, but, PFIso has a fixed cone size of $R=0.4$ cm. PFIso is defined in Equation \ref{pfiso} where $p_{T,\text{ch. had} }^{PV}$ is the transverse momentum of tracks associated with the primary vertex that are inside the cone, $E_{T,\text{neut. had}}$ is the transverse energy from energy deposits not associated with tracks that are inside the cone, and $p_{T, \text{ch. had}}^{PU}$ is the transverse momentum of tracks not associated with the primary vertex that are inside the cone. Unlike PFIso, MiniIso uses the same cone-based particle summation strategy, but the cone sizes varies inversely with lepton \pt as shown in Equation \ref{isoeq}.

\begin{equation}\label{pfiso}
\sum p_{T,\text{ch. had} }^{PV} + \text{max}(0, \sum E_{T,\text{neut. had}}) - 0.5*\sum p_{T, \text{ch. had}}^{PU}
\end{equation}

\begin{equation}
R_{\text{miniIso}}=
    \begin{cases}
      0.2 & \pt^\ell < 50 \text{GeV}\\
      \frac{10}{\pt^\ell} & 50 \text{GeV} \leq \pt^\ell \leq 200 \text{GeV} \\
      0.05 & \pt^\ell > 200 \text{GeV}
    \end{cases}
    \label{isoeq}
\end{equation}


%Mini isolation also includes effective area pile-up corrections provided in a look up table of bins of \pt and $\eta$ in the CMSSW Producer/Ntuplizing stage. The implementation of mini-isolation and their corrections utilize the same IsoValueMap producer as used in NANO AOD as of \url{CMSSW_10_6_X}.


The explicit flavor independent formulas for Gold, Silver, and Bronze can be generalized by the product of three components which are the measured efficiencies of the three previously mentioned quantities. The efficiencies take the form of conditional probabilities to be measured independently in sequence relative to each other:
\begin{equation}\label{eq:efflep_general}
\begin{split}
\epsilon_{\Gold}& = \epsilon_{\ID}\times \epsilon_{\Isolated|\ID} \times \epsilon_{\Prompt|(\ID \cap \Isolated)} \\
\epsilon_{\Silver}& = \epsilon_{\ID} \times \epsilon_{\Isolated|\ID} \times (1-\epsilon_{\Prompt|(\ID \cap \Isolated)}) \\
\epsilon_{\Bronze}& = 1-(\epsilon_{\ID} \times \epsilon_{\Isolated|\ID)} )
\end{split}
\end{equation}

The subscript for an efficiency, e.g. $\epsilon_{\Prompt|(\ID \cap \Isolated)}$, reads as the efficiency to pass the SIP3D requirement given the lepton passes the Id and Isolation requirements. From equation \ref{eq:efflep_general} the Gold, Silver, and Bronze efficiencies can be read off as Gold passes all criteria, Silver fails only the SIP3D requirement, and Bronze fails either the Id or isolation and is agnostic to SIP3D. While isolation and vertexing requirements are physically uncorrelated, there is an intersection between the two, meaning a lepton can be both prompt and isolated. This intersection then demands the necessity for conditional efficiencies.  The order of the conditional efficiencies is also chosen to minimize the number of measured efficiencies by reusing efficiencies across Gold, Silver, and Bronze.  





\begin{table}[htbp]
\centering
\caption{\label{tab:veryloose} The criteria that define the minimum requirements for an accepted lepton. The electron and muon requirements are equivalent in terms of pseudo-rapidity, vertexing, and isolation but vary in \pt threshold and the MVA VLooseFO working point. The MVA VLooseFO ID also varies between years. The parameters $d_{xy}$ and $d_{z}$ are the transverse and longitudinal impact parameters, respectively.}

\begin{tabular}{c|c|c}
\hline
Criteria & Electron & Muon \\
\hline
\hline
\pt & $\geq 5$ GeV & $\geq 3$ GeV \\

$|\eta|$ & $<2.4$ & $<2.4$ \\
\hline

$\text{IP}_{3D}/\sigma_{\text{IP}_{3D}}$ & $<8$ & $<8$ \\

$|d_{xy}|$ & $<0.05$ cm & $<0.05$ cm \\

$|d_z|$ & $<0.1$ cm & $<0.1$ cm \\

\hline
$\text{PFIso}_{\text{abs}}$ & $<20 + (300/\pt)$ GeV & $<20 + (300/\pt)$ GeV \\

\hline
MVA VLooseFO ID & \checkmark  & --\\
\end{tabular}
\end{table}


The advantage of having multiple lepton quality categories allows for robust sensitivity to a wide range of signal processes. This strategy boosts the overall statistics and provides control regions for multiple scenarios. %but also provide fake rich selection in control region that helps stabilize the overall fit and extract fake rates into the sensitive regions.  
The populations of different truth matched objects per quality are shown in Figure \ref{andresPurity} and the MC efficiency for Gold, Silver, and Bronze with truth matched objects is shown in Figure \ref{andresEff}.  The Gold region is mainly populated by prompt and isolated leptons that are produced within the primary vertex. This region also coincides with the signature of many targeted electroweakino models. The Silver selection accommodates both leptonically decaying taus, providing an ideal region for stau's, and assists in recovering efficiency of isolated b decays. The Bronze selection is rich in fake leptons and provides the best regions to extract fake rates and anchor the fit through a surplus of events. 



\FigThreeScale{Lep_Obj_plots/gold_ele_TTJets_Fall17.pdf}{Lep_Obj_plots/silver_ele_TTJets_Fall17.pdf}{Lep_Obj_plots/bronze_ele_TTJets_Fall17.pdf}{Gold (Top-Left), Silver (Top-Right) and Bronze (Bottom) MC truth matching in $t\bar{t}$ sample for 2017. Each figure demonstrates the MC truth composition of each electron quality where a reconstructed electron is $\Delta R$ matched to a either a signal electron from a prompt $W$ boson decay, a leptonic tau decay, b-quark jet, c-quark jet, or light quark jet \cite{AN}.}{andresPurity}{0.49}{0.49}{0.49}


\FigTwoScale{Lep_Obj_plots/sigEff_ele_TTJets_Fall17.pdf}{Lep_Obj_plots/sigEff_mu_TTJets_Fall17.pdf}{Gold, Silver, and Bronze efficiency on truth matched prompt signal leptons from $W$ bosons in tt+jets. The fake efficiencies are also shown for each lepton quality with leptons truth matched to any other source \cite{AN}.}{andresEff}{0.49}{0.49}


\FloatBarrier

\section{Tag-and-Probe Methodology}
An important element of a lepton based search is properly modeling the efficiency of selected leptons. A purely Monte-Carlo driven approach is inadequate in perfectly describing nuances in data due to imperfections in modeling. Instead of trying to model exactly all physics and detector effects with simulation, the efficiencies can be directly measured from data by using the Tag-and-Probe method. Then the efficiency ratio between data and MC can be used to multiply the MC as calibration to match data. 

The Tag-and-Probe method is used to measure lepton selection criteria by using a well known resonance such as a $Z$, $J/\psi$, or $\Upsilon$ and counting the number of probes that pass that criteria. Each counted instance of the Tag-and-Probe consists of two selected leptons. One of the selected leptons is the tag and the other is the probe.  The tag passes tight selection requirement to give high confidence that it isn't a fake lepton. Fake leptons fall into two possible categories: reducible and irreducible. A reducible fake lepton is a particle that fakes the signature of a lepton such as a charged pion. An irreducible fake lepton is an actual lepton which coincidentally passes some selection criteria but is not the targeted leptons of interest e.g. an isolated muon from a jet accompanying a leptonic Z decay of interest.  The  second lepton in the Tag-and-Probe is the probe. The probe is subjected to the selection criteria whose efficiency is being measured. The invariant mass of the pair of leptons is calculated and required to fall within a defined range around the resonance. A particular event may have multiple lepton pairs but the tag and the probe are not allowed to switch positions and be counted twice, as double counting would lead to a bias in the efficiency measurement \cite{AN111-2009}. To avoid bias, the tag and probe are required to be the opposite charge and same flavor where the tag is randomly selected. If multiple same flavor lepton pairs occur in single event i.e. there are multiple probes to a single tag, the treatment for selecting the pairs differs between electrons and muons. There is no specific study which led to justifying the differing arbitration approaches in flavors, only that the choice reflects the default choices implemented in the existing code bases.  For muons, no arbitration is used, all pairs are utilized which means an additional pair not truly from the resonance will then contribute as combinitorial background in a single event. For electrons, only a single probe is selected per event which has the highest \pt. The selected probes can either pass or fail their selection which leads to the formation of three distributions, one with a passing probe, one with a failing probe, and one with all probes. An example of all three distributions is shown in Figure \ref{tnpexp}.  The probability of observing $k$ passing probes in $n$ Tag-and-Probe pair trials is dependent on the selection efficiency $\varepsilon$ and can be expressed as a likelihood from the binomial probability density $P(k|\varepsilon,n) = \binom{n}{k}\varepsilon^k(1-\varepsilon)^{n-k}$. The MLE estimator for efficiency is then the fraction of passing probes to the total number of pairs, or $\varepsilon = k/n$. Technical documentation for the Tag-and-Probe in CMS is scarce, but, an early strategy for fitting efficiency is defined in \cite{Berryhill_2010}. The legacy code base as of \url{CMSSW_10_6_X}  uses a binned maximum likelihood between the observed passing probes and failing probes where the efficiency extracted is an explicit fit parameter. The two simultaneously fit functions are:
\begin{equation}
	N^{\text{Pass}} = N_{\text{Total}} (\varepsilon \cdot f^{\text{sig}}_{\text{All}} ) +  \varepsilon_{\text{bkg}} \cdot (1-f^{\text{sig} }_{\text{All}}) )
\end{equation} 
\begin{equation}
	N^{\text{Fail}} = N_{\text{Total}} ( (1-\varepsilon) \cdot f^{\text{sig}}_{\text{All}} +   (1-\varepsilon_{\text{bkg}}) \cdot (1-f^{\text{sig}}_{\text{All}}) )
\end{equation}

$N^{\text{Pass/Fail}}$ is the total number of observed probes that either pass or fail the selection criteria while $N_{\text{Total}}$ is the total number of Tag-and-Probe pairs.
The binomial estimator for efficiency, $\varepsilon$, enters the fit functions as the first term but is accompanied by a second term that describes the background contribution with its own efficiency $\varepsilon_{\text{bkg}}$.  The term $f^{\text{sig}}_{\text{All}}$ is the fraction of background subtracted signal events over the allowed dilepton mass range.  $f^{\text{sig}}_{\text{All}}$ depends on the defined signal and background pdfs. The nominal pdfs chosen for reported fits uses a 5 parameter Voigtian+Voigtian signal model which share a common mean but use independent width parameters, in conjunction with an Exponential background model. 


\FigureThree{PassingProbes_Med16_lowpt.png}{FailingProbes_Med16_lowpt.png}{AllProbes_Med16_lowpt.png}{Example Tag-and-Probe Z di-muon fits for passing,failing, and all probes with the Medium Id, $|\eta|<1.2$, and $p_T < 20$ GeV   }{tnpexp}



%The distributions are fit simultaneously with a combined signal and background model. To extract the probe criteria efficiecy we divide the resonance distributions by the all probes distribution. The uncertainty on an efficiency is a combination of a statistical and systematic uncertainties. The systematic uncertainties are defined by repeating the simultaneous fit with varying mass windows, number of bins, and fit models and measuring the maximum spread of the central values.


%The fit code is located here: \url{https://github.com/cms-sw/cmssw/blob/CMSSW_10_6_X/PhysicsTools/TagAndProbe/src/TagProbeFitter.cc} around line 600. confirm this with the fit model from the electron paper. \cite{Berryhill_2010}

\FloatBarrier

\section{The Electron Tag-and-Probe }

The electron Tag-and-Probe is performed with the $Z$ resonance over a range of bins divided by $p_T$ and $\eta$. The selected binnings follow the $\pt$ and $\eta$ binning conventions from the electron physics object group and are $ p_T \in [5, 10, 20, 30, 40, 70, 100]$ and $|\eta| \in [ 0, 0.6, 1.4, 2.4]$. The electron Tag-and-Probe tools uses a centrally curated CMSSW PhysicsTools in \url{CMSSW_10_2_X}. The software pipeline consists of two steps, an ntuplizing stage and a fitting stage. The  Ntupilizing stage selects Tag-and-Probe pairs along with all potential variables of interest and loads them onto an ntuple using \url{TnPTreeProducer} \cite{ElTnPGit}. The samples used in the Ntuplizing stage are listed in Table \ref{tab:electronTnPSamples}. In the fitting stage, a random subset of of TnP pairs are sampled with \url{TnPTreeAnalyzer} \cite{ElTnPAnaGit}. The analyzer performs all of the fitting and efficiency measurements according to the specified selection criteria. 

%notes to add back in later 2016B, 2017C, 2018A
\begin{table}
\caption{ Data and MC samples for each year used by the electron Tag-and-Probe. }
\label{tab:electronTnPSamples}
\scriptsize
\begin{adjustwidth}{-0.5in}{-1in}
\begin{tabular}{cc|c}
\hline
Type & Year & Sample Name \\ 
\hline
\hline 
Data & 2016 & \tiny \url{/SingleElectron/Run2016-17Jul2018_ver2-v1/MINIAOD}  \\  
Data & 2017 & \tiny \url{/SingleElectron/Run2017-31Mar2018-v1/MINIAOD} \\  
Data & 2018 & \tiny \url{/EGamma/Run2018-17Sep2018-v2/MINIAOD} \\ 
\hline 
MC & 2016 & \tiny \url{/DYJetsToLL_M-50_TuneCUETP8M1_13TeV-madgraphMLM-pythia8/RunIISummer16MiniAODv3-PUMoriond17_94X_mcRun2_asymptotic_v3_ext2-v2/MINIAODSIM} \\ 
MC & 2017 & \tiny \url{/DYJetsToLL_M-50_TuneCP5_13TeV-madgraphMLM-pythia8/RunIIFall17MiniAODv2-PU2017RECOSIMstep_12Apr2018_94X_mc2017_realistic_v14_ext1-v1/MINIAODSIM} \\ 
MC & 2018 & \tiny \url{/DYJetsToLL_M-50_TuneCP5_13TeV-madgraphMLM-pythia8/RunIIAutumn18MiniAOD-102X_upgrade2018_realistic_v15-v1/MINIAODSIM} \\
\hline
\end{tabular} 
\end{adjustwidth}
\end{table}




A general selection is applied for electron TnP candidates that differs between the tag and probe, but, both depend on super cluster (SC) kinematics. The super clusters are expected to fall within the calorimeter acceptance which includes vetoing super clusters in the end-cap gaps. The invariant mass of the electron of the pair also is required to fall within a specified Z-window. The selection specifics are listed in Table \ref{tab:eleTnPSelect}.  The tag is also required to pass a trigger requirement to reflect the inherit trigger bias which is not applied in simulation by default. The triggers selected are HLT electron collections and are grouped by specific paths and filters. The tag electrons are matched to trigger objects in the path/filter combination and passed based on the OR of triggers in the collection. The chosen trigger combinations are \url{HLT_Ele27_eta2p1_WPTight_Gsf_v*}, \url{HLT_Ele32_WPTight_Gsf_L1DoubleEG_v*}, \url{HLT_Ele32_WPTight_Gsf_v*} for 2016 through 2018 respectively.\\

\begin{table}
\caption{Kinematic requirements for the electron Tag-and-Probe components.}
\label{tab:eleTnPSelect}
\begin{adjustwidth}{-0.45in}{-1in}
\begin{tabular}{|c|c|c|c|}
\hline 
\multicolumn{4}{|c|}{Tag-and-Probe Electron Candidate Selection Criteria} \\ 
\hline 
Tag & Probe & Super Cluster & Pair \\ 
\hline 
$|\eta_{SC}| \leq 2.1$ & $|\eta_{SC}| \leq 2.5$  & $|\eta|<2.5 $ & $50 \text{GeV} < m_{ee} < 130 \text{GeV} $ \\
veto $ 1.4442 \leq |\eta_{SC}| \leq 1.566 $ & $E_{ECAL}\sin(\theta_{SC}) > 5.0 $ GeV & $E_T > 5.0 $ GeV &  \\
 $\pt \geq 30.0$ GeV &  &  &  \\
 Passes Tight Id &  &  & \\
\hline 
\end{tabular} 
\end{adjustwidth}
\end{table}



%\begin{center}
%\begin{tabular}{@{}l@{}} 
%\tabitem 2016: \url{HLT_Ele27_eta2p1_WPTight_Gsf_v*} \\
%\tabitem 2017: \url{HLT_Ele32_WPTight_Gsf_L1DoubleEG_v*}\\
%\tabitem 2018: \url{HLT_Ele32_WPTight_Gsf_v*} \\
%\end{tabular} 
%\end{center}


%\end{center}
The measurements of the Gold, Silver, and Bronze efficiencies components, and VeryLoose, are shown in Figure \ref{fig:el2deff}. The relative efficiencies per component range from approximately $75\%$ to $95\%$ with a slight dependence on $|\eta|$. The largest combined systematic and statistical errors are $O(4\%)$ and occur in data with the lowest \pt bins. The data and MC agreement is within a few percent for both the Id, Isolation, and SIP3D. The efficiency components combined into Gold, Silver, and Bronze is shown in Figure \ref{fig:elgsb}. The range of efficiencies for each quality ranking are $(50-70)\%$, $(10-20)\%$, and $(10-30)\%$ for Gold, Silver, and Bronze respectively. The Gold, Silver, and Bronze data to MC ranges around a few percent, but large discrepancies can be seen at the highest and lowest \pt bins. More granular Id measurements could be obtained by using a different resonance such as $J/\psi \rightarrow ee$ for the lower \pt ranges, however, data triggers with electrons for $J/\psi$ are not available.

\FigFour{Lep_Obj_plots/h_2017_0_redo4-16-23_eleff.pdf}{Lep_Obj_plots/h_2017_1_redo4-16-23_eleff.pdf}{Lep_Obj_plots/h_2017_2_redo4-16-23_eleff.pdf}{Lep_Obj_plots/h_2017_3_redo4-16-23_eleff.pdf}{The 2017 individually measured electron efficiencies for Very Loose, ID, Isolation, and Impact parameter in bins of $p_T$ and $|\eta|$.}{fig:el2deff}


\FigThreeScale{Lep_Obj_plots/h_ElectronRedo_4-16-23_2017_1.pdf}{Lep_Obj_plots/h_ElectronRedo_4-16-23_2017_2.pdf}{Lep_Obj_plots/h_ElectronRedo_4-16-23_2017_3.pdf}{The 2017 combined electron efficiencies for Gold, Silver, and Bronze. }{fig:elgsb}{0.49}{0.49}{0.49}

\FloatBarrier
\section{The Muon Tag-and-Probe}
The muon Tag-and-Probe tools also uses the same centrally curated CMSSW PhysicsTools in \url{CMSSW_10_6_X}. The software pipeline is identical to electrons in that it consists of an ntuplizing \cite{MuTnPTwiki} and fitting \cite{MuTnPAnaTwiki} stage. The code bases for muons and electrons are separate but functionally identical. The samples chosen for Z measurements are shown in Table \ref{tab:mutnpsamples}. The $J/\psi$ ntuples are available from a central repository of standard Tag-and-Probe selection variables which use the pre-ultra legacy samples for each year \cite{MuTnPCentralSamps} and are used for low $p_T$ Id efficiency measurement. The muon Tag-and-Probe efficiencies are measured above 20 GeV using the Z boson and below 20 GeV with the $J/\psi$ meson. The $\eta$ bins are divided into a central and forward regions around the end-caps at $|\eta| = 2.1$. In total there are three sets of binnings: The low \pt $J/\psi$ binning $J/\psi^{L}$ for muon Id below 20 GeV, the high \pt Z binning $Z^{H}$ above 20 GeV, and the low \pt Z binning $Z^{L}$ used to extrapolate isolation and impact parameter efficiencies down to 3 GeV.  The explicit bin edges for each range are defined in Table \ref{tab:mubin}.


%notes to add in later 2016C, 2017C, 2018A
\begin{table}
\caption{Data and MC samples for each year used by the electron Tag-and-Probe.}
\label{tab:mutnpsamples}
\scriptsize
\begin{adjustwidth}{-0.5in}{-1in}
\begin{tabular}{cc|c}
\hline 
Type & Year & Sample Name \\ 
\hline 
\hline
Data & 2016 & \tiny \url{/SingleMuon/Run2016-17Jul2018-v1/MINIAOD}  \\  
Data & 2017 & \tiny \url{/SingleMuon/Run2017-31Mar2018-v1/MINIAOD} \\  
Data & 2018 & \tiny \url{/SingleMuon/Run2018-17Sep2018-v2/MINIAOD} \\ 
\hline 
MC & 2016 & \tiny \url{/DYJetsToLL_M-50_TuneCUETP8M1_13TeV-madgraphMLM-pythia8/RunIISummer16MiniAODv3-PUMoriond17_94X_mcRun2_asymptotic_v3_ext2-v2/MINIAODSIM} \\ 
MC & 2017 & \tiny \url{/DYJetsToLL_M-50_TuneCP5_13TeV-madgraphMLM-pythia8/RunIIFall17MiniAODv2-PU2017RECOSIMstep_12Apr2018_94X_mc2017_realistic_v14_ext1-v1/MINIAODSIM} \\ 
MC & 2018 & \tiny \url{/DYJetsToLL_M-50_TuneCP5_13TeV-madgraphMLM-pythia8/RunIIAutumn18MiniAOD-102X_upgrade2018_realistic_v15-v1/MINIAODSIM} \\ 
\hline
\end{tabular} 
\end{adjustwidth}
\end{table}

\begin{table}
\caption{Names of centrally produced $J/\psi$ Tag-and-Probe trees.}
\label{tab:jpsimutnpsamples}
\scriptsize
\centering
\begin{tabular}{cc|c}
\hline 
Type & Year & Sample Name \\ 
\hline 
\hline
Data & 2016 & \tiny \url{TnPTreeJPsi_LegacyRereco07Aug17_Charmonium_Run2016Bver2_GoldenJSON.root}  \\  
Data & 2017 & \tiny \url{TnPTreeJPsi_17Nov2017_Charmonium_Run2017Cv1_Full_GoldenJSON.root} \\  
Data & 2018 & \tiny \url{TnPTreeJPsi_Charmonium_Run2018Dv2_GoldenJSON.root} \\ 
\hline 
MC & 2016 & \tiny \url{TnPTreeJPsi_80X_JpsiToMuMu_JpsiPt8_Pythia8.root} \\ 
MC & 2017 & \tiny \url{TnPTreeJPsi_94X_JpsiToMuMu_Pythia8.root} \\ 
MC & 2018 & \tiny \url{TnPTreeJPsi_102XAutumn18_JpsiToMuMu_JpsiPt8_Pythia8.root} \\ 
\hline
\end{tabular} 
\end{table}

\begin{table}
\centering
\caption{Muon bin edges for the $J/\psi$ case, high $p_T$ Z case, and low $p_T$ fitting case $Z^{L}$. }
\label{tab:mubin}
\begin{tabular}{c|c|c}
\hline 
\multicolumn{3}{c}{Muon Binning} \\ 
\hline 
Range & $p_T$ GeV & $|\eta|$ \\ 
\hline 
$J/\psi^{L}$ & [3.0, 4.0,  5.0, 6.0, 7.0, 9.0, 14.0,  20.0] & [0, 1.2, 2.4] \\ 

$Z^{H}$ &  [10, 20, 30, 40, 60, 100] & [0, 1.2, 2.4] \\ 

$Z^{L}$ & [6,8,10,14,18,22,28,32,38,44,50] & [0, 1.2, 2.4] \\ 
\hline 
\end{tabular} 
\end{table}

%\begin{itemize}
%\item $J/\psi^{L}$  
%	\begin{itemize}
%		\item[] $p_T \in [3.0, 4.0,  5.0, 6.0, 7.0, 9.0, 14.0,  20.0]$
%		\item[]  $\eta| \in [0, 1.2, 2.4]$
%	\end{itemize}
%\item $Z^{H}$
%	\begin{itemize}
%		\item[] $ p_T \in [10, 20, 30, 40, 60, 100]$
%		\item[] $|\eta| \in [ 0, 1.2, 2.4]$
%	\end{itemize}
%\item $Z^{L}$
%	\begin{itemize}
%		\item[] $ p_T \in [6,8,10,14,18,22,28,32,38,44,50]$ 
%		\item[] $|\eta| \in [0, 1.2, 2.4]$
%	\end{itemize}
%\end{itemize}


Isolation and impact parameters are unable to be measured using the $J/\psi$. About $30\%$ of prompt $J/\psi$ are produced from $\chi_c$ and $\Psi(2S)$ inside a jet and likely will be unisolated \cite{Lansberg:2006dh}. Similarly another $10\%$ of all $J/\psi$ are produced within b-jets and leading to  non-prompt  unisolated events \cite{LHCb:2013itw}.

The criteria for the tag or probe vary between $Z$ and $J/\psi$ but are identical across the two $Z$ ranges. The selections follow the standard criteria defined from the centrally produced muon Tag-and-Probe efficiencies and are described in Table \ref{tab:mutnpselect}.\\

\begin{table}
\small
\caption{Kinematic requirements for the muon Tag-and-Probe components.}
\centering
\begin{tabular}{|c|c|c|}
\hline 
\multicolumn{3}{|c|}{Tag-and-Probe Muon Candidate Selection Criteria} \\ 
\hline 
%\multicolumn{3}{|c|}{$J/\psi$} \\
\hline
Tag & Probe & Pair \\ 
\hline 
%isGlobalMuon & Matches hltTracksIter   & $2.8 \text{GeV} < m_{\mu\mu} < 3.4 \text{GeV} $ \\
%numberOfMatchedStations$>1$  & OR &  $|z_{\mu_1} - z_{\mu_2}| <1 $ cm\\
% $\pt > 5$ GeV & Matches hltMuTrackJpsiEffCtfTrackCands &    \\
% Matches hltIterL3MuonCandidates &   & \\
%\hline
\hline
passes tightID & No requirement & $m_{\mu\mu} > 60$ GeV \\
$\sum \pt^{ch} / \pt < 0.2$ & &  $|z_{\mu_1} - z_{\mu_2}| <4 $ cm \\
$\pt > 15$ GeV &   &   \\
\hline 
\end{tabular} 
\label{tab:mutpnselect}
\end{table}

The muon data will also has an implicit selection due to triggering. To reflect the trigger bias in MC, the tag is required to pass a chosen trigger in the efficiency denominator in addition to HLT object matching. The triggers available vary from year to year for $Z$ using \url{IsoTkMu22} in 2016 and \url{isoMu24eta2p1} in 2017 and 2018. $J/\psi$ uses the same trigger for all years, which is \url{Mu7p5Tk2}.\\

%\begin{center}
%\begin{tabular}{@{}l@{}} 
%\tabitem $J/\psi$ 2016,2017,2018: \url{Mu7p5Tk2} \\
%\tabitem $Z$ 2016: \url{IsoTkMu22}\\
%\tabitem $Z$ 2017, 2018: \url{isoMu24eta2p1} \\
%%\end{tabular} 
%\end{center}





 The Gold, Silver, and Bronze efficiency definitions used the same bin ranges $J/\psi^L$, $Z^L$, and $Z^H$ shown in Table \ref{tab:mubin}. The explicit form for Gold Silver and Bronze based on $p_T$ range and as a function of conditional efficiency components of ID, Isolation, and SIP3D are shown in table \ref{tab:gsbeqs}. The low \pt muons include the Id measured by $J/\psi$ as well as the extrapolated efficiencies from SIP3D and isolation fits in $Z_{L}$. The high \pt muons are composed of all the factors directly measured in $Z_H$.
\begin{table}
\caption{The formulation of Gold, Silver, and Bronze efficiencies as a function of conditional Tag-and-Probe efficiency components.} 
\label{tab:gsbeqs}
\begin{itemize}
\item[] $\pt \in [3,20)$
\end{itemize}
\begin{equation}\label{eq:effcomp_J}
\begin{split}
\epsilon_{\Gold}& = \epsilon_{\ID}^{J/\psi}\times \epsilon_{\Isolated|\ID}^{Z_L} \times \epsilon_{\Prompt|(\ID \cap \Isolated)}^{Z_L} \\
\epsilon_{\Silver}& = \epsilon_{\ID}^{J/\psi} \times \epsilon_{\Isolated|\ID}^{Z_L} \times (1-\epsilon_{\Prompt|(\ID \cap \Isolated)}^{Z_L}) \\
\epsilon_{\Bronze}& = 1-(\epsilon_{\ID}^{J/\psi} \times \epsilon_{\Isolated|\ID}^{Z_L})
\end{split}
\end{equation}
%\quad \quad \\
\begin{itemize}
\item[] $\pt \in [20,100 ]$
\end{itemize}
\begin{equation}\label{eq:effcomp_Z}
\begin{split}
\epsilon_{\Gold}& = \epsilon_{\ID}^{Z_H}\times\epsilon_{\Isolated|\ID}^{Z_H}\times\epsilon_{\Prompt|(\ID \cap \Isolated)}^{Z_H} \\
\epsilon_{\Silver}& = \epsilon_{\ID}^{Z_H}\times\epsilon_{\Isolated|\ID}^{Z_H}\times(1-\epsilon_{\Prompt|(\ID \cap \Isolated)}^{Z_H})\\
\epsilon_{\Bronze}& = 1-(\epsilon_{\ID}^{Z_H} \times \epsilon_{\Isolated|\ID}^{Z_H})
\end{split}
\end{equation}

\end{table}

 The 2017 efficiencies for Id, Isolation, SIP3D, and VeryLoose are shown in Figure \ref{fig:jpsiZ17-ideff-ratio}. The efficiencies for the other years are nearly identical. The overlapping bins of efficiency between $J/\psi$ and $Z$ do not all match within statistical uncertainties. However, the average deviation of the efficiency central values are $0.02\%$ for MC and $1\%$ for data.  The relative efficiencies per component range from approximately $88\%$ to $98\%$ and are fairly uniform between the barrel and end-caps. The efficiencies for the isolation ranges from $(90 - 95)\%$ where the end-caps generally are about $5\%$ more efficient. As for SIP3D, the efficiency ranges from about $(80 - 93)\%$ with another $5\%$ $|\eta|$ based efficiency gap, however, in the SIP3D case, the barrel is more efficient as opposed to isolation.  The extrapolation of the vertexing and isolation efficiencies below 20 GeV is done by fitting a quadratic polynomial to the efficiencies on the $Z_L$ interval.  Both data and MC are shown in Figure \ref{fig:eff17-extraps}.  The errors for each bin are the combined statistical and systematic errors from Table \ref{tab:musyst} and are adjusted before the polynomial fit. Any efficiencies below 20 GeV are then reported from the fit model. The fit errors are the 68\% confidence interval combined with the systematic errors. The worst observed right tail P-value from all fits is $\approx 2\%$, the median P-value from the Figure \ref{fig:eff17-extraps} is $84\%$. The fits in each year behave qualitatively the same as 2017.
The product of the efficiency components into their corresponding Gold, Silver, and Bronze category is shown in Figure \ref{fig:2017-mu1-gsbvl}. The range of efficiencies for each quality ranking are $(70 - 80)\%$, $(5 - 15)\%$, and $(4 - 20)\%$ for Gold, Silver, and Bronze respectively. The Data and MC agreement for all three ranks is better than electrons with the largest discrepancy in Gold being $2\%$ and the average deviation in Silver and Bronze begin approximately $(5-10)\%$.

%The very loose and the efficiency components combined into Gold, Silver, and Bronze are summarized in Figure \ref{fig:2017-mu1-gsbvl}, the other years are included in %the appendix. The tool used to store/calculate efficiencies can be found at \url{https://github.com/Jphsx/LepTool}.  The cumulative efficiency for a muon also includes %the efficiency of the VeryLoose selection and is defined as:
%\begin{equation}
%\epsilon_\mu = \epsilon_{\text{VeryLoose}} \times \epsilon_{\text{Gold/Silver/Bronze}}
%\end{equation}



\FigFour{Lep_Obj_plots/h_2017_0_mueff.pdf}{Lep_Obj_plots/h_2017_1_mueff.pdf}{Lep_Obj_plots/h_2017_2_mueff.pdf}{Lep_Obj_plots/h_2017_3_mueff.pdf}{Tag-and-Probe efficiencies for the muon Very Loose, Id, Isolation, and SIP3D in 2017. Uses the efficiency shape fit from Figure \ref{fig:eff17-extraps} to extrapolate isolation and SIP3D to low $p_T$. Also includes the $J/\psi$ contribution below 20 GeV. Low $p_T$ binning does not reflect actual bin granularity, bins are merged for readability. }{fig:jpsiZ17-ideff-ratio}

\FigureTwo{effFit_2.pdf}{effFit_5.pdf}%
          {fig:eff17-extraps}{The fitted muon isolation and SIP3D efficiencies for 2017. Includes both data and MC which are separated between barrel and end-cap.  }



\FigureThree{h_2017_1_mu.pdf}{h_2017_2_mu.pdf}{h_2017_3_mu.pdf}{The combined efficiency components from equations \ref{eq:effcomp_J} and \ref{eq:effcomp_Z} and Very Loose for 2017. The low-\pt region ($<20$ GeV) includes the contributions from $J/\psi$ as well as the isolation and SIP3D extrapolations. Propagated errors are treated as uncorrelated.}{fig:2017-mu1-gsbvl}

\FloatBarrier
\section{Lepton Systematics and Scale Factors}

The systematic error for the electron and muon efficiencies are derived by varying the Tag-and-Probe signal and background models, shifting the mass window , increasing and decreasing the number of bins used in the fit, varying the selection on the tag, and testing MC samples with different generators. The systematic error is defined as the maximum spread in efficiencies between the efficiency variations with an example spread shown in Figure \ref{fig:systspread}.  Rather than compute the systematic error for every bin, similarities between neighboring bins motivates using a simplified bin approach which was chosen qualitatively by the background shape. The shape of the \pt based mass distributions is illustrated in Figure \ref{fig:systplots}. The same $\eta$ bins are utilized according to lepton flavor, but the \pt bins are consolidated into a high and low bin pivoting on $20$ GeV. A high and low systematic is derived for each selection criteria per flavor per year and is applied to the efficiencies that fall within the corresponding \pt and $\eta$ range for modeling shape systematics. The tag systematics are computed by varying the $p_T$ requirement on the tag and are found to be $<1\%$. The MC systematics are computed by comparing ZDY MC sample with Z $p_T$ ranging between 100-250 GeV against an inclusive sample and is found to be about $1\%$. 
%todo syst spread fig
%todo syst tnp dist
\FigureStack{muons_lowpt_syst.png}{muons_highpt_syst.png}%
	{fig:systplots}{Tag-and-Probe di-muon mass distributions for both passing and failing probes. The top set of plots consist of probes below 20 GeV and the bottom set are about 20 GeV.}
	


Scale factors are derived bin by bin for each criteria per flavor per year by finding the ratio of efficiencies in data to Monte Carlo. The scale factor error is propagated by combining both the statistical error from the Tag-and-Probe in quadrature with the systematic error. The full 2017 set of shape modeling systematics for electrons and muons is shown in Table \ref{tab:elesysts} and Table \ref{tab:musysttable} and are nearly identical to the systematics derived in the other years. Additional scale factors are also needed to adjust the differences between samples which are either created with a full simulation or fast simulation. The fast to full factor is obtained by extracting the efficiency ratio between  full and fast sim $t\bar{t}$ MC.

\FigureOne{2017data_medID_msyst.pdf}{Example systematic spread from various fit models and binnings for muons. Includes the four combinations of regions either low or high $p_T$ and central and forward $|\eta|$.}{fig:systspread}
	
\begin{table}
\centering
\caption{The electron modeling systematic error derived from the Tag-and-Probe for 2017 data and split into $p_T$ and $|\eta|$ regions. }
\label{tab:elesysts}
\begin{tabular}{|c|ccc|}
\hline
ID & $0\leq |\eta|<0.8$ & $0.8\leq |\eta|<1.479$ & $|\eta|\geq1.479$ \\
\hline
$\pt < 20$ [GeV] & 0.003 & 0.001 & 0.005 \\
$\pt \geq 20$ [GeV] & 0.001 & 0.001 & 0.002  \\
 &  &  &    \\
\hline
Iso $|$ ID  &  &  &   \\
\hline
$\pt < 20$ [GeV] & 0.002 & 0.003 & 0.003   \\
$\pt \geq 20$ [GeV] & 0.001 & 0.001 & 0.002 \\
 &  &  &   \\
\hline
SIP $|$ Iso $\cap$ ID &  &  &  \\
\hline
$\pt < 20$ [GeV]& 0.006 & 0.004 & 0.007 \\
$\pt \geq 20$ [GeV]& 0.002 & 0.002 & 0.0006  \\
\hline
VeryLoose &  &  &  \\
\hline
$\pt < 20$ [GeV]& 0.002 & 0.007 & 0.03 \\
$\pt \geq 20$ [GeV]& 0.003 & 0.0001 & 0.0007 \\
\hline
\end{tabular}
\end{table}


\begin{table}[htbp]
\centering
\caption{The muon modeling systematic error derived from the Tag-and-Probe data and split into \pt and $|\eta|$ regions. }
\label{tab:musysttable}
\begin{tabular}{|c|cc|}
\hline
ID & $|\eta|<1.2$ & $|\eta|\geq 1.2$  \\
\hline
$\pt < 20$ [GeV](J) & 0.001 & 0.001  \\

$\pt \geq 20$ [GeV](Z) &  0.001& 0.0003 \\

 &  & \\
\hline
Iso $|$ ID  &  &  \\
\hline
$\pt < 20$ [GeV]  & 0.007 & 0.004  \\

$\pt \geq 20$ [GeV] & 0.007 & 0.002  \\

 &  &  \\
\hline
SIP $|$ Iso $\cap$ ID &  &  \\
\hline
$\pt < 20$ [GeV]& 0.005 & 0.003 \\

$\pt \geq 20$ [GeV]& 0.001 & 0.002 \\
 & &  \\
\hline
Very Loose & &  \\
\hline
$\pt < 20 $ [GeV]  & 0.001 & 0.0003 \\
$\pt \geq 20$ [GeV]  & 0.001 & 0.001 \\
\hline
\end{tabular}
\label{tab:musyst}
\end{table}




%\setcounter{secnumdepth}{3}
\setcounter{tocdepth}{3}
\setlength{\parskip}{\smallskipamount}
\setlength{\parindent}{0pt}


\makeatletter


\providecommand{\tabularnewline}{\\}


\makeatother

\chapter{Modeling}

\section{Introduction}
%introduce actual fit, fit regions, systematics treatment, results of fit stages
The modeling for this analysis uses the standard counting experiment approach with a Poisson likelihood. The MC model is data driven such that background composes most of the regions and tunes the data and MC agreement. This agreement translates into well constrained background predictions in sensitive regions with sparse background with an ABCD-like approach. To achieve a robust fit model three stages of fits are performed, one with the Control Region (CR) which has no signal presence, a Validation Region (VR) which is a partially unblinded region with mild sensitivity designed to validate the CR model and demonstrate reasonable modeling in regions untouched by the CR , and finally SR which is comprised of the high $R_{ISR}$ bins which are sensitive to all signal regions. Following the full fit, limits can be calculated which test the hypotheses of background only model versus background plus signal model. 


\section{Fit Strategy and Fit Region Definitions}
CR VR SR

\section{Fit Implementation and Model Defintion}
%Poisson likelihood
The fitting framework is provided by the HiggsCombine tool which generates datacards that encodes all the components of the fit into a standard format that is processed by CombineHarvester and RooFit/RooStats packages. The fit and its components can be represented by a Poisson likelihood defined as:
\begin{equation}
\label{eq:fit}
\mathcal{L}(\vec{\alpha}|\vec{x}) = \bigg[ \prod_i^N \text{Pois}(x_i|\lambda_i(\vec{\alpha})) \bigg] \bigg[\prod_j^M \pi_j(\alpha_j) \bigg]
\end{equation}
The contents of equation \ref{eq:fit} extend over the range of all $N$ analysis bins where each $i$-th bin is composed of a count of observed events $x_i$ and expected events $\lambda_i$. The expected events are subject to the set of nuisance parameters $\vec{\alpha}$ of which some are conditioned by prior probability distributions $\pi_j(\alpha_j)$. The ideal model for $\lambda(\vec{\alpha})$ is found by maximizing the likelihood \ref{eq:fit} with the minimum set of nuisance parameters $\vec{\alpha}$ by fitting the stages of fit regions such that the model is sensitive to signals and the signal + background hyptothesis.  There are three types of nuisances implemented in the fit, freely floating rate parameters, log-normal constrained parameters, and shape parameters.  Freely floating paramters contribute to a factor $\kappa$, with a starting value of 1,  that is applied to the expected bin yield $\lambda$ to adjust the bin yield by some fraction with respect to the nominal value. The free parameters have no associated penalty with their adjustment and are fully determined by data. Individual bins $i$ are mapped together by common processes $k$ which are all associated under a common nuisance $j$. The selection of processes associated to a nuisance parameter can either be the contribution from a background process or associated fake leptons. The definition of a free rate parameter can then be defined as 
\begin{equation}
\label{eq:rateparam}
\kappa_{ijk}(\alpha_j) = \alpha_j
\end{equation}  
The log-normal parameters also functions of a $\kappa$ factor that is applied to an expected events of the associated bin. The log-normal parameter is different from the freely floating paramters in such that it is penalized for moving from the nominal value with based on normally distributed prior $\pi(\alpha_j)$. Along with a prior associated uncertainty on a process $j$ of nuisance $k$, that is $\sigma_{jk}$, the log-normal factors are defined as
\begin{equation}
\label{eq:logparam}
\kappa_{ijk}(\alpha_j) = (1+\sigma_{ijk})^{\alpha_j}
\end{equation}

The third type of nuisance is different from the first two such that it does not associate with and modify the process components for a particular bin, instead, the shape nuisances adjust expected bin yields based on the underlying shapes of the $R_{ISR}$ and $M_\perp$ distributions. The $kappa$ factor for a bin yield is then a function of up and down variations of one of the kinematic variables which are also encoded with a normally distributed prior $\pi(\alpha_j)$. The $\kappa$ definition is based on the interpolation $-1<\alpha_j<1$ and is written as follwing based on a predefined shape treatment \cite{combine shapes}
\begin{equation}
\label{eq:shapeparam}
\kappa_{ijk}(\alpha_j)= 1 + \frac{1}{2}((\delta^+ - \delta^-)\alpha_j + \frac{1}{8}(\delta^+ + \delta^-)(3\alpha_j^6-10\alpha_j^4+15\alpha_j^2))
\end{equation}
the $\delta^\pm$ components are ratios of the up and down shape variations, $\lambda^{up/down}$, to the nominal shape< $\lambda^{nominal}$ such that $\delta^+ = \lambda^{up}/\lambda^{nominal}$ and $\delta^- = \lambda^{down}/\lambda^{nominal}.$

From the Likelihood Equation \ref{eq:fit} which is composed of the three types of nuisances from equations \ref{eq:rateparam}, \ref{eq:logparam}, \ref{eq:shapeparam}. These nuisances are mapped to either a set of processes or shapes and also mapped to a set of associated bins. The product of the three $\kappa$ factors multiply the nominal expectation $\lambda$ such that they maximize the likelihood and in turn the agreement between the observed data $\vec{x}$ and $\vec{\lambda}$. 


\section{Development of Modeling Systematics}
definition of sytematics
poisson pulls

\section{Region Results}

CR and VR plots 
pull dists



\setcounter{secnumdepth}{3}
\setcounter{tocdepth}{3}
\setlength{\parskip}{\smallskipamount}
\setlength{\parindent}{0pt}


\makeatletter


\providecommand{\tabularnewline}{\\}


\makeatother


\chapter{Results}


\section{Asymptotic Limits}

This search is designed to be generically sensitive to SUSY with an emphasis compressed scenarios. The consequence is sensitive to a wide variety of models and final states, so, we present upper exlusion limits on the cross sections for stop, electroweakino, and slepton processes. The results use the full Run II dataset alongside the full SM MC background which is combined with the data driven fit model described in the previous chapter. The limits are calculated using the asypmtotic method for profile-likelihood test statistic \cite{whatever AN cites here} 

\FigOne{Results_figs/allyear_limit_t2tt_138.pdf}{t2tt run2}{fig:limt2tt}

\FigOne{Results_figs/allyear_limit_tchiwz_138.pdf}{tchiwz run2}{fig:limtchiwz}

\FigOne{Results_figs/allyear_limit_tslepslep_138.pdf}{tslep run2}{fig:limslep}

\FigOne{Results_figs/T2bW_dm_xs_LV1.pdf}{t2bw 17}{fig:limt2bw}

\FigOne{Results_figs/can_dM2_WW.pdf}{ww 17}{fig:limtchiww}

\section{Model Dependent Interpretation}
The TChiWZ limits from the previous section use a simplified model which is described in Chapter 1 section ?. The decay kinematics for the MC use a flat matrix element meaning the W and Z boson 3 body decays are uniform in phase space. However, it is well known that the 3 body decays are model dependent and can by categorized into two distinct scenarios that correspond to the sign of the eigenstates of the neutralino mass matrix. The two eigenstates 


\section{Summary}


\setcounter{secnumdepth}{3}
\setcounter{tocdepth}{3}
\setlength{\parskip}{\smallskipamount}
\setlength{\parindent}{0pt}


\makeatletter


\providecommand{\tabularnewline}{\\}


\makeatother


\chapter{Summary}


This dissertation outlines a general search for supersymmetric particles in compressed scenarios. The search is performed with proton-proton collisions at $\sqrt{s} = 13$ TeV and the CMS detector using the full Run II data-set with integrated luminosity of 138 fb$^{-1}$. A general compressed SUSY topology is identified as ISR recoiling against an energetic invisible and soft visible system. This topology is organized into a set of decay trees by labeling topological components as either part of the ISR or Sparticle systems. The sparticle system is further subdivided into a di-sparticle system with visible and invisible components partitioned into each. The decay tree reference frames are approximated using a rule set from the Recursive Jigsaw Reconstruction framework that guides the optimal partitioning for each reference frame. Following the construction of each event's decay tree, kinematic mass sensitive variables are derived to discriminate against backgrounds. Each event is then further categorized in bins of the mass sensitive kinematic variables $R_{ISR}$ and $M_\perp$ as well as physics object counts such as lepton multiplicity, jet multiplicity, b-tagging, and other complementary variables. The organization of all categories and the optimization of those categories has been shown to contain complementary regions that act as cross constraints for the fit. The lepton selection is also defined and the efficiencies of that selection are modeled with the Tag-and-Probe method. The Tag-and-Probe measurements are used to  model and correct systematic effects through scale factors for each component of the gold, silver, and bronze lepton selections in each year separately. The Tag-and-Probe scale factors and systematics are a minor contribution to the overall fit which is composed of over 200 nuisance parameters. The nuisance parameters are derived by studying control region fits and statistical metrics such as the Poisson likelihood, z-score, or chi-squared. The set of nuisances is found to be satisfactory in describing both the control region fit and the validation region fit and does not reject a signal hypothesis in a signal injected fit. From the establishment of this fit, expected limits are shown for processes involving the production of stops, electroweakinos, and sleptons which extend the current exclusion reach significantly.   



  
\global\long\def\bibname{References}%

%\bibliographystyle{plain}
\bibliographystyle{unsrt}
\bibliography{Biblio/allcites}


%\appendix
%\include{Appendix1/Appendix1}
\end{document}
