



\setcounter{secnumdepth}{3}
\setcounter{tocdepth}{3}
\setlength{\parskip}{\smallskipamount}
\setlength{\parindent}{0pt}


\makeatletter


\providecommand{\tabularnewline}{\\}


\makeatother

%\usepackage{babel}
%\begin{document}

\chapter{The Standard Model and Supersymmetry}

\begin{chapterabstract}
Introduce SM, talk about electroweak origins,susy origins
\end{chapterabstract}

\section{Introduction}

c1 introduction section

\section{The Standard Model}

The standard model (SM) is a collection of theories which explain the most fundamental particles as we know them. It is a gauge quantum field theory between SU(3)xSU(2)xU(1) groups and uses a Lagrangian density to describe fundamental particles  dynamics and interactions. These particles fall into two separate categories, fermions and bosons both which can be divided into two subgroups. The fermions consist of leptons and quarks. There are three flavors of leptons, the electron $(e)$, the muon $(\mu)$, and the tau $\tau)$. For each flavor there is a charged particle and pairing neutral neutrino. The $e$ and $\mu$ are also generally considered as "light" leptons for there small mass relative to the $\tau$ and the term lepton, depending on context, often refers to only the charged particles. As for the quarks, there are three generations in mass in 3 pairs of quarks. The quarks have complementary fractional electric charges. The lighest set of quarks are the up $(u)$ and down $(d)$ quarks, followed by the charm $(c)$ and strange $(s)$, then the bottom $(b)$ and extremely massive top quark $(t)$.  The bosons are comprised of the vector bosons - the photon $(\gamma)$, gluon $(g)$, the $W^\pm$, and the $Z^0$ - along with the singular scalar boson the Higgs $(h)$. The elementary pariticles are summarized in Figure \ref{fig:smfig}.

\FigOne{Intro_figs/Standard_Model_of_Elementary_Particles.png}{particles figure cite wiki}{fig:smfig}


The SM Lagrangian is composed of constituent sectors which describe diffrent groups/fields/particles. The main SM sectors are, the quantum chromodynamics (QCD) sector, the electroweak sector, the Higgs sector, and the Yukawa sector. QCD describes colored interactions of quarks mediated by gluons with the strong force. The electroweak sector unifies both the electromagnetic and weak interactions via exchange of W or Z bosons as well as electromagnetic interactions via $\gamma$. The Higgs sector introduces the complex scalar higgs field (citation needed). Interaction of bosons with the Higgs field causes the bosons to have mass and the Yukawa coupling describes the interaction of fermions with the higgs field which also allows the fermions to have mass (citation needed).



The SM is a asymmetric chiral theory, meaning that spinors with different chiralities can
transform differently under the Lorentz group and the EW gauge group
SU(2)L*U(1) (cite slides).  The consequence of chiral properties is limiting the possible helicity combinations between interaction vertices (cite thompson?). Helcity is the defined by the projection of a particles spin ontion its direction of motio (thompson). A particle is considred right-handed if the direction of its spin and motion is parallel. It is left-handed if spin and motion antiparallel. (cite wikipedia helicity). This peculiar property shows up with the W boson, which only couples to left handed particles or right handed antiparticles.

\section{Supersymmetry}

base super symmetery at lagrange level

Supersymmetry is an extension of the standard model. It includes an generator that acts as a rotation of the spin between bosons and fermions. This symmetry would generates super partners for each particle which differ in spin by a half integer. (elaborate on generating super partners fermion vs boson?) Thus there is a particle super partner (sparticle) matched to every particle. The super partners are denoted with an "s" prefix e.g. sfermion, squark, selectron. Supersymmetry is an extremely expansive model and intractable to experimentally test without significant well motivated simplifications. The most experimentally common is the Minimally Super Symmetric Standard Model (MSSM). The MSSM contains the smallest number of new particle states and new interactions which are consistent with phenomenology (cite howie direct weak scale book). The MSSM is still experimentally inaccesible due to the presence of over 100 parameters, a small change in parameter space could completely change the overall structure of masses and couplings significantly morphing the experimental signature. To tackle such a vast parameter space further simplification is needed, an example of a simpler model the phenomological MSSM (pMSSM). This contains 19 parameters which includes mass parameters for each generation of squark and slepton, parameters to control the mixing of super fields, and dials for the higgs.  More commonly the MSSM is boiled down into 5 parameters in Gravity -mediate supersymmetry breaking models or (minimal supbergravity mSUGRA). It can also be reduced, which is what our MC is made from, into 4 paramters M1,M2, $\mu$, $\tan\beta$. These come from Realistic simplified gaugino-higgsino models in MSSM (cite benjamin Fuks papaer). A direct quote from the paper describing the 4 paramters: "that are the off-diagonal Higgs(ino) mass parameter, the ratio of the vacuum expectation values of the neutral components of the two doublets of Higgs fields and the two soft
supersymmetry-breaking electroweak gaugino mass parameters, respectively."

What particles are in susY?
For each elementary standard model particle there is a super partner. For the quarks and leptons, the pairing is simple, there is just the equivalent slepton and squark partner. The gauge bosons are slightly more complicated, these are generally denoted with an "ino" suffix. There are also 3 super fields which mix in specific quantities to yield varying instances of particles with particular properties. These mixings define the characteristics of the model point by influecing things like decay mode, cross seection, and couplings.  (WhY?) There are four neutralinos $\chi^0_i$ and two charginos $\chi^\pm_j$. There are also 4 Hiigs bosons, a charged pair $H^\pm$ and a neutral pair $H^0_{u,d}$. (Why?)  The electroweakinos, i.e chargino or neutralinos, increase in mass with increasing index but the structure of reletavie masses depends specifically on the model. The $\chi^0_0$ is generally the lightest supersymmetric partilce (LSP) and in many popular models is stable. The instances of stable LSP depend on R-parity conservation. (Define R-parity conservation) If this is violated the LSP will decay into SM particles.



Incldue a plot with mass hierarchies. Since there are so many possible parameters, varying sets of paramters can produce significant diffences in experimental signatures and topologies. Typically for a model we decouple specific sectors when generatting monte carlo, For instance if we are searching for sleptons, the squark or electroweakino sector will be chosen to be significantly heavier (out of current experimental range) effectively decoupling it from the slepton sector. Then from a simplified model with everything else decoupled we scan various topologies with particular mass values. 


SUSY Chirality!

MSSM how many paramaters, what about msugra and all of that, what are the main interesting parameters??


talk about the higgsino double and the particles that arise from this model

Talk about and define r-parity, what is rparity pair production


simplified models

pmssm

what are the important parameters and specific particles?
