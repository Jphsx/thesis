



\setcounter{secnumdepth}{3}
\setcounter{tocdepth}{3}
\setlength{\parskip}{\smallskipamount}
\setlength{\parindent}{0pt}


\makeatletter


\providecommand{\tabularnewline}{\\}


\makeatother

%\usepackage{babel}
%\begin{document}

\chapter{The Standard Model and Supersymmetry}

\begin{chapterabstract}
Introduce SM, talk about electroweak origins,susy origins
\end{chapterabstract}

\section{Introduction}

c1 introduction section

\section{The Standard Model}

The standard model (SM) is a collection of theories which explain the most fundamental particles as we know them. Overall it is a gauge quantum field theory between SU(3)xSU(2)xU(1) groups which describes the fundamental particles  dynamics and interactions.

\FigOne{Intro_figs/Standard_Model_of_Elementary_Particles.png}{descriptions}{fig:smfig}

The SM is represented with a Lagrangian density composed of a sum of constituent lagrangian densities representing diffrent groups/fields/particles. The main SM sectors are, the quantum electrodynamic sector $\mathcal{L}_{QED}$, the quantum chromodynamics sector $\mathcal{L}_{QCD}$, the electroweak sector $\mathcal{L}_{EW}$, the Higgs sector $\mathcal{L}_{H}$, and the Yukawa sector $\mathcal{L}_{Yukawa}$.  QED explains the interactions of charged particles through photon exchange. QCD explains the strong interaction of quarks mediated by gluons. The electroweak sector unifies both the electromagnetic and weak interactions between all particles via exchange of W or Z bosons. 
The Higgs sector inculdes the higgs field, the interaction of bosons with this field causes the bosons to have mass. The Yukawa sector describes the interaction of fermions with the higgs field.

\section{Supersymmetry}

base super symmetery at lagrange level

Supersymmetry is an extension of the standard model. It includes an generator that acts as a rotation of the spin between bosons and fermions. This symmetry would generates super partners for each particle which differs by a half integer in spin. Thus there is a particle super partner (sparticle) matched to every particle and denoted with an "s" indicating superparnter e.g. sfermion, squark, selectron. 


simplified models

pmssm
