



\setcounter{secnumdepth}{3}
\setcounter{tocdepth}{3}
\setlength{\parskip}{\smallskipamount}
\setlength{\parindent}{0pt}


\makeatletter


\providecommand{\tabularnewline}{\\}


\makeatother

%\usepackage{babel}
%\begin{document}

\chapter{The Standard Model and Supersymmetry}

\begin{chapterabstract}
Here we outline the fundamental concepts of particle physics, we introduce the set of fundamental particles, fields, and interactions which are described by the standard model. Since the standard model does not solve all of the problems of particle physics I introduce supersymmetry, a well motivated extension to the standard model. The motivation for the SUSY extension is reserved for the following chapter.

\end{chapterabstract}

\section{Introduction}

c1 introduction section
maybe a little history here

\section{The Standard Model}

The Standard Model(SM) is a collection of adhoc theories used to predict and reproduce experimental data. The theory itself incorporates four major concepts: Quantum Field theory, the Dirac equation, the gauge principle, and the higgs mechanism. These four principles are constrained by physical data and describe the set of elementary particles, known as fermions and bosons. The SM generally refers to the SM Lagrangian, an equation with different sectors that describe different subsets of particles, fields, and their interactions. The SM Lagrangian itself consists of 26 free parameters which are input by hand. These parameters are: the masses of the 12 fermions, 3 coupling constants that describe gauge interactions: $g, g', g_s$; 2 parameters to desribe the Higgs potential $\mu, \lambda$ which are the higgs mass $m_h$ and the vacuum expectation value (vev), and 9 mixing angles which describe the  PMNS and CKM matrices i.e. the mixing of different fermionic fields. The 12 fermion paramters are subdivided by three neutrinos $m_\nu^i$, three charged leptons $m_\ell^\pm$, and six quarks $m_q$; 

Quantum field theory (QFT) provides a descriptions for both known and theoretical particles by combining quantum theory, the field concept, and relativity (cite peskin?). The gauge theory aspect describes the exact nature of QFT interactions and provides the mechanisms for the electromagnetic, strong, weak forces.  We know of four gauge fields:  $\vec{G}$ which transforms under $SU(3)$ and governs strong interactions, $\vec{W}$ and $B$ which transform under $SU(2)\timesU(1)$ and governs electromagnetic and weak interactions. The combination of the gauge fields and fermion fields along with the Dirac equation yields eigenstates that represent fermionic matter particles. In general these particles would be massless if not for the inclusion of the complex scalar Higgs field. The Yukawa copuling to the higgs field
The spontaneous symmetry breaking of the Higgs field, due to the Yukawa coupling, creates a non zero vev which is responsible for generating the masses of the electroweak gauge bosons. Additionally, the interaction between the fermionic fields and the non zero vev generates the masses of SM fermions.

The set of standard model elementary particles is divided into two subgroups: fermions and bosons.  The fermions consist of both charged and neutral leptons as well as fractionally charged quarks. There are three flavors of charged leptons, the electron $(e)$, the muon $(\mu)$, and the tau $\tau)$. Each charged particle and pairing neutral neutrino $\nu_\ell$. The $e$ and $\mu$ are also generally considered as "light" leptons for there small mass relative to the $\tau$. The term lepton, depending on context, often refers to only the charged particles. As for the quarks, there are also three generations of pairs of quarks.. The lighest set of quarks are the up $(u)$ and down $(d)$ quarks, followed by the charm $(c)$ and strange $(s)$, and lastly the bottom $(b)$ and extremely massive top quark $(t)$.  The bosons are the force carrying particles which represent the gauge fields. They are comprised of the vector bosons - the photon $(\gamma)$, gluon $(g)$, the $W^\pm$, and the $Z^0$ - along with the singular scalar boson the Higgs $(h)$. The elementary pariticles are summarized in Figure \ref{fig:smfig}.

\FigOne{Intro_figs/Standard_Model_of_Elementary_Particles.png}{particles figure cite wiki}{fig:smfig}


%The SM Lagrangian is composed of constituent sectors which describe diffrent groups/fields/particles. The main SM sectors are, the quantum chromodynamics (QCD) sector, the electroweak sector, the Higgs sector, and the Yukawa sector. QCD describes colored interactions of quarks mediated by gluons with the strong force. The electroweak sector unifies both the electromagnetic and weak interactions via exchange of W or Z bosons as well as electromagnetic interactions via $\gamma$. The Higgs sector introduces the complex scalar higgs field (citation needed). Interaction of bosons with the Higgs field causes the bosons to have mass and the Yukawa coupling describes the interaction of fermions with the higgs field which also allows the fermions to have mass (citation needed).



The SM is a asymmetric chiral theory, meaning that spinors with different chiralities can
transform differently under the Lorentz group and the EW gauge group
SU(2)L*U(1) (cite slides).  The consequence of chiral properties is limiting the possible helicity combinations between interaction vertices (cite thompson?). Helcity is the defined by the projection of a particles spin ontion its direction of motion (thompson). A particle is considred right-handed if the direction of its spin and motion is parallel. It is left-handed if spin and motion antiparallel. (cite wikipedia helicity). This peculiar property shows up with the W boson, which only couples to left handed particles or right handed antiparticles.

\section{Supersymmetry}

Supersymmetry is an extension of the standard model. It adds an generator that rotates the spin between bosons and fermions. This then introduces a bosonic degree of freedom for every fermionic degree of freedom (cite run2 susy paper) and generates super partners for each particle which differ in spin by a half integer.  Thus there is a mirrored symmetry of elementary particles with a super symmetric partner and are refered to as sparticles. Each fermionic sparticle carries the same as its fermion partner but with an "s" prefix e.g. sfermion, squark, selectron. As for the bosonic sparticles, the gauge $B$ and $\vec{W}$  fields are accompanied by the the super fields - Higgsino $\tilde{H}$, Bino $\tilde{B}$, and Wino $\tilde{W}$. The B and $\vec{W}$ SM fields mix and yield mass eigenstates for $\gamma, Z, W^\pm$. The Higgsino,Bino, and Wino mix to produce four neutral and two charged eignestates, the neutralinos ($\tilde{\chi}^0_1, \tilde{\chi}^0_2, \tilde{\chi}^0_3, \tilde{\chi}^0_4$)  and charginos ($\tilde{\chi}^\pm_1, \tilde{\chi}^\pm_2$). SUSY also requires an additional higgs doublet to give mass to up-type and down-type fermions which leads to five higgs boson states with 2 Charged higgs and 3 neutral higgs with the lightest neutral higgs representing the SM higgs boson. The second higgs doublet also introduces another vev and the ratio of the two is commonly denoted as $v_1/v_2 = \tan \beta$. With the introduction to a new set of particles we denote a new bookkeeping parameter similar to lepton number or baryon number conservation which is called R-parity where the total number of SM particles (+1) and sparticles (-1) are conserved. R-parity conservation then requires sparticles to be produced in pairs. If R-parity is violated, the common consequence is that the lightest supersymmetric particle (LSP) is unstable. 


Supersymmetry is an extremely expansive model and intractable to experimentally test without significant well motivated simplifications. The most experimentally common is the Minimally Super Symmetric Standard Model (MSSM). The MSSM contains the smallest number of new particle states and new interactions which are consistent with phenomenology (cite howie direct weak scale book). The MSSM is still experimentally inaccesible due to the presence of over 100 parameters, a small change in parameter space could completely change the overall structure of masses and couplings significantly morphing the experimental signature. To tackle such a vast parameter space further simplification is needed, an example of a simpler model the phenomological MSSM (pMSSM). This contains 19 parameters which includes mass parameters for each generation of squark and slepton, parameters to control the mixing of super fields, and dials for the higgs.  More commonly the MSSM is boiled down into 5 parameters in Gravity -mediate supersymmetry breaking models or (minimal supbergravity mSUGRA). It can also be reduced, which is what our MC is made from, into 4 paramters M1,M2, $\mu$, $\tan\beta$. These come from Realistic simplified gaugino-higgsino models in MSSM (cite benjamin Fuks papaer). A direct quote from the paper describing the 4 paramters: "that are the off-diagonal Higgs(ino) mass parameter, the ratio of the vacuum expectation values of the neutral components of the two doublets of Higgs fields and the two soft
supersymmetry-breaking electroweak gaugino mass parameters, respectively."

%What particles are in susY?
%For each elementary standard model particle there is a super partner. For the quarks and leptons, the pairing is simple, there is just the equivalent slepton and squark partner. The gauge bosons are slightly more complicated, these are generally denoted with an "ino" suffix. There are also 3 super fields which mix in specific quantities to yield varying instances of particles with particular properties. These mixings define the characteristics of the model point by influecing things like decay mode, cross seection, and couplings.  (WhY?) There are four neutralinos $\chi^0_i$ and two charginos $\chi^\pm_j$. There are also 4 Hiigs bosons, a charged pair $H^\pm$ and a neutral pair $H^0_{u,d}$. (Why?)  The electroweakinos, i.e chargino or neutralinos, increase in mass with increasing index but the structure of reletavie masses depends specifically on the model. The $\chi^0_0$ is generally the lightest supersymmetric partilce (LSP) and in many popular models is stable. The instances of stable LSP depend on R-parity conservation. (Define R-parity conservation) If this is violated the LSP will decay into SM particles.



Incldue a plot with mass hierarchies. Since there are so many possible parameters, varying sets of paramters can produce significant diffences in experimental signatures and topologies. Typically for a model we decouple specific sectors when generatting monte carlo, For instance if we are searching for sleptons, the squark or electroweakino sector will be chosen to be significantly heavier (out of current experimental range) effectively decoupling it from the slepton sector. Then from a simplified model with everything else decoupled we scan various topologies with particular mass values. 


SUSY Chirality!

MSSM how many paramaters, what about msugra and all of that, what are the main interesting parameters??


talk about the higgsino double and the particles that arise from this model

Talk about and define r-parity, what is rparity pair production


simplified models

pmssm

what are the important parameters and specific particles?
