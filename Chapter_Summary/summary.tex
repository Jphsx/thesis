\setcounter{secnumdepth}{3}
\setcounter{tocdepth}{3}
\setlength{\parskip}{\smallskipamount}
\setlength{\parindent}{0pt}


\makeatletter


\providecommand{\tabularnewline}{\\}


\makeatother


\chapter{Summary}


This dissertation outlines a general search for supersymmetric particles in compressed scenarios. The search is performed with proton-proton collisions at $\sqrt{s} = 13$ TeV and the CMS detector using the full Run II data-set with integrated luminosity of 138 fb$^{-1}$. A general compressed SUSY topology is identified as ISR recoiling against an energetic invisible and soft visible system. This topology is organized into a set of decay trees by labeling topological components as either part of the ISR or Sparticle systems. The sparticle system is further subdivided into a di-sparticle system with visible and invisible components partitioned into each. The decay tree reference frames are approximated using a rule set from the Recursive Jigsaw Reconstruction framework that guides the optimal partitioning for each reference frame. Following the construction of each event's decay tree, kinematic mass sensitive variables are derived to discriminate against backgrounds. Each event is then further categorized in bins of the mass sensitive kinematic variables $R_{ISR}$ and $M_\perp$ as well as physics objects counts such as lepton multiplicity, jet multiplicity, b-tagging, and other complementary variables. The organization of all categories and the optimization of those categories have been shown to contain complementary regions that act as cross constraints for the fit. The lepton selection is also defined and the efficiencies of that selection are modeled with the Tag-and-Probe method. The Tag-and-Probe measurements are used to  model and correct systematic effects through scale factors for each component of the gold, silver, and bronze lepton selections in each year separately. The Tag-and-Probe scale factors and systematics are a minor contribution to the overall fit which is composed of over 200 nuisance parameters. The nuisance parameters are derived by studying control region fits and statistical metrics such as the Poisson likelihood, z-score, or chi-squared. The set of nuisances is found to be satisfactory in describing both the control region fit and the validation region fit and does not reject a signal hypothesis in a signal injected fit. From the establishment of this fit, expected limits are shown for processes involving the production of stops, electroweakinos, and sleptons which extend the current exclusion reach significantly.   

