
\setcounter{secnumdepth}{3}
\setcounter{tocdepth}{3}
\setlength{\parskip}{\smallskipamount}
\setlength{\parindent}{0pt}


\makeatletter


\providecommand{\tabularnewline}{\\}


\makeatother

%\usepackage{babel}
%\begin{document}

\chapter{The Tag-and-Probe}

\begin{chapterabstract}
The tag-and-probe is a method used to measure the selection efficiencies of an object using data. In the context of this compressed SUSY analysis, the Tag-and-probe measures the efficiencies separately of each light lepton selection critera. The total lepton selection efficiency is then computed by combining the factorized efficiency components. The same general method is used for both electrons and muons, however, Muons utilize  the jpsi di-muon trigger which grant a more robust efficiency measurement from data at lower muon pT.
\end{chapterabstract}

\section{Introduction and Methodology}
An important element of a lepton based search for new physics is properly modeling the efficiency of leptons that are selected for that search. A purely Monte-Carlo driven approach is inadequate in actually describing nuances in data due to imperfections in modeling. Instead of trying to model perfectly all physics and detector effects with a simulation, the efficiencies can be directly measured from data by using the Tag-and-Probe method. 

The tag and probe method is used to measure a selection criteria by using a well known resonance such as $Z$, $J/\psi$, or $\Upsilon$. Each counted instance of the tag and probe consists of two selected leptons one which is the tag and the other is the probe.  The tag passes tight requirement so we are confident that it isn't a fake lepton. Fake leptons fall into two possible categories: reducible and irreducible. A reducible fake lepton is a particle that fakes the signature of a lepton such as a charged pion. An irreducible fake lepton is an actual lepton which coincidentally passes some selection criteria but is not the targeted leptons of interest e.g. an isolated muon from a jet accompanying a lepton Z decay of interest.  The  second lepton in the tag and probe is the probe. The probe is subjected to the selection criteria whose efficiency is being measured. The invariant mass of lepton pair is calculated and required to fall within a defined range around the resonance. A particular event may have multiple lepton pairs, the tag and probe are required to be oppositely charged and same flavor. If multiple pairs are formed from a single event all pairs are used, the pair that is not from the resonance contributes as combinitorial background. From the pairing the probe can either pass or fail its selection thus three distributions of the same resonance are formed, one with a passing probe, one with a failing probe, and one with all probes. The distributions are fit simultaneously with a combined signal and background model. To extract the probe criteria efficiecy we divide the resonance distributions by the all probes distribution. The uncertainty on an efficiency is a combination of a statistical and systematic uncertainties. The systematic uncertainties are defined by repeating the simultaneous fit with varying mass windows, number of bins, and fit models and measuring the maximum spread of the central values.

Work this into above paragraph maybe:
The probability of observing $k$ passing probes in $n$ tag and probe pair trials is dependent on the selection efficiency $\varepsilon$ and can be expressed as a binomial likelihood $P(k|\varepsilon,n) = \binom{n}{k}\varepsilon^k(1-\varepsilon)^{n-k}$. From this the MLE estimator for efficiency is the fraction of passing probes from the total number of pairs, or $\varepsilon = k/n$.
The fit code is located here: \url{https://github.com/cms-sw/cmssw/blob/CMSSW_10_6_X/PhysicsTools/TagAndProbe/src/TagProbeFitter.cc} around line 600. confirm this with the fit model from the electron paper. \cite{Berryhill_2010}

The fit partiallly defined in \cite{Berryhill_2010} is a simulataneous (un?) binned maximum likelihood between the observed passing probes and failing probes. The efficiency extracted is an explicit fit parameter. The two simultaneously fit functions are:
\begin{equation}
	N^{\text{Pass}} = N_{\text{Total}} \cdot \varepsilon \cdot f^{\text{sig}}_{\text{All}} + N_{\text{Total}}  \cdot \varepsilon_{\text{bkg}} \cdot (1-f^{\text{sig}}_{\text{All}})
\end{equation} 
\begin{equation}
	N^{\text{Fail}} = N_{\text{Total}} \cdot (1-\varepsilon) \cdot f^{\text{sig}}_{\text{All}} + N_{\text{Total}}  \cdot (1-\varepsilon_{\text{bkg}}) \cdot (1-f^{\text{sig}}_{\text{All}})
\end{equation}



For selecting leptons the minimium requirment is called VeryLoose, the criteria defining VeryLoose is defined in Table X.  The set of VeryLoose leptons is further subdivided by quality into three mutually exclusive categories: Gold, Silver, and Bronze. Each category has a measure of three main quantities: the quality of the pre-determined id (citation about ID here), the "promptness" or distance of the lepton production point from a primary interaction point, and the isolation which is a measure of the density of particles in a cone around the lepton.  Aside from the lepton ID, promptness is measured by the significance of the 3D impact parameter and isolation is measured in two similar quantities PFIso and MiniIso. Both isolations are an energy sum of neighboring particles inside a cone. PFIso has a fixed cone size of 0.5? and miniIso cone sizes varies inversely with lepton \pt

\begin{table}[htbp]
\centering
\caption{\label{tab:veryloose} The collection of criteria that define the minimum requirements for a lepton object to be considered a lepton. The electron and muon requirements are equivalent in terms of pseudorapidity, vertexing, and isolation but vary in \pt threshold and the MVA VLooseFO working point. The MVA VLooseFO ID is also dependent on year. TODO: REMAKE CAPTION }

\begin{tabular}{c|c|c}
\hline
Criteria & Electron & Muon \\
\hline
\hline
\pt & $\geq 5$ GeV & $\geq 3$ GeV \\

$|\eta|$ & $<2.4$ & $<2.4$ \\
\hline

$\text{IP}_{3D}/\sigma_{\text{IP}_{3D}}$ & $<8$ & $<8$ \\

$|d_{xy}|$ & $<0.05$ cm & $<0.05$ cm \\

$|d_z|$ & $<0.1$ cm & $<0.1$ cm \\

\hline
$\text{PFIso}_{\text{abs}}$ & $<20 + (300/\pt)$ GeV & $<20 + (300/\pt)$ GeV \\

\hline
MVA VLooseFO ID & \checkmark  & --\\
\end{tabular}
\end{table}

\section{Electron Tag-and-Probe }

The electron tag and probe is done by using Z

\section{Muon Tag-and-Probe}

\section{Lepton Systematics}
