
\setcounter{secnumdepth}{3}
\setcounter{tocdepth}{3}
\setlength{\parskip}{\smallskipamount}
\setlength{\parindent}{0pt}


\makeatletter


\providecommand{\tabularnewline}{\\}


\makeatother

%\usepackage{babel}
%\begin{document}

\chapter{Compressed SUSY Search}

%\begin{chapterabstract}
%This chapter summarizes the approach for a compressed susy search and pertinent sensitive kinematic variables that the analysis is based on.
%\end{chapterabstract}

\section{Compressed Search Introduction}
In accordance with the strong experimental and phenomenological motivation for compressed spectra accompanied by the pursuit to comprehensively test SUSY, we conduct a search designed to be generic sensitive to many final states that involve missing energy. The targeted processes include, but are not limited to the production of stops, electroweakinos, and sleptons with diagrams included in Figure \ref{susyfeyn}.  
\FigFour{RJR_figs/T2tt.pdf}{RJR_figs/TChiWW.pdf}{RJR_figs/TChiWZ.pdf}{RJR_figs/TSlepSlep.pdf}{diagrams}{fig:susyfeyn}
The common thread between between all of these processes is a pair produced visible system alongside a massive invisible system. In the case of a compressed scenario, most of the energy available in the system is used by the rest mass of the LSP. These small mass splittings leads to low momemtum visible products that are difficult to reconstruct or are undetectable. In the case of intermediate massive particles, such as W or Z boson, these are forced off-shell so the visible products receive even less momentum, as the available energy goes into the mass of the intermediate particle. In order to identify these type of events we study cases with significant initial-state radiation (ISR). The ISR system recoils against, or boosts, the sparticle system, leading to high missing transverse momentum which is a tractable experimental signature. A depiction of this type of event is shown in Figure \ref{fig:isrrecoil}. With the ISR assisted pair produced topology we categorize and subdivide the visible and invisible system using Recursive Jigsaw Reconstruction (RJR) to approximate various rest frames. From those rest frames, we compute a basis of kinematic observables that describe features aid in the discrimination of compressed SUSY against SM processes.

\FigOne{RJR_figs/ISRrecoil.png}{recoiling system}{fig:isrrecoil}

%mentally all supersymmetry searches tend to exhibit their weakest limits in corridor regions with low mass differ-
%ences; if supersymmetry is to be tested comprehensively this region must also be explored. Phenomenologically,
%the lowest lying states in the electroweakino sector \\
%Given the lack of compelling conclusive evidence against supersymmetry, we conduct a broad search
%with many final states that involve missing energy. This is well motivated and does not rely strongly on theoretical
%assumptions.


%A compressed system is defined by a sparticle such as a neutralino 2 or stop in which the mass difference with this particle and the lightest supersymmetrical particle is small. The mass difference is considered small when the sparticle decays to intermediate standard model particles like W,Z,t such that the intermediate particle is forced off shell. For example the smallest targeted mass splittings can range between 3 to 10 GeV in neutralino 2 to W/Z decays. The intermediate decays will be difficult to detect or separate from standard model backgrounds. To assist in identifying compressed topologies we look for ISR assisted events. The signature of the event then becomes an ISR jet back to back with sparticle system which consists of mostly missing transverse energy from the LSP and soft SM particles.


\section{ RJR Reconstruction}
 The ISR assisted topology involves a collimated invisible and soft visible system together recoiling against another visible system of ISR jets. Each event is organized by imposing a decay tree onto the visible $(V)$ and invisible objects $(I)$ and assigning the visible components to either the ISR or sparticle $(S)$ side of the event. From the initial assignment, the sparticle system is further subdivided into subsystems A and B where each sub system has a visible $(V_{a/b})$ and invisible  $(I_{a/b})$ component. The three decay trees that contains these sets of objects in different reference frames are illustrated in Figure \ref{fig:decaytrees}. 
\FigThree{RJR_figs/tree_CM.pdf}{RJR_figs/tree_ISR.pdf}{RJR_figs/tree_S.pdf}{decay trees}{fig:decaytrees}
In order to resolve the two invisible four momenta, kinematic and combinatoric unknowns need to be estimated. For example, the visible products are indistinguishable, so, there needs to be a metric which dictates the assignment of the visible objects to either the ISR or sparticle system. Similarly, the sparticle subsystem partitioning, both visible and invisible, must be determined when only the transverse invisible momentum is known. Thus, the combinatoric assignment and the estimated four momenta depend on each other, so, we  apply a set of rules to simultaneously determine both.  RJR provides the framework to organize and evaluate each event. The set of rules used by this analysis are as follows:
\begin{itemize}
\item[1.] Assign charged leptons to the S system
\begin{itemize}
	\item Target leptonic sparticle signatures and fixing the leptonic system to always recoil against ISR
\end{itemize}
\item[2.] Fully determine the set of $\{V,ISR\}$ objects by assigning other visible objects to either the S or ISR systems by maximizing the momentum of the sparticle system in the CM frame
\begin{equation}
\{V,ISR\} =  \underset{V,ISR}{\arg\max} \, p_S^{CM}
\end{equation}
\item[3.] The visible S system objects are assigned to $V_a$ or $V_b$ by minimizing the mass of the sparticle subsystems i.e. grouping objects that traveling in similar directions
\begin{equation}
\{V_a,V_b\} = \underset{V_a,V_b}{\arg\min} \, M_{P_a}^2 + M_{P_b}^2
\end{equation}
\item[4.] Adjust the total mass of the invisible system according the visible systems where the individual invisible masses are constrained to zero.
\begin{equation}
M_I^2 = M_V^2 - 4M_{V_a}M_{V_b}
\end{equation}
\item[5.] Estimate the longitudinal component of invisible momentum by minimizing CM mass, i.e. determining the transverse mass of V+I systems
\begin{equation}
 \vec{\beta}_{CM,z}^{\text{lab}} = \underset{\vec{\beta}_{CM,z}^{\text{lab}} } {\arg\min} \, M_{CM}
\end{equation}
\item[6.] Determine the full kinematics of $I_A$ and $I_B$ and momentum partitioning  by evaluating the S frame velocities also through minimizing the mass of the sparticle subsystems
\begin{equation}
\vec{\beta}_{P_a}^S, \, \vec{\beta}_{P_b}^S = \underset{\vec{\beta}_{P_a}^S, \, \vec{\beta}_{P_b}^S}{\arg\min} \, M_{P_a}^2 + M_{P_b}^2
\end{equation}
\end{itemize}

By recursively iterating through the various combinations of objects and determining  the four momenta of the groupings from Figure \ref{fig:decaytrees}, the optimal organization for the event is determined that satisfies the aforementioned RJR prescription. The consequences of this organization is discussed in the following section where we construct a basis of kinematic variables to exploit characteristics of compressed SUSY and discriminate against SM processes.



%An isr assisted event is divided into multiple reference frames. The CM frame consists of the particles measured in the lab e.g. the isr jet against the met system. The sparticle frame consists two subsystems A and B. THe sparticles are expected to be pair produced if r parity is conserved.

%The kinematic variables the form basis of the search is RISR and MPERP.
%Risr. RISR is process independent and peaks at the ratio of sparticle/lsp masses. mperp is the transverse mass of the sparticle frame with respect to the sparticle frame boost axis.  
\section{Compressed Kinematics}

The main observables are designed to be sensitive to the properties of compressed SUSY. One of these properties is the mass of the invisible particle in the event. With the ISR assisted system recoiling against a massive invisible particle, the invisible gets a large momentum kick from ISR.  This characterisic can be exploited with the variable $R_{ISR}$ which is defined as:
\begin{equation}
R_{ISR} = \frac{|\vec{p}_I^{CM} \cdot \hat{p}_{ISR}^{CM}|}{|\vec{p}_{ISR}^{CM}|} \sim \frac{m_I}{m_P}
\end{equation}

In a compressed scenario, the fraction of the invisible momemtum to the total ISR kick would be expected to be close to one, as the visible system uses a small fraction of the momentum of the event and the ISR vs invisible system would be nearly anti parallel. The peak of the $R_{ISR}$ distribution can be approximated by the ratio of the true invisible mass to its sparticle parent mass.  This behavior of $R_{ISR}$ is that is spread in the distribution is reduced and the peak approaches one as the level of compression increases. Similarly the $R_{ISR}$ behavior doesn't depend specifically on the underlying SUSY process, but only that there is a heavy invisible system recoiling against ISR.  SM backgrounds do not exhibit the same behavior in $R_{ISR}$, so, the result is strong discrimination between compressed SUSY and SM at high $R_{ISR}$ which actually improves as the mass splittings $\Delta m = m_P - m_I$ decrease. An illustration of the $R_{ISR}$ shapes comparing signal to background is show in Figure \ref{fig:risrshape}

\FigTwo{RJR_figs/RISR_1L_T2tt.png}{RJR_figs/RISR_1L_bkg.png}{Risr distributions 1L selection di stop}{fig:risrshape}

Another observable which is uncorrelated with ISR but also sensitive to compressed topologies is $M_\perp$. $M_\perp$ is constructed from the average squared masses of the sparticle subsytems $M_{P_{a/b}}$ and is explicitly defined as
\begin{equation}
M_\perp = \sqrt{\frac{M_{P_a\perp}^2 + M_{P_b\perp}^2}{2}}
\end{equation}
The indivdual invisible masses are constrained to zero, and, in the case of massive invisible particles is an incorrect assumption. The consequence is then that $M_\perp$ is sensitive to the inherent mass splittings between the parent sparticle and LSP. The behavior of the $M_\perp$ distribution is that it exhibits a kinematic endpoint or edge at the $\Delta m = m_P - m_I$ and gives the strongest discrimination against SM processes at larger values of $M_\perp$, which may still be compressed e.g. in case of intermediate top quarks. An example of the shape of the $M_\perp$ distributions in with signal versus background is shown in Figure \ref{fig:mperpshape}

\FigTwo{RJR_figs/Mperp_0L_T2tt.png}{RJR_figs/Mperp_0L_bkg.png}{mperp shapes}{fig:mperpshape}

The combination of both $R_{ISR}$ and $M_\perp$ form a 2-D plane in which to conduct a "bump hunt" which is illustrated in Figure \ref{fig:risrperp2d}. Here the localization of the bump in the plane depends on the sparticle masses, the larger both masses the less over spread in the distribution. Also the smaller the mass splitting between sparticles, the peak of the distribution is pushed sharply towards 1 in $R_{ISR}$ and the stronger the discrimination against SM backgrounds becomes. Binning in this 2D plane gives us the most sensitive signal region at high $R_{ISR}$  while the opposite, low $R_{ISR}$ provides a background rich region to constrain the background yields in the sensitive region. 

\FigTwo{RJR_figs/RISR_v_Mperp_2L_TChiWZ_250_240.png}{RJR_figs/RISR_v_Mperp_2L_ttbar.png}{2d plots}{fig:risrmperp2d}


Two additional kinematic variables are utilized to aid in discrimination against SM backgrounds, both are less powerful than $R_{ISR}$ and $M_\perp$ but still very useful. The first quantity is complementary to $R_{ISR}$ and is the transverse momentum of the ISR system $p_T^{ISR}$. The more momentum in the ISR system means the sparticle system gets kicked harder - leading to better resolution in the $R_{ISR}$ distribution from the $p_T^{ISR}$ $R_{ISR}$ correlation. Fortunately the $R_{ISR}$  distributions from backgrounds are anti correlated with $p_T^{ISR}$ which means that the combination of both high  $p_T^{ISR}$ $R_{ISR}$ provides a rich signal region. Both correlations for signal and background can are visualized in Figure \ref{fig:ptisrrisr}

\FigTwo{RJR_figs/PTISR_v_RISR_2L_T2tt_500_480-gif-converted-to.pdf}{RJR_figs/PTISR_v_RISR_2L_Wjets-gif-converted-to.pdf}{ptisr risr distributions}{fig:ptisrrisr}

 The other kinematic variable is $\gamma_\perp$ which is a measure of the symmetry of the di-sparticle system and is defined as:
\begin{equation}
\gamma_\perp = \frac{2M_\perp}{M_{S\perp}}
\end{equation}
Here $M_S$ is the mass composed of the transverse four momenta of all objects from  both sparticle subsystems. The behavior of the mass ratio $\gamma_\perp$ is that it tends to larger values asymmetry in the final state, as illustrated in Figure \ref{fig:gammat}. This is useful because it better isolates signals and backgrounds with have pairs of W or Z bosons. Since both complementary variables $p_T^{ISR}$ and $\gamma_\perp$ don't have the discriminating power or $R_{ISR}$ and $M_\perp$ we categorize events. With $p_T^{ISR}$ we have high or low categories, where the lower edge and low to high pivot depends on the lepton and jet multiplicity of the event. In the case of $\gamma_\perp$ we also construct high and low categories about the value $\gamma_\perp= 0.5$. The combination of high and low and the divides selected for each pair of categories is designed to create more signal sensitive high regions with background yields that are constrained by low background rich categories.

\FigTwo{RJR_figs/gammaT_2L_bkg.png}{RJR_figs/gammaT_2L_TChiWZ-gif-converted-to.pdf}{gamT dist}{fig:gammat}
  


%The analysis is generalized to deal with a broad range of signal models but the three targeted compressed signal processes incldue stop, neutralino/chargino, slepton production. the signals include T2tt, T2bW, TChiWZ, TChiWW, TSlepSlep
