
\setcounter{secnumdepth}{3}
\setcounter{tocdepth}{3}
\setlength{\parskip}{\smallskipamount}
\setlength{\parindent}{0pt}


\makeatletter


\providecommand{\tabularnewline}{\\}


\makeatother

%\usepackage{babel}
%\begin{document}

\chapter{Compressed SUSY Search}

%\begin{chapterabstract}
%This chapter summarizes the approach for a compressed susy search and pertinent sensitive kinematic variables that the analysis is based on.
%\end{chapterabstract}

\section{ Data and Simulation}
The analysis involves the full Run II data-set which is divided by the years 2016,2017,2018 and has a cumulative integrated luminosity of $138 \text{fb}^{-1}$. Each year is comprised of $36.31 \, \text{fb}^{-1} \, \pm 1.2\%$ \cite{CMS:2021xjt}, $41.48 \, \text{fb}^{-1} \, \pm 2.3\%$ \cite{CMS:2018elu}, $59.83 \, \text{fb}^{-1} \, \pm2.5\%$ \cite{CMS:2019jhq} in 2016, 2017, and 2018 respectively. The data is modeled by MC that represents the full SM background and is qualitatively grouped by process and final state. These grouping of SM backgrounds is defined in table \ref{tab:bkgsigtab}.

%insert bg list and description
\begin{table}
\caption{Labels for SM background aggregation and relevant signal processes with their CMS shorthand labeling.}
\begin{tabular}{c|c}
\hline 
Bkg. Label & Bkg. Composition \\ 
\hline 
\hline

W + jets & \makecell{Single W boson, a dominant background that composes \\ about $50\%$ of the total background} \\ 
  & \\
tt+jets & \makecell{$t\bar{t}$ which can be accompanied by a W,Z,h, or $\gamma$, \\ the other dominant background composes about $50\%$ of the total background} \\ 
  & \\
ZDY & Z+jets and Drell Yan, an intermediate background \\ 
Di-boson (DB)& WW,ZZ,WZ,Wh,Zh,  an intermediate background \\ 

ST & Single top processes including tW, a rare background \\ 

Tri-boson (TB) & WWW,ZZZ,WWZ, WZZ, WZ$\gamma$, WW$\gamma$, a rare background  \\ 
\hline 
Signal Label & Signal Composition \\
\hline
\hline
T2tt & $pp \rightarrow \tilde{t} \tilde{\bar{t}}; \, \, \tilde{t}\rightarrow t \tilde{\chi}_1^0$ \\
T2bW & $pp \rightarrow \tilde{t} \tilde{\bar{t}}; \, \, \tilde{t}\rightarrow b \tilde{\chi}_1^\pm; \, \, \tilde{\chi}_1^\pm \rightarrow W^\pm \tilde{\chi}^0_1$ \\
TChiWZ & $pp\rightarrow \tilde{\chi}_2^0 \tilde{\chi}_1^\pm; \, \, \tilde{\chi}_2^0 \rightarrow Z \tilde{\chi}^0_1; \, \, \tilde{\chi}_1^\pm \rightarrow W^\pm \tilde{\chi}^0_1$ \\
TSlepSlep & $pp\rightarrow \tilde{\ell} \tilde{\ell}; \, \, \tilde{\ell} \rightarrow \ell \tilde{\chi}^0_1$ \\
TChipmWW & $pp\rightarrow \tilde{\chi}_1^\pm \tilde{\chi}_1^\mp; \, \,  \tilde{\chi}_1^\pm \rightarrow W^\pm \tilde{\chi}^0_1$  \\
\end{tabular} \\
\label{tab:bkgsigtab}
\end{table}

The signals being addressed in this search include multiple sparticle processes and final states. A list of the signals and their CMS shorthand that will be used throughout this text is also provided in table \ref{tab:bkgsigtab}. Each signal is produced according to an mass grid for each year.  The raw number of events per mass points and grid spacings for the signals shown are displayed in Figure \ref{fig:grids} with all years combined. 

%insert signal list and description
\FigFive{Analysis_figs/T2tt_EventCount.pdf}{Analysis_figs/TChiWZ_EventCount.pdf}{Analysis_figs/TSlepSlep_EventCount.pdf}{Analysis_figs/TChipmWW_EventCount.pdf}{Analysis_figs/TSlepSlep_EventCount.pdf}{Mass grids and associated numbers of events per grid point for each the SUSY processes T2tt, T2bW, TChiWZ, TChipmWW, and TSlepSlep.}{fig:grids}

The majority of signal and backgrounds use the MadGraph \cite{Alwall:2011uj} generator to model at LO and NLO. The ST backgrounds use PowHEG 2.0 \cite{Alioli:2010xd} to model at NLO. Parton shower and fragmentation for all samples is done with PYTHIA 8 \cite{Sjostrand:2014zea}. Each year is subjected to an underlying event tune with CUETP8M1 for 2016, CP2 for 2017 and 2018 signals and CP5, for 2017 and 2018 backgrounds \cite{CMS:2015wcf}\cite{CMS:2019csb}. The detector conditions and response are simulated for all samples with GEANT4 \cite{GEANT4:2002zbu}.

\section{Compressed Signal Overview}
In accordance with strong experimental and phenomenological motivation for compressed spectra and the intention to comprehensively test SUSY, we conduct a search designed to be generically sensitive to many compressed final states. The kinematic techniques in this chapter is based largely on the previous work by C. Rogan and the Recursive Jigsaw Reconstruction (RJR) framework \cite{AN}\cite{PhysRevD.96.112007}. The targeted processes include, but are ultimately not limited to the production of stops, electroweakinos, and sleptons whose diagrams are included in Figure \ref{fig:susyfeyn}.  
\FigFive{RJR_figs/T2tt.pdf}{RJR_figs/T6bbWW.pdf}{RJR_figs/TChiWW.pdf}{RJR_figs/TChiWZ.pdf}{RJR_figs/TSlepSlep.pdf}{Feynman diagrams for SUSY pair produce processes. Top row involves di-stop pairs where the intermediate state on the top left undergoes stop to top decay and the top right instead undergoes a stop to chargino decay. The middle left diagram show Chargino pairs decaying to two W boson and the middle right is a Neutralino and Chargino pair which decay into a W and Z boson final state. The bottom diagram shows di-slepton production with sleptons decaying directly to SM leptons}{fig:susyfeyn}
Each of the processes in \ref{fig:susyfeyn} has a pair produced visible system alongside a massive invisible system. For these systems to be compressed, most of the energy available in the sparticle decay is used by the rest mass of the LSP. These small mass splittings lead to low momentum visible products that are difficult to reconstruct or are undetectable. In the case of intermediate massive particles, such as W or Z boson, these are forced off-shell so the visible products receive even less momentum because the available energy goes into the mass of the intermediate particle. In order to identify these type of events we study cases with significant initial-state radiation (ISR). The ISR system recoils against, or boosts, the sparticle system leading to high missing transverse momentum which is a tractable experimental signature. A depiction of this type of event is shown in Figure \ref{fig:isrrecoil} with objects grouped into either the ISR or sparticle system. The initial sparticle system is then further divided into two sparticle rest frames according to RJR rules. From these rest frames, we compute a basis of kinematic observables that aid in the discrimination of compressed SUSY against SM processes.

\FigOneScale{RJR_figs/ISRrecoil.png}{Illustration of a visible ISR system recoiling against a Sparticle system which can be decomposed into soft visible subsystem and massive invisible subsystem}{fig:isrrecoil}{0.4}

%mentally all supersymmetry searches tend to exhibit their weakest limits in corridor regions with low mass differ-
%ences; if supersymmetry is to be tested comprehensively this region must also be explored. Phenomenologically,
%the lowest lying states in the electroweakino sector \\
%Given the lack of compelling conclusive evidence against supersymmetry, we conduct a broad search
%with many final states that involve missing energy. This is well motivated and does not rely strongly on theoretical
%assumptions.


%A compressed system is defined by a sparticle such as a neutralino 2 or stop in which the mass difference with this particle and the lightest supersymmetrical particle is small. The mass difference is considered small when the sparticle decays to intermediate standard model particles like W,Z,t such that the intermediate particle is forced off shell. For example the smallest targeted mass splittings can range between 3 to 10 GeV in neutralino 2 to W/Z decays. The intermediate decays will be difficult to detect or separate from standard model backgrounds. To assist in identifying compressed topologies we look for ISR assisted events. The signature of the event then becomes an ISR jet back to back with sparticle system which consists of mostly missing transverse energy from the LSP and soft SM particles.


\section{ RJR Methodology}
 The ISR assisted topology involves a collimated invisible and soft visible system recoiling against another visible system of ISR jets. Each event is organized by imposing a decay tree onto the visible $(V)$ and invisible objects $(I)$ and assigning the visible components to either the ISR or sparticle $(S)$ side of the event. From the initial assignment, the sparticle system is further subdivided into subsystems A and B where each sub system has a visible $(V_{a/b})$ and invisible  $(I_{a/b})$ component. The three decay trees that contains these sets of objects in and reference frames are illustrated in Figure \ref{fig:decaytrees}. 
\FigThreeScale{RJR_figs/tree_CM.pdf}{RJR_figs/tree_ISR.pdf}{RJR_figs/tree_S.pdf}{A depiction of the three reference frames used in this RJR analysis. The top left shows the lab frame which is composed of the visible and invisible systems. The top right shows the CM frame in which the visible objects are divided into either ISR or visible objects associated with the sparticle system. The bottom figure further breaks down the sparticle system into a system composed of sparticle pairs of which both have their own visible and invisible subsystem.}{fig:decaytrees}{0.4}{0.4}{0.4}
In order to resolve the two invisible four momenta, kinematic and combinatoric unknowns need to be estimated. The visible products are indistinguishable, so, a metric must dictates the assignment of the visible objects to either the ISR or sparticle system. Similarly, the object partitioning, both visible and invisible, must be determined when only the transverse invisible momentum is known. To fully solve the system the objects need to beg assigned and the momentum of each needs to be estimated. The combinatoric assignment and the estimated four momenta depend on each other, so, we  apply a set of rules to simultaneously determine both.  RJR provides the framework and rules to organize and evaluate each event \cite{PhysRevD.96.112007}. The set of rules used by this analysis are as follows :
\begin{itemize}
\item[1.] Assign charged leptons to the S system
\begin{itemize}
	\item Targets leptonic sparticle signatures and fixing the leptonic system to always recoil against ISR
\end{itemize}
\item[2.] Fully determine the set of $\{V,ISR\}$ objects by assigning other visible objects to either the S or ISR systems by maximizing the momentum of the sparticle system,$p_S^{CM}$ , in the CM frame
\begin{equation}
\{V,ISR\} =  \underset{V,ISR}{\arg\max} \, p_S^{CM}
\end{equation}
\item[3.] The visible S system objects are assigned to $V_a$ or $V_b$ by minimizing the mass of the sparticle subsystems i.e. grouping objects that traveling in similar directions
\begin{equation}
\{V_a,V_b\} = \underset{V_a,V_b}{\arg\min} \, M_{P_a}^2 + M_{P_b}^2
\end{equation}
\item[4.] Adjust the total mass of the invisible system according to the visible systems with the individual invisible masses set to zero.
\begin{equation}
M_I^2 = M_V^2 - 4M_{V_a}M_{V_b}
\end{equation}
\item[5.] Estimate the longitudinal component of invisible momentum by minimizing CM mass, i.e. determining the transverse mass of V+I systems
\begin{equation}
 \vec{\beta}_{CM,z}^{\text{lab}} = \underset{\vec{\beta}_{CM,z}^{\text{lab}} } {\arg\min} \, M_{CM}
\end{equation}
\item[6.] Determine the full kinematics of $I_A$ and $I_B$ and momentum partitioning  by evaluating the S frame velocities through minimizing the mass of the sparticle subsystems
\begin{equation}
\vec{\beta}_{P_a}^S, \, \vec{\beta}_{P_b}^S = \underset{\vec{\beta}_{P_a}^S, \, \vec{\beta}_{P_b}^S}{\arg\min} \, M_{P_a}^2 + M_{P_b}^2
\end{equation}
\end{itemize}

By recursively iterating through the various combinations of objects and determining  the four momenta of the groupings from Figure \ref{fig:decaytrees}, the optimal organization for the event is determined that satisfies the aforementioned RJR prescription. The consequences of this organization is discussed in the following section where we construct a basis of kinematic variables to exploit characteristics of compressed SUSY and discriminate against SM processes.



%An isr assisted event is divided into multiple reference frames. The CM frame consists of the particles measured in the lab e.g. the isr jet against the met system. The sparticle frame consists two subsystems A and B. THe sparticles are expected to be pair produced if r parity is conserved.

%The kinematic variables the form basis of the search is RISR and MPERP.
%Risr. RISR is process independent and peaks at the ratio of sparticle/lsp masses. mperp is the transverse mass of the sparticle frame with respect to the sparticle frame boost axis.  
\section{Compressed Kinematics}

The main observables are designed to be sensitive to the properties of compressed SUSY which is that there are massive invisible particles in the event. The ISR assisted system recoiling against a massive invisible particle boosts the momentum of the invisible system.  This feature can be exploited through a ratio of these momenta, defined as $R_{ISR}$:
\begin{equation}
R_{ISR} = \frac{|\vec{p}_I^{CM} \cdot \hat{p}_{ISR}^{CM}|}{|\vec{p}_{ISR}^{CM}|} \sim \frac{m_I}{m_P}
\end{equation}

In a compressed scenario, the fraction of the invisible momentum to the total ISR kick is expected to be near one due to the visible system using a small fraction of the momentum and making the ISR and invisible system anti-parallel. The peak of the $R_{ISR}$ distribution can be approximated by the ratio of the true invisible mass, $m_I$ to its sparticle parent mass, $m_P$.  The $R_{ISR}$  width contracts with a mean approaching one as the level of compression increases due to the diminishing available phase space for the visible sparticle objects that can offset the $R_{ISR}$ ratio. The $R_{ISR}$ behavior doesn't depend specifically on an underlying SUSY process, but only that there is a heavy invisible system recoiling against ISR.  SM backgrounds do not exhibit the same behavior in $R_{ISR}$ so, the result is strong discrimination between compressed SUSY and SM at high $R_{ISR}$.  This discrimination actually improves as the mass splittings $\Delta m = m_P - m_I$ decreases or as $m_P$ increases such that the ratio $m_I/m_P$ can get closer to one. An illustration of the $R_{ISR}$ shapes comparing signal to background is show in Figure \ref{fig:risrshape}

\FigTwoScale{RJR_figs/RISR_1L_T2tt.png}{RJR_figs/RISR_1L_bkg.png}{$R_{ISR}$ shapes with a basic 1 lepton selection for stop pairs on the left comparing  a range of mass splittings. The right distribution is the $R_{ISR}$ shape for the components of the full SM background with the same basic 1 lepton selection}{fig:risrshape}{0.49}{0.49}

Another observable which is uncorrelated with ISR but also sensitive to compressed topologies is $M_\perp$. $M_\perp$ is constructed from the average squared masses of the sparticle subsystems $M_{P_{a/b}}$ and is explicitly defined as
\begin{equation}
M_\perp = \sqrt{\frac{M_{P_a\perp}^2 + M_{P_b\perp}^2}{2}}
\end{equation}
The individual invisible masses are set to zero, and is an incorrect assumption in the case of massive invisible particles. The consequence is then that the boost to the sparticle rest frames is sensitive to the inherent mass splittings between the parent sparticle and LSP. The behavior of the $M_\perp$ distribution is that it exhibits a kinematic endpoint or edge at the $\Delta m = m_P - m_I$ and gives the strongest discrimination against SM processes at larger values of $M_\perp$ around 100 GeV, which is still be compressed in cases associated with top quarks. An example of the shape of the $M_\perp$ distributions in with signal versus background is shown in Figure \ref{fig:mperpshape}

\FigTwoScale{RJR_figs/Mperp_0L_T2tt.png}{RJR_figs/Mperp_0L_bkg.png}{$M_\perp$ shapes with no leptons for di stop pairs on the left comparing a range of mass splittings. The right distributions is the $M_{\perp}$ shape for the components of the full SM background which also is absent of reconstructed leptons. }{fig:mperpshape}{0.49}{0.49}

The combination of both $R_{ISR}$ and $M_\perp$ form a mass sensitive 2D plane in which to conduct a bump hunt. An example showing the 2D localizations of signal and background is illustrated in Figure \ref{fig:risrperp2d} where the most sensitive region is high $R_{ISR}$ and low $M_\perp$ with sensitivity that improves as mass splittings decrease. Not only is the sensitive signal region at high $R_{ISR}$ useful, the low $R_{ISR}$ region provides a high statistics background rich region to constrain the background yields in the sensitive region and cross-constrain processes in different categories. 

\FigTwoScale{RJR_figs/RISR_v_Mperp_2L_TChiWZ_250_240.png}{RJR_figs/RISR_v_Mperp_2L_ttbar.png}{2D distributions of $R_{ISR}$ and ${M_\perp}$ with a basic 2 lepton selection for Neutralino-Chargino pairs with LSP and NLSP mass splitting of 10 GeV on the left and tt+jets background on the right. The bump from the left distribution develops a more resolved peaked and better signal to background at its center in cases with increasing compression.}{fig:risrmperp2d}{0.49}{0.49}


Two additional kinematic variables are utilized to aid in discrimination against SM backgrounds, both are less powerful than $R_{ISR}$ and $M_\perp$ but still very useful. The first quantity is complementary to $R_{ISR}$ and is the transverse momentum of the ISR system $p_T^{ISR}$. The more momentum in the ISR system the larger the sparticle boost - leading to better resolution in the $R_{ISR}$ distribution. Fortunately the $R_{ISR}$ distributions from backgrounds are anti-correlated with $p_T^{ISR}$, so an event with the combination of both high  $p_T^{ISR}$ $R_{ISR}$ has the highest sensitivity. Both correlations for signal and background are visualized in Figure \ref{fig:ptisrrisr}

\FigTwoScale{RJR_figs/PTISR_v_RISR_2L_T2tt_500_480-gif-converted-to.pdf}{RJR_figs/PTISR_v_RISR_2L_Wjets-gif-converted-to.pdf}{2D distributions with $R_{ISR}$ versus $p_T^{ISR}$ with simple 2 lepton selection and compressed di stop model with a 20 GeV mass splitting on the left and W+jets background on the right. }{fig:ptisrrisr}{0.49}{0.49}

 The other complementary kinematic variable used in this search is $\gamma_\perp$,  a measure of the symmetry of the di-sparticle system and is defined as:
\begin{equation}
\gamma_\perp = \frac{2M_\perp}{M_{S\perp}}
\end{equation}
$M_S$ is the mass of the transverse four momenta of all objects in the sparticle. The behavior of $\gamma_\perp$ is that it tends to larger values for events with asymmetry in the final state, as illustrated in Figure \ref{fig:gammat}. This is useful for isolating signals and backgrounds with semi-leptonic decays of pairs of $W$ or $Z$ bosons. Since both complementary variables $p_T^{ISR}$ and $\gamma_\perp$ don't have the discriminating power of $R_{ISR}$ and $M_\perp$ they are used to categorize events. The $p_T^{ISR}$ categories are separated into high or low with the lower edge and low to high edge depending on the lepton and jet multiplicity. For $\gamma_\perp$ categorization we also construct high and low categories around $\gamma_\perp= 0.5$. The combination of high and low and the divides selected for each pair of categories is designed to create more signal sensitive high regions with background yields that are constrained by low background rich categories.

\FigTwoScale{RJR_figs/gammaT_2L_TChiWZ-gif-converted-to.pdf}{RJR_figs/gammaT_2L_bkg.png}{Distributions of $\gamma_\perp$ for Neutralino and Chargino pairs over a range of mass splittings on the left and the comparison of SM background components on the right.}{fig:gammat}{0.49}{0.49}
  


%The analysis is generalized to deal with a broad range of signal models but the three targeted compressed signal processes incldue stop, neutralino/chargino, slepton production. the signals include T2tt, T2bW, TChiWZ, TChiWW, TSlepSlep
