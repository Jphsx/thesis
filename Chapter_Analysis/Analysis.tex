
\setcounter{secnumdepth}{3}
\setcounter{tocdepth}{3}
\setlength{\parskip}{\smallskipamount}
\setlength{\parindent}{0pt}


\makeatletter


\providecommand{\tabularnewline}{\\}


\makeatother

%\usepackage{babel}
%\begin{document}

\chapter{Analysis Description }

\section{Introduction and Strategy}
The full analysis is built on the compressed kinematics and RJR strategy described in the previous chapter. The goal of this chapter is to build on this strategy with specific details about the objects and categorization used to potentially discover SUSY. This includes the description of events selected to analyze and their associated physics objects. This being a general search, it casts a wide net to capture a variety of signatures and final states with the consequence being a large number of categories and bins. Finally, I will discuss the data driven strategy to constrain and predict background events in the most sensitive regions with a series of fits to construct and test a fit model.

 
\section{Event Selection and Physics Objects}
 For events to be qualified for analysis, they pass a handful of selected triggers and a preselection that reflects the compressed kinematic description provided in the previous chapter. The triggers used PFMET and PFMHT cross triggers. PFMET is particle flow missing transverse energy which is expected to capture the $p_T^{\text{miss}}$ from the LSP. The threshold for the PFMET trigger is that the missing transverse energy is above 120 GeV. The definition of missing momentum is the negative sum of all visible momentum and is as follows: 
\begin{equation}
p_T^{\text{miss}} = -\sum{p_T^{\text{vis}}}
\end{equation} 
PFMHT is expected to trigger on events with significant jet activity which are likely candidates for ISR events. The threshold for PFMHT is that the scalar sum of all visible transverse momentum is above 120 GeV.  In general, the compressed SUSY topology is a somewhat rare organization of an event, due to the uncommon nature of these events, the MC modeling does not always sufficiently and precisely describe data, so,  a data driven approach is utilized for physics objects to compensate for any disagreement. This approach compares the overall efficiency or behavior of selected objects and computes data driven scale factors, while also providing a platform to model and understand systematic effects. These scale factor calibrations are then applied to MC to bring data and MC into agreement. The instance of scale factor generation arises in the modeling of the trigger efficiency, with efficiency defined as events that pass the trigger and preselection versus only preselection. The comparisons of data and MC missing $E_T$ efficiency are shown in Figure \ref{fig:metsf} with the efficiency shapes modeled by a Gaussian CDF.  

%insert trigger turn ons
\FigOneScale{Analysis_figs/SF_Plot_HT-Le600--SingleMuontrigger-E1--Nmu-E1_SingleMuon_2016.pdf}{Example MET trigger efficiency comparison for Single Muon triggers between data and MC  \cite{AN}.}{fig:metsf}{0.75}

The preselection is a set of carefully studied criteria to remove background and mismodeled events while selecting events with objects associated with signals. The preselection consists of the criteria listed in Table \ref{tab:presel} and introduces three quantities $p_T^{\text{CM}}$ which is the vector sum of the transverse momentum of the CM frame, $\Delta \phi_{\vec{p}_T^{\text{miss}}, V}$ the angle between the visible and invisible system, and $\Delta\phi_{CM,I}$ the angle between $p_T^{\text{CM}}$ and the invisible system.

%insert preselection table  
\begin{table}
\caption{Kinematic and combinatorial event requirements.}
\begin{tabular}{c|c}
\hline 
\multicolumn{2}{|c|}{Preselection Requirements} \\ 
\hline 
Criteria & Description \\ 
\hline 
\hline
$N_V \geq 1$ & At least one visible object assigned to the S system \\ 
$N_j^{ISR} \geq 1$ & At least one jet assigned to the ISR system \\ 

$p_T^{\text{miss}} > 150$ GeV &\makecell{ Minimum transverse missing energy based on trigger efficiency}  \\ 

$p_T^{ISR} > 250 $ GeV & \makecell{Minimum ISR kick to resolve massive invisible particles} \\ 

$R_{ISR}$ > 0.5 & Target Massive LSPs \\ 

$|\Delta \phi_{\vec{p}_T^{miss}, V}| < \pi/2$ &  Ensures visible and invisible system are traveling in the same direction \\ 

$p_T^{CM} < 200$ GeV  & Rejects mismodeled events \\ 

veto $f(\Delta\phi_{CM,I}, p_T^{CM})$& 2D function to also reject mismodeled events - See Fig \ref{fig:cleancut}\\
\hline 
\end{tabular} \\
\label{tab:presel}
\end{table}

\FigOneScale{Analysis_figs/cleaningCuts.pdf}{Cleaning cuts designed to veto events in the red regions with significantly large data to MC $R_{ISR}$ ratios. The accepted region corresponds to events passing the last two criteria of Table \ref{tab:presel} \cite{AN}.}{fig:cleancut}{0.75}

The preselection forms the basis for an event to be analyzed. Following preselection, the physics objects can be selected, classified, and categorized. The possible object composition can consist of jets, b-tagged jets, soft secondary vertices (SVs), and leptons. The discussion of the leptons and their classification will be reserved for the following chapter alongside the calculation of lepton scale factors. A summary of these physics objects and their kinematic requirements are listed in Table \ref{tab:physicsobjects}. The AK4 jets are clustered with the anti-$k_t$ algorithm and $\Delta R = 0.4$ \cite{Cacciari:2008gp} and selected based on working points (WP) defined by their physics object group which are standard objects used in CMS physics analysis \cite{CMS:2010xta}. The b-tagging is done by the MVA based DeepJet NN \cite{Stoye:2018qgr} which only identifies b-jets $p_T \geq 20$ GeV. SVs above 20 GeV are tagged using Inclusive Secondary Vertex Finder \cite{CMS:2011yuk}. A complementary SV tagger was developed which efficiently extends the SV tagging range $2 \leq p_T \leq 20$ GeV for final state topologies with very soft b-jets \cite{erich}.


\begin{table}
\centering
\caption{Kinematic requirements and working points for visible physics objects.}
\begin{tabular}{c|c}
\hline 
\multicolumn{2}{c}{Visible Physics Objects} \\ 
\hline 
\hline
Jets & \makecell{AK4 PF Jets \\ Tight ID \\ $p_T^{jet} > 20$ GeV \\ $|\eta| <2.4$} \\ 
\hline
B-tagged Jets & \makecell{AK4 PF Jets \\ DeepJet Medium WP \\ $p_T^{b-jet} > 20 $ GeV}  \\ 
\hline
SVs & $2<p_T^{SV}<20$ GeV \\ 
\hline
Leptons & \makecell{Very Loose ID \\ $p_T^{\mu^\pm} > 3$ GeV \\ $p_T^{e^\pm} > 5 $ GeV \\ Gold/Silver/Bronze quality classes} \\ 
\hline 
\end{tabular} 
\label{tab:physicsobjects}
\end{table}

\section{Categorization}

Once an event passes the preselection and the leptons are classified, the event is then categorized based on object composition, object multiplicity, and kinematic characteristics. The goal of categorization is to finely split up background and create generically sensitive regions for any type of signal. The regions with no signal function as control regions and constrain the background prediction in the sensitive regions. The most fundamental categories for the analysis are the lepton and jet multiplicities, events can contain either 0, 1, 2, or 3 leptons and are accompanied by a range of sparticle system jets (S jets) from $0\leq N_{S_{jet}} \leq 5$ where the upper limit of S jet counting is dependent on the lepton multiplicity. The type of jets are also counted i.e. whether or not the jets have been tagged as b-jets. The b-jet counting is dependent on the lepton multiplicity and restricted by the maximum number of S jets per jet multiplicity category.  ISR system can have jets can have 0 or $\geq 1$ b-jets and the S system can be composed of 0, 1, or $\geq2$ b-jets. SV tagging is similar to ISR jet counting in the S system, but, SVs are not counted in the ISR system. Events with SVs are further classified based on their orientation being either central or forward with respect to $|\eta|=1.5$ with the forward range at of maximum of $|\eta|<2.4$. The counting of b-jets and SVs is very important because it creates categories that isolate tt+jets and stop signals and similarly the opposite regions are created with no b-jets that isolate backgrounds like W+jets and signals such as TChiWZ. The benefit of having different object counting regions is that they can cross-constrain each other and is illustrated in Figure \ref{fig:bcount}, showing the presence and absence of tt+jets or W+jets backgrounds based on S system b-jet counting.
\FigTwo{Analysis_figs/NbS_v_NbISR_0L_ttbar-gif-converted-to.pdf}{Analysis_figs/NbS_v_NbISR_1L_Wjets-gif-converted-to.pdf}{Example of relative background presence for tt+jets (left) and W+jets (right) with b-tag counting.}{fig:bcount}

Aside from jet counting, there is lepton and kinematic categorization. The electrons and muons are separated by flavor, charge, quality, and $Z/\text{no} Z$ candidate.  A $Z$ candidate is defined as an opposite sign, same flavor pair (OSSF) in events with 0 S jets and OSSF in the same A or B hemisphere in events with S jets. A compressed scenario expects an off-shell $Z$ so there is no mass requirement on the OSSF lepton pair in a $Z$ category. The complementary compressed kinematic observables $p_T^{ISR}$ and $\gamma_\perp$ both have a high and low category. A complete table of all the categories is included in Table \ref{tab:cats}. 

\begin{table}
\caption{Organization of categories across all lepton multiplicities. The bracketed jet ranges imply counting for all integer numbers of jets within the inclusively listed edges. The b-tag counting is limited based on the allowed number of jets in the category. The SV and $\gamma_\perp$ splitting indicates two categories of either forward and central $\eta$ or low and high, respectively. The numbers listed in $p_T^{ISR}$ define the low $p_T^{ISR}$ category edges while the high $p_T^{ISR}$ bin is inclusive for everything above the low bin upper edge. The checkmarks indicate regions that are split by either $\gamma_\perp$ or $SV_\eta$. }
\begin{adjustwidth}{-0.7in}{-1in}

\begin{tabular}{|c|c|c|c|c|c|c|c|c|c|}
\multicolumn{7}{c|}{Obj. Combinatorics} & \multicolumn{3}{c}{Obj. Kinematics} \\
\hline 
$N_\ell$ &  $\ell$ Type & $\ell$ Quality & $N_{jets}^{S}$ & $N_{b-tag}^{S}$ & $N_{SV}^S$ & $N_{b-tag}^{ISR}$ & $SV_\eta$ & $\gamma_\perp$ & $p_T^{ISR}$ \\ 
\hline
\hline 
0 & - & - & 0 & 0 & $[1,\geq 2]$ & - & $\checkmark$ & - & $\geq 350$ \\ 
0 & - & - & 1 & - & $\geq 1$ &    - & $\checkmark$ & - & $\geq 400$ \\
0 & - & - & 1 & - &  0       &    - &  -           & - & $[400,\geq 550]$ \\
0 & - & - & $[2,\geq 5]$ & $[0,\geq 2]$ & 0 & $[0,\geq 1]$ & - & $\checkmark$ & $[350,\geq500]$ \\
\hline 
1 & $e^+,e^-,\mu^+,\mu^-$& Gold & 0 & 0 & 0 & $[0,\geq 1]$ & - & $\checkmark$ & $[350, \geq 500]$ \\
1 & $\ell^+, \ell^-$  & Gold & 0 & 0 & 1 & - & $\checkmark$ & - & $\geq 350$ \\
1 & $e, \mu$ & Gold & 1 & 0 &  $\geq 1$ & - & $\checkmark$ & - & $\geq 350$ \\
1 & $\ell$ & Gold & $[1,\geq 4]$ & $[0,\geq 2]$ & 0 & $[0,\geq 1]$ & - & $\checkmark$ & $[350, \geq 500]$ \\
1 & $e, \mu$ & Silver/Bronze & $[0,\geq 1]$ & 0 & 1 & - & $\checkmark$ & - & $\geq 350$ \\
1 & $e, \mu$ & Silver/Bronze & $[2, \geq 4]$ & - & - & - & - & $\checkmark$ & $[350, \geq 500]$ \\
\hline
2 & $e^\pm e^\mp, \mu^\pm \mu^\mp, e^\pm \mu^\mp $ & Gold & 0 & 0 & 0 & $[0,\geq 1]$ & - & $\checkmark$ & $[250,\geq350]$  \\
2 & $Z, no Z$ & Gold & $[1,\geq 2]$ & $[0, \geq 1]$& - & $[0,\geq 1]$ & - & $\checkmark$ & $[250,\geq350]$  \\
2 & $\ell^\pm \ell^\pm$ & Gold & $[0,\geq 2]$ & - & - &  - & - & - & $[250,\geq350]$ \\
2 & $ee, \mu\mu, e\mu$ & Silver/Bronze & $[0,\geq 2]$ & - & - & - & - & - & $\geq350$ \\
2 & $\ell \ell$ & Gold/Silver/Bronze & 0 & 0 & $\geq 1 $ & - & $\checkmark$ & - & $\geq 250$\\
\hline
3 & $Z, no Z$ & Gold/Silver/Bronze & $[0,\geq 1]$ & - & - & - & - & - & $\geq 250$ \\
3 & $\ell^\pm \ell^\pm \ell^\pm$ & Gold/Silver/Bronze & - & - & -& - & - & -& $\geq 250$ \\
\hline
\end{tabular} 
\end{adjustwidth}
\label{tab:cats}
\end{table}



Each category is divided into a 2D set of $(R_{ISR}, M_\perp)$ bins. The number of bins and bin edges are optimized for each combination of lepton and jet multiplicity. An example binning of the sensitive variables with the total SM background in each lepton multiplicity is shown in Figure \ref{fig:kinbin} and a full table of all binnings is included in Table \ref{tab:allkinbin}.
\FigFour{Analysis_figs/0L_2J_totalBkg.png}{Analysis_figs/1L_G_1J_1sv_totalBkg.png}{Analysis_figs/2L_G_0J_totalBkg.png}{Analysis_figs/3L_B_1J_totalBkg.png}{The dotted lines show example 2D $R_{ISR}$ and $M_\perp$ binning for each lepton multiplicity with the full SM background.}{fig:kinbin}

% insert RISR MP bins table here
\begin{table}
\caption{The complete set of $R_{ISR}$ and $M_\perp$ bin edges. A set of 2D bins have been designed for each allowed lepton and jet multiplicity.}
\begin{adjustwidth}{-0.5in}{-1in}
%\begin{singlespace}
\begin{tabular}{c|cc|cc|cc|cc|}
- & \multicolumn{2}{c}{0L} & \multicolumn{2}{c}{1L} & \multicolumn{2}{c}{2L} & \multicolumn{2}{c}{3L}   \\
\hline
$N_{jets}^S$ & $R_{ISR}$ & $M_\perp$ & $R_{ISR}$ & $M_\perp$ & $R_{ISR}$ & $M_\perp$ & $R_{ISR}$ & $M_\perp$ \\
\hline
0 & [0.95,0.985] & [$\geq 0$]  &[0.9,0.96] & [$\geq 0$]  &[0.6,0.7] & [$0,\geq 50$]  &[0.6,0.7] & [$\geq 0$]  \\
0 & [0.985,1] & [$0,5,\geq 10$]  &[0.96,0.98] & [$0,\geq 10$]  & [0.7,0.8]  & [$0,\geq 40$] &[0.7,0.8] & [$\geq 0$]  \\
0 &  &  &[0.98,1] & [$0,5,\geq 10$]  &  [0.8,0.9]  & [$0,\geq 30$] &[0.8,0.9] & [$\geq 0$]  \\
0 &  &   &        &                &[0.9,0.95] &  [$0,\geq 20$]  &[0.9,1] & [$\geq 0$]  \\
0 &   &   &   &   & [0.95,1] & [$0,\geq 15$]  & &   \\
\hline
1 & [0.8,0.9] & [$\geq 0$]  &[0.65,0.75] & [$0,\geq 50$]  &[0.5,0.6] & [$0,\geq 100$]  &[0.55,0.7] & [$\geq 0$]  \\
1 & [0.9,0.93] & [$0,\geq 20$]  &[0.75,0.85] & [$0,\geq 40$]  &[0.6,0.7] & [$0,\geq 80$]  &[0.7,0.85] & [$\geq 0$]  \\
1 & [0.93,0.96] & [$0,\geq 20$]  & [0.85,0.9] & [$0,\geq 30$] &[0.7,0.8]  & [$0,\geq 60$]  &[0.85,1] & [$\geq 0$]  \\
1 & [0.96,1] & $0,\geq 15$]   &[0.9,0.95] & [$0,\geq 20$]   &[0.8,0.9] & [$0,\geq 40$]   & &   \\
1 &  &   &[0.95,1]   & [$\geq 0$]   &[0.9,1]& [$0,\geq 30$]   & &   \\
\hline
2 & [0.65,0.75] & [$0,\geq 50$]  &[0.55,0.65] & [$0,\geq 110$]  &[0.5,0.65] & [$0,\geq 100$]  & &   \\
2 & [0.75,0.85] & [$0,\geq 40$]  &[0.65,0.75] & [$0,\geq 90$]  &[0.65,0.75] & [$0,\geq 80$]  & &   \\
2 & [0.85,0.9] & [$0,\geq 30$]  &[0.75,0.85] & [$0,\geq 70$]  &[0.75,0.85] & [$0,\geq 60$]  & &   \\
2 & [0.9,0.95] & [$0,\geq 20$]  &[0.85,0.9] & [$0,\geq 50$]  &[0.85,1] & [$\geq 0$]  & &   \\
2 & [0.95,1] & [$\geq 0$]  &[0.9,1] & [$\geq 0$]  &  &   &  &   \\
\hline
3 & [0.55,0.65] & [$0,\geq 110$]  &[0.55,0.65] & [$0,\geq 150$]  &[0.5,0.65] & [$0,\geq 130$]  & &   \\
3 & [0.65,0.75] & [$0,\geq 90$]  &[0.65,0.75] & [$0,\geq 100$]  &[0.65,0.8] & [$0,\geq 100$]  & &   \\
3 & [0.75,0.85] & [$0,\geq 70$]  &[0.75,0.85] & [$0,\geq 80$]  &[0.8,1] & [$\geq 0$]  & &   \\
3 & [0.85,0.9] & [$0,\geq 50$]  &[0.85,1] & [$\geq 0$]  &  &   &  &   \\
3 & [0.9,1] & [$\geq 0$]  &  &   &   &   &  &   \\
\hline
4 & [0.55,0.65] & [$0,\geq 150$]  &[0.5,0.6] & [$0,\geq 210$]  &  &   &  &   \\
4 & [0.65,0.75] & [$0,\geq 100$]  &[0.6,0.7] & [$0,\geq 180$]  &  &   &  &   \\
4 & [0.75,0.85] & [$0,\geq 80$]  &[0.7,0.8] & [$0,\geq 150$]  &  &   &  &   \\
4 & [0.85,1] & [$\geq 0$]  &[0.8,1] & [$\geq 0$]  &  &   &  &   \\
\hline
5 & [0.5,0.6] & [$0,\geq 210$]  &  &   &   &   &  &   \\
5 & [0.6,0.7] & [$0,\geq 180$]  &  &   &   &   &  &   \\
5 & [0.7,0.8] & [$0,\geq 150$]  &  &   &   &   &  &   \\
5 & [0.8,1] & [$\geq 0$]  &  &   &   &   &  &   \\
\hline
\end{tabular}
\end{adjustwidth}
\label{tab:allkinbin}
\end{table}

In total, there are 392 categories and 3093 total bins, the number of categories contributing to each lepton multiplicity is 84 categories from 0L, 178 categories from 1L, 115 categories from 2L, and 15 categories from 3L. The optimization of all the numerous bins and categories was a large undertaking and was performed using two statistical metrics: the Cousins Z-binomial method \cite{Cousins_2008} which is a measure of the signal to background significance with an assumption on systematic error, and the median asymptotic limit based on a profile likelihood ratio \cite{Cowan:2010js}, comparing the signal+background versus background only hypotheses but the observed data is taken as the nearest floor rounded integer from the MC model. Both metrics were tested with the presences of multiple signal types being either, stop, slepton, or electroweakinos. The optimization efforts targeted $R_{ISR}$, $M_\perp$, $p_T^{ISR}$, $\gamma_\perp$, the number of bins and bin edges, as well as object counting such as the  maximum allowed number of S jets or counted b-jets per lepton multiplicity. The categories and bins were also consolidated to guarantee non-zero expected background events in every bin. An example of this study is the flavor consolidation of two lepton categories with $N_{jets}^S > 0$ from the original implementation with flavor separated categories. This consolidation instance was found to have too few events with electrons which led to pathologically good exclusion limits from the empty bins. Similarly 2L $N_{jets}^S \geq 3$  was reduced to $N_{jets}^S \geq 2$ to boost the statistics of high jet multiplicity events in 2L. Example limits comparing many different optimization tests are illustrated in Figure \ref{fig:opto}.

%1d limit
\FigThreeScale{Analysis_figs/optoTChiWZ.pdf}{Analysis_figs/TChiWZ_consol_opto.png}{Analysis_figs/T2bWopto.png}{Example internal sensitivity optimization plots. The y-axis is the median asymptotic limit with smaller values indicating stronger expected exclusion of the signal grid points on the x-axis. The x-axis label conventions are $m_{NLSP}\_m_{LSP}\_(m_{NLSP}-m_{LSP})$. The top distribution illustrates tests performed with TChiWZ $M_\perp$ bin edges and b combinatorics. The middle distribution shows tests performed on TChiWZ with bin consolidation and jet multiplicity consolidation. The bottom distribution tests b-tag counting and $\gamma_T$ bin optimization on T2tt. }{fig:opto}{0.7}{0.7}{0.7}
