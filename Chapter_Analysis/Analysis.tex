
\setcounter{secnumdepth}{3}
\setcounter{tocdepth}{3}
\setlength{\parskip}{\smallskipamount}
\setlength{\parindent}{0pt}


\makeatletter


\providecommand{\tabularnewline}{\\}


\makeatother

%\usepackage{babel}
%\begin{document}

\chapter{Analysis Description }

\section{Introduction and Strategy}
The full analysis is built on the compressed kinematics described in the previous chapter. This goal of this chapter is to outline specfic details and the strategies used to potentially discover SUSY. This includes the description of events selected to analyze and the objects that compose these events. This search casts a wide net to capture a wide variety of signatures and final states, the consequence of this strategy yields a large number of categories and bins that cover many multiplicites of lepton and jet final states. Finally, I will discuss the data driven approach to constrain and predict background events in the most sensitive regions by conducting a series of fits to construct a robust fit model.


\section{ Data and Simulation}
The analysis invovles the full Run II dataset which is divided into three subsets by the years 2016,2017,2018 and total integrated luminosity of $138 \text{fb}^{-1}$. Each year is comprised of $36.31 \, \text{fb}^{-1} \, \pm 1.2\%$ (cite lumi 17 003), $41.48 \, \text{fb}^{-1} \, \pm 2.3\%$ (cite lumi 17 004), $59.83 \, \text{fb}^{-1} \, \pm2.5\%$ (cite lumi 18-002) in 2016, 2017, and 2018 respectively. The data is modeled by MC that represents the full SM background and is qualitatively grouped by process and final state. These grouping of SM backgrounds is defined in table X

%insert bg list and description

The signals that will be addressed in this work include mutiple different sparticles and final states, which also will include multiple interpretation of their analysis results. A list of the signals is provided in table y. Each signal is produced according to an (LSP,NLSP) mass grid where the events number of raw events per mass points and grid spacings for the signal shown in table y are displayed in Figure Z.

%insert signal list and description

The majority of signal and backgrounds use the MadGraph \cite{madgraph} generator to model at LO and NLO. The ST backgrounds use PowHEG 2.0 \cite{powheg} to model at NLO. Parton shower and fragmentaion for all samples is done with PYTHIA 7 \cite{pythia}. Each year is subjected to an underlying event tunes with CUETP8M1 for 2016, CP2 for 2017 and 2018 signals and CP5, for 2017 and 2018 backgrounds \cite{erich tunes}. The detector conditions and response are simulated for all samples with GEANT4 \cite{geant}.
 
\section{Event Selection and Physics Objects}
 For events to be qualified for analysis, they pass a handful of selected triggers and preselection which reflect the compressed kinematic description provided in the previous chapter. The triggers used PFMET PFMHT cross triggers. PFMET is the particle flow missing transverse energy which is expected to capture the $p_T^{Miss}$ from the LSP. PFMHT is expected to trigger on events with significant jet activity and multiplicity which are likely candidates for ISR events.  In general, the compressed SUSY topology is are somewhat rare organization of an event, due to the uncommon nature of these types of events the MC modeling does not always sufficiently and precisely describe data, so,  a data driven approach is utilized for physics objects to compensate for any disagreement. This approach compares the overall efficiency or behavior of selected objects and computes data driven scale factors, while also providing a platform to model and understand systematic effects. These calibrations are then applied to MC to bring data and MC into agreement. The instance of Scale factor generation arises in the modeling of the efficiencies of the trigger turn ons, with efficiency defined as events that pass the trigger and preselection versus only preselection. The comparisons of data with  which is shown in Figure X with the efficiency shapes modeled by a Gaussian CDF.  

%insert trigger turn ons

The aforementioned pre selection which is a set of carefully studied criteria to remove background and mismodeled events as well as select events with objects that are expected to populated targeted final states. The pre selection consists of the criteria listed in the following Table Z:

%insert preselection table  

The object composition for each event 



\section{Categorization and Fit Strategy}

category and bin optimization
