
\setcounter{secnumdepth}{3}
\setcounter{tocdepth}{3}
\setlength{\parskip}{\smallskipamount}
\setlength{\parindent}{0pt}


\makeatletter


\providecommand{\tabularnewline}{\\}


\makeatother

%\usepackage{babel}
%\begin{document}

\chapter{Analysis Description }

\section{Introduction and Strategy}
The full analysis is built on the compressed kinematics described in the previous chapter. This goal of this chapter is to outline specfic details and the strategies used to potentially discover SUSY. This includes the description of events selected to analyze and the objects that compose these events. This search casts a wide net to capture a wide variety of signatures and final states, the consequence of this strategy yields a large number of categories and bins that cover many multiplicites of lepton and jet final states. Finally, I will discuss the data driven approach to constrain and predict background events in the most sensitive regions by conducting a series of fits to construct a robust fit model.


\section{ Data and Simulation}
The analysis invovles the full Run II dataset which is divided into three subsets by the years 2016,2017,2018 and total integrated luminosity of $138 \text{fb}^{-1}$. Each year is comprised of $36.31 \, \text{fb}^{-1} \, \pm 1.2\%$ (cite lumi 17 003), $41.48 \, \text{fb}^{-1} \, \pm 2.3\%$ (cite lumi 17 004), $59.83 \, \text{fb}^{-1} \, \pm2.5\%$ (cite lumi 18-002) in 2016, 2017, and 2018 respectively. The data is modeled by MC that represents the full SM background and is qualitatively grouped by process and final state. These grouping of SM backgrounds is defined in table \ref{tab:bkgsigtab} along with the partnered signal models.

%insert bg list and description
\begin{table}
\label{tab:bkgsigtab}
\caption{table caption}
\begin{tabular}{c|c}
\hline 
Bkg. Label & Bkg. Composition \\ 
\hline 
\hline

W + jets & \makecell{Single W boson, a dominant background that composes \\ about $50\%$ of the total background} \\ 
  & \\
tt+jets & \makecell{$t\bar{t}$ which can be accompanied by a W,Z,h, or $\gamma$, \\ the other dominant background composes about $50\%$ of the total background} \\ 
  & \\
ZDY & Z+jets and Drell Yan, an intermediate background \\ 
Di-boson (DB)& WW,ZZ,WZ,Wh,Zh,  an intermediate background \\ 

ST & Single top processes including tW, a rare background \\ 

Tri-boson (TB) & WWW,ZZZ,WWZ, WZZ, WZ$\gamma$, WW$\gamma$, a rare background  \\ 
\hline 
Signal Label & Signal Composition \\
\hline
\hline
T2tt & $pp \rightarrow \tilde{t} \tilde{\bar{t}}; \, \, \tilde{t}\rightarrow t \tilde{\chi}_1^0$ \\
TChiWZ & $pp\rightarrow \tilde{\chi}_2^0 \tilde{\chi}_1^\pm; \, \, \tilde{\chi}_2^0 \rightarrow Z \tilde{\chi}^0_1; \, \, \tilde{\chi}_1^\pm \rightarrow W^\pm \tilde{\chi}^0_1$ \\
TSlepSlep & $pp\rightarrow \tilde{\ell} \tilde{\ell}; \, \, \tilde{\ell} \rightarrow \ell \tilde{\chi}^0_1$ \\
TChipmWW & $pp\rightarrow \tilde{\chi}_1^\pm \tilde{\chi}_1^\mp; \, \,  \tilde{\chi}_1^\pm \rightarrow W^\pm \tilde{\chi}^0_1$  \\
\end{tabular} \\
\end{table}

The signals that will be addressed in this work include mutiple different sparticles and final states, which also will include multiple interpretation of their analysis results. A list of the signals is provided in table y. Each signal is produced according to an (LSP,NLSP) mass grid for each year.  The raw number of events per mass points and grid spacings for the signals shown in table \ref{tab:bkgsigtab} are displayed in Figure \ref{fig:grids} with all years combined. 

%insert signal list and description
\FigFour{Analysis_figs/T2tt_EventCount.pdf}{Analysis_figs/TChiWZ_EventCount.pdf}{Analysis_figs/TSlepSlep_EventCount.pdf}{Analysis_figs/TChipmWW_EventCount.pdf}{grids}{fig:grids}

The majority of signal and backgrounds use the MadGraph \cite{madgraph} generator to model at LO and NLO. The ST backgrounds use PowHEG 2.0 \cite{powheg} to model at NLO. Parton shower and fragmentaion for all samples is done with PYTHIA 7 \cite{pythia}. Each year is subjected to an underlying event tunes with CUETP8M1 for 2016, CP2 for 2017 and 2018 signals and CP5, for 2017 and 2018 backgrounds \cite{erich tunes}. The detector conditions and response are simulated for all samples with GEANT4 \cite{geant}.
 
\section{Event Selection and Physics Objects}
 For events to be qualified for analysis, they pass a handful of selected triggers and preselection which reflect the compressed kinematic description provided in the previous chapter. The triggers used PFMET PFMHT cross triggers. PFMET is the particle flow missing transverse energy which is expected to capture the $p_T^{Miss}$ from the LSP. PFMHT is expected to trigger on events with significant jet activity and multiplicity which are likely candidates for ISR events.  In general, the compressed SUSY topology is are somewhat rare organization of an event, due to the uncommon nature of these types of events the MC modeling does not always sufficiently and precisely describe data, so,  a data driven approach is utilized for physics objects to compensate for any disagreement. This approach compares the overall efficiency or behavior of selected objects and computes data driven scale factors, while also providing a platform to model and understand systematic effects. These calibrations are then applied to MC to bring data and MC into agreement. The instance of Scale factor generation arises in the modeling of the efficiencies of the trigger turn ons, with efficiency defined as events that pass the trigger and preselection versus only preselection. The comparisons of data with  which is shown in Figure \ref{fig:metsf} with the efficiency shapes modeled by a Gaussian CDF.  

%insert trigger turn ons
\FigOne{Analysis_figs/SF_Plot_HT-Le600--SingleMuontrigger-E1--Nmu-E1_SingleMuon_2016.pdf}{met trig sf}{fig:metsf}

The aforementioned pre selection which is a set of carefully studied criteria to remove background and mismodeled events as well as select events with objects that are expected to populated targeted final states. The pre selection consists of the criteria listed in the following Table Z:

%insert preselection table  
\begin{tabular}{c|c}
\hline 
\multicolumn{2}{|c|}{Preselection Requirements} \\ 
\hline 
Criteria & Description \\ 
\hline 
\hline
$N_V \geq 1$ & At least one visible object assigned to the S system \\ 
$N_j^{ISR} \geq 1$ & At least one jet assigned to the ISR system \\ 

$p_T^{miss} > 150$GeV &\makecell{ Minimum transverse missing energy based on trigger efficiency}  \\ 

$p_T^{ISR} > 250 $GeV & \makecell{Minimum ISR kick to resolve massive invisible particles} \\ 

$R_{ISR}$ > 0.5 & Target Massive LSPs \\ 

$|\Delta \phi_{\vec{p}_T^{miss}, V}| < \pi/2$ &  Ensures visible and invisible system are traveling in the same direction \\ 

$p_T^{CM} < 200$  & Rejects mismodeled events \\ 

veto $f(\Delta\phi_{CM,I}, p_T^{CM})$& 2D function to also reject mismodeled events - See Fig \ref{fig:cleancut}\\
\hline 
\end{tabular} \\

\FigOne{Analysis_figs/cleaningCuts.pdf}{cleaning cuts}{fig:cleancut}

The pre selection forms the basis for an event to be analyzed. Following preselection, the physics objects can be selected, classified, and categorized. The possible object composition can consist of jets, b-tagged jets, soft secondary vertices (SVs), and leptons. The discussion of the leptons and their classification will be reserved for the following chapter alongside the calculation of lepton scale factors. A summary of these physics objects and their kinematic requirements are listed in the following Table X. The various types of jets involve working points (WP) from their respective physics object groups and are standard objects used in CMS physics analysis (cite jet stuff ak4 and ids). The b-tagging is done by a standard NN based tagger which only identifies b-jets down to 20 GeV. A complementary SV tagger was developed specifically for this analysis and a detailed description of this tool is described in this analysis's sister thesis (cite erich thesis). The purpose of the SV tagger is to efficiently extend the b-tagging range down to 2 GeV because a final state topology with something like compressed stops often includes soft b-jets.


\begin{table}
\centering
\label{tab:physicsobjects}
\caption{table caption2}
\begin{tabular}{c|c}
\hline 
\multicolumn{2}{c}{Visible Physics Objects} \\ 
\hline 
\hline
Jets & \makecell{AK4 PF Jets \\ Tight ID \\ $p_T^{jet} > 20$ GeV \\ $|\eta|>2.4$} \\ 
\hline
B-tagged Jets & \makecell{AK4 PF Jets \\ DeepJet Medium WP \\ $p_T^{b-jet} > 20 $ GeV}  \\ 
\hline
SVs & $2<p_T^{SV}<20$ GeV \\ 
\hline
Leptons & \makecell{Very Loose ID \\ $p_T^{\mu^\pm} > 3$ GeV \\ $p_T^{e^\pm} > 5 $GeV \\ Gold/Silver/Bronze quality classes} \\ 
\hline 
\end{tabular} 
\end{table}

\section{Categorization}

Once an event passes all the preselection and the objects are classified, the event is then categorized based on object composition, object multplicity, and kinematic characteristics. The goal of categorization is to finely split up background and create generically sensitive regions for any type of signal. The regions where signal are not localized, function as control regions which can constrain the back ground prediction in the sensitive regions. The most fundamental categories for the analysis are the lepton and jet multiplicites, events can contain either 0,1,2, or 3 leptons and accompanied by a range of sparticle system jets (S jets) from $0\geq N_{S_{jet}} \geq 5$ where the upper limit of jet S jet counting is dependent on the lepton multiplicity. The type of jets are also counted i.e. whether or not any of the present jets have been tagged as b-jets. The b-jet counting is dependent on the system and lepton multiplicity - ISR jets can have 0 or $\geq 1$ b-jets and the S system can be composed of 0, 1, or $\geq2$ b-jets. SV tagging is similar but SVs are not counted in the ISR system. Events with SVs are special, however, wheret the SV can be further classified based on its direction, central or forward. The counting of b-jets and SVs are very important because these create categories that isolate tt+jets and stop signals with the presence of b-jets and similarly regions are created with the absence of b-jets that isolate backgrounds like w+jets and signals like TChiWZ. The benefit of both regions is that they can cross-constrain each other and is illustrated in Figure \ref{fig:bcount}, showing the presence and absence of tt+jets or W+jets backgrounds based on S system b-jet counting.
\FigTwo{Analysis_figs/NbS_v_NbISR_0L_ttbar-gif-converted-to.pdf}{Analysis_figs/NbS_v_NbISR_1L_Wjets-gif-converted-to.pdf}{ttbar bjet count}{fig:bcount}

Aside from jet counting, there is further classification with leptons which includes flavor, charge, quality, and the presence or absence of an opposite sign same flavor Z candidate. The remaining classifications are the previously metioned complementary compressed kinematic observables $p_T^{ISR}$ and $\gamma_\perp$ where both have a high and low category.    The categorization does not include all possible combinations all aforementioned categorical labels but a subset of all possible combinations that reflects the available statistics and presence of backgrounds. A  table of the categorization is included in Table \ref{tab:cats}. 

\begin{table}
\label{tab:cats}
\caption{cats, includes ptisr bin edges}
\begin{adjustwidth}{-0.7in}{-1in}

\begin{tabular}{|c|c|c|c|c|c|c|c|c|c|}
\multicolumn{7}{c|}{Obj. Combinatorics} & \multicolumn{3}{c}{Obj. Kinematics} \\
\hline 
$N_\ell$ &  $\ell$ Type & $\ell$ Quality & $N_{jets}^{S}$ & $N_{b-tag}^{S}$ & $N_{SV}^S$ & $N_{b-tag}^{ISR}$ & $SV_\eta$ & $\gamma_\perp$ & $p_T^{ISR}$ \\ 
\hline
\hline 
0 & - & - & 0 & 0 & $[1,\geq 2]$ & - & $\checkmark$ & - & $\geq 350$ \\ 
0 & - & - & 1 & - & $\geq 1$ &    - & $\checkmark$ & - & $\geq 400$ \\
0 & - & - & 1 & - &  0       &    - &  -           & - & $[400,\geq 550]$ \\
0 & - & - & $[2,\geq 5]$ & $[0,\geq 2]$ & 0 & $[0,\geq 1]$ & - & $\checkmark$ & $[350,\geq500]$ \\
\hline 
1 & $e^+,e^-,\mu^+,\mu^-$& Gold & 0 & 0 & 0 & $[0,\geq 1]$ & - & $\checkmark$ & $[350, \geq 500]$ \\
1 & $\ell^+, \ell^-$  & Gold & 0 & 0 & 1 & - & $\checkmark$ & - & $\geq 350$ \\
1 & $e, \mu$ & Gold & 1 & 0 &  $\geq 1$ & - & $\checkmark$ & - & $\geq 350$ \\
1 & $\ell$ & Gold & $[1,\geq 4]$ & $[0,\geq 2]$ & 0 & $[0,\geq 1]$ & - & $\checkmark$ & $[350, \geq 500]$ \\
1 & $e, \mu$ & Silver/Bronze & $[0,\geq 1]$ & 0 & 1 & - & $\checkmark$ & - & $\geq 350$ \\
1 & $e, \mu$ & Silver/Bronze & $[2, \geq 4]$ & - & - & - & - & $\checkmark$ & $[350, \geq 500]$ \\
\hline
2 & $e^\pm e^\mp, \mu^\pm \mu^\mp, e^\pm \mu^\mp $ & Gold & 0 & 0 & 0 & $[0,\geq 1]$ & - & $\checkmark$ & $[250,\geq350]$  \\
2 & $Z, no Z$ & Gold & $[1,\geq 2]$ & $[0, \geq 1]$& - & $[0,\geq 1]$ & - & $\checkmark$ & $[250,\geq350]$  \\
2 & $\ell^\pm \ell^\pm$ & Gold & $[0,\geq 2]$ & - & - &  - & - & - & $[250,\geq350]$ \\
2 & $ee, \mu\mu, e\mu$ & Silver/Bronze & $[0,\geq 2]$ & - & - & - & - & - & $\geq350$ \\
2 & $\ell \ell$ & Gold/Silver/Bronze & 0 & 0 & $\geq 1 $ & - & $\checkmark$ & - & $\geq 250$\\
\hline
3 & $Z, no Z$ & Gold/Silver/Bronze & $[0,\geq 1]$ & - & - & - & - & - & $\geq 250$ \\
3 & $\ell^\pm \ell^\pm \ell^\pm$ & Gold/Silver/Bronze & - & - & -& - & - & -& $\geq 250$ \\
\hline
\end{tabular} 
\end{adjustwidth}
\end{table}



Each category is then further divided into 2-D set of $(R_{ISR}, M_\perp)$ bins, where the number of bins and bin edges has been optimized for each combination of lepton and jet multiplicities. An example of binning of the sensitive variables for each lepton multiplicity is shown in Figure \ref{kinbin} and a full table of all binnings are included in Table X.
\FigFour{Analysis_figs/0L_2J_totalBkg.png}{Analysis_figs/1L_G_1J_1sv_totalBkg.png}{Analysis_figs/2L_G_0J_totalBkg.png}{Analysis_figs/3L_B_1J_totalBkg.png}{example 2d binning}{fig:kinbin}

% insert RISR MP bins table here
\begin{table}
\begin{adjustwidth}{-0.5in}{-1in}
%\begin{singlespace}
\begin{tabular}{c|cc|cc|cc|cc|}
- & \multicolumn{2}{c}{0L} & \multicolumn{2}{c}{1L} & \multicolumn{2}{c}{2L} & \multicolumn{2}{c}{3L}   \\
\hline
$N_{jets}^S$ & $R_{ISR}$ & $M_\perp$ & $R_{ISR}$ & $M_\perp$ & $R_{ISR}$ & $M_\perp$ & $R_{ISR}$ & $M_\perp$ \\
\hline
0 & [0.95,0.985] & [$\geq 0$]  &[0.9,0.96] & [$\geq 0$]  &[0.6,0.7] & [$0,\geq 50$]  &[0.6,0.7] & [$\geq 0$]  \\
0 & [0.985,1] & [$0,5,\geq 10$]  &[0.96,0.98] & [$0,\geq 10$]  & [0.7,0.8]  & [$0,\geq 40$] &[0.7,0.8] & [$\geq 0$]  \\
0 &  &  &[0.98,1] & [$0,5,\geq 10$]  &  [0.8,0.9]  & [$0,\geq 30$] &[0.8,0.9] & [$\geq 0$]  \\
0 &  &   &        &                &[0.9,0.95] &  [$0,\geq 20$]  &[0.9,1] & [$\geq 0$]  \\
0 &   &   &   &   & [0.95,1] & [$0,\geq 15$]  & &   \\
\hline
1 & [0.8,0.9] & [$\geq 0$]  &[0.65,0.75] & [$0,\geq 50$]  &[0.5,0.6] & [$0,\geq 100$]  &[0.55,0.7] & [$\geq 0$]  \\
1 & [0.9,0.93] & [$0,\geq 20$]  &[0.75,0.85] & [$0,\geq 40$]  &[0.6,0.7] & [$0,\geq 80$]  &[0.7,0.85] & [$\geq 0$]  \\
1 & [0.93,0.96] & [$0,\geq 20$]  & [0.85,0.9] & [$0,\geq 30$] &[0.7,0.8]  & [$0,\geq 60$]  &[0.85,1] & [$\geq 0$]  \\
1 & [0.96,1] & $0,\geq 15$]   &[0.9,0.95] & [$0,\geq 20$]   &[0.8,0.9] & [$0,\geq 40$]   & &   \\
1 &  &   &[0.95,1]   & [$\geq 0$]   &[0.9,1]& [$0,\geq 30$]   & &   \\
\hline
2 & [0.65,0.75] & [$0,\geq 50$]  &[0.55,0.65] & [$0,\geq 110$]  &[0.5,0.65] & [$0,\geq 100$]  & &   \\
2 & [0.75,0.85] & [$0,\geq 40$]  &[0.65,0.75] & [$0,\geq 90$]  &[0.65,0.75] & [$0,\geq 80$]  & &   \\
2 & [0.85,0.9] & [$0,\geq 30$]  &[0.75,0.85] & [$0,\geq 70$]  &[0.75,0.85] & [$0,\geq 60$]  & &   \\
2 & [0.9,0.95] & [$0,\geq 20$]  &[0.85,0.9] & [$0,\geq 50$]  &[0.85,1] & [$\geq 0$]  & &   \\
2 & [0.95,1] & [$\geq 0$]  &[0.9,1] & [$\geq 0$]  &  &   &  &   \\
\hline
3 & [0.55,0.65] & [$0,\geq 110$]  &[0.55,0.65] & [$0,\geq 150$]  &[0.5,0.65] & [$0,\geq 130$]  & &   \\
3 & [0.65,0.75] & [$0,\geq 90$]  &[0.65,0.75] & [$0,\geq 100$]  &[0.65,0.8] & [$0,\geq 100$]  & &   \\
3 & [0.75,0.85] & [$0,\geq 70$]  &[0.75,0.85] & [$0,\geq 80$]  &[0.8,1] & [$\geq 0$]  & &   \\
3 & [0.85,0.9] & [$0,\geq 50$]  &[0.85,1] & [$\geq 0$]  &  &   &  &   \\
3 & [0.9,1] & [$\geq 0$]  &  &   &   &   &  &   \\
\hline
4 & [0.55,0.65] & [$0,\geq 150$]  &[0.5,0.6] & [$0,\geq 210$]  &  &   &  &   \\
4 & [0.65,0.75] & [$0,\geq 100$]  &[0.6,0.7] & [$0,\geq 180$]  &  &   &  &   \\
4 & [0.75,0.85] & [$0,\geq 80$]  &[0.7,0.8] & [$0,\geq 150$]  &  &   &  &   \\
4 & [0.85,1] & [$\geq 0$]  &[0.8,1] & [$\geq 0$]  &  &   &  &   \\
\hline
5 & [0.5,0.6] & [$0,\geq 210$]  &  &   &   &   &  &   \\
5 & [0.6,0.7] & [$0,\geq 180$]  &  &   &   &   &  &   \\
5 & [0.7,0.8] & [$0,\geq 150$]  &  &   &   &   &  &   \\
5 & [0.8,1] & [$\geq 0$]  &  &   &   &   &  &   \\
\hline
\hline5 &  & &  & &  & &  & \\




\end{tabular}
\end{adjustwidth}
\end{table}

Overall there are 392 categories and 3XXX total bins, the category population is comprised of A categories from 0L, B categories from 1L, C categories from 2L, and D categories from 3L. The optimization of all the numberous bins and categories was a large undertaking and was performed using two statisical metrics: (1) The Z-binomial \cite{zbi} which is a measure of the significance of the ratio of signal to background with some assumption on systematic error. (2)  Median asymptotic limit based on a profile likelihood ratio, under signal and background hypotheses but where observed data is identically the MC model. Both metrics were tested with the presences of multiple signal types being either, stop, slepton, or electroweakino. The optimization efforts targeted $(R_{ISR}, M_\perp)$ number of bins and edges, the $p_T^{ISR}$ and $\gamma_\perp$ number of bins and edges. The categories and bins were also optimally consolidated to guarantee non zero expected background events in every bin. An example of this study is the flavor consolidation to 2L Z*/no Z 1J from previously flavor separated categories, in this instance the it was found that there were too few Z* candidates with electrons leading to pathologically improved limits due to nearly empty bins in the limit fit. Similarly 2L $\geq 3$ was reduced to $\geq 2$ to boost the statisitcs of high jet multiplicity events in 2L. An example limit comparison is illustrated in Figure \ref{fig:opto}.

%1d limit
\FigThree{Analysis_figs/optoTChiWZ.pdf}{Analysis_figs/TChiWZ_consol_opto.png}{Analysis_figs/T2bWopto.png}{Three optimization}{fig:opto}
