
\setcounter{secnumdepth}{3}
\setcounter{tocdepth}{3}

\setlength{\parindent}{1 em}


\makeatother


\chapter{Motivating the Search for SUSY}

\begin{chapterabstract}
First im opening up with the basic motivations for susy, like solving the hierarchty problem and dark matter candidate, next we look at a theoretical motivation for SUSY via the higgs mass. Motivate simplified models with naturalness etc, talk about how susy needs to be at a few TeV scale to work out. Then we visit two recent experimental measurements which strongly motivate the search for susy and more speficically this body of work.
\end{chapterabstract}

\section{Introduction}
susy is dark matter candidate, susy solve hierarchy problem


\section{Motivating SUSY}
blah summarize higgs mass, g-2 , and w mass measurement


\subsection{Stabilizing the Higgs mass}
UV problem with the higgs and large scale corrections to the mass
\subsection{Muon anomalous magnetic moment}
g-2 overview, measurement, could point to susy or could deal with QCD lattice and weird diagrams

g-2 is an experiment designed to measure the anomolaus magnetic dipole moment of the muon. The spin magnetic moment of a charged, spin-1/2 particle that does not possess any internal structure (a Dirac particle) is given by (wiki direct quote \url{https://en.wikipedia.org/wiki/G-factor_(physics)}) ${\displaystyle {\boldsymbol {\mu }}=g{e \over 2m}\mathbf {S} }$. where g is the particles g-factor, $\mu$ is the magetic moment, $m$ is the particle mass and $S$ is the spin. The g-factor in quantum electrodynamics is close to 2 so typically the reported measurement is the difference from 2 or g-2 or as a signficance $a_\mu = g-2/2$. The difference from 2 arises from higher order contributions in quantum field theory


What is the g-factor
The quantity $g$ is a factor in the electromagnetic coupling of charged particles to a photon. The factor largely depends on the tree level lepton-photon coupling but gets small quantum corrections from higher order loops, the largest being the single photon loop or Schwinger term. There are three types of corrections applied when determining the SM g-factor for light leptons -- QED, Electroweak, and Hadronic. Corrections due to the Higgs are neglected due to the mass disparity $m_h >> m_{e,\mu}$ and the mass dependence in the Higgs coupling. This causes the Higgs contributions to be unnecessary at our current level of precision. If the g-factor is calculated to first order in QED the g-factor is exactly 2, when accounting for quantum corrections the g-factor deviates slightly from 2. Experimentally the deviations from 2 are studied, which are written in the form $a_\ell = \frac{g-2}/2$ and referred to as $(g-2)$. These small contribtuions are interesting because if they were to deviate from the SM prediction, it would be an indication of new particles interacting with the SM leptons.  There are three possible g-factors we can measure, one for each generation of lepton: $a_e, a_\mu, a_\tau$. The current best candidate to both test the SM and search for new physics is by measuring $(g-2)_\mu$ or $a_\mu$ because of  experimental precision potential. The electron measurement is already known to the highest precision and is expected to have the smallest contributions from new physics (cite). The $(g-2)_\tau$ is not yet experimentally tractable to measure with precisions that are competitive with $e$ and $\mu$, so $(g-2)_\mu$ has been measured at both at Brookhaven National Lab (BNL) and again at Fermi National Accelerator Laboratory (FNAL).

What is the current calculation and currrent status of (g-2)mu
The currently accepted best SM prediction of $a_\mu$ from (CITE) includes QED, Electroweak(EW) and Hadronic contributions and is reported as $a_\mu^{SM} = a_\mu^{QED}+ a_\mu^{EW}+a_\mu^{\text{Hadronic}} = 116 591 810(43) \times 10^{-11}$. For each of the $a_\mu$ components, the QED compenent enters at the $O(10^{-3})$ and is known to $O(10^{-11})$. the EW component enters the sum at $O(10^{-9})$ and is known to $O(10^{-10})$. Finally the most complicated component, hadronic, contributes at $O(10^{-8})$ and is known up to $O(10^{-9})$, the  main sub components that contribute to the $a_\mu^{\text{Hadronic}}$ is the Hadronic vacuum polarization and light by light scattering, diagrams illustrated in Figure X. The hadronic precision is constrained by data driven measurements and computation approaches -- QCD lattice theory, this error dominates the overall uncertainty of $a_\mu$. The BNL measurement of $a_\mu$ yields a difference with the SM prediction of $\Delta a_\mu := a_\mu^{BNL} - a_\mu^{SM} = 279(76) \times 10^{-11}$ which is a significance of $3.7\sigma$. The most recent $a_\mu$ measurement from FNAL confirms the BNL measurement within $1\sigma$ and the combined experimental average increases the SM deviation with a significance of $4.2\sigma$.

What could this deviation mean?
The $4.2\sigma$ significance is a compelling sign for potential new physics, but can be somewhat explained by improvements in QCD lattice calculations of the HVP and LBL contributions. There is also a calculation which resolves this tension up to $1.2\sigma$ (cite BMW). The more interesting explanation is that the tension could be due to effects of new particles. There are several models which can quantify the $a_\mu$ SM deviation, those being Super Symmetry, Dark Matter (DM mediator Dark photon), Lepto quarks, 2 Higgs doublet models.



\subsection{W mass measusremnt}
the most recent w mass measurement yielded a heavy W, this higher mass is more favorable for light higgsino and compressed susy models

\section{The current status of SUSY}
drop the most recent limits here, start with multi TeV excluded gluino and squark models which leaves the a good place to search in the weak scale sector with electoweakinos. Talk about electroweak limits and how alot of these are excluded already one of the remaining places to search is the compressed corridor where mass splittings are small. link this limit motivation with how both g-2 and W mass favor compressed scenarios



