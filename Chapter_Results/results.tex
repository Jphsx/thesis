\setcounter{secnumdepth}{3}
\setcounter{tocdepth}{3}
\setlength{\parskip}{\smallskipamount}
\setlength{\parindent}{0pt}


\makeatletter


\providecommand{\tabularnewline}{\\}


\makeatother


\chapter{Results}


\section{Asymptotic Limits}

This search is designed to be generically sensitive to SUSY with an emphasis compressed scenarios. The consequence is sensitive to a wide variety of models and final states, so, we present upper exlusion limits on the cross sections for stop, electroweakino, and slepton processes. The results use the full Run II dataset alongside the full SM MC background which is combined with the data driven fit model described in the previous chapter. The limits are calculated using the asypmtotic method for profile-likelihood test statistic \cite{whatever AN cites here} 

\FigOne{Results_figs/allyear_limit_t2tt_138.pdf}{t2tt run2}{fig:limt2tt}

\FigOne{Results_figs/allyear_limit_tchiwz_138.pdf}{tchiwz run2}{fig:limtchiwz}

\FigOne{Results_figs/allyear_limit_tslepslep_138.pdf}{tslep run2}{fig:limslep}

\FigOne{Results_figs/T2bW_dm_xs_LV1.pdf}{t2bw 17}{fig:limt2bw}

\FigOne{Results_figs/can_dM2_WW.pdf}{ww 17}{fig:limtchiww}

\section{Model Dependent Interpretation}
The TChiWZ limits from the previous section use a simplified model which is described in Chapter 1 section ?. The decay kinematics for the MC use a flat matrix element meaning the W and Z boson 3 body decays are uniform in phase space. However, it is well known that the 3 body decays are model dependent and can by categorized into two distinct scenarios that correspond to the sign of the eigenstates of the neutralino mass matrix. The two eigenstates 


\section{Summary}
