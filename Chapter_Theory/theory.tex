



\setcounter{secnumdepth}{3}
\setcounter{tocdepth}{3}
\setlength{\parskip}{\smallskipamount}
\setlength{\parindent}{0pt}


\makeatletter


\providecommand{\tabularnewline}{\\}


\makeatother

%\usepackage{babel}
%\begin{document}

\chapter{The Standard Model and Supersymmetry}

%\begin{chapterabstract}
%Here we outline the fundamental concepts of particle physics, we introduce the set of fundamental particles, fields, and interactions which %are described by the standard model. Since the standard model does not solve all of the problems of particle physics I introduce %supersymmetry, a well motivated extension to the standard model. The implications of SUSY model space on various phenomoglly of processes %as well as experimental observations.  The motivation for the SUSY extension is reserved for the following chapter.

%\end{chapterabstract}

\section{Introduction}

The fundamental building blocks of matter and their interactions expressed through three of the four fundamental forces of nature via the Standard Model (SM). The fourth force, or gravity, is left to General Relativity. The SM  is the culmination of over a century of work by many scientists, and has its roots in the late 19th century. The first force is based on the theory of strong interactions, formulated as quantum chromodynamics by Murray Gell-Mann and others in the 1960s, which provides a description of how protons and neutrons are held together in the nucleus of an atom. The theory describing weak interactions was developed by Enrico Fermi in the 1930s, which was then combined with electromagnetic interactions in electroweak theory by Sheldon Glashow, Abdus Salam, and Steven Weinberg in the 1960s.
In this chapter, concepts of the Standard Model are introduced, including the fundamental particles, fields, and their basic properties and interactions. Then expanding from the core SM we will discuss an extension of the Standard Model with supersymmetry, which proposes a new symmetry between fermions and bosons. Finally, we delve into the specifics of simplified models of supersymmetry and the challenges associated with detecting these models experimentally.



\section{The Standard Model}

The Standard Model is a collection of adhoc theories used to predict and reproduce experimental data. The theory itself incorporates four major concepts: Quantum Field theory (QFT), the Dirac equation, the gauge principle, and the Higgs mechanism. These four principles are constrained by physical data and describe the set of elementary particles, known as fermions and bosons. The SM generally refers to the SM Lagrangian, an equation with different sectors that describe different subsets of particles, fields, and their interactions. The SM Lagrangian itself consists of 26 free parameters which are input by hand. These parameters are: the masses of the 12 fermions, 3 coupling constants that describe gauge interactions: $g, g', g_s$, 2 parameters to desribe the Higgs potential i.e. the higgs mass $m_h$ and the vacuum expectation value (vev), and 9 mixing angles which describe the  PMNS and CKM matrices or the mixing of different fermionic fields. The 12 fermion paramters are subdivided by three neutrinos $m_{\nu_i}$, three charged leptons $m_{\ell_i}^\pm$, and six quarks $m_{q_i}$; 

QFT provides a description for both known and theoretical particles by combining quantum theory, the field concept, and relativity (cite peskin). The gauge theory aspect describes the exact nature of QFT interactions and provides the mechanisms for the electromagnetic, strong, and weak forces.  We know of three gauge fields:  $\vec{G}$ which transforms under $SU(3)$ and govern strong interactions, $\vec{W}$ and $B$ which transform under $SU(2)_L \times U(1)$ and govern electromagnetic and weak interactions. The combination of the gauge fields and fermion fields along with the Dirac equation yields eigenstates that represent fermionic matter particles. These particles would be massless if not for the inclusion of the complex scalar Higgs field.  The spontaneous symmetry breaking of the Higgs field, due to the Yukawa coupling, creates the non-zero vev responsible for generating the masses of the electroweak gauge bosons. Additionally, the interaction between the fermionic fields and the non zero-vev generates the masses of SM fermions.

The set of standard model elementary particles is divided into two subgroups: fermions and bosons.  The fermions consist of both charged and neutral leptons as well as fractionally charged quarks. There are three flavors of charged leptons $(\ell)$, the electron $(e)$, the muon $(\mu)$, and the tau $(\tau)$. Each charged lepton has a flavor pairing neutral neutrino $\nu_\ell$. The $e$ and $\mu$ are also generally considered as "light" leptons due to their small mass relative to the $\tau$. The term lepton, depending on context, often refers to only the charged particles. As for the quarks, there are also three generations of pairs of quarks.. The lighest set of quarks are the up $(u)$ and down $(d)$ quarks, followed by the charm $(c)$ and strange $(s)$, and lastly the bottom $(b)$ and extremely massive top quark $(t)$.  The bosons are the force carrying particles which represent the gauge fields. They are comprised of the vector bosons - the photon $(\gamma)$, gluon $(g)$, the $W^\pm$, and the $Z^0$ - along with the singular scalar boson the Higgs $(h)$. The elementary particles masses, generations, and spins are summarized in Figure \ref{fig:smfig}.

\FigOne{Intro_figs/Standard_Model_of_Elementary_Particles.png}{particles figure cite wiki}{fig:smfig}


%The SM Lagrangian is composed of constituent sectors which describe diffrent groups/fields/particles. The main SM sectors are, the quantum chromodynamics (QCD) sector, the electroweak sector, the Higgs sector, and the Yukawa sector. QCD describes colored interactions of quarks mediated by gluons with the strong force. The electroweak sector unifies both the electromagnetic and weak interactions via exchange of W or Z bosons as well as electromagnetic interactions via $\gamma$. The Higgs sector introduces the complex scalar higgs field (citation needed). Interaction of bosons with the Higgs field causes the bosons to have mass and the Yukawa coupling describes the interaction of fermions with the higgs field which also allows the fermions to have mass (citation needed).



The SM is an asymmetric chiral theory, combining three groups $SU(3)_L \times SU(2)_L \times U(1)$. The $L$, or left handed, subscript indicates that mirrored fields (with different chiralities)  transform differently under the Lorentz group and the EW gauge group (cite slides).  The consequence of chiralilty is that the possible combinations between interaction vertices is limited(cite thompson). This peculiar property shows up with the $W$ boson, which only couples to left handed particles or right handed antiparticles. Extensions of the standard model also often extend chiral or symmetrical properties.  %Helcity is the defined by the projection of a particles spin ontion its direction of motion (thompson). A particle is considred right-handed if the direction of its spin and motion is parallel. %It is left-handed if spin and motion antiparallel. (cite wikipedia helicity). This peculiar property shows up with the W boson, which only couples to left handed particles or right handed antiparticles.

\section{Supersymmetry}

Supersymmetry (SUSY) is an extension of the standard model. It adds a generator that rotates the spin between bosons and fermions. This then introduces a bosonic degree of freedom for every fermionic degree of freedom (cite run2 susy paper) which generates a super partner for each particle with spin differing by a half integer.  The resulting set of mirrored elementary particles are referred to as sparticles. Each bosonic sparticle carries the same name as its fermion partner but with an "s" prefix e.g. sfermion, squark, selectron. As for the bosons, with the gauge fields $B$ and $\vec{W}$, these are accompanied by three super symmetric fields - the Higgsino $\tilde{H}$, Bino $\tilde{B}$, and Wino $\tilde{W}$. The mixture of the B and $\vec{W}$ SM fields can be represented by particle matrix. One can obtain the  mass eigenstates representing the SM particles $\gamma, \, \, Z, \, \, W^\pm$ through the diagnolization of particle matrix. Similarly, the Higgsino, Bino, and Wino mix to produce four neutral and two charged eignestates, the neutralinos ($\tilde{\chi}^0_1, \tilde{\chi}^0_2, \tilde{\chi}^0_3, \tilde{\chi}^0_4$)  and charginos ($\tilde{\chi}^\pm_1, \tilde{\chi}^\pm_2$) (cite erich's 43 susy matrix eigenstates). SUSY also requires an additional Higgs doublet to give mass to up-type and down-type fermions, (cite this run 2 paper.. reference chasing) leading to five higgs boson states consisting of two charged Higgs and three neutral Higgs. The lightest neutral higgs of the three neutral options represents the SM Higgs boson. The full set of SM particles alonglide their SUSY partners are illustrated in Figure \ref{smandsusyfig}. The addition of another higgs doublet also introduces a second vev. The ratio between the two vev's  is commonly denoted as $v_1/v_2 = \tan \beta$ and is an important parameter in experimental searches. Another important bookeeping parameter, similar to lepton number or baryon number conservation, is R-parity. This parameter tallies the total number of SM particles (+1) and sparticles (-1) and expects the net total between particles to be conserved in the initial and final states. R-parity conservation then requires sparticles to be produced in pairs. If R-parity is violated, the common consequence is that the lightest supersymmetric particle (LSP) is unstable. 

\FigOne{Intro_figs/elementary_sparticles.png}{stolen from this springer thesis book, probably make my own figure later \url{https://link.springer.com/chapter/10.1007/978-3-030-25988-4_4}}{fig:smandsusyfig}



Supersymmetry is an extremely expansive model and intractable to experimentally test without significant well motivated simplifications. The most experimentally common simplified SUSY model is the Minimally Super Symmetric Standard Model (MSSM). The MSSM contains the smallest number of new particle states and new interactions which are consistent with phenomenology (cite howie direct weak scale book). The MSSM is still experimentally inaccesible due to the presence of over 100 parameters, where small changes in parameter space can completely morph the model structure and experimental signatures. To reduce the problem's dimensionality, further simplification is needed, resulting in a popular simplified model: the phenomological MSSM (pMSSM). The pMSSM contains 19 parameters which include the masses of each generation of squark and slepton, parameters to control the mixing of $\tilde{H}, \tilde{W}, \tilde{B}$, and dials for the higgs doublet(cite what wiki cites).  The pMSSM is still borderline too complicated to attack directly, so, the pMSSM is boiled down into a simplified model of four parameters $M_1$,$M_2$, $\mu$, and $\tan\beta$. $M_1$ and $M_2$ are the gaugino mass parameters, $\mu$ is the Higgsino mass parameter, and $\tan\beta$ is the previously mentioned vev ratio (cite Fuks paper).  A model point from this four parameter space is referred to as Realistic simplified gaugino-higgsino model, and targets specific regions of MSSM parameter space and experimental topologies.

To effectively grasp the structure of SUSY and various models, either in the pMSSM or simplified models, there are a couple key elements to condsider. The first elements is the mass scale of the relative SUSY sectors i.e. how massive are the gauginos versus sleptons versus squarks. If the mass scales are well separated, the sectors are effectively decoupled. If the mass scales are similar then it may introduce complicated cross-talk between sectors. In an electroweak SUSY search with a 4 parameter simplified model, the model can be further simplified by assuming squarks and slepton masses sit at the several TeV scale while the targeted electroweak-inos are at detectable  TeV and sub-TeV scale.  By decoupling sectors outside the sector-of-interest we remove the interaction between these groups, so, if sleptons are decoupled from the gauginos complicated dependencies, like cascading decays are avoided. The other key element is the composition of the LSP, typically $\tilde{\chi}^0_1$. Each unique model point is composed of a specific mixing of $\tilde{H},\tilde{W},\tilde{B}$ with an LSP that reflects that mixing. The model point is denoted by the field that dominates the overall mix, so a Higgsino model has an LSP composed of mostly $\tilde{H}$ (cite mixing altas paper?). The characteristic take away from simplified model types is that H,W,B  can control the nature of the model by governing the overall cross sections for sparticles, the topological infrastructure, and how the sparticles interact and amongst themselves and SM particles. Two pMSSM examples comparing the mass structure between two arbitrary mass points of a Wino model versus Higgsino model is shown in Figure \ref{fig:mass_modelpoint}. For both models the Higgs and slepton sectors are decoupled at a multi-TeV scale while the squark and gaugino sectors are at an accesible TeV and sub-TeV scale. Note that small changes in pMSSM model space results in differing LSP content and large variations in the relative mass structure and orderings. The difference in cross sections between the same two model points for gaugino pair production combinations are show in in Figure \ref{fig:xsec_modelpoint}. This relative differences in cross section illustrates that the same tweak in parameter space can induce order of magnitude changes sparticle production.

\FigTwo{Intro_figs/wino_modelpoint_mass.png}{Intro_figs/higgsino_modelpoint_mass.png}{mass structure winno vs higgsino modelpoints}{fig:mass_modelpoint}


\FigTwo{Intro_figs/wino_modelpoint_xsec.png}{Intro_figs/higgsino_modelpoint_xsec.png}{xsec strucutre wino vs higgsino modelpoints}{fig:xsec_modelpoint}
%Example of differing model points


In addition to the mass structure and cross sections, the decay nature of H/W/B models also varies. The variation in decay modes has a significant impact on the experimental channels and signatures of interest. In an experimental search we would expect the heavier sparticles to decay to both SM particles along with the LSP. If the LSP happens to be close in mass to its parent, say O(100) GeV or less, the model would be considered as a compressed scenario. This scenario is considered compressed because the observable energy of the SM particle involved in a sparticle decay is compressed to a very small amount due to the majority of the available energy being used by the rest mass of the sparticles. Of the 3 types of models, the most likely candidates for compression are the Higgsino-like and Bino-like models. Wino models by far have the largest cross sections but are the least likely to have compressed states. Particularly interesting topologies for these compressed models involve decay signatures of processes like $\tilde{\chi}^0_2 \rightarrow Z^*\tilde{\chi}^0_1 $, $\tilde{\chi}^0_2\rightarrow \tilde{\chi}^\pm_1 \tilde{\chi}^0_1 $, $\tilde{\chi}^\pm_1\rightarrow W^\pm \tilde{\chi}^0_1$, $\tilde{t}\rightarrow t \tilde{\chi}^0_1$, $\tilde{\ell}\rightarrow\ell \tilde{\chi}^0_1$. The nature of sparticle decay is not only dependent on the H/W/B nature of the model but also on the degree of compression. Figure \ref{fig:n2decaymodes} shows the average decay modes for H W or B from a selection of pMSSM models (cite atlas pmssm paper).

\FigThree{Intro_figs/wino_n2decaymodes.png}{Intro_figs/bino_n2decaymodes.png}{Intro_figs/higgsino_n2decaymodes.png}{N2 BFs}{fig:n2decaymodes}  

Note that between each model type in Figure \ref{fig:n2decaymodes} the $Z^*$ and $W^\pm$ modes can be highly suppressed or enhanced. In some cases even, specific modes like $\tilde{\chi}^0_2\rightarrow \tilde{\chi}_1^\pm W^\mp$ can be either kinematically forbidden, or excluded to streamline MC production and enhance the statistical power of different targeted final states. Alongside the decay specific complications, the phase space of the final state particles is model dependent. For instance, in the case of $\tilde{\chi}^0_2 \rightarrow Z^*\tilde{\chi}^0_1 $ the shape of $Z$ dilepton mass distribution $m_{\ell\ell}$  changes depending on the sign of the gaugino eigenstates. Experimentally this problem is divided into two possible scenarios: cases where the eigenstates are the same sign and cases where the eigenstates are the opposite sign. The distribution that showcases the $m_{\ell\ell}$ differences under two different model interpretations is shown in Figure \ref{fig:atlasmllwbh}. Overall, with the complications of model dependent decays, inherently rare production, varying mass orderings, and relative scale between sectors, the search for SUSY is an extraordinary challenge. To discover SUSY one should design a search to encompass a large generalized model space and target generic features rather than highly specific corners. 


\FigOne{Intro_figs/atlas_wbh_mll.png}{mll reweight with w/b or H interpretations from altas paper in grahams talk}{fig:atlasmllwbh}

%What particles are in susY?
%For each elementary standard model particle there is a super partner. For the quarks and leptons, the pairing is simple, there is just the equivalent slepton and squark partner. The gauge bosons are slightly more complicated, these are generally denoted with an "ino" suffix. There are also 3 super fields which mix in specific quantities to yield varying instances of particles with particular properties. These mixings define the characteristics of the model point by influecing things like decay mode, cross seection, and couplings.  (WhY?) There are four neutralinos $\chi^0_i$ and two charginos $\chi^\pm_j$. There are also 4 Hiigs bosons, a charged pair $H^\pm$ and a neutral pair $H^0_{u,d}$. (Why?)  The electroweakinos, i.e chargino or neutralinos, increase in mass with increasing index but the structure of reletavie masses depends specifically on the model. The $\chi^0_0$ is generally the lightest supersymmetric partilce (LSP) and in many popular models is stable. The instances of stable LSP depend on R-parity conservation. (Define R-parity conservation) If this is violated the LSP will decay into SM particles.



%Incldue a plot with mass hierarchies. Since there are so many possible parameters, varying sets of paramters can produce significant diffences in experimental signatures and topologies. Typically for a model we decouple specific sectors when generatting monte carlo, For instance if we are searching for sleptons, the squark or electroweakino sector will be chosen to be significantly heavier (out of current experimental range) effectively decoupling it from the slepton sector. Then from a simplified model with everything else decoupled we scan various topologies with particular mass values. 

%SUSY chirality

%Talk about higgsino/wino bino model structure decay modes etc
