

\setcounter{secnumdepth}{3}
\setcounter{tocdepth}{3}
\setlength{\parskip}{\smallskipamount}
\setlength{\parindent}{0pt}


\makeatletter


\providecommand{\tabularnewline}{\\}


\makeatother

%\usepackage{babel}
%\begin{document}

\chapter{The Standard Model and Supersymmetry}

%\begin{chapterabstract}
%Here we outline the fundamental concepts of particle physics, we introduce the set of fundamental particles, fields, and interactions which %are described by the standard model. Since the standard model does not solve all of the problems of particle physics I introduce %supersymmetry, a well motivated extension to the standard model. The implications of SUSY model space on various phenomoglly of processes %as well as experimental observations.  The motivation for the SUSY extension is reserved for the following chapter.

%\end{chapterabstract}

\section{Introduction}

The fundamental building blocks of matter and their interactions are expressed through three of the four fundamental forces of nature via the Standard Model (SM). The SM  is the culmination of over a century of work by many theorists and experimentalists, with roots in the late 19th century. The experimental starting point begins with the establishment of the sub-atomic world through the discovery of the electron by JJ Thomson in 1897 \cite{Thomson:1897cm}, then the proton by Ernest Rutherford in 1917 \cite{Rutherford:1911zz}, followed by further development with the solution to the mystery of beta decay by the prediction of the neutrino by Enrico Fermi in 1934 \cite{Fermi:1934hr}, and its discovery in 1957 by Clyde Cowan and Frederick Reines \cite{Reines:1956rs}.  After the neutrino,  the theory of strong interactions is formulated as quantum chromodynamics by Murray Gell-Mann and others in the 1960s, providing a description of how protons and neutrons are held together in the nucleus of an atom \cite{GellMann:1964nj} ushering in the modern Standard Model era. In this era, the theory describing weak interactions from Enrico Fermi in the 1930s is combined with electromagnetic interactions in electroweak theory by Sheldon Glashow, Abdus Salam, and Steven Weinberg in the 1960s \cite{GLASHOW1961579}\cite{Salam:1968rm}\cite{Weinberg:1967tq} and later confirmed by the UA1 and UA2 experiments with the discovery of the W and Z bosons in 1983\cite{arnison1983experimental}\cite{glashow1984future}. Following the establishment of electroweak theory, the term Standard Model is coined in the 1970s describing the collection particle physics theories over the last century. The most recent milestone added to the SM, is the discovery of the Higgs boson in 2012 \cite{hCMS:2012qbp}\cite{ATLAS:2012yve}.  Outlining the inner-workings from this history of particle physics, this chapter will introduce concepts of the Standard Model, the fundamental particles, fields, their basic properties, and interactions. Expanding from the  SM we will discuss potential new physics with an extension of the SM which proposes a new symmetry between fermions and bosons called supersymmetry. Finally, we delve into the specifics of simplified models of supersymmetry and the challenges associated with experimental discovery.



\section{The Standard Model}

The Standard Model is a collection of theories used to predict and reproduce experimental data. The theory itself incorporates four major concepts: Quantum Field theory (QFT), the Dirac equation, the gauge principle, and the Higgs mechanism. These four principles are constrained by physical data and describe the set of elementary particles, known as fermions and bosons. The SM generally refers to the SM Lagrangian, an equation with different sectors that describe different subsets of particles, fields, and their interactions. The SM Lagrangian itself consists of 26 free parameters which are input by hand. These parameters are: the masses of the 12 fermions, 3 coupling constants, $g, g', g_s$, to describe gauge interactions,  2 parameters to describe the Higgs potential, $m_h, v$, being the Higgs mass $m_h$ and the vacuum expectation value (vev), and 9 mixing angles describing mixing of different fermionic fields. The 12 fermion mass parameters are subdivided by three neutrinos $m_{\nu_i}$, three charged leptons $m_{\ell_i}^\pm$, and six quarks $m_{q_i}$ \cite{Thomson:2013zua}.

QFT provides a description for both known and theoretical particles by combining quantum theory, the field concept, and relativity \cite{Peskin:1995ev}. The gauge theory aspect describes the exact nature of QFT interactions and provides the mechanisms for the electromagnetic, strong, and weak forces.  We know of three gauge fields:  $\vec{G}$ which transforms under $SU(3)$ and governs strong interactions, $\vec{W}$ and $B$ which transform under $SU(2)_L \times U(1)$ and govern electromagnetic and weak interactions. The combination of the gauge fields, fermion fields, and the Dirac equation yields eigenstates that represent fermionic matter particles. These matter particles would technically be massless if not for the inclusion of the complex scalar Higgs field.  The spontaneous symmetry breaking of the Higgs field, due to the Yukawa coupling, creates the non-zero vev responsible for generating the masses of the electroweak gauge bosons and SM fermions \cite{Higgs:1966ev}\cite{Bernardi:2008zz}.

The set of standard model elementary particles is divided into two subgroups: fermions and bosons.  The fermions consist of both charged and neutral leptons as well as fractionally charged quarks. There are three flavors of charged leptons $(\ell)$, the electron $(e)$, the muon $(\mu)$, and the tau $(\tau)$. Each charged lepton has a flavor pairing neutral neutrino $\nu_\ell$.  As for the quarks, there are also three generations of pairs of quarks.. The lightest set of quarks are the up $(u)$ and down $(d)$ quarks, followed by the charm $(c)$ and strange $(s)$, and lastly the bottom $(b)$ and extremely massive top quark $(t)$.  The bosons are the force carrying particles which represent the gauge fields. They are comprised of the vector bosons - the photon $(\gamma)$, gluon $(g)$, the $W^\pm$, and the $Z^0$ - along with the singular scalar Higgs boson $(h)$ \cite{ParticleDataGroup:2020ssz}. The elementary particles masses, generations, and spins are summarized in Figure \ref{fig:smfig}.

%The $e$ and $\mu$ are also generally considered as "light" leptons due to their small mass relative to the $\tau$. The term lepton, depending on context, often refers to only the charged particles.

\FigOneScale{Intro_figs/Standard_Model_of_Elementary_Particles_updated.png}{The elementary particles of the standard model which includes the color coded categories of fermions and bosons as well as their nominal mass (or mass upper limit), charge, and spin.\cite{particle-physics-wikipedia}}{fig:smfig}{0.65}


%The SM Lagrangian is composed of constituent sectors which describe diffrent groups/fields/particles. The main SM sectors are, the quantum chromodynamics (QCD) sector, the electroweak sector, the Higgs sector, and the Yukawa sector. QCD describes colored interactions of quarks mediated by gluons with the strong force. The electroweak sector unifies both the electromagnetic and weak interactions via exchange of W or Z bosons as well as electromagnetic interactions via $\gamma$. The Higgs sector introduces the complex scalar higgs field (citation needed). Interaction of bosons with the Higgs field causes the bosons to have mass and the Yukawa coupling describes the interaction of fermions with the higgs field which also allows the fermions to have mass (citation needed).



The SM is an asymmetric chiral theory, combining three groups $SU(3)_C \times SU(2)_L \times U(1)$. The $L$, or left handed, subscript indicates that mirrored fields (with different chiralities)  transform differently under the Lorentz group and the EW gauge group.  The consequence of chirality is that the possible combinations between interaction vertices is limited \cite{Thomson:2013zua}. This peculiar property shows up with the $W$ boson, which only couples to left handed particles or right handed antiparticles. Extensions of the standard model also often extend chiral or symmetrical properties.  %Helcity is the defined by the projection of a particles spin ontion its direction of motion (thompson). A particle is considred right-handed if the direction of its spin and motion is parallel. %It is left-handed if spin and motion antiparallel. (cite wikipedia helicity). This peculiar property shows up with the W boson, which only couples to left handed particles or right handed antiparticles.

\section{Supersymmetry}

Supersymmetry (SUSY) is an extension of the standard model that adds a generator rotating the spin between bosons and fermions. This introduces a bosonic degree of freedom for every fermionic degree of freedom, and in turn, a super partner for each particle differing by spin one-half \cite{Baer:2007izw}.  The resulting set of mirrored elementary particles are referred to as sparticles. Each bosonic sparticle carries the same name as its fermion partner but with an `s' prefix e.g. sfermion, squark, selectron. As for the bosons, with the gauge fields $B$ and $\vec{W}$, these are accompanied by three super symmetric fields - the Higgsino $\tilde{H}$, and gauginos represented by the Bino $\tilde{B}$, and Wino $\tilde{W}$. The set of boson superpartners can be obtained with the same approach as the SM where the mixture of the B and $\vec{W}$ SM fields are represented by a particle matrix.  The diagonalization of the particle matrix leads to mass eigenstates representing the SM particles $\gamma, \, \, Z, \, \, W^\pm$. Similarly, the Higgsino, Bino, and Wino super fields mix to produce four neutral and two charged eigenstates, the neutralinos ($\tilde{\chi}^0_1, \tilde{\chi}^0_2, \tilde{\chi}^0_3, \tilde{\chi}^0_4$)  and charginos ($\tilde{\chi}^\pm_1, \tilde{\chi}^\pm_2$) \cite{DJOUADI_2008}. SUSY also requires an additional Higgs doublet to give mass to up-type and down-type fermions,  leading to five Higgs boson states consisting of two charged Higgs and three neutral Higgs \cite{Adam:2021rrw}. The lightest neutral Higgs of the two scalar neutral ones is usually assumed to be the experimentally discovered Higgs boson. The full set of SM particles alongside their SUSY partners are illustrated in Figure \ref{fig:smandsusyfig}. The addition of another Higgs doublet also introduces a second vev. The ratio between the two vevs is denoted as $v_2/v_1 = \tan \beta$ and is an important parameter in experimental searches. An important bookkeeping parameter, similar to lepton number or baryon number conservation, is R-parity. This parameter tallies the total number of SM particles (+1) and sparticles (-1) and expects the product among particles to be conserved in the initial and final states. R-parity conservation then requires sparticles to be produced in pairs. If R-parity is violated, the common consequence is that the lightest supersymmetric particle (LSP) is unstable or models allow protons to decay \cite{Farrar:1978xj}. 

\FigOneScale{Intro_figs/smandsusy.png}{The elementary particles of the standard model with their supersymmetric partners.}{fig:smandsusyfig}{0.8}



Supersymmetry is an expansive model and intractable to experimentally test without significant well motivated simplifications. The most experimentally common simplified SUSY model is the Minimally Supersymmetric Standard Model (MSSM). The MSSM contains the smallest number of new particle states and new interactions which are consistent with phenomenology \cite{Baer:2007izw}. The MSSM, in full generality, is experimentally inaccessible due to the presence of over 100 parameters. To reduce the problem's dimensionality, further simplification is needed, resulting in a popular simplified model: the phenomenological MSSM (pMSSM). The pMSSM contains 19 parameters which include the masses of each generation of squark and slepton, parameters to control the mixing of $\tilde{H}, \tilde{W}, \tilde{B}$, and dials for the Higgs doublet \cite{MSSMWorkingGroup:1998fiq}.  The pMSSM is still borderline too complicated to attack directly, so, the pMSSM is boiled down into a simplified model of four parameters $M_1$,$M_2$, $\mu$, and $\tan\beta$. $M_1$ and $M_2$ are the gaugino mass parameters, $\mu$ is the Higgsino mass parameter, and $\tan\beta$ is the previously mentioned vev ratio \cite{Fuks_2018}.  A model point from this four parameter space is referred to as Realistic simplified gaugino-higgsino model, and targets specific regions of MSSM parameter space and experimental topologies. Realistic simplified models are typically used in most experimental searches, including this one.

To effectively grasp the structure of SUSY and various models, either in the pMSSM or simplified models, there are a couple of key elements to consider. First is the mass scale of the relative SUSY sectors i.e. how massive are the electroweakinos versus the sleptons versus the squarks. If the mass scales are well separated, the sectors are effectively decoupled. If the mass scales are similar then it may introduce complicated cross-talk between sectors. A SUSY search with a 4 parameter simplified model can be further simplified by assuming squarks and slepton masses sit at the several TeV scale while the targeted electroweak-inos are at detectable TeV and sub-TeV scale.  By decoupling sectors, outside the sector-of-interest, we remove the interaction between these groups, so for example, if sleptons are decoupled from the charginos then complicated dependencies, like cascading between the two decays are avoided. The other key element is the dominant type of super field coupling or the composition of the LSP, typically $\tilde{\chi}^0_1$.  A model point can be denoted by the field that dominates the overall LSP mixing, i.e. a Higgsino model LSP would be composed of a majority $\tilde{H}$ \cite{ATLAS:2015wrn}. The characteristic take away from simplified model types is that $\tilde{H},\tilde{W},\tilde{B}$ controls the nature of the model by governing the cross sections, topological infrastructure, and how the sparticles interact amongst themselves and SM particles. Two pMSSM examples comparing the mass structure between two arbitrary mass points of a Wino model versus Higgsino model are shown in Figure \ref{fig:mass_modelpoint}. For both models in Figure \ref{fig:mass_modelpoint} the Higgs and slepton sectors are decoupled at a multi-TeV scale while the squark and gaugino sectors are at an accessible TeV and sub-TeV scale. Note that small changes in pMSSM model space results in differing electroweakino content and large variations in the relative mass structure and orderings. The difference in cross sections between the same two model points for neutralino/chargino pair production combinations are show in in Figure \ref{fig:xsec_modelpoint}. This relative difference in cross section illustrates that the same tweak in parameter space can induce order of magnitude changes in sparticle production. The cross-section difference can also be generalized in terms of the dominant model couplings with the Wino/Bino cross-section being a factor of 8 larger than Higgsino cross-section.

\FigTwoScale{Intro_figs/wino_modelpoint_mass.png}{Intro_figs/higgsino_modelpoint_mass.png}{Left: Wino-like LSP from PMSSM model point 18898934. Right: Higgsino-like LSP from PMSSM model 6755879 \cite{ATLAS:2015wrn}. Both models have a somewhat similar relative mass structure but order of magnitude differences in the Higgs, squark, and slepton sectors.}{fig:mass_modelpoint}{0.49}{0.49}


\FigTwoScale{Intro_figs/wino_modelpoint_xsec.png}{Intro_figs/higgsino_modelpoint_xsec.png}{Comparison of the Wino LSP and Higgsino LSP models from Figure \ref{fig:mass_modelpoint} sparticle pair production cross sections.}{fig:xsec_modelpoint}{0.75}{0.75}
%Example of differing model points


In addition to the mass structure and cross sections, the decay specifics of each gaugino and Higgsino mixing also varies. The variation in decay modes has a significant impact on the experimental channels and signatures of interest. In an experimental search we would expect the heavier sparticles to decay to both SM particles along with the LSP. If the LSP happens to be close in mass to its parent, say O(100) GeV or less, the model would be considered as a compressed scenario. This scenario is considered compressed because the observable energy of the SM particle involved in a sparticle decay is compressed to a very small amount due to the majority of the available energy being used by the rest mass of the sparticles.  Particularly interesting topologies for these compressed models involve decay signatures of processes like $\tilde{\chi}^0_2 \rightarrow Z^*\tilde{\chi}^0_1 $, $\tilde{\chi}^0_2\rightarrow \tilde{\chi}^\pm_1 W^\mp $, $\tilde{\chi}^\pm_1\rightarrow W^\pm \tilde{\chi}^0_1$, $\tilde{t}\rightarrow t \tilde{\chi}^0_1$, $\tilde{\ell}\rightarrow\ell \tilde{\chi}^0_1$. To further complicate the experimental topologies, the nature of sparticle decay is not only dependent on the superfield mixing but also on the level compression. Figure \ref{fig:n2decaymodes} shows the average decay modes for $\tilde{H}, \tilde{W},$ or $\tilde{B}$ LSPs as a function of mass splittings from a selection of pMSSM models \cite{ATLAS:2015wrn}.

\FigThreeScale{Intro_figs/winoN2decay.pdf}{Intro_figs/binoN2decay.pdf}{Intro_figs/hinoN2decay.pdf}{Example average neutralino branching fractions for the combined set of pMSSM model points from \cite{ATLAS:2015wrn} for $\Delta m < 50$ GeV. The LSP content for each figure is Wino-like top-left, Bino-like top-right, and Higgsino-like bottom. Branching fractions are shown as a function of the mass splitting $\Delta m = m_{\tilde{\chi}_2^0 } - m_{\tilde{\chi}_1^0}$ . The model with Wino LSP is subject to a $W$ channel enhancement while $Z$ is suppressed. In the Bino case the W channel is suppressed and the Z is enhanced. The Higgsino LSP models can be be dominated by either the W or Z channel depending on the mass splitting.}{fig:n2decaymodes}{0.49}{0.49}{0.49}


Note that among each model type in Figure \ref{fig:n2decaymodes} the $Z^*$ and $W^\pm$ modes can be highly suppressed or enhanced. In some cases even, specific modes like $\tilde{\chi}^0_2\rightarrow \tilde{\chi}_1^\pm W^\mp$ may be kinematically forbidden. Even when kinematically allowed such decay possibilities have often been neglected. Alongside these branching fraction complications, the decay phase space of the final state particles is also model dependent. For instance, in the case of $\tilde{\chi}^0_2 \rightarrow Z^*\tilde{\chi}^0_1 $ the shape of the $Z$ di-lepton mass distribution $m_{\ell\ell}$  changes depending on the sign of the $\tilde{\chi}_2^0$ and $\tilde{\chi}_1^0$ eigenstates. Experimentally this problem is then divided into two possible scenarios: cases where the eigenstates are the same sign and cases where the eigenstates are the opposite sign. The distribution that showcases the $m_{\ell\ell}$ differences under two different model interpretations is shown in Figure \ref{fig:atlasmllwbh} from a previous ATLAS search. Overall, with the complications of model dependent decays, inherently rare production, varying mass orderings, and relative scale between sectors, the search for SUSY is an extraordinary challenge. To discover SUSY, a reasonable approach is to try to be generically sensitive to all of the formerly discussed complicated nuances. 


\FigOneScale{Intro_figs/atlas_wbh_mll.png}{Example reweighing of simplified model from an ATLAS search which covers the different dilepton mass line shapes in the same-sign and opposite-sign eigenstate cases \cite{ATLAS:2019lng}.}{fig:atlasmllwbh}{0.7}

%What particles are in susY?
%For each elementary standard model particle there is a super partner. For the quarks and leptons, the pairing is simple, there is just the equivalent slepton and squark partner. The gauge bosons are slightly more complicated, these are generally denoted with an "ino" suffix. There are also 3 super fields which mix in specific quantities to yield varying instances of particles with particular properties. These mixings define the characteristics of the model point by influecing things like decay mode, cross seection, and couplings.  (WhY?) There are four neutralinos $\chi^0_i$ and two charginos $\chi^\pm_j$. There are also 4 Hiigs bosons, a charged pair $H^\pm$ and a neutral pair $H^0_{u,d}$. (Why?)  The electroweakinos, i.e chargino or neutralinos, increase in mass with increasing index but the structure of reletavie masses depends specifically on the model. The $\chi^0_0$ is generally the lightest supersymmetric partilce (LSP) and in many popular models is stable. The instances of stable LSP depend on R-parity conservation. (Define R-parity conservation) If this is violated the LSP will decay into SM particles.



%Incldue a plot with mass hierarchies. Since there are so many possible parameters, varying sets of paramters can produce significant diffences in experimental signatures and topologies. Typically for a model we decouple specific sectors when generatting monte carlo, For instance if we are searching for sleptons, the squark or electroweakino sector will be chosen to be significantly heavier (out of current experimental range) effectively decoupling it from the slepton sector. Then from a simplified model with everything else decoupled we scan various topologies with particular mass values. 

%SUSY chirality

%Talk about higgsino/wino bino model structure decay modes etc
