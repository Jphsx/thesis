\setcounter{secnumdepth}{3}
\setcounter{tocdepth}{3}
\setlength{\parskip}{\smallskipamount}
\setlength{\parindent}{0pt}


\makeatletter


\providecommand{\tabularnewline}{\\}


\makeatother

\chapter{Modeling}

\section{Introduction}
%introduce actual fit, fit regions, systematics treatment, results of fit stages
The modeling for this analysis uses the standard ounting experiment approach with a Poisson likelihood. The MC model is data driven such that background composes most of the regions and tunes the data and MC agreement. This agreement translates into well constrained background predictions in sensitive regions with sparse background with an ABCD-like approach. To achieve a robust fit model three stages of fits are performed, one with the Control Region (CR) which has no signal presence, a Validation Region (VR) which is a partially unblinded region with mild sensitivity designed to validate the CR model and demonstrate reasonable modeling in regions untouched by the CR , and finally SR which is comprised of the high $R_{ISR}$ bins which are sensitive to all signal regions. Following the full fit, limits can be calculated which test the hypotheses of background only model versus background plus signal model. 
\section{Fit Defintion}
%Poisson likelihood



\section{Fit Stragegy and Fit Region Definitions}
CR VR SR

\section{Treatment of systematics}
definition of sytematics
poisson pulls

\section{Region Results}

CR and VR plots 
pull dists

