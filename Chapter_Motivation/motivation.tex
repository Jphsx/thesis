
\setcounter{secnumdepth}{3}
\setcounter{tocdepth}{3}

\setlength{\parindent}{1 em}


\makeatother


\chapter{Motivating the Search for SUSY}

%\begin{chapterabstract}
%Introduce some of the issues of the SM and opening up with the basic motivations for susy, like solving the hierarchty problem and dark matter candidate, next we look %at a theoretical motivation for SUSY via the higgs mass. Motivate simplified models with naturalness etc, talk about how susy needs to be at a few TeV scale to work %out. Then we visit two recent experimental measurements which strongly motivate the search for susy and more speficically this body of work.
%\end{chapterabstract}

\section{Introduction}

The SM is a remarkable theory which describes a wide variety of sub-atomic phenomenon and has consistently held up to tests over many orders of magnitude in energy. However, it's not a perfect theory. There are a few  experimental and theoretical problems that the SM can not yet explain like: how to incorporate gravity,  how can we explain neutrino mass and mass orderings, why is the universe made up of matter and not antimatter?  
A significant problem in physics in general is one that connects both cosmology and particle physics and is what is dark matter. Cold dark matter (CDM) is a type of matter that is thought to have played a crucial role in the formation of large-scale structures in the Universe, such as galaxies and galaxy clusters \cite{Garrett:2010hd}. The evidence for CDM begins with observations of Zwicky in 1933 where he found that the observed motion of the galaxies in the Coma Cluster could not be explained by the gravitational interactions visible matter in the cluster, but could be explained by adding additonal invisble mass in the form of ``dark matter" \cite{Zwicky:1933gu}. Additional evidence for CDM has compiled over the years such as: gravitational lensing with observed data that disfavors being completely explained by effects black holes or condensed baryonic matter \cite{Massey:2007lens}, large scale structure formation where CDM can explain the formation and evolution of galaxy clusters \cite{Springel:2005}, and temperature fluctuations in the cosmic microwave background suggest the universe is composed of approximately 85\% dark matter \cite{Planck:2018vyg}. Depsite overwhelming evidences for the existence of CDM there are no suitable SM dark matter candidates to explain the abundance this potential cosmic particle. SUSY offers many attractive solutions with the introduction of new particles. One such new particle serves as a mediator of gravitational interactions, the gravitino. Other new particles can explain dark matter directly via massive invisible particles which act as candidates, such as the neutralino, $\chi_1^0$. The neutralino can handle the DM problem with models capable of producing the expected relic DM density of the universe and, in fact, the CDM relic density is used to constrain the SUSY model space and simplify searches. Aside from these leading motivations, other more detailed motivations will be discussed in this chapter, the first being the ``naturalness problem" with its theoretically aesthetic improvement which adds a symmetry to protect against divergent terms in the perturbative expansion of the Higgs mass. The next motivations are experimental, where SUSY offers an explanation to the signficant deviation observed in the muon $(g-2)$ factor from recent FNAL result, as well as the deviation observed in the W boson mass at CDF. It should be noted that the divergent higgs mass - known as the hierarchy problem - satisfies most SUSY scenarios up to the few TeV scale \cite{Barbieri:1987fn}, but, the two latter experimental measurements motivate searching for SUSY in compressed scenarios.



\subsection{Stabilizing the Higgs mass}

An aesthetic attribute of theoretical models is naturalness, we should expect a model to function "naturally" if the ratio of free parameters in a model are of $O(1)$. Large swings between parameters would be considered fine-tuning and could indicate issues with the underlying theory. So, naturally, if fine-tuning exists in a model, it strongly motivates building extensions to the model to eliminate fine-tuning. %. We would expect with some improved theory with new physics one would balance out finely tuned parameters providing a natural solution to whatever is being modeled. 
One such fine tuning arises in the hierarchy problem, specifically in the Higgs self interaction terms. The SM Higgs Lagrangian terms that involve self interaction are illustrated in equation \ref{eq-higgslagrange}.
\begin{equation}
\label{eq-higgslagrange}
\mathcal{L}=\frac{gm_h}{4M_W}H^3 - \frac{g^2m_h^2}{32M_W^2}H^4
\end{equation}
$H$ represents the scalar higgs field, $m_h$ the higgs mass, and $m_W$ the W boson  mass. A correction to the higgs mass can be calculated using standard perturbation theory by evaluating the second term of the Higgs Lagrangian \cite{Baer:2007izw}. 
\begin{equation}
\begin{split}
\Delta m_h^2 = \langle H | \frac{g^2m_h^2}{32M_W^2} H^4 | H  \rangle = 12\frac{g^2m_h^2}{32M_W^2}\int \frac{d^4 k}{(2\pi)^2} \frac{i}{k^2 - m_h^2}\\
= 12\frac{g^2m_h^2}{32M_W^2} \frac{1}{16\pi^2}\big( \Lambda^2 - m_h^2\log\frac{\Lambda^2}{m_h^2} + O(\frac{1}{\Lambda^2})\big)
\end{split} 
\end{equation}
 
 Here the intergal term is the propagator for the exchange of a virtual Higgs and is integrated over phase space. The $\Lambda$ is known as the scale cutoff parameter and should be interpreted as the scale at which the SM breaks down, possibly near the planck scale $O(10^{19})$GeV. Notice the leading term $\Lambda^2$ indicates that the expansion is quadratically divergent. The divergent mass correction means there needs to be extremely large cancellations, around 20 orders of magnitude, to maintain $\Delta m_h \propto O(m_h)$. This divergent phenomenon can also be observed with fermion masses, but, chiral symmetry protects the fermion mass from divergence by cancelling out high order $\Lambda$ terms. SUSY offers a similar protection to the Higgs mass by introducing a symmetry with the additional fermionic and bosonic degrees of freedom leading to similar cancellations and a more natural model. 


\subsection{The Muon Anomalous Magnetic Moment}

A major experimental motivation for SUSY lies within the measurement of the muon anomalous magnetic moment.  Multiple meausurements between two labs, Brookhaven National Lab (BNL) and Fermi National Accelerator Lab (FNAL) have shown significant disagreement with the SM. These experiments measure the muon $g$ factor, or specifically, its deviation from two, $(g-2)_\mu$ .  The $g$ factor is related to the electromagnetic coupling of charged particles with the photon and largely depends on the tree level lepton-photon coupling, but, gets small quantum corrections from higher order loops. The largest correction being the single photon loop shown in Figure \ref{fig:gm2fig}. To predict the $g$ factor, an SM calculation is performed with three types of quantum corrections: Quantum Electrodynamic (QED), Electroweak (EW), and Hadronic. Corrections from the Higgs are neglected because the effects are not experimentally observable. %the mass disparity $m_h >> m_{e,\mu}$ and the mass dependence in the Higgs coupling  effects that are smaller than what is experimentally observable. 
The g-factor prediction starts at exactly 2, with QED, and then involves quantum corrections up to $O(10^{-11})$. The prediction is compared with an experimental measurement at a very high level of precision. If the observation were to deviate from the SM prediction, it can indicate new and unaccounted physics interactions with the SM leptons.
The current best $a_\mu = \frac{g-2}{2}$ prediction is reported as $a_\mu= a_\mu^{QED}+ a_\mu^{EW}+a_\mu^{\text{Hadronic}} =  116 591 810(43) \times 10^{-11}$ \cite{Muong-2:2021ojo}.
\FigOne{Motivation_figs/g2_diagrams_pdg.png}{figures from https://www.particlebites.com/?p=8972 which cites pdg \cite{ParticleDataGroup:2020ssz} }{fig:gm2fig}
 For each of the $a_\mu$ components, the QED compenent enters at the $O(10^{-3})$ and is known to $O(10^{-11})$. the EW component enters the sum at $O(10^{-9})$ and is known to $O(10^{-10})$. Finally the most complicated hadronic component, contributes at $O(10^{-8})$ and is known up to $O(10^{-9})$. The hadronic contributions arise from Hadronic vacuum polarization(HVP) and light by light scattering with the  HVP diagram also illustrated in Figure \ref{fig:gm2fig}. The $a_\mu^{\text{Hadronic}}$ precision dominiates the overall $a_\mu$ error and is constrained by data driven measurements alongside the limitations of the computational approach with QCD lattice theory. The BNL measurement of $a_\mu$ yields a difference with the SM prediction of $\Delta a_\mu := a_\mu^{BNL} - a_\mu^{SM} = 279(76) \times 10^{-11}$ which carries significance of $3.7\sigma$. The most recent $a_\mu$ measurement from FNAL confirms the BNL measurement within $1\sigma$ and the combined experimental average increases the SM deviation with a significance of $4.2\sigma$ \cite{Muong-2:2021ojo}.


%Experimentally the deviations from 2 are the most interesting, and are written in the form $a_\ell = \frac{g-2}/2$ and referred to as $(g-2)_\ell$. These small contribtutions are interesting because they encapsulate the current theory and provide a test bed for our current understanding.  If observations were to deviate from the SM prediction, it would be an indication of new and unaccounted physics interactions with the SM leptons. The $g$ factor can be extracted by measuring the anomalous magnetic moment of any generation of charged lepton.  The current best candidate to both test the SM and search for new physics is by measuring $(g-2)_\mu$ or $a_\mu$ because of  experimental precision potential. The electron measurement is already known to the highest precision and is expected to have the smallest contributions from new physics (cite youtube citation). The $(g-2)_\tau$ is not yet experimentally tractable competitive precision to $\mu$ or $e$.%so $(g-2)_\mu$ has been measured at both at Brookhaven National Lab (BNL) and again at Fermi National Accelerator Laboratory (FNAL).
%The current best SM prediction of $a_\mu$ from (CITE g-2 collab) includes QED, Electroweak(EW) and Hadronic contributions and is reported as $a_\mu^{SM} = a_\mu^{QED}+ a_\mu^{EW}+a_\mu^{\text{Hadronic}} = 116 591 810(43) \times 10^{-11}$. For each of the $a_\mu$ components, the QED compenent enters at the $O(10^{-3})$ and is known to $O(10^{-11})$. the EW component enters the sum at $O(10^{-9})$ and is known to $O(10^{-10})$. Finally the most complicated component, hadronic, contributes at $O(10^{-8})$ and is known up to $O(10^{-9})$, the  main sub components that contribute to the $a_\mu^{\text{Hadronic}}$ is the Hadronic vacuum polarization and light by light scattering, diagrams illustrated in Figure X. The hadronic precision is constrained by data driven measurements and computation approaches -- QCD lattice theory, this error dominates the overall uncertainty of $a_\mu$. The BNL measurement of $a_\mu$ yields a difference with the SM prediction of $\Delta a_\mu := a_\mu^{BNL} - a_\mu^{SM} = 279(76) \times 10^{-11}$ which is a significance of $3.7\sigma$. The most recent $a_\mu$ measurement from FNAL confirms the BNL measurement within $1\sigma$ and the combined experimental average increases the SM deviation with a significance of $4.2\sigma$.

%What could this deviation mean?
The $4.2\sigma$  is a compeling sign for new physics, but not a smoking gun. It is possible to reduced or eliminate the discrepancy by improving the calculations of the HVP and LBL contributions. New and updated calculations are be done to attempt to resolve the discrepancy. Some new calcuations reduces the discrepancy by a few $\sigma$, but do not fully resolve the differences between observations and theory. If computational improvements can't bring the theory into focus, new particles introduce  quantum corrections that will bring experiment and theory into agreement. Several models qualify and successfully explain the $a_\mu$ SM deviation, one being SUSY, where for example, contributes additional diagrams via the smuon-muon coupling illustrated in Figure \ref{fig:gm2susy}.

\FigOne{Motivation_figs/g2_susy_loop.png}{SUSY diagrams explaining g-2 from svens talk\cite{SvenTalkgm2}}{fig:gm2susy}

%g-2 is an experiment designed to measure the anomolaus magnetic dipole moment of the muon. The spin magnetic moment of a charged, spin-1/2 particle that does not possess any internal structure (a Dirac particle) is given by (wiki direct quote \url{https://en.wikipedia.org/wiki/G-factor_(physics)}) ${\displaystyle {\boldsymbol {\mu }}=g{e \over 2m}\mathbf {S} }$. where g is the particles g-factor, $\mu$ is the magetic moment, $m$ is the particle mass and $S$ is the spin. The g-factor in quantum electrodynamics is close to 2 so typically the reported measurement is the difference from 2 or g-2 or as a signficance $a_\mu = g-2/2$. The difference from 2 arises from higher order contributions in quantum field theory

\subsection{The W boson mass}
%the most recent w mass measurement yielded a heavy W, this higher mass is more favorable for light higgsino and compressed susy models

%What is the W boson
The W boson is an important and peculiar particle, it is the electrically charged boson and couples only with left handed particles. The decay modes follow two channels: (1) the hadronic mode with different flavor quark pairs and (2) the leptonic mode with a charged lepton and neutrino. Measuring the W mass directly is challenging at the LHC due to either high levels of QCD di-jet background or missing energy from the neutrino. The mass parameter itself, $m_W$, underpins many important parameters in the SM as well. In fact, $m_W$ is related to the Higgs vev, which implies that coupling of the higg field to all particles is effectively tuned by $m_W$. Similary, $m_W$ is related to the $g$ factor from $(g-2)_\ell$ and both of these parameters together can be used to constrain new physics. The W mass can also be paramterized at tree level in terms the fine structure constant $\alpha$, the Fermi constant $G_\mu$ and the Z-boson mass $m_Z$, with higher order radiative corrections coming from $\Delta r$  \cite{Awramik:2003rn}.
\begin{equation}
\label{eq:mwequation}
m_W^2 = m_Z^2\Bigg(\frac{1}{2} + \sqrt{\frac{1}{4} - \frac{\pi\alpha}{\sqrt{2}G_\mu m_{Z}^2 }(1+\Delta r) } \Bigg)
\end{equation}

%What is the current status of the W boson?
There is no exact SM prediction of the W mass, but, since there is an interdependence of many parameters such as $v$, $m_z$, $G_\mu$,$\alpha$ , the SM "prediction" is constrained by experimentally measured parameters. The most recent measurement of $m_W$ was performed by CDF II at the Tevatron where $m_W$ was obtained by fitting the kinematic distributions of light leptonic decays recoiling against a system of jets. This measurement is $50\%$ more precise than the previous measurement by ATLAS and heavier than the SM prediction. The combination of a large deviation with very small error bars results in a significance of $7\sigma$ \cite{CDF:2022hxs}.  

%What could this deviation mean?
If we believe the CDF measurement, and follow up experiments confirm the excess in the W mass, it is definite sign of new physics. The new physics would express itself as new particles in the radiative corrections via equation \ref{eq:mwequation}. There are numerous SUSY models that could explain the excessive mass of the W boson, but in general, a slightly heavier W favors light SUSY models which is illustrated in Figure \ref{fig:cdfw}. A light SUSY implies models that are characterized by electro-weak scale SUSY particles among which  can favor compressed scenarios. The available parameter space, with an abundance of models,  which can satisfy heavy $m_W$ and deviations in $(g-2)_\mu$ is shown in Figure \ref{fig:gm2mw}.
% Due to the interdependence of $m_W$ and $(g-2)_\mu$, these parameters both constrain compressed SUSY and spotlight a critical area to search. An example of model points of sleptons and gaug-ino models which satisfy the newest $g-2$ and W mass constraints in shown in FIGURE Z (cite Wmass and g-2 sven paper)
\FigOne{Motivation_figs/wmass_cdf.png}{plot from CDF paper \cite{CDF:2022hxs}}{fig:cdfw}
\FigOne{Motivation_figs/gm2_mw.png}{sven plot from paper \cite{Bagnaschi:2022qhb}}{fig:gm2mw}
\section{The current status of SUSY}
%drop the most recent limits here, start with multi TeV excluded gluino and squark models which leaves the a good place to search in the weak scale sector with electoweakinos. Talk about electroweak limits and how alot of these are excluded already one of the remaining places to search is the compressed corridor where mass splittings are small. link this limit motivation with how both g-2 and W mass favor compressed scenarios

There have been many searches for SUSY particles, starting from searches at LEP and still ongoing at the LHC today with both the ATLAS and CMS experiments. There is no observed evidence of SUSY yet, but, there is still not enough lack of observation to fully reject the SUSY hypothesis. The most widely searched region SUSY space is related to strong production of SUSY particles. The large expected cross sections compared to other sectors gauginos/sleptons offers the most low hanging fruit for potential discovery.
 %ut there is still plenty of room to keep searching. 
%The strong production of SUSY has the largest excluding limits due to the large expected cross sections compared to other sectors gauginos/sleptons.
Simplified models search in ATLAS and CMS have excluded $\tilde{g}$ and $\tilde{q}$ (excluding stop) up to around 2 TeV with the most recent limits are shown in Figure \ref{fig:run2stronglim} and \ref{fig:run2squarklim} alongside exlusions of stop squarks around 1 TeV shown in \ref{fig:stoplim}. The area inside the lines in Figure \ref{fig:run2stronglim} indicate that the 2-D mass points of the sparticle and LSP pair are ruled at a 95\% confidence level for the associated simplified model.
\FigTwo{Motivation_figs/gluinoPair_lim.png}{Motivation_figs/atlas_gluinopair_lim.png}{gluino limits \cite{cmslims}\cite{atlaslims}}{fig:run2stronglim}

\FigTwo{Motivation_figs/squarkPair_lim.png}{Motivation_figs/altas_squark_lim.png}{squark limits \cite{cmslims}\cite{atlaslims}}{fig:run2squarklim}

\FigTwo{Motivation_figs/cms_stop_limit.png}{Motivation_figs/Atlas_stop_limit.png}{top limits \cite{cmslims}\cite{atlaslims}}{fig:stoplim}

Simlarly the CMS and ATLAS electroweak limits are shown in Figure \ref{fig:ewlims}. Note that the electro weak limits are just now reaching the TeV scale while SUSY remains valid at the few TeV level, so, there is plenty of room in the 2-D mass plane to either discover or exclude SUSY by adding more data. One particluar simplified model which is nearly undaddressed by both CMS and ATLAS is the di-chargino production associated with final states consistent with two oppositely charged W bososns.  Most simplified models assume a mass degeneracy with $m_{\tilde{\chi}^0_2} = m_{\tilde{\chi}^\pm_1}$ but, there is no reason to believe that $\tilde{\chi}^0_2$ can be decoupled from $\tilde{\chi}^\pm_1$. In this particular case, the limits would not apply in excluding charginos of any mass. So, it is important to address this nearly untouched final state. The slepton limits are for both CMS and ATLAS are showin in Figure \ref{fig:elplims} which are generally the weakest limits of all the aforementioned processes, but, potentially the most important in association with $(g-2)_\mu$.



\FigTwo{Motivation_figs/c1n2_lim.png}{Motivation_figs/Atlas_chiwz_lim.png}{ chargino limits \cite{cmslims}\cite{atlaslims}}{fig:ewlims}

\FigTwo{Motivation_figs/slep_lim.png}{Motivation_figs/Atlas_slepton_lim.png}{ slep limits \cite{cmslims}\cite{atlaslims}}{fig:sleplims}

For all of the previously presented limits from CMS and ATLAS, excluding gluinos and squarks, there is a common thread which is the weakest region of the limit is the compressed region. The compressed region varies from process to process, such as compressed relative to stop production is such that the mass difference $\Delta m = m_{NLSP} - m_{LSP}$ is less than the SM top mass $\Delta m < m_t$. In for compression in eletroweakinos the associatied $\Delta m$ is below the W or Z pole. For sleptons there is no intermediate heavy particle like a W,Z or t so the compressed region is more ambiguous, here compressed could be considered "soft" which could be interpreted as $O(20-30)$ GeV or less. Now for each of the compressed scenarios, most are unadressed by CMS, but there are dedicated compressed searches in ATLAS. Each ATLAS result when combining different search results, has a common feature which is a large gap between the compressed and non compressed results. So, based on all of the leading SUSY results we see there is strong motivation to created confirmative compressed results for ATLAS with a CMS compressed search. The compressed results from ATLAS also have room to be improved as well as the gaps between results can be filled with a generic search. 

