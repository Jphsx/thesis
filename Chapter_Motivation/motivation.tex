
\setcounter{secnumdepth}{3}
\setcounter{tocdepth}{3}

\setlength{\parindent}{1 em}


\makeatother


\chapter{Motivating the Search for SUSY}

%\begin{chapterabstract}
%Introduce some of the issues of the SM and opening up with the basic motivations for susy, like solving the hierarchty problem and dark matter candidate, next we look %at a theoretical motivation for SUSY via the higgs mass. Motivate simplified models with naturalness etc, talk about how susy needs to be at a few TeV scale to work %out. Then we visit two recent experimental measurements which strongly motivate the search for susy and more speficically this body of work.
%\end{chapterabstract}

\section{Introduction}

The SM is a remarkable theory that has consistently held up to tests over many orders of magnitude in energy. The SM, however, is not a perfect theory in such that there are a few  experimental and theoretical problems that the SM can not yet explain. For instance, the current SM doesn't have an explanation for neutrino mass which is implied by observations of flavor oscillations (cite solar and atmo neutrino data). There is also not a theoretically proven understanding of the all fundamental particle masses and mixing as well as generational mass orderings. Observations of from the relic microwave background suggest the existence of cold dark matter (cite DM observation) but there are currently no SM particles which make a suitable dark matter candidate. SUSY offers many attractive solutions to SM problems. Fore example, specifically can handle the DM problem because it offers a DM candidate in the form of the LSP $\chi_1^0$, in fact, the expected relic DM density of the universe is used to constrain the SUSY model space and provide more well motivated search regions. Three motivations which will be discussed in the following sections, the first being is the  natural and theoretically aesthetic improvement by adding a symmetry to protect against divergent terms in the perturbative expansion of the higgs mass, second is an explanation to the signficant deviation observed of the muon $(g-2)$ factor from recent FNAL result, and lastly another explanation to the deviation observed in the W mass at CDF. It should be noted that the divergent higgs mass - known as the hierarchy problem - satisfies most SUSY topologies up to the few TeV scale, but, the two latter experimental measurements not only motivate searching for SUSY but point specifically compressed SUSY scenarios.



\subsection{Stabilizing the Higgs mass}

A commonly pursued aesthetic attribute of theoretical models is naturalness, typically we expect a model to function "naturally" if the ratio of free parameters in a model are of $O(1)$, large swings in parameters would be considered "fine tuning". Fine tuning is only an aesthetic problem, but could indicate issues with the underlying theory. We would expect with some improved theory with new physics one would balance out finel tuned parameters providing a natural solution to whatever is being modeled. One such fine tuning arises in the hierarchy problem, from Higgs self interaction terms. This self interaction is illustrated in equation Xbelow with the SM Higgs Lagrangian.
\begin{equation}
\mathcal{L}=\frac{gm_h}{4M_W}H^3 - \frac{g^2m_h^2}{32M_W^2}H^4
\end{equation}
$H$ represents the scalar higgs field, $m_h$ the higgs mass, and $m_W$ the W mass. A correction to the higgs mass can be calculated using standard perturbation theory by evaluating the second term of the Higgs Lagrangian. (cite baer)
\begin{equation}
\begin{split}
\Delta m_h^2 = \langle H | \frac{g^2m_h^2}{32M_W^2} H^4 | H  \rangle = 12\frac{g^2m_h^2}{32M_W^2}\int \frac{d^4 k}{(2\pi)^2} \frac{i}{k^2 - m_h^2}\\
= 12\frac{g^2m_h^2}{32M_W^2} \frac{1}{16\pi^2}\big( \Lambda^2 - m_h^2\log\frac{\Lambda^2}{m_h^2} + O(\frac{1}{\Lambda^2})\big)
\end{split} 
\end{equation}
 
 Here the intergal term is the propagator (cite propagator stuff??) for the exchange of a virtual Higgs and integrated phase space. The $\Lambda$ is known as the scale cutoff parameter and should be interpreted as the scale at which the SM breaks down, possibly near the planck scale $O(10^{19})$GeV. Notice the leading term $\Lambda^2$ which indicates a that the expansion is quadratically divergent. This divergence means there would need need to be extremely large finely tuned cancellations, around 20 orders of magnitude, to maintain $\Delta m_h \propto O(m_h)$. This divergent phenomenon can also be observed with fermion masses, but, chiral symmetry protects the fermion mass from divergence by cancelling out high order $\Lambda$ terms. SUSY offers a similar protection to the Higgs mass by introducing a symmetry with the additional fermionic and bosonic degrees of freedom leading to natural structure for the higgs boson. 


\subsection{The Muon Anomalous Magnetic Moment}

A very interesting experimental motivation for SUSY lies within the measurement of the muon anomalous magnetic moment, multiple meausurements at two different labs BNL and FNAL have shown significant disagreement with the standard model. These experiments measure the muon $g$ factor or specicially its deviation from two, $(g-2)_\mu$ .  The $g$ factor is related to the electromagnetic coupling of charged particles to a photon. The factor largely depends on the tree level lepton-photon coupling but gets small quantum corrections from higher order loops, the largest being the single photon loop or Schwinger term shown in Figure X. The SM calculation of the $g$ factor includes three types of corrections -- QED, Electroweak, and Hadronic. Corrections due to the Higgs are neglected due to the mass disparity $m_h >> m_{e,\mu}$ and the mass dependence in the Higgs coupling which has effects that are smaller than what is experimentally observable. To first order in QED, the g-factor is exactly 2, when accounting for quantum corrections the g-factor deviates very slightly from 2. Experimentally the deviations from 2 are the most interesting, and are written in the form $a_\ell = \frac{g-2}/2$ and referred to as $(g-2)_\ell$. These small contribtutions are interesting because they encapsulate the current theory and provide a test bed for our current understanding.  If observations were to deviate from the SM prediction, it would be an indication of new and unaccounted physics interactions with the SM leptons. The $g$ factor can be extracted by measuring the anomalous magnetic moment of any generation of charged lepton.  The current best candidate to both test the SM and search for new physics is by measuring $(g-2)_\mu$ or $a_\mu$ because of  experimental precision potential. The electron measurement is already known to the highest precision and is expected to have the smallest contributions from new physics (cite youtube citation). The $(g-2)_\tau$ is not yet experimentally tractable competitive precision to $\mu$ or $e$.%so $(g-2)_\mu$ has been measured at both at Brookhaven National Lab (BNL) and again at Fermi National Accelerator Laboratory (FNAL).
The currently accepted best SM prediction of $a_\mu$ from (CITE g-2 collab) includes QED, Electroweak(EW) and Hadronic contributions and is reported as $a_\mu^{SM} = a_\mu^{QED}+ a_\mu^{EW}+a_\mu^{\text{Hadronic}} = 116 591 810(43) \times 10^{-11}$. For each of the $a_\mu$ components, the QED compenent enters at the $O(10^{-3})$ and is known to $O(10^{-11})$. the EW component enters the sum at $O(10^{-9})$ and is known to $O(10^{-10})$. Finally the most complicated component, hadronic, contributes at $O(10^{-8})$ and is known up to $O(10^{-9})$, the  main sub components that contribute to the $a_\mu^{\text{Hadronic}}$ is the Hadronic vacuum polarization and light by light scattering, diagrams illustrated in Figure X. The hadronic precision is constrained by data driven measurements and computation approaches -- QCD lattice theory, this error dominates the overall uncertainty of $a_\mu$. The BNL measurement of $a_\mu$ yields a difference with the SM prediction of $\Delta a_\mu := a_\mu^{BNL} - a_\mu^{SM} = 279(76) \times 10^{-11}$ which is a significance of $3.7\sigma$. The most recent $a_\mu$ measurement from FNAL confirms the BNL measurement within $1\sigma$ and the combined experimental average increases the SM deviation with a significance of $4.2\sigma$.
The standard model is a consistent and remarkable theory that holds over many orders of magnitude in energy

What could this deviation mean?
The $4.2\sigma$ significance is a compelling sign for potential new physics, but can be somewhat explained by improvements in QCD lattice calculations of the HVP and LBL contributions. There is also a calculation which resolves this tension up to $1.2\sigma$ (cite BMW). The more interesting explanation is that the tension could be due to effects of new particles. There are several models which can quantify the $a_\mu$ SM deviation, those being Super Symmetry, Dark Matter (DM mediator Dark photon), Lepto quarks, 2 Higgs doublet models.

%g-2 is an experiment designed to measure the anomolaus magnetic dipole moment of the muon. The spin magnetic moment of a charged, spin-1/2 particle that does not possess any internal structure (a Dirac particle) is given by (wiki direct quote \url{https://en.wikipedia.org/wiki/G-factor_(physics)}) ${\displaystyle {\boldsymbol {\mu }}=g{e \over 2m}\mathbf {S} }$. where g is the particles g-factor, $\mu$ is the magetic moment, $m$ is the particle mass and $S$ is the spin. The g-factor in quantum electrodynamics is close to 2 so typically the reported measurement is the difference from 2 or g-2 or as a signficance $a_\mu = g-2/2$. The difference from 2 arises from higher order contributions in quantum field theory

\subsection{W mass measusremnt}
the most recent w mass measurement yielded a heavy W, this higher mass is more favorable for light higgsino and compressed susy models

What is the W boson
The W boson is an very important and peculiar particle, it is the electrically charged boson that mediates flavor change but only with left handed particles.  The mass of the W-boson underpins many important parameters in the SM. In fact, $m_W$ is related to the vacuum expectation value $v$ of the Higgs, this implies that scale of mass depedent coupling of the higg field to all particles is tuned by the mass of the W. Similary the W mass is related to the $g$ factor from $(g-2)_\ell$ 

What is the current status of the W boson?
Since there is interdependence of many paramters such as $v$, $m_z$ there is no exact SM prediction of the W mass, rather a value that is constrained by experimatally measured parameters. at tree level $m_W$ can be parameterized with $m_Z$ which can be precisely measured in the visible leptonic and hadronic modes, $G_F$ or the fermi constant which is also related to $v$ and can be measured precisely with muon lifetime, and $\alpha$ the fine structure constant (cite sm w paper). The most recent expereimental measurment of $m_W$ was performed by CDF II at the tevatron. The mass was obtained by fitting the kinematic distributions from light leptonic decays recoiling against a system of jets. This measurement is $50\%$ more precise than the previous measurement by ATLAS (cite atlas) and heavier than the SM prediction. The combination deviation and precision results in a $7\sigma$ signficance with the SM (cite CDFII).  


What could this deviation mean?
Possible early sign of new physics would be slight inconsistencies between measurements and different SM obervables, and the $7\sigma$ deviation of the W mass, if correct, is a very strong indicator of new physics. There are a slew of SUSY models that could explain the excessive mass of the W boson. (CDF lightsusy) In general, a slightly heavier W favors light SUSY models. This means electro weak scale SUSY particles in the cases of light wino/bino and higgsino models, many of which favor compressed mass scenraios. Due to the interdependence of $m_W$ and $(g-2)_\mu$, these parameters both constraint compressed SUSY and spotlight a critical area to search. An example of model points of sleptons and gaug-ino models which satisfy the newest $g-2$ and W mass constraints in shown in FIGURE Z (cite Wmass and g-2 sven paper)


\section{The current status of SUSY}
drop the most recent limits here, start with multi TeV excluded gluino and squark models which leaves the a good place to search in the weak scale sector with electoweakinos. Talk about electroweak limits and how alot of these are excluded already one of the remaining places to search is the compressed corridor where mass splittings are small. link this limit motivation with how both g-2 and W mass favor compressed scenarios

There have been many searches for SUSY particles from starting with LEP and still ongoing at the LHC today. As of yet there is no evidence for SUSY particles but there is still plenty of room to keep searching. The currently most stringent limits are in the gluino/squark sector because of large expected cross sections compared to gauginos/sleptons. These simplified models have been excluded up to X GeV masses. The most recent limits are shown in fig XYZ.  The current best slepton and electroweak limits are shown in Figure ZYZ. Note that the electro weak limits are just reaching the TeV scale while SUSY remains valid at the few TeV level. Specifically the weakest limits are in the compressed regions for electroweakinos and sleptons. 



