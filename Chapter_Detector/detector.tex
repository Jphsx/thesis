\setcounter{secnumdepth}{3}
\setcounter{tocdepth}{3}
\setlength{\parskip}{\smallskipamount}
\setlength{\parindent}{0pt}


\makeatletter


\providecommand{\tabularnewline}{\\}


\makeatother


\chapter{The CMS experiment}

\section{Introduction} The Compact Muon Solenoid experiment consists of a detector which is a part of the Large Hadron Collider (LHC). The detector encapsulates two synchronous bunches of high energy protons  which counter rotate through the LHC accelerator ring.  Protons collide at the center of the detector with a large energy and the expectation that more massive and potentially new particles can be produced. Each particle produced in the collision can either decay, interact, or escape detection. The particles that interact are measured by the detector, where different detector layers specialize in measuring certain classes of particles. Then, from energy and momentum measurements, the initial interaction and everything in between is reconstructed.   

%Here talk about accelrator concept + hardware, detector concept + hardware, the anatomy of physics events the interpretation of the events through measureed observables, 
%we accelerate particles to collided them and produce new particles


\section{The Large Hadron Collider}
The LHC is a 27 km circumference circular collider designed to collide proton beams with a center-of-mass energy of 14 TeV and an instantaneous luminosity of $10^{34} \, \text{cm}^{-2}\text{s}^{-1}$\cite{Evans:2008zzb}. The main accelerator ring consists of two counter rotating proton beams, which are encased in an ultra high vacuum to prevent unintended interactions. The beams are accelerated with cryogenic electro-magnets which operate at $-273$ \degree C and are cooled by liquid helium. There are two types of magnets present, 1232 dipole magnets which bend the beam around the ring and 392 quadrapole magnets which focus the beams.  The beam itself is structured with proton bunches, with each bunch spaced $25$ ns apart and about 2800 bunches per beam. The periods of recording collision data are referred to as runs, of which there are two completed runs denoted as Run I and Run II. The runs have integrated luminosities of $58 \, \text{fb}^{-1}$ and $138\, \text{fb}^{-1}$ with operational center of mass energies of 8 TeV and 13 TeV, respectively. There is also an expected cumulative integrated luminosity of up to $400 \,\text{fb}^{-1}$ which includes Run I, Run II, and the presently ongoing Run III. Around the accelerator ring there are 4 detectors at different points, CMS, ATLAS, ALICE, and LHCb shown in Figure \ref{fig:lhcmachine}.

\FigOneScale{Detector_figs/LHCMachine.png}{Layout of the LHC with the red clockwise beam and blue counter clockwise beam. The beams collide at the interaction regions of the four experiments: CMS, ATLAS, ALICE, and LHCb. \cite{Evans:2008zzb}}{fig:lhcmachine}{0.5}
%%TODO add bunch structure 23ns info

%machine layout and goals: \\
%rings magnets, bends beam, quadrapole focuses beam, rf cavities add energy to beam


\section{The CMS Detector}

The CMS detector is a hermetic shell that surrounds the interaction point of the two proton beams. The beams collide at the center of the detector illustrated in Figure \ref{fig:det}. and produce outgoing showers of particles that travel into the detector. The observable outgoing particles, depending on the type of particle, are then measured in one of the specialized concentric layers of the detector. The initial transverse depiction of sub atomic interaction, and intermediate particles, can then be reconstructed from the energy and momentum measured in the detector. The total longitudinal momentum is not reconstructable for two reasons: first being that the momentum fraction of the initial partons is unknown and second is that some particles travel along the beam line outside the detector acceptance.  There is also an abundance of collisions seen by the detector that are not recorded. Instead, interesting events, say due to the presence of a muon or large missing energy, trigger the detector to take a snapshot resulting in a permanent record of the event.
\FigOneScale{Detector_figs/detector_old.png}{Cross-sectional view of the CMS detector and main detector components \cite{CMS:2008xjf}}{fig:det}{0.7}

The coordinate system adopted for CMS is right-handed with the z axis pointing along the beam line and x axis pointing toward the center of the ring. The origin is defined as the nominal collision point. The angular coordinates are defined as $\phi$ in the x-y plane, the polar angle $\theta$ measured from the z axis, and pseudo-rapidity $\eta =-\log(\theta/2)$. The momentum and energy transverse to the beam direction are denoted as $p_T$ and $E_T$ and are computed from the x and y components \cite{CMS:2017lum}. 

A particle's traversal through the detector, as shown in Figure \ref{fig:cmsdet}, is as follows: Particles are produced post-collision at a primary interaction point, or primary vertex. Other interactions can occur nearby in the same snapshot and are denoted as pile-up, a form of noise obfuscating the primary interaction. From either primary or secondary vertices, both charged and neutral particles traverse the first region of the detector, the silicon tracker. The vertex selected as the primary vertex, from the set of approximately simultaneously created vertices, is the vertex with the largest scalar sum of transverse momentum from tracks consistent with that vertex. The tracker consists of concentric thin silicon sensors with coverage provided in the barrel up to $|\eta|<1.2$ and end-cap sensors which extend the acceptance range to $|\eta| < 2.5$. The tracker is also divided into an inner pixel tracker and outer silicon strip tracker. The inner pixel tracker is designed to have high spatial resolution to determine particle origins while being radiation tolerant. The silicon strip tracker complements the pixel tracker by adding significantly more coverage but with less spatial resolution. All tracker sensors  register hits from only charged particles with the sequence of hits being connected into a track that represents the path and origin of the charged particle.  The next stop is the Electromagnetic Calorimeter (ECAL), which consists of scintillating $\text{PbWO}_4$ crystals that are designed to stop and measure the energy deposits of photon and electrons. These different energy deposits are distinguished by tracks that seed ECAL super clusters. Particles that do not stop inside the ECAL, encounter the hadronic calorimeter (HCAL). The HCAL consists of brass and plastic scintillators that stop the remaining massive particles and measure their energy. The ECAL and HCAL acceptance extends to $|\eta| <3.0$. The last two regions of the detector are only seen by muons and are the centerpieces of CMS. First is the solenoidal magnet, which generates a 3.8 T uniform magnetic field throughout all of the inner regions of the detector. The magnetic field allows the measurement of two important observables: charge and momentum.  A charged particle's path will bend in the presence of a magnetic field, and the clockwise or counter clockwise trajectory indicates the charge, while the curvature of the bend determines the momentum. The outer-most part of the detector is the muon chambers, which similar to the tracker, register a sequence of hits via drift tubes or cathode strips. The tracks in both the tracker and muon chambers can then be combined to precisely measure the momentum of the muon. The muon chamber also has interspersed resistive plate chambers which act as a hardware level muon trigger. The tracker performance has an efficiency above $99\%$ for muons with $p_T > 1$ GeV and momentum resolution of about $2\%$ for charged tracks in the barrel with $p_T < 100$ GeV. The particles which register in the ECAL and HCAL, have their the energy deposits clustered into jets. All particles and jets are then reconstructed according to the particle flow method \cite{CMS:2017yfk} \cite{CMS:2008xjf} \cite{CMS:2017lum}.

\FigOneScale{Detector_figs/CMSParticleDetectionSummary.png}{A Transverse slice of the CMS detector that illustrates the components that specialize in detecting specific particles \cite{CMS:2017yfk}.}{fig:cmsdet}{0.85}



%\section{The Anatomy of a physics event}
%	Reconstruction of particles
%	observables
