\setcounter{secnumdepth}{3}
\setcounter{tocdepth}{3}
\setlength{\parskip}{\smallskipamount}
\setlength{\parindent}{0pt}


\makeatletter


\providecommand{\tabularnewline}{\\}


\makeatother


\chapter{The CMS experiment}

\section{Introduction} The Compact Muon Solenoid (CMS) experiment consists of a detector housed at the Large Hadron Collider (LHC). Two synchronous bunches of high energy protons counter rotate through the LHC accelerator ring and collide at a center point in the CMS detector.  The protons are collided with a significantly large energy with the expectation that more massive and potentially new particles can be produced. The intermediate particles decay or interact and can then measured by the detector, where different regions of the detector specialize in the measurement of specific signatures or features of different particles. The overall p-p collision is then reconstructed or essentially reverse engineered through final state particles interpeted through observable quantities such as energy and momentum.   

Here talk about accelrator concept + hardware, detector concept + hardware, the anatomy of physics events the interpretation of the events through measureed observables, 
we accelerate particles to collided them and produce new particles


\section{The Large Hadron Collider}
The LHC is designed to collide proton beams with a centre-of-mass energy of 14 TeV and an instantaneous luminosity of $1034 \text{cm}^{-2}\text{s}^{-1}$.(cite lhc paper direct quote). The main accelerator ring consists of two counter rotating proton beams which are incased in an ultra high vacuum to prevent interactions. The beams are accelrated with cryogenic magnets which consist of dipole and quadrapole magnets

machine layout and goals: \\
rings magnets, bends beam, quadrapole focuses beam, rf cavities add energy to beam


\section{The CMS Detector}


\section{The Anatomy of a physics event}
	Reconstruction of particles
	observables
