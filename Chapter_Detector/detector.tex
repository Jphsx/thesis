\setcounter{secnumdepth}{3}
\setcounter{tocdepth}{3}
\setlength{\parskip}{\smallskipamount}
\setlength{\parindent}{0pt}


\makeatletter


\providecommand{\tabularnewline}{\\}


\makeatother


\chapter{The CMS experiment}

\section{Introduction} The Compact Muon Solenoid (CMS) experiment consists of a detector housed at the Large Hadron Collider (LHC). Two synchronous bunches of high energy protons counter rotate through the LHC accelerator ring and collide at a center point in the CMS detector.  The protons are collided with a significantly large energy with the expectation that more massive and potentially new particles can be produced. The intermediate particles decay or interact and can then measured by the detector, where different regions of the detector specialize in the measurement of specific signatures or features of different particles. The overall p-p collision is then reconstructed or essentially reverse engineered through final state particles interpeted through observable quantities such as energy and momentum.   

%Here talk about accelrator concept + hardware, detector concept + hardware, the anatomy of physics events the interpretation of the events through measureed observables, 
we accelerate particles to collided them and produce new particles


\section{The Large Hadron Collider}
The LHC is a circular collider designed to collide proton beams with a centre-of-mass energy of 14 TeV and an instantaneous luminosity of $1034 \text{cm}^{-2}\text{s}^{-1}$.(cite lhc paper direct quote). The main accelerator ring consists of two counter rotating proton beams which are incased in an ultra high vacuum to protect the beam from interactions. The beams are accelerated with cryogenic electro-magnets which operate at -273C and are cooled by liquid helium. There are two types of magnets present, 1232 dipole magnets which bend the beam around the ring and 392 quadrapole magnets which focus the beams.the beams are structured etc bunches TALK AbOUt BEAMS HERE. Currently there are two completed data taking periods at the LHC denoted as Run I, and Run II. The integrated luminosity respectively per run is $58\text{fb}^{-1}$ and $138\text{fb}^{-1}$ with an expected cumulative integrated luminosity of up to $500\text{fb}^{-1}$ Including the presently ongoing Run III.  

%%TODO add bunch structure 23ns info

%machine layout and goals: \\
%rings magnets, bends beam, quadrapole focuses beam, rf cavities add energy to beam


\section{The CMS Detector}

The CMS detector is a hermetic shell that encapsulates the two counter rotating proton beams. The beams collide at the center of the detector and produce outgoing showers of particles that travel transverse to the beam axis. The observable outgoing particles, depending on the type of particle, are then measured in one of the specialized concentric layers of the detector. The initial transverse depiction of p-p interaction and intermediate particles can then be reconstructed from the energy measured in the detector. The total final state energy longitudinal to the beam axis is unknown because the momentum fraction of the initial quarks is unknown as well as some final state particles travel outside detector acceptance along the beam-line.  The detector also not generally record every collision but works on a triggering system meaning that the detector records and event and if the event is sufficiently interesting, say due the presence of a muon, a snap shot of the event is taken and the event is permanently recorded.

The reconstruction story of a particle traversing the detector is as follows. Primary particles are produced at a primary interaction point, other less energetic p-p interactions can occur in the same detector snapshot and are considered the underlying-event or pile-up which could be considered as a form of noise obfuscating the primary interaction. Both Charged and neutral particles traverse the first region of the detector, the silicon tracker. The silicon tracker consists of concentric thin electronic sensors that register "hits" from only charged particles. Each sequence of hits can be connected into a track which is a reconstruction of the path and origin of the charged particle.  The next region encounted by traveling particles is the Electromagnetic Calorimeter (ECAL). In the ECAL consists of scintillating pbW04 crystals that stop electrons and photons and measure their deposited energy. The energy deposits from photons and electrons are distinguished by tracks that connect to ECAL showers. Following the ECAL, is the hadronic calorimeter (HCAL) which consists of brass and plastic scintillators. The HCAL stops the heavy, remaining particles which are yet to interact and measures their energy. The last two regions of the detector are generally only traversed by the muon and a the centerpieces of CMS. The muons first stop would be through the solenoidal magnet, which generates a 4 Tesla uniform magnetic field throughout all of the inner regions of the detector. The magnetic field serves two purposes, first charge particles path bends in the presence of a magnetic field and that bend is either clockwise or counter clockwise around the beam axis depending on the charge, and thereby distinguishing the charge of the particle.  Secondly the momentum of tracks is measured their radius of curvature, or how much they bend in the magnetic field. The final stop for a muon is the muon chamber, which similar to the tracker, registers a sequence of hits via drift tubes or cathode strips. The tracks in both the tracker and muon chambers can then be combined to precisely measure the momemntum of the muon. resistive plate chambers reduntant triggering system 




%\section{The Anatomy of a physics event}
%	Reconstruction of particles
%	observables
